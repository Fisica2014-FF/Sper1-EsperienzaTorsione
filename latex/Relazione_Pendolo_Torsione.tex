\documentclass[10pt,a4paper]{article} % Prepara un documento con un font grande

\input{./preamboli_e_stili/pacchetti.tex}

\DeclareGraphicsExtensions{.pdf, .png, .jpg} % Se due immagini hanno lo stesso nome sceglile secondo l'ordine di filetype qui
\graphicspath{ {./img/} }					 % Path delle immagini 

\input{./preamboli_e_stili/titolo_Pendolo_Torsione.tex}
\input{./preamboli_e_stili/stili_float.tex}


%////////////////////////////////////////////////////////////////////////////////////////////////////////////////////////////
%////////////////////////////////////////////////////////////////////////////////////////////////////////////////////////////
% Fine dei dati iniziali per il latex: il documento finale inizierà da qui
\begin{document}

\maketitle % Produce il titolo a partire dai comandi \title, \author e \date

\vspace{ \stretch{1} }
\begin{center}
	\includegraphics[width=0.8\textwidth]{risonanza}
\end{center}

\vspace{ \stretch{1} }
% Le varie sezioni
%\section{Obiettivi}
\begin{abstract}
	\noindent
	\input{./sezioni/obiettivi.tex}
\end{abstract}

\newpage



\microtypesetup{protrusion=false} % disables protrusion locally in the document
\tableofcontents % prints Table of Contents
\microtypesetup{protrusion=true} % enables protrusion

%\begin{multicols}{2}

\section{Apparato strumentale}
	L’apparato strumentale consiste in un cilindro in plexiglass al cui interno è posto un peso in acciaio di massa: $(115.5 /pm 0.1) g$
 e diametro: $(22.7 /pm 0.1)mm$ e altezza: $(34.0 /pm 0.1)mm$ collegato ad un filo di acciaio armonico, materiale dotato discrete
 capacità elastiche, ed immerso in un liquido. Il filo è inoltre collegato ad una piattaforma rotante azionata da un motore di
 diametro circa 8cm che, una volta azionato, induce un’oscillazione forzata del corpo formato dal filo più il pesetto. Il range delle
 frequenze alla quale il motore può essere indotto ad oscillare risulta compreso tra 0.8 e 1.2 Hz. Di suddetta oscillazione è
 possibile modificare il periodo e l’ampiezza. Il tutto viene controllato e registrato mediante l’interfaccia fornita da un computer.
 I dati vengono acquisiti in intervalli di 0.05 secondi permettendo una frequenza di rilevamento di 20 dati per secondo.
 L’interfaccia permette di visualizzare valori della frequenza, dell’ampiezza e della fase (espressa in gradi). Inoltre sono presenti
 diversi grafici che mostrano, in tempo reale, la curva di risonanza, il moto del pesetto, segnandone la posizione. Inoltre la
 presenza del pulsante offset permetteva di tarare l’apparato dopo ogni misurazione, al fine di limitare errori sistematici.




\section{Metodologia di misura}
	Per poter stimare la frequenza di risonanza si è proceduto azionando il motore e mettendo in oscillazione la piattaforma rotante.
Partendo dalle informazioni fornite e dall’apparecchiatura si è deciso di porre l’ampiezza a 10 millesimi di giro e di variare gli
intervalli delle frequenze di 0.020 Hz e acquisendo campioni di misure per un tempo totale di circa 10 secondi durante la fase a
regime, interrompendo la misurazione per eseguire lo store dei dati così ricavati, spegnendo il motore e intraprendendo una seconda
fase di registrazione dati per una durata di circa 20 secondi per la fase di smorzamento, al termine del quale è stato  eseguito
nuovamente lo store dei dati. Le misure sonos tate prese solo non appena è stato evidente dai grafici di riferimento il carattere
periodico del moto del pendolo. 

L'apparato strumentale è stato ricalibrato prima delle prese dati della giornata per permetterne un
funzionamento ottimale. L' ampiezza massima di oscillazione della forzante è stata scelta per permettere oscillazioni abbastanza
ampie da studiare ma non così ampie da rendere caotico il moto del pendolo, costringendolo a muoversi sul piano perpendicolare
all'asse di oscillazione. Attraverso questo metodo è stato possibile ottenere una panoramica del comportamento oscillatorio del 
materiale di studio e identificare efficacemente il settore in cui avveniva il fenomeno di risonanza. Tale settore è stato poi 
sondato ricorrendo al metodo di bisezione restringendosi in un intorno di valori della frequenza e aumentando l’esposizione 
dell’acquisizione dati. 

In questa fase (che si concentra nell’intervallo tra 0.965 e 0.970 Hz) sono stati registrati valori per la durata di circa 100
secondi per la fase a regime e di 40 secondi circa per la fase in smorzamento, una volta spento il motore. Questo offre agli
sperimentatori la possibilità di analizzare una serie di campioni più concentrati avente intervalli di frequenza di 0.001 Hz,
permettendo di stimare il più efficacemente possibile la frequenza di risonanza.


%\newpage
\section{Presentazione dei dati}			
	\subsection{Tabelle}
	%\begin{multicols}{2}
	%\begin{tabella}
%	\centering
%	\input{./tabelle/dati_temperature.tex}
%	\caption{Temperature delle isoterme}
%	\label{tab:ad}
%\end{tabella}

	%\end{multicols}
	\clearpage
	\subsection{Grafici}
	%\begin{grafico}
%  \centering
%\begin{tikzpicture}[gnuplot]
%% generated with GNUPLOT 4.6p3 (Lua 5.1; terminal rev. 99, script rev. 100)
%% mar 27 mag 2014 18:54:53 CEST
\path (0.000,0.000) rectangle (12.500,8.750);
\gpcolor{color=gp lt color axes}
\gpsetlinetype{gp lt axes}
\gpsetlinewidth{1.00}
\draw[gp path] (1.196,0.763)--(11.947,0.763);
\gpcolor{color=gp lt color border}
\gpsetlinetype{gp lt border}
\draw[gp path] (1.196,0.763)--(1.376,0.763);
\draw[gp path] (11.947,0.763)--(11.767,0.763);
\node[gp node right] at (1.012,0.763) { 0.4};
\gpcolor{color=gp lt color axes}
\gpsetlinetype{gp lt axes}
\draw[gp path] (1.196,1.759)--(11.947,1.759);
\gpcolor{color=gp lt color border}
\gpsetlinetype{gp lt border}
\draw[gp path] (1.196,1.759)--(1.376,1.759);
\draw[gp path] (11.947,1.759)--(11.767,1.759);
\node[gp node right] at (1.012,1.759) { 0.6};
\gpcolor{color=gp lt color axes}
\gpsetlinetype{gp lt axes}
\draw[gp path] (1.196,2.755)--(11.947,2.755);
\gpcolor{color=gp lt color border}
\gpsetlinetype{gp lt border}
\draw[gp path] (1.196,2.755)--(1.376,2.755);
\draw[gp path] (11.947,2.755)--(11.767,2.755);
\node[gp node right] at (1.012,2.755) { 0.8};
\gpcolor{color=gp lt color axes}
\gpsetlinetype{gp lt axes}
\draw[gp path] (1.196,3.751)--(11.947,3.751);
\gpcolor{color=gp lt color border}
\gpsetlinetype{gp lt border}
\draw[gp path] (1.196,3.751)--(1.376,3.751);
\draw[gp path] (11.947,3.751)--(11.767,3.751);
\node[gp node right] at (1.012,3.751) { 1};
\gpcolor{color=gp lt color axes}
\gpsetlinetype{gp lt axes}
\draw[gp path] (1.196,4.748)--(11.947,4.748);
\gpcolor{color=gp lt color border}
\gpsetlinetype{gp lt border}
\draw[gp path] (1.196,4.748)--(1.376,4.748);
\draw[gp path] (11.947,4.748)--(11.767,4.748);
\node[gp node right] at (1.012,4.748) { 1.2};
\gpcolor{color=gp lt color axes}
\gpsetlinetype{gp lt axes}
\draw[gp path] (1.196,5.744)--(11.947,5.744);
\gpcolor{color=gp lt color border}
\gpsetlinetype{gp lt border}
\draw[gp path] (1.196,5.744)--(1.376,5.744);
\draw[gp path] (11.947,5.744)--(11.767,5.744);
\node[gp node right] at (1.012,5.744) { 1.4};
\gpcolor{color=gp lt color axes}
\gpsetlinetype{gp lt axes}
\draw[gp path] (1.196,6.740)--(11.947,6.740);
\gpcolor{color=gp lt color border}
\gpsetlinetype{gp lt border}
\draw[gp path] (1.196,6.740)--(1.376,6.740);
\draw[gp path] (11.947,6.740)--(11.767,6.740);
\node[gp node right] at (1.012,6.740) { 1.6};
\gpcolor{color=gp lt color axes}
\gpsetlinetype{gp lt axes}
\draw[gp path] (1.196,7.736)--(4.591,7.736);
\draw[gp path] (11.763,7.736)--(11.947,7.736);
\gpcolor{color=gp lt color border}
\gpsetlinetype{gp lt border}
\draw[gp path] (1.196,7.736)--(1.376,7.736);
\draw[gp path] (11.947,7.736)--(11.767,7.736);
\node[gp node right] at (1.012,7.736) { 1.8};
\gpcolor{color=gp lt color axes}
\gpsetlinetype{gp lt axes}
\draw[gp path] (1.514,0.616)--(1.514,8.381);
\gpcolor{color=gp lt color border}
\gpsetlinetype{gp lt border}
\draw[gp path] (1.514,0.616)--(1.514,0.796);
\draw[gp path] (1.514,8.381)--(1.514,8.201);
\node[gp node center] at (1.514,0.308) { 0.02};
\gpcolor{color=gp lt color axes}
\gpsetlinetype{gp lt axes}
\draw[gp path] (2.779,0.616)--(2.779,8.381);
\gpcolor{color=gp lt color border}
\gpsetlinetype{gp lt border}
\draw[gp path] (2.779,0.616)--(2.779,0.796);
\draw[gp path] (2.779,8.381)--(2.779,8.201);
\node[gp node center] at (2.779,0.308) { 0.04};
\gpcolor{color=gp lt color axes}
\gpsetlinetype{gp lt axes}
\draw[gp path] (4.043,0.616)--(4.043,8.381);
\gpcolor{color=gp lt color border}
\gpsetlinetype{gp lt border}
\draw[gp path] (4.043,0.616)--(4.043,0.796);
\draw[gp path] (4.043,8.381)--(4.043,8.201);
\node[gp node center] at (4.043,0.308) { 0.06};
\gpcolor{color=gp lt color axes}
\gpsetlinetype{gp lt axes}
\draw[gp path] (5.307,0.616)--(5.307,6.969);
\draw[gp path] (5.307,8.201)--(5.307,8.381);
\gpcolor{color=gp lt color border}
\gpsetlinetype{gp lt border}
\draw[gp path] (5.307,0.616)--(5.307,0.796);
\draw[gp path] (5.307,8.381)--(5.307,8.201);
\node[gp node center] at (5.307,0.308) { 0.08};
\gpcolor{color=gp lt color axes}
\gpsetlinetype{gp lt axes}
\draw[gp path] (6.571,0.616)--(6.571,6.969);
\draw[gp path] (6.571,8.201)--(6.571,8.381);
\gpcolor{color=gp lt color border}
\gpsetlinetype{gp lt border}
\draw[gp path] (6.571,0.616)--(6.571,0.796);
\draw[gp path] (6.571,8.381)--(6.571,8.201);
\node[gp node center] at (6.571,0.308) { 0.1};
\gpcolor{color=gp lt color axes}
\gpsetlinetype{gp lt axes}
\draw[gp path] (7.836,0.616)--(7.836,6.969);
\draw[gp path] (7.836,8.201)--(7.836,8.381);
\gpcolor{color=gp lt color border}
\gpsetlinetype{gp lt border}
\draw[gp path] (7.836,0.616)--(7.836,0.796);
\draw[gp path] (7.836,8.381)--(7.836,8.201);
\node[gp node center] at (7.836,0.308) { 0.12};
\gpcolor{color=gp lt color axes}
\gpsetlinetype{gp lt axes}
\draw[gp path] (9.100,0.616)--(9.100,6.969);
\draw[gp path] (9.100,8.201)--(9.100,8.381);
\gpcolor{color=gp lt color border}
\gpsetlinetype{gp lt border}
\draw[gp path] (9.100,0.616)--(9.100,0.796);
\draw[gp path] (9.100,8.381)--(9.100,8.201);
\node[gp node center] at (9.100,0.308) { 0.14};
\gpcolor{color=gp lt color axes}
\gpsetlinetype{gp lt axes}
\draw[gp path] (10.364,0.616)--(10.364,6.969);
\draw[gp path] (10.364,8.201)--(10.364,8.381);
\gpcolor{color=gp lt color border}
\gpsetlinetype{gp lt border}
\draw[gp path] (10.364,0.616)--(10.364,0.796);
\draw[gp path] (10.364,8.381)--(10.364,8.201);
\node[gp node center] at (10.364,0.308) { 0.16};
\gpcolor{color=gp lt color axes}
\gpsetlinetype{gp lt axes}
\draw[gp path] (11.629,0.616)--(11.629,6.969);
\draw[gp path] (11.629,8.201)--(11.629,8.381);
\gpcolor{color=gp lt color border}
\gpsetlinetype{gp lt border}
\draw[gp path] (11.629,0.616)--(11.629,0.796);
\draw[gp path] (11.629,8.381)--(11.629,8.201);
\node[gp node center] at (11.629,0.308) { 0.18};
\draw[gp path] (1.196,8.381)--(1.196,0.616)--(11.947,0.616)--(11.947,8.381)--cycle;
\node[gp node right] at (10.479,8.047) {'TEST_isobara2' using (1.0/$3):($1)};
\gpcolor{color=gp lt color 0}
\gpsetlinetype{gp lt plot 0}
\draw[gp path] (10.663,8.047)--(11.579,8.047);
\draw[gp path] (7.749,5.943)--(4.960,5.943)--(4.957,5.943)--(6.778,5.943)--(5.876,5.943)%
  --(6.139,5.943)--(6.691,5.943)--(6.697,5.943)--(6.690,5.943)--(5.705,5.943)--(6.665,5.943)%
  --(6.909,5.943)--(6.927,5.943)--(6.906,5.943)--(6.915,5.943)--(6.964,5.943)--(6.958,5.943)%
  --(6.088,5.943)--(6.091,5.943)--(8.074,5.943)--(8.073,5.943)--(8.072,5.943)--(8.073,5.943)%
  --(6.482,5.943)--(6.488,5.943)--(7.433,5.943)--(7.757,5.948)--(4.965,5.948)--(6.797,5.948)%
  --(6.793,5.948)--(6.783,5.948)--(6.804,5.948)--(6.801,5.948)--(5.879,5.948)--(5.881,5.948)%
  --(6.144,5.948)--(5.706,5.948)--(5.705,5.948)--(6.672,5.948)--(6.680,5.948)--(6.942,5.948)%
  --(6.959,5.948)--(6.970,5.948)--(6.091,5.948)--(8.079,5.948)--(8.096,5.948)--(8.084,5.948)%
  --(6.489,5.948)--(7.455,5.948)--(7.447,5.948)--(7.478,5.948)--(7.463,5.948)--(4.970,5.953)%
  --(6.809,5.953)--(5.885,5.953)--(6.150,5.953)--(6.702,5.953)--(5.711,5.953)--(6.966,5.953)%
  --(6.971,5.953)--(6.983,5.953)--(6.099,5.953)--(8.096,5.953)--(6.493,5.953)--(7.477,5.953)%
  --(4.975,5.958)--(5.892,5.958)--(5.887,5.958)--(6.155,5.958)--(6.706,5.958)--(5.717,5.958)%
  --(5.720,5.958)--(6.982,5.958)--(6.998,5.958)--(6.120,5.958)--(6.108,5.958)--(8.109,5.958)%
  --(8.105,5.958)--(7.477,5.958);
\gpcolor{color=gp lt color border}
\node[gp node right] at (10.479,7.739) {'TEST_isobara1' using (1.0/$3):($1)};
\gpcolor{color=gp lt color 1}
\gpsetpointsize{4.00}
\gppoint{gp mark 2}{(4.687,4.499)}
\gppoint{gp mark 2}{(4.644,4.499)}
\gppoint{gp mark 2}{(4.640,4.499)}
\gppoint{gp mark 2}{(4.160,4.499)}
\gppoint{gp mark 2}{(4.155,4.499)}
\gppoint{gp mark 2}{(4.159,4.499)}
\gppoint{gp mark 2}{(5.096,4.499)}
\gppoint{gp mark 2}{(5.095,4.499)}
\gppoint{gp mark 2}{(5.095,4.499)}
\gppoint{gp mark 2}{(4.269,4.499)}
\gppoint{gp mark 2}{(4.270,4.499)}
\gppoint{gp mark 2}{(5.253,4.499)}
\gppoint{gp mark 2}{(4.545,4.499)}
\gppoint{gp mark 2}{(3.835,4.499)}
\gppoint{gp mark 2}{(3.834,4.499)}
\gppoint{gp mark 2}{(3.832,4.499)}
\gppoint{gp mark 2}{(6.466,4.499)}
\gppoint{gp mark 2}{(6.462,4.499)}
\gppoint{gp mark 2}{(6.460,4.499)}
\gppoint{gp mark 2}{(4.895,4.499)}
\gppoint{gp mark 2}{(5.248,4.499)}
\gppoint{gp mark 2}{(5.258,4.499)}
\gppoint{gp mark 2}{(5.238,4.499)}
\gppoint{gp mark 2}{(5.242,4.499)}
\gppoint{gp mark 2}{(3.483,4.503)}
\gppoint{gp mark 2}{(4.691,4.503)}
\gppoint{gp mark 2}{(4.163,4.503)}
\gppoint{gp mark 2}{(4.165,4.503)}
\gppoint{gp mark 2}{(5.096,4.503)}
\gppoint{gp mark 2}{(4.274,4.503)}
\gppoint{gp mark 2}{(4.786,4.503)}
\gppoint{gp mark 2}{(4.553,4.503)}
\gppoint{gp mark 2}{(3.837,4.503)}
\gppoint{gp mark 2}{(3.836,4.503)}
\gppoint{gp mark 2}{(6.463,4.503)}
\gppoint{gp mark 2}{(6.463,4.503)}
\gppoint{gp mark 2}{(4.902,4.503)}
\gppoint{gp mark 2}{(6.052,4.508)}
\gppoint{gp mark 2}{(3.485,4.508)}
\gppoint{gp mark 2}{(4.695,4.508)}
\gppoint{gp mark 2}{(4.651,4.508)}
\gppoint{gp mark 2}{(4.650,4.508)}
\gppoint{gp mark 2}{(4.649,4.508)}
\gppoint{gp mark 2}{(4.170,4.508)}
\gppoint{gp mark 2}{(4.166,4.508)}
\gppoint{gp mark 2}{(5.100,4.508)}
\gppoint{gp mark 2}{(5.099,4.508)}
\gppoint{gp mark 2}{(5.099,4.508)}
\gppoint{gp mark 2}{(5.101,4.508)}
\gppoint{gp mark 2}{(5.100,4.508)}
\gppoint{gp mark 2}{(4.290,4.508)}
\gppoint{gp mark 2}{(4.283,4.508)}
\gppoint{gp mark 2}{(4.280,4.508)}
\gppoint{gp mark 2}{(4.287,4.508)}
\gppoint{gp mark 2}{(5.258,4.508)}
\gppoint{gp mark 2}{(5.257,4.508)}
\gppoint{gp mark 2}{(4.554,4.508)}
\gppoint{gp mark 2}{(3.838,4.508)}
\gppoint{gp mark 2}{(6.464,4.508)}
\gppoint{gp mark 2}{(6.465,4.508)}
\gppoint{gp mark 2}{(4.901,4.508)}
\gppoint{gp mark 2}{(5.267,4.508)}
\gppoint{gp mark 2}{(5.259,4.508)}
\gppoint{gp mark 2}{(6.054,4.513)}
\gppoint{gp mark 2}{(3.511,4.513)}
\gppoint{gp mark 2}{(3.504,4.513)}
\gppoint{gp mark 2}{(3.500,4.513)}
\gppoint{gp mark 2}{(3.491,4.513)}
\gppoint{gp mark 2}{(3.498,4.513)}
\gppoint{gp mark 2}{(3.493,4.513)}
\gppoint{gp mark 2}{(3.505,4.513)}
\gppoint{gp mark 2}{(3.501,4.513)}
\gppoint{gp mark 2}{(3.495,4.513)}
\gppoint{gp mark 2}{(3.488,4.513)}
\gppoint{gp mark 2}{(3.503,4.513)}
\gppoint{gp mark 2}{(4.699,4.513)}
\gppoint{gp mark 2}{(4.725,4.513)}
\gppoint{gp mark 2}{(4.716,4.513)}
\gppoint{gp mark 2}{(4.713,4.513)}
\gppoint{gp mark 2}{(4.729,4.513)}
\gppoint{gp mark 2}{(4.711,4.513)}
\gppoint{gp mark 2}{(4.707,4.513)}
\gppoint{gp mark 2}{(4.705,4.513)}
\gppoint{gp mark 2}{(4.702,4.513)}
\gppoint{gp mark 2}{(4.652,4.513)}
\gppoint{gp mark 2}{(4.652,4.513)}
\gppoint{gp mark 2}{(4.651,4.513)}
\gppoint{gp mark 2}{(4.175,4.513)}
\gppoint{gp mark 2}{(4.173,4.513)}
\gppoint{gp mark 2}{(5.103,4.513)}
\gppoint{gp mark 2}{(4.293,4.513)}
\gppoint{gp mark 2}{(4.297,4.513)}
\gppoint{gp mark 2}{(5.261,4.513)}
\gppoint{gp mark 2}{(4.789,4.513)}
\gppoint{gp mark 2}{(4.557,4.513)}
\gppoint{gp mark 2}{(3.841,4.513)}
\gppoint{gp mark 2}{(3.841,4.513)}
\gppoint{gp mark 2}{(3.841,4.513)}
\gppoint{gp mark 2}{(6.465,4.513)}
\gppoint{gp mark 2}{(6.465,4.513)}
\gppoint{gp mark 2}{(5.273,4.513)}
\gppoint{gp mark 2}{(11.121,7.739)}
\gpcolor{color=gp lt color border}
\node[gp node right] at (10.479,7.431) {'TEST_isobara3' using (1.0/$3):($1)};
\gpcolor{color=gp lt color 2}
\gppoint{gp mark 3}{(3.911,2.257)}
\gppoint{gp mark 3}{(4.591,2.257)}
\gppoint{gp mark 3}{(4.594,2.257)}
\gppoint{gp mark 3}{(3.430,2.257)}
\gppoint{gp mark 3}{(3.431,2.257)}
\gppoint{gp mark 3}{(3.430,2.257)}
\gppoint{gp mark 3}{(3.457,2.257)}
\gppoint{gp mark 3}{(3.471,2.257)}
\gppoint{gp mark 3}{(3.971,2.257)}
\gppoint{gp mark 3}{(3.892,2.257)}
\gppoint{gp mark 3}{(3.296,2.257)}
\gppoint{gp mark 3}{(3.297,2.257)}
\gppoint{gp mark 3}{(3.713,2.257)}
\gppoint{gp mark 3}{(3.712,2.257)}
\gppoint{gp mark 3}{(3.710,2.257)}
\gppoint{gp mark 3}{(3.488,2.257)}
\gppoint{gp mark 3}{(4.932,2.257)}
\gppoint{gp mark 3}{(3.945,2.257)}
\gppoint{gp mark 3}{(3.946,2.257)}
\gppoint{gp mark 3}{(3.942,2.257)}
\gppoint{gp mark 3}{(3.947,2.257)}
\gppoint{gp mark 3}{(3.946,2.257)}
\gppoint{gp mark 3}{(3.455,2.257)}
\gppoint{gp mark 3}{(3.594,2.257)}
\gppoint{gp mark 3}{(3.910,2.262)}
\gppoint{gp mark 3}{(4.597,2.262)}
\gppoint{gp mark 3}{(3.433,2.262)}
\gppoint{gp mark 3}{(3.433,2.262)}
\gppoint{gp mark 3}{(3.458,2.262)}
\gppoint{gp mark 3}{(3.459,2.262)}
\gppoint{gp mark 3}{(3.470,2.262)}
\gppoint{gp mark 3}{(3.468,2.262)}
\gppoint{gp mark 3}{(3.468,2.262)}
\gppoint{gp mark 3}{(3.975,2.262)}
\gppoint{gp mark 3}{(3.896,2.262)}
\gppoint{gp mark 3}{(3.295,2.262)}
\gppoint{gp mark 3}{(3.718,2.262)}
\gppoint{gp mark 3}{(3.722,2.262)}
\gppoint{gp mark 3}{(3.720,2.262)}
\gppoint{gp mark 3}{(3.716,2.262)}
\gppoint{gp mark 3}{(3.719,2.262)}
\gppoint{gp mark 3}{(3.716,2.262)}
\gppoint{gp mark 3}{(3.489,2.262)}
\gppoint{gp mark 3}{(4.938,2.262)}
\gppoint{gp mark 3}{(4.945,2.262)}
\gppoint{gp mark 3}{(3.940,2.262)}
\gppoint{gp mark 3}{(3.938,2.262)}
\gppoint{gp mark 3}{(3.936,2.262)}
\gppoint{gp mark 3}{(3.938,2.262)}
\gppoint{gp mark 3}{(3.910,2.267)}
\gppoint{gp mark 3}{(4.597,2.267)}
\gppoint{gp mark 3}{(3.434,2.267)}
\gppoint{gp mark 3}{(3.465,2.267)}
\gppoint{gp mark 3}{(3.460,2.267)}
\gppoint{gp mark 3}{(3.459,2.267)}
\gppoint{gp mark 3}{(3.462,2.267)}
\gppoint{gp mark 3}{(3.467,2.267)}
\gppoint{gp mark 3}{(3.463,2.267)}
\gppoint{gp mark 3}{(3.465,2.267)}
\gppoint{gp mark 3}{(3.976,2.267)}
\gppoint{gp mark 3}{(3.464,2.267)}
\gppoint{gp mark 3}{(3.976,2.267)}
\gppoint{gp mark 3}{(3.295,2.267)}
\gppoint{gp mark 3}{(3.293,2.267)}
\gppoint{gp mark 3}{(3.724,2.267)}
\gppoint{gp mark 3}{(3.491,2.267)}
\gppoint{gp mark 3}{(3.500,2.267)}
\gppoint{gp mark 3}{(4.943,2.267)}
\gppoint{gp mark 3}{(3.934,2.267)}
\gppoint{gp mark 3}{(3.931,2.267)}
\gppoint{gp mark 3}{(3.933,2.267)}
\gppoint{gp mark 3}{(3.931,2.267)}
\gppoint{gp mark 3}{(3.936,2.267)}
\gppoint{gp mark 3}{(3.930,2.267)}
\gppoint{gp mark 3}{(3.457,2.267)}
\gppoint{gp mark 3}{(3.596,2.267)}
\gppoint{gp mark 3}{(3.593,2.267)}
\gppoint{gp mark 3}{(3.909,2.272)}
\gppoint{gp mark 3}{(3.907,2.272)}
\gppoint{gp mark 3}{(3.905,2.272)}
\gppoint{gp mark 3}{(4.599,2.272)}
\gppoint{gp mark 3}{(4.602,2.272)}
\gppoint{gp mark 3}{(3.435,2.272)}
\gppoint{gp mark 3}{(3.466,2.272)}
\gppoint{gp mark 3}{(3.465,2.272)}
\gppoint{gp mark 3}{(3.463,2.272)}
\gppoint{gp mark 3}{(3.976,2.272)}
\gppoint{gp mark 3}{(3.903,2.272)}
\gppoint{gp mark 3}{(3.293,2.272)}
\gppoint{gp mark 3}{(3.725,2.272)}
\gppoint{gp mark 3}{(3.506,2.272)}
\gppoint{gp mark 3}{(3.498,2.272)}
\gppoint{gp mark 3}{(3.495,2.272)}
\gppoint{gp mark 3}{(3.503,2.272)}
\gppoint{gp mark 3}{(3.494,2.272)}
\gppoint{gp mark 3}{(3.504,2.272)}
\gppoint{gp mark 3}{(3.497,2.272)}
\gppoint{gp mark 3}{(4.944,2.272)}
\gppoint{gp mark 3}{(3.929,2.272)}
\gppoint{gp mark 3}{(3.928,2.272)}
\gppoint{gp mark 3}{(3.460,2.272)}
\gppoint{gp mark 3}{(3.599,2.272)}
\gppoint{gp mark 3}{(11.121,7.431)}
\gpcolor{color=gp lt color border}
\node[gp node right] at (10.479,7.123) {'TEST_isobara4' using (1.0/$3):($1)};
\gpcolor{color=gp lt color 3}
\gppoint{gp mark 4}{(9.770,7.138)}
\gppoint{gp mark 4}{(8.132,7.138)}
\gppoint{gp mark 4}{(8.137,7.138)}
\gppoint{gp mark 4}{(9.320,7.138)}
\gppoint{gp mark 4}{(9.335,7.138)}
\gppoint{gp mark 4}{(8.202,7.138)}
\gppoint{gp mark 4}{(8.217,7.138)}
\gppoint{gp mark 4}{(8.193,7.138)}
\gppoint{gp mark 4}{(9.846,7.138)}
\gppoint{gp mark 4}{(9.866,7.138)}
\gppoint{gp mark 4}{(9.971,7.138)}
\gppoint{gp mark 4}{(9.947,7.138)}
\gppoint{gp mark 4}{(7.887,7.138)}
\gppoint{gp mark 4}{(9.773,7.143)}
\gppoint{gp mark 4}{(8.144,7.143)}
\gppoint{gp mark 4}{(8.152,7.143)}
\gppoint{gp mark 4}{(9.352,7.143)}
\gppoint{gp mark 4}{(9.974,7.143)}
\gppoint{gp mark 4}{(9.946,7.143)}
\gppoint{gp mark 4}{(9.898,7.143)}
\gppoint{gp mark 4}{(9.998,7.143)}
\gppoint{gp mark 4}{(9.984,7.143)}
\gppoint{gp mark 4}{(9.928,7.143)}
\gppoint{gp mark 4}{(7.887,7.143)}
\gppoint{gp mark 4}{(8.756,7.143)}
\gppoint{gp mark 4}{(9.802,7.148)}
\gppoint{gp mark 4}{(9.805,7.148)}
\gppoint{gp mark 4}{(9.788,7.148)}
\gppoint{gp mark 4}{(9.782,7.148)}
\gppoint{gp mark 4}{(9.822,7.148)}
\gppoint{gp mark 4}{(8.169,7.148)}
\gppoint{gp mark 4}{(8.162,7.148)}
\gppoint{gp mark 4}{(8.176,7.148)}
\gppoint{gp mark 4}{(9.368,7.148)}
\gppoint{gp mark 4}{(9.380,7.148)}
\gppoint{gp mark 4}{(9.367,7.148)}
\gppoint{gp mark 4}{(8.228,7.148)}
\gppoint{gp mark 4}{(8.230,7.148)}
\gppoint{gp mark 4}{(8.224,7.148)}
\gppoint{gp mark 4}{(10.024,7.148)}
\gppoint{gp mark 4}{(9.993,7.148)}
\gppoint{gp mark 4}{(10.002,7.148)}
\gppoint{gp mark 4}{(10.014,7.148)}
\gppoint{gp mark 4}{(9.998,7.148)}
\gppoint{gp mark 4}{(10.030,7.148)}
\gppoint{gp mark 4}{(7.899,7.148)}
\gppoint{gp mark 4}{(7.894,7.148)}
\gppoint{gp mark 4}{(7.905,7.148)}
\gppoint{gp mark 4}{(8.767,7.148)}
\gppoint{gp mark 4}{(9.847,7.153)}
\gppoint{gp mark 4}{(9.840,7.153)}
\gppoint{gp mark 4}{(9.834,7.153)}
\gppoint{gp mark 4}{(8.192,7.153)}
\gppoint{gp mark 4}{(8.178,7.153)}
\gppoint{gp mark 4}{(8.181,7.153)}
\gppoint{gp mark 4}{(9.397,7.153)}
\gppoint{gp mark 4}{(9.386,7.153)}
\gppoint{gp mark 4}{(9.384,7.153)}
\gppoint{gp mark 4}{(10.068,7.153)}
\gppoint{gp mark 4}{(10.100,7.153)}
\gppoint{gp mark 4}{(10.077,7.153)}
\gppoint{gp mark 4}{(10.065,7.153)}
\gppoint{gp mark 4}{(7.936,7.153)}
\gppoint{gp mark 4}{(7.918,7.153)}
\gppoint{gp mark 4}{(8.776,7.153)}
\gppoint{gp mark 4}{(11.121,7.123)}
\gpcolor{color=gp lt color border}
\gpsetlinetype{gp lt border}
\draw[gp path] (1.196,8.381)--(1.196,0.616)--(11.947,0.616)--(11.947,8.381)--cycle;
%% coordinates of the plot area
\gpdefrectangularnode{gp plot 1}{\pgfpoint{1.196cm}{0.616cm}}{\pgfpoint{11.947cm}{8.381cm}}
\end{tikzpicture}
%% gnuplot variables

%\caption{Alcune isobare estratte dai grafici}
%\label{img:isob}
%\end{grafico}
\begin{grafico} \centering \begin{tikzpicture}[gnuplot]
%% generated with GNUPLOT 4.6p0 (Lua 5.1; terminal rev. 99, script rev. 100)
%% Tue 10 Jun 2014 06:41:32 PM CEST
\path (0.000,0.000) rectangle (12.500,8.750);
\gpcolor{color=gp lt color border}
\gpsetlinetype{gp lt border}
\gpsetlinewidth{1.00}
\draw[gp path] (1.688,1.840)--(1.868,1.840);
\node[gp node right] at (1.504,1.840) {-0.03};
\draw[gp path] (1.688,2.695)--(1.868,2.695);
\node[gp node right] at (1.504,2.695) {-0.02};
\draw[gp path] (1.688,3.550)--(1.868,3.550);
\node[gp node right] at (1.504,3.550) {-0.01};
\draw[gp path] (1.688,4.405)--(1.868,4.405);
\node[gp node right] at (1.504,4.405) { 0};
\draw[gp path] (1.688,5.260)--(1.868,5.260);
\node[gp node right] at (1.504,5.260) { 0.01};
\draw[gp path] (1.688,6.115)--(1.868,6.115);
\node[gp node right] at (1.504,6.115) { 0.02};
\draw[gp path] (1.688,6.970)--(1.868,6.970);
\node[gp node right] at (1.504,6.970) { 0.03};
\draw[gp path] (1.688,0.985)--(1.688,1.165);
\node[gp node center] at (1.688,0.677) { 0};
\draw[gp path] (3.740,0.985)--(3.740,1.165);
\node[gp node center] at (3.740,0.677) { 5};
\draw[gp path] (5.792,0.985)--(5.792,1.165);
\node[gp node center] at (5.792,0.677) { 10};
\draw[gp path] (7.843,0.985)--(7.843,1.165);
\node[gp node center] at (7.843,0.677) { 15};
\draw[gp path] (9.895,0.985)--(9.895,1.165);
\node[gp node center] at (9.895,0.677) { 20};
\draw[gp path] (1.688,7.152)--(1.688,1.808);
\draw[gp path] (1.688,0.985)--(9.936,0.985);
\node[gp node center,rotate=-270] at (0.246,4.405) {Ampiezza [???]};
\node[gp node center] at (6.817,0.215) {Tempo $[s]$};
\node[gp node center] at (6.817,8.287) {Dati decadimento 0.900d};
\gpcolor{rgb color={0.000,0.000,1.000}}
\gpsetlinetype{gp lt plot 0}
\draw[gp path] (1.688,4.320)--(1.709,5.175)--(1.729,5.859)--(1.750,6.543)--(1.770,7.056)%
  --(1.791,7.141)--(1.811,7.141)--(1.832,6.799)--(1.852,6.286)--(1.873,5.602)--(1.893,4.747)%
  --(1.914,3.978)--(1.934,3.208)--(1.955,2.524)--(1.975,2.011)--(1.996,1.840)--(2.016,1.840)%
  --(2.037,2.011)--(2.057,2.439)--(2.078,3.037)--(2.098,3.807)--(2.119,4.662)--(2.139,5.346)%
  --(2.160,6.030)--(2.180,6.628)--(2.201,6.970)--(2.221,7.056)--(2.242,6.885)--(2.263,6.543)%
  --(2.283,5.944)--(2.304,5.260)--(2.324,4.491)--(2.345,3.721)--(2.365,2.866)--(2.386,2.439)%
  --(2.406,2.097)--(2.427,1.840)--(2.447,1.840)--(2.468,2.182)--(2.488,2.695)--(2.509,3.379)%
  --(2.529,4.063)--(2.550,4.833)--(2.570,5.517)--(2.591,6.286)--(2.611,6.628)--(2.632,6.885)%
  --(2.652,6.885)--(2.673,6.714)--(2.693,6.201)--(2.714,5.688)--(2.734,5.004)--(2.755,4.234)%
  --(2.775,3.379)--(2.796,2.952)--(2.816,2.439)--(2.837,2.097)--(2.858,1.926)--(2.878,2.097)%
  --(2.899,2.439)--(2.919,2.952)--(2.940,3.550)--(2.960,4.405)--(2.981,5.089)--(3.001,5.773)%
  --(3.022,6.201)--(3.042,6.714)--(3.063,6.799)--(3.083,6.714)--(3.104,6.457)--(3.124,6.030)%
  --(3.145,5.431)--(3.165,4.747)--(3.186,3.978)--(3.206,3.379)--(3.227,2.781)--(3.247,2.268)%
  --(3.268,2.011)--(3.288,2.182)--(3.309,2.268)--(3.329,2.610)--(3.350,3.208)--(3.370,3.978)%
  --(3.391,4.576)--(3.412,5.260)--(3.432,5.944)--(3.453,6.372)--(3.473,6.628)--(3.494,6.714)%
  --(3.514,6.543)--(3.535,6.286)--(3.555,5.688)--(3.576,5.175)--(3.596,4.491)--(3.617,3.892)%
  --(3.637,3.208)--(3.658,2.610)--(3.678,2.353)--(3.699,2.182)--(3.719,2.268)--(3.740,2.439)%
  --(3.760,2.952)--(3.781,3.550)--(3.801,4.149)--(3.822,4.833)--(3.842,5.517)--(3.863,6.030)%
  --(3.883,6.372)--(3.904,6.543)--(3.924,6.543)--(3.945,6.372)--(3.965,6.030)--(3.986,5.517)%
  --(4.007,4.918)--(4.027,4.405)--(4.048,3.636)--(4.068,3.037)--(4.089,2.610)--(4.109,2.353)%
  --(4.130,2.268)--(4.150,2.353)--(4.171,2.695)--(4.191,3.208)--(4.212,3.721)--(4.232,4.405)%
  --(4.253,5.089)--(4.273,5.602)--(4.294,6.030)--(4.314,6.286)--(4.335,6.543)--(4.355,6.457)%
  --(4.376,6.201)--(4.396,5.859)--(4.417,5.260)--(4.437,4.747)--(4.458,4.063)--(4.478,3.465)%
  --(4.499,3.037)--(4.519,2.610)--(4.540,2.439)--(4.561,2.353)--(4.581,2.610)--(4.602,2.952)%
  --(4.622,3.294)--(4.643,3.978)--(4.663,4.576)--(4.684,5.175)--(4.704,5.688)--(4.725,6.115)%
  --(4.745,6.372)--(4.766,6.457)--(4.786,6.201)--(4.807,6.030)--(4.827,5.602)--(4.848,5.089)%
  --(4.868,4.405)--(4.889,3.892)--(4.909,3.294)--(4.930,2.952)--(4.950,2.524)--(4.971,2.439)%
  --(4.991,2.524)--(5.012,2.866)--(5.032,3.123)--(5.053,3.721)--(5.073,4.234)--(5.094,4.747)%
  --(5.115,5.260)--(5.135,5.944)--(5.156,6.115)--(5.176,6.201)--(5.197,6.372)--(5.217,6.115)%
  --(5.238,5.773)--(5.258,5.431)--(5.279,4.833)--(5.299,4.234)--(5.320,3.636)--(5.340,3.123)%
  --(5.361,2.866)--(5.381,2.695)--(5.402,2.524)--(5.422,2.695)--(5.443,2.952)--(5.463,3.379)%
  --(5.484,3.807)--(5.504,4.491)--(5.525,4.918)--(5.545,5.431)--(5.566,5.773)--(5.586,6.201)%
  --(5.607,6.115)--(5.627,6.201)--(5.648,5.944)--(5.668,5.602)--(5.689,5.175)--(5.710,4.662)%
  --(5.730,4.063)--(5.751,3.550)--(5.771,3.123)--(5.792,2.781)--(5.812,2.524)--(5.833,2.695)%
  --(5.853,2.866)--(5.874,3.208)--(5.894,3.636)--(5.915,4.063)--(5.935,4.491)--(5.956,5.089)%
  --(5.976,5.431)--(5.997,5.944)--(6.017,6.115)--(6.038,6.201)--(6.058,5.944)--(6.079,5.944)%
  --(6.099,5.431)--(6.120,4.918)--(6.140,4.405)--(6.161,3.978)--(6.181,3.465)--(6.202,3.123)%
  --(6.222,2.866)--(6.243,2.781)--(6.264,2.695)--(6.284,3.123)--(6.305,3.379)--(6.325,3.807)%
  --(6.346,4.320)--(6.366,4.747)--(6.387,5.260)--(6.407,5.688)--(6.428,6.030)--(6.448,6.201)%
  --(6.469,6.030)--(6.489,6.115)--(6.510,5.688)--(6.530,5.175)--(6.551,4.747)--(6.571,4.320)%
  --(6.592,3.892)--(6.612,3.379)--(6.633,3.037)--(6.653,2.866)--(6.674,2.866)--(6.694,2.866)%
  --(6.715,3.123)--(6.735,3.465)--(6.756,3.892)--(6.776,4.405)--(6.797,4.833)--(6.818,5.260)%
  --(6.838,5.602)--(6.859,5.944)--(6.879,5.944)--(6.900,5.944)--(6.920,5.688)--(6.941,5.431)%
  --(6.961,5.004)--(6.982,4.576)--(7.002,4.063)--(7.023,3.636)--(7.043,3.294)--(7.064,3.037)%
  --(7.084,2.866)--(7.105,2.866)--(7.125,3.037)--(7.146,3.294)--(7.166,3.721)--(7.187,4.149)%
  --(7.207,4.576)--(7.228,5.089)--(7.248,5.431)--(7.269,5.688)--(7.289,5.859)--(7.310,5.944)%
  --(7.330,5.859)--(7.351,5.602)--(7.371,5.175)--(7.392,4.918)--(7.413,4.405)--(7.433,3.978)%
  --(7.454,3.550)--(7.474,3.294)--(7.495,2.952)--(7.515,2.952)--(7.536,3.037)--(7.556,3.208)%
  --(7.577,3.379)--(7.597,3.892)--(7.618,4.320)--(7.638,4.747)--(7.659,5.089)--(7.679,5.517)%
  --(7.700,5.773)--(7.720,5.773)--(7.741,5.773)--(7.761,5.688)--(7.782,5.346)--(7.802,5.004)%
  --(7.823,4.662)--(7.843,4.320)--(7.864,3.721)--(7.884,3.465)--(7.905,3.037)--(7.925,3.037)%
  --(7.946,3.123)--(7.967,3.037)--(7.987,3.294)--(8.008,3.636)--(8.028,4.063)--(8.049,4.491)%
  --(8.069,4.833)--(8.090,5.260)--(8.110,5.517)--(8.131,5.773)--(8.151,5.859)--(8.172,5.859)%
  --(8.192,5.517)--(8.213,5.346)--(8.233,4.918)--(8.254,4.491)--(8.274,4.063)--(8.295,3.807)%
  --(8.315,3.379)--(8.336,3.208)--(8.356,3.123)--(8.377,3.037)--(8.397,3.208)--(8.418,3.465)%
  --(8.438,3.807)--(8.459,4.234)--(8.479,4.576)--(8.500,4.918)--(8.520,5.260)--(8.541,5.602)%
  --(8.562,5.773)--(8.582,5.773)--(8.603,5.602)--(8.623,5.346)--(8.644,5.175)--(8.664,4.747)%
  --(8.685,4.320)--(8.705,4.063)--(8.726,3.721)--(8.746,3.465)--(8.767,3.208)--(8.787,3.208)%
  --(8.808,3.208)--(8.828,3.294)--(8.849,3.550)--(8.869,3.978)--(8.890,4.405)--(8.910,4.576)%
  --(8.931,5.004)--(8.951,5.431)--(8.972,5.602)--(8.992,5.688)--(9.013,5.602)--(9.033,5.517)%
  --(9.054,5.260)--(9.074,5.004)--(9.095,4.576)--(9.116,4.320)--(9.136,3.892)--(9.157,3.550)%
  --(9.177,3.294)--(9.198,3.208)--(9.218,3.123)--(9.239,3.294)--(9.259,3.465)--(9.280,3.807)%
  --(9.300,4.063)--(9.321,4.405)--(9.341,4.747)--(9.362,5.175)--(9.382,5.346)--(9.403,5.431)%
  --(9.423,5.602)--(9.444,5.517)--(9.464,5.431)--(9.485,5.175)--(9.505,4.833)--(9.526,4.491)%
  --(9.546,4.149)--(9.567,3.807)--(9.587,3.550)--(9.608,3.294)--(9.628,3.294)--(9.649,3.294)%
  --(9.670,3.379)--(9.690,3.550)--(9.711,3.892)--(9.731,4.234)--(9.752,4.491)--(9.772,5.004)%
  --(9.793,5.260)--(9.813,5.431)--(9.834,5.431)--(9.854,5.602)--(9.875,5.431)--(9.895,5.260)%
  --(9.916,5.089)--(9.936,4.662);
\gpcolor{color=gp lt color border}
\node[gp node right] at (10.479,7.491) {Massimi e minimi};
\gpcolor{rgb color={1.000,0.000,0.000}}
\gpsetpointsize{4.00}
\gppoint{gp mark 3}{(1.821,7.152)}
\gppoint{gp mark 3}{(2.239,7.059)}
\gppoint{gp mark 3}{(2.663,6.917)}
\gppoint{gp mark 3}{(3.083,6.799)}
\gppoint{gp mark 3}{(3.511,6.717)}
\gppoint{gp mark 3}{(3.935,6.564)}
\gppoint{gp mark 3}{(4.360,6.553)}
\gppoint{gp mark 3}{(4.781,6.468)}
\gppoint{gp mark 3}{(5.215,6.374)}
\gppoint{gp mark 3}{(5.614,6.229)}
\gppoint{gp mark 3}{(6.053,6.211)}
\gppoint{gp mark 3}{(6.469,6.201)}
\gppoint{gp mark 3}{(6.889,5.987)}
\gppoint{gp mark 3}{(7.330,5.944)}
\gppoint{gp mark 3}{(7.731,5.805)}
\gppoint{gp mark 3}{(8.182,5.869)}
\gppoint{gp mark 3}{(8.592,5.794)}
\gppoint{gp mark 3}{(9.013,5.688)}
\gppoint{gp mark 3}{(9.447,5.606)}
\gppoint{gp mark 3}{(9.875,5.602)}
\gppoint{gp mark 3}{(2.027,1.819)}
\gppoint{gp mark 3}{(2.457,1.808)}
\gppoint{gp mark 3}{(2.878,1.926)}
\gppoint{gp mark 3}{(3.290,2.009)}
\gppoint{gp mark 3}{(3.723,2.178)}
\gppoint{gp mark 3}{(4.150,2.268)}
\gppoint{gp mark 3}{(4.576,2.342)}
\gppoint{gp mark 3}{(4.991,2.439)}
\gppoint{gp mark 3}{(6.684,2.845)}
\gppoint{gp mark 3}{(7.115,2.845)}
\gppoint{gp mark 3}{(7.525,2.909)}
\gppoint{gp mark 3}{(7.936,2.984)}
\gppoint{gp mark 3}{(8.394,3.033)}
\gppoint{gp mark 3}{(8.797,3.176)}
\gppoint{gp mark 3}{(9.235,3.119)}
\gppoint{gp mark 3}{(9.639,3.261)}
\gppoint{gp mark 3}{(11.121,7.491)}
\gpcolor{color=gp lt color border}
\gpsetlinetype{gp lt border}
\draw[gp path] (1.688,7.152)--(1.688,1.808);
\draw[gp path] (1.688,0.985)--(9.936,0.985);
%% coordinates of the plot area
\gpdefrectangularnode{gp plot 1}{\pgfpoint{1.688cm}{0.985cm}}{\pgfpoint{11.947cm}{7.825cm}}
\end{tikzpicture}
%% gnuplot variables
 \caption{Grafico 0.900dgdecad.tex} \label{gr:0.900dgdecad.tex} \end{grafico}
\begin{grafico} \centering \begin{tikzpicture}[gnuplot]
%% generated with GNUPLOT 4.6p0 (Lua 5.1; terminal rev. 99, script rev. 100)
%% Tue 10 Jun 2014 09:42:46 PM CEST
\path (0.000,0.000) rectangle (12.500,8.750);
\gpcolor{color=gp lt color border}
\gpsetlinetype{gp lt border}
\gpsetlinewidth{1.00}
\draw[gp path] (1.688,1.607)--(1.868,1.607);
\node[gp node right] at (1.504,1.607) {-0.05};
\draw[gp path] (1.688,2.229)--(1.868,2.229);
\node[gp node right] at (1.504,2.229) {-0.04};
\draw[gp path] (1.688,2.850)--(1.868,2.850);
\node[gp node right] at (1.504,2.850) {-0.03};
\draw[gp path] (1.688,3.472)--(1.868,3.472);
\node[gp node right] at (1.504,3.472) {-0.02};
\draw[gp path] (1.688,4.094)--(1.868,4.094);
\node[gp node right] at (1.504,4.094) {-0.01};
\draw[gp path] (1.688,4.716)--(1.868,4.716);
\node[gp node right] at (1.504,4.716) { 0};
\draw[gp path] (1.688,5.338)--(1.868,5.338);
\node[gp node right] at (1.504,5.338) { 0.01};
\draw[gp path] (1.688,5.960)--(1.868,5.960);
\node[gp node right] at (1.504,5.960) { 0.02};
\draw[gp path] (1.688,6.581)--(1.868,6.581);
\node[gp node right] at (1.504,6.581) { 0.03};
\draw[gp path] (1.688,7.203)--(1.868,7.203);
\node[gp node right] at (1.504,7.203) { 0.04};
\draw[gp path] (1.688,0.985)--(1.688,1.165);
\node[gp node center] at (1.688,0.677) { 0};
\draw[gp path] (3.740,0.985)--(3.740,1.165);
\node[gp node center] at (3.740,0.677) { 5};
\draw[gp path] (5.792,0.985)--(5.792,1.165);
\node[gp node center] at (5.792,0.677) { 10};
\draw[gp path] (7.843,0.985)--(7.843,1.165);
\node[gp node center] at (7.843,0.677) { 15};
\draw[gp path] (9.895,0.985)--(9.895,1.165);
\node[gp node center] at (9.895,0.677) { 20};
\draw[gp path] (1.688,7.452)--(1.688,1.467);
\draw[gp path] (1.688,0.985)--(10.942,0.985);
\node[gp node center,rotate=-270] at (0.246,4.405) {Ampiezza [giri]};
\node[gp node center] at (6.817,0.215) {Tempo $[s]$};
\node[gp node center] at (6.817,8.287) {Dati decadimento 0.920d};
\gpcolor{rgb color={0.000,0.000,1.000}}
\gpsetlinetype{gp lt plot 0}
\draw[gp path] (1.688,4.405)--(1.709,3.472)--(1.729,2.850)--(1.750,2.166)--(1.770,1.856)%
  --(1.791,1.669)--(1.811,1.793)--(1.832,2.229)--(1.852,2.850)--(1.873,3.534)--(1.893,4.343)%
  --(1.914,5.213)--(1.934,6.022)--(1.955,6.519)--(1.975,7.017)--(1.996,7.265)--(2.016,7.265)%
  --(2.037,6.892)--(2.057,6.333)--(2.078,5.773)--(2.098,4.902)--(2.119,4.094)--(2.139,3.286)%
  --(2.160,2.664)--(2.180,2.166)--(2.201,1.918)--(2.221,1.793)--(2.242,2.042)--(2.263,2.477)%
  --(2.283,3.099)--(2.304,3.845)--(2.324,4.654)--(2.345,5.462)--(2.365,6.084)--(2.386,6.706)%
  --(2.406,7.079)--(2.427,7.141)--(2.447,6.954)--(2.468,6.644)--(2.488,6.084)--(2.509,5.400)%
  --(2.529,4.592)--(2.550,3.845)--(2.570,3.099)--(2.591,2.540)--(2.611,2.104)--(2.632,1.856)%
  --(2.652,1.980)--(2.673,2.291)--(2.693,2.726)--(2.714,3.472)--(2.734,4.156)--(2.755,4.902)%
  --(2.775,5.586)--(2.796,6.270)--(2.816,6.706)--(2.837,6.954)--(2.858,6.954)--(2.878,6.768)%
  --(2.899,6.395)--(2.919,5.835)--(2.940,5.089)--(2.960,4.467)--(2.981,3.659)--(3.001,2.975)%
  --(3.022,2.477)--(3.042,2.166)--(3.063,2.042)--(3.083,2.166)--(3.104,2.540)--(3.124,3.099)%
  --(3.145,3.721)--(3.165,4.405)--(3.186,5.089)--(3.206,5.773)--(3.227,6.333)--(3.247,6.706)%
  --(3.268,6.830)--(3.288,6.830)--(3.309,6.581)--(3.329,6.084)--(3.350,5.462)--(3.370,4.965)%
  --(3.391,4.218)--(3.412,3.472)--(3.432,2.850)--(3.453,2.415)--(3.473,2.166)--(3.494,2.166)%
  --(3.514,2.353)--(3.535,2.788)--(3.555,3.286)--(3.576,3.908)--(3.596,4.654)--(3.617,5.338)%
  --(3.637,5.897)--(3.658,6.395)--(3.678,6.706)--(3.699,6.830)--(3.719,6.644)--(3.740,6.333)%
  --(3.760,5.835)--(3.781,5.338)--(3.801,4.592)--(3.822,3.908)--(3.842,3.286)--(3.863,2.850)%
  --(3.883,2.415)--(3.904,2.291)--(3.924,2.353)--(3.945,2.602)--(3.965,2.975)--(3.986,3.534)%
  --(4.007,4.218)--(4.027,4.902)--(4.048,5.462)--(4.068,6.084)--(4.089,6.395)--(4.109,6.644)%
  --(4.130,6.644)--(4.150,6.457)--(4.171,6.146)--(4.191,5.649)--(4.212,5.027)--(4.232,4.343)%
  --(4.253,3.783)--(4.273,3.224)--(4.294,2.726)--(4.314,2.477)--(4.335,2.415)--(4.355,2.540)%
  --(4.376,2.788)--(4.396,3.286)--(4.417,3.908)--(4.437,4.467)--(4.458,5.089)--(4.478,5.711)%
  --(4.499,6.208)--(4.519,6.457)--(4.540,6.581)--(4.561,6.581)--(4.581,6.333)--(4.602,5.960)%
  --(4.622,5.400)--(4.643,4.840)--(4.663,4.218)--(4.684,3.597)--(4.704,3.099)--(4.725,2.726)%
  --(4.745,2.540)--(4.766,2.477)--(4.786,2.664)--(4.807,2.975)--(4.827,3.534)--(4.848,4.032)%
  --(4.868,4.654)--(4.889,5.276)--(4.909,5.773)--(4.930,6.146)--(4.950,6.395)--(4.971,6.457)%
  --(4.991,6.395)--(5.012,6.146)--(5.032,5.711)--(5.053,5.151)--(5.073,4.592)--(5.094,3.970)%
  --(5.115,3.410)--(5.135,2.975)--(5.156,2.726)--(5.176,2.602)--(5.197,2.602)--(5.217,2.850)%
  --(5.238,3.224)--(5.258,3.721)--(5.279,4.281)--(5.299,4.902)--(5.320,5.400)--(5.340,5.835)%
  --(5.361,6.208)--(5.381,6.395)--(5.402,6.395)--(5.422,6.270)--(5.443,5.960)--(5.463,5.524)%
  --(5.484,4.965)--(5.504,4.343)--(5.525,3.783)--(5.545,3.348)--(5.566,2.913)--(5.586,2.726)%
  --(5.607,2.602)--(5.627,2.850)--(5.648,3.037)--(5.668,3.410)--(5.689,3.908)--(5.710,4.467)%
  --(5.730,5.027)--(5.751,5.524)--(5.771,5.897)--(5.792,6.146)--(5.812,6.270)--(5.833,6.270)%
  --(5.853,6.022)--(5.874,5.649)--(5.894,5.276)--(5.915,4.716)--(5.935,4.156)--(5.956,3.659)%
  --(5.976,3.224)--(5.997,2.913)--(6.017,2.726)--(6.038,2.788)--(6.058,2.913)--(6.079,3.224)%
  --(6.099,3.659)--(6.120,4.156)--(6.140,4.654)--(6.161,5.151)--(6.181,5.649)--(6.202,5.960)%
  --(6.222,6.208)--(6.243,6.208)--(6.264,6.084)--(6.284,5.835)--(6.305,5.586)--(6.325,5.089)%
  --(6.346,4.592)--(6.366,4.032)--(6.387,3.534)--(6.407,3.161)--(6.428,2.913)--(6.448,2.850)%
  --(6.469,2.850)--(6.489,3.099)--(6.510,3.348)--(6.530,3.845)--(6.551,4.343)--(6.571,4.840)%
  --(6.592,5.276)--(6.612,5.649)--(6.633,5.960)--(6.653,6.146)--(6.674,6.146)--(6.694,6.022)%
  --(6.715,5.711)--(6.735,5.338)--(6.756,4.840)--(6.776,4.343)--(6.797,3.908)--(6.818,3.472)%
  --(6.838,3.161)--(6.859,2.975)--(6.879,2.913)--(6.900,3.037)--(6.920,3.224)--(6.941,3.597)%
  --(6.961,4.032)--(6.982,4.467)--(7.002,4.965)--(7.023,5.338)--(7.043,5.773)--(7.064,6.022)%
  --(7.084,6.084)--(7.105,6.146)--(7.125,5.835)--(7.146,5.524)--(7.166,5.089)--(7.187,4.716)%
  --(7.207,4.218)--(7.228,3.783)--(7.248,3.410)--(7.269,3.161)--(7.289,3.037)--(7.310,3.037)%
  --(7.330,3.099)--(7.351,3.410)--(7.371,3.783)--(7.392,4.218)--(7.413,4.654)--(7.433,5.151)%
  --(7.454,5.462)--(7.474,5.773)--(7.495,6.022)--(7.515,6.022)--(7.536,5.897)--(7.556,5.711)%
  --(7.577,5.400)--(7.597,4.965)--(7.618,4.529)--(7.638,4.032)--(7.659,3.659)--(7.679,3.348)%
  --(7.700,3.161)--(7.720,3.037)--(7.741,3.099)--(7.761,3.286)--(7.782,3.597)--(7.802,3.908)%
  --(7.823,4.343)--(7.843,4.778)--(7.864,5.151)--(7.884,5.524)--(7.905,5.835)--(7.925,5.960)%
  --(7.946,5.960)--(7.967,5.773)--(7.987,5.586)--(8.008,5.213)--(8.028,4.778)--(8.049,4.343)%
  --(8.069,3.970)--(8.090,3.597)--(8.110,3.348)--(8.131,3.161)--(8.151,3.161)--(8.172,3.224)%
  --(8.192,3.410)--(8.213,3.721)--(8.233,4.094)--(8.254,4.467)--(8.274,4.965)--(8.295,5.338)%
  --(8.315,5.649)--(8.336,5.773)--(8.356,5.897)--(8.377,5.835)--(8.397,5.711)--(8.418,5.400)%
  --(8.438,5.027)--(8.459,4.654)--(8.479,4.218)--(8.500,3.908)--(8.520,3.534)--(8.541,3.348)%
  --(8.562,3.224)--(8.582,3.224)--(8.603,3.348)--(8.623,3.534)--(8.644,3.908)--(8.664,4.218)%
  --(8.685,4.654)--(8.705,5.089)--(8.726,5.400)--(8.746,5.586)--(8.767,5.773)--(8.787,5.773)%
  --(8.808,5.711)--(8.828,5.524)--(8.849,5.276)--(8.869,4.902)--(8.890,4.529)--(8.910,4.156)%
  --(8.931,3.845)--(8.951,3.534)--(8.972,3.348)--(8.992,3.286)--(9.013,3.348)--(9.033,3.472)%
  --(9.054,3.721)--(9.074,4.032)--(9.095,4.405)--(9.116,4.840)--(9.136,5.151)--(9.157,5.400)%
  --(9.177,5.649)--(9.198,5.773)--(9.218,5.773)--(9.239,5.586)--(9.259,5.462)--(9.280,5.151)%
  --(9.300,4.716)--(9.321,4.405)--(9.341,4.094)--(9.362,3.721)--(9.382,3.472)--(9.403,3.348)%
  --(9.423,3.348)--(9.444,3.410)--(9.464,3.597)--(9.485,3.845)--(9.505,4.156)--(9.526,4.592)%
  --(9.546,4.902)--(9.567,5.213)--(9.587,5.462)--(9.608,5.649)--(9.628,5.711)--(9.649,5.649)%
  --(9.670,5.524)--(9.690,5.276)--(9.711,4.965)--(9.731,4.654)--(9.752,4.281)--(9.772,3.970)%
  --(9.793,3.659)--(9.813,3.472)--(9.834,3.410)--(9.854,3.410)--(9.875,3.534)--(9.895,3.721)%
  --(9.916,3.970)--(9.936,4.281)--(9.957,4.654)--(9.977,5.027)--(9.998,5.276)--(10.018,5.462)%
  --(10.039,5.649)--(10.059,5.649)--(10.080,5.586)--(10.100,5.400)--(10.121,5.213)--(10.141,4.902)%
  --(10.162,4.529)--(10.182,4.218)--(10.203,3.908)--(10.223,3.659)--(10.244,3.534)--(10.265,3.472)%
  --(10.285,3.472)--(10.306,3.597)--(10.326,3.783)--(10.347,4.218)--(10.367,4.467)--(10.388,4.778)%
  --(10.408,5.089)--(10.429,5.276)--(10.449,5.524)--(10.470,5.586)--(10.490,5.586)--(10.511,5.462)%
  --(10.531,5.213)--(10.552,5.027)--(10.572,4.716)--(10.593,4.405)--(10.613,4.094)--(10.634,3.845)%
  --(10.654,3.597)--(10.675,3.534)--(10.695,3.472)--(10.716,3.534)--(10.736,3.721)--(10.757,4.032)%
  --(10.777,4.218)--(10.798,4.529)--(10.819,4.840)--(10.839,5.151)--(10.860,5.338)--(10.880,5.462)%
  --(10.901,5.586)--(10.921,5.462)--(10.942,5.338);
\gpcolor{color=gp lt color border}
\node[gp node right] at (10.479,7.491) {Punti tangenza};
\gpcolor{rgb color={1.000,0.000,0.000}}
\gpsetpointsize{4.00}
\gppoint{gp mark 3}{(1.796,1.467)}
\gppoint{gp mark 3}{(2.006,7.452)}
\gppoint{gp mark 3}{(2.221,1.793)}
\gppoint{gp mark 3}{(2.427,7.141)}
\gppoint{gp mark 3}{(2.632,1.856)}
\gppoint{gp mark 3}{(2.847,7.048)}
\gppoint{gp mark 3}{(3.063,2.042)}
\gppoint{gp mark 3}{(3.278,6.954)}
\gppoint{gp mark 3}{(3.494,2.166)}
\gppoint{gp mark 3}{(3.699,6.830)}
\gppoint{gp mark 3}{(3.904,2.291)}
\gppoint{gp mark 3}{(4.119,6.737)}
\gppoint{gp mark 3}{(4.335,2.415)}
\gppoint{gp mark 3}{(4.550,6.706)}
\gppoint{gp mark 3}{(4.766,2.477)}
\gppoint{gp mark 3}{(4.971,6.457)}
\gppoint{gp mark 3}{(5.176,2.602)}
\gppoint{gp mark 3}{(5.391,6.457)}
\gppoint{gp mark 3}{(5.607,2.602)}
\gppoint{gp mark 3}{(5.812,6.270)}
\gppoint{gp mark 3}{(6.028,2.726)}
\gppoint{gp mark 3}{(6.243,6.208)}
\gppoint{gp mark 3}{(6.448,2.850)}
\gppoint{gp mark 3}{(6.664,6.208)}
\gppoint{gp mark 3}{(6.879,2.913)}
\gppoint{gp mark 3}{(7.084,6.084)}
\gppoint{gp mark 3}{(7.300,3.006)}
\gppoint{gp mark 3}{(7.515,6.022)}
\gppoint{gp mark 3}{(7.720,3.037)}
\gppoint{gp mark 3}{(7.936,6.053)}
\gppoint{gp mark 3}{(8.151,3.161)}
\gppoint{gp mark 3}{(8.356,5.897)}
\gppoint{gp mark 3}{(8.572,3.161)}
\gppoint{gp mark 3}{(8.787,5.773)}
\gppoint{gp mark 3}{(8.992,3.286)}
\gppoint{gp mark 3}{(9.198,5.773)}
\gppoint{gp mark 3}{(9.413,3.317)}
\gppoint{gp mark 3}{(9.628,5.711)}
\gppoint{gp mark 3}{(9.844,3.348)}
\gppoint{gp mark 3}{(10.059,5.649)}
\gppoint{gp mark 3}{(10.265,3.472)}
\gppoint{gp mark 3}{(10.480,5.649)}
\gppoint{gp mark 3}{(10.695,3.472)}
\gppoint{gp mark 3}{(11.121,7.491)}
\gpcolor{color=gp lt color border}
\gpsetlinetype{gp lt border}
\draw[gp path] (1.688,7.452)--(1.688,1.467);
\draw[gp path] (1.688,0.985)--(10.942,0.985);
%% coordinates of the plot area
\gpdefrectangularnode{gp plot 1}{\pgfpoint{1.688cm}{0.985cm}}{\pgfpoint{11.947cm}{7.825cm}}
\end{tikzpicture}
%% gnuplot variables
 \caption{Grafico 0.920dgdecad.tex} \label{gr:0.920dgdecad.tex} \end{grafico}
\begin{grafico} \centering \begin{tikzpicture}[gnuplot]
%% generated with GNUPLOT 4.6p0 (Lua 5.1; terminal rev. 99, script rev. 100)
%% Tue 10 Jun 2014 07:45:02 PM CEST
\path (0.000,0.000) rectangle (12.500,8.750);
\gpcolor{color=gp lt color border}
\gpsetlinetype{gp lt border}
\gpsetlinewidth{1.00}
\draw[gp path] (1.504,0.985)--(1.684,0.985);
\node[gp node right] at (1.320,0.985) {-1};
\draw[gp path] (1.504,2.695)--(1.684,2.695);
\node[gp node right] at (1.320,2.695) {-0.5};
\draw[gp path] (1.504,4.405)--(1.684,4.405);
\node[gp node right] at (1.320,4.405) { 0};
\draw[gp path] (1.504,6.115)--(1.684,6.115);
\node[gp node right] at (1.320,6.115) { 0.5};
\draw[gp path] (1.504,7.825)--(1.684,7.825);
\node[gp node right] at (1.320,7.825) { 1};
\draw[gp path] (1.504,0.985)--(1.504,1.165);
\node[gp node center] at (1.504,0.677) { 0};
\draw[gp path] (2.548,0.985)--(2.548,1.165);
\node[gp node center] at (2.548,0.677) { 2};
\draw[gp path] (3.593,0.985)--(3.593,1.165);
\node[gp node center] at (3.593,0.677) { 4};
\draw[gp path] (4.637,0.985)--(4.637,1.165);
\node[gp node center] at (4.637,0.677) { 6};
\draw[gp path] (5.681,0.985)--(5.681,1.165);
\node[gp node center] at (5.681,0.677) { 8};
\draw[gp path] (6.726,0.985)--(6.726,1.165);
\node[gp node center] at (6.726,0.677) { 10};
\draw[gp path] (7.770,0.985)--(7.770,1.165);
\node[gp node center] at (7.770,0.677) { 12};
\draw[gp path] (8.814,0.985)--(8.814,1.165);
\node[gp node center] at (8.814,0.677) { 14};
\draw[gp path] (9.858,0.985)--(9.858,1.165);
\node[gp node center] at (9.858,0.677) { 16};
\draw[gp path] (10.903,0.985)--(10.903,1.165);
\node[gp node center] at (10.903,0.677) { 18};
\draw[gp path] (1.504,7.825)--(1.504,0.985)--(11.921,0.985);
\node[gp node center,rotate=-270] at (0.246,4.405) {Ampiezza [???]};
\node[gp node center] at (6.725,0.215) {Tempo $[s]$};
\node[gp node center] at (6.725,8.287) {Dati decadimento 0.940d};
\gpcolor{rgb color={0.000,0.000,1.000}}
\gpsetlinetype{gp lt plot 0}
\draw[gp path] (1.504,4.569)--(1.530,4.504)--(1.556,4.439)--(1.582,4.357)--(1.608,4.275)%
  --(1.635,4.213)--(1.661,4.172)--(1.687,4.149)--(1.713,4.142)--(1.739,4.159)--(1.765,4.210)%
  --(1.791,4.272)--(1.817,4.337)--(1.843,4.412)--(1.870,4.487)--(1.896,4.555)--(1.922,4.600)%
  --(1.948,4.634)--(1.974,4.648)--(2.000,4.634)--(2.026,4.590)--(2.052,4.542)--(2.078,4.484)%
  --(2.104,4.408)--(2.131,4.326)--(2.157,4.261)--(2.183,4.213)--(2.209,4.172)--(2.235,4.149)%
  --(2.261,4.159)--(2.287,4.193)--(2.313,4.237)--(2.339,4.289)--(2.366,4.364)--(2.392,4.439)%
  --(2.418,4.504)--(2.444,4.559)--(2.470,4.610)--(2.496,4.634)--(2.522,4.627)--(2.548,4.607)%
  --(2.574,4.573)--(2.601,4.521)--(2.627,4.449)--(2.653,4.374)--(2.679,4.313)--(2.705,4.255)%
  --(2.731,4.203)--(2.757,4.169)--(2.783,4.166)--(2.809,4.179)--(2.835,4.210)--(2.862,4.258)%
  --(2.888,4.323)--(2.914,4.391)--(2.940,4.453)--(2.966,4.514)--(2.992,4.573)--(3.018,4.610)%
  --(3.044,4.617)--(3.070,4.614)--(3.097,4.597)--(3.123,4.552)--(3.149,4.491)--(3.175,4.426)%
  --(3.201,4.361)--(3.227,4.299)--(3.253,4.237)--(3.279,4.196)--(3.305,4.179)--(3.332,4.179)%
  --(3.358,4.193)--(3.384,4.227)--(3.410,4.289)--(3.436,4.343)--(3.462,4.408)--(3.488,4.473)%
  --(3.514,4.535)--(3.540,4.576)--(3.566,4.600)--(3.593,4.607)--(3.619,4.603)--(3.645,4.573)%
  --(3.671,4.518)--(3.697,4.467)--(3.723,4.405)--(3.749,4.343)--(3.775,4.275)--(3.801,4.227)%
  --(3.828,4.203)--(3.854,4.183)--(3.880,4.190)--(3.906,4.213)--(3.932,4.255)--(3.958,4.306)%
  --(3.984,4.364)--(4.010,4.432)--(4.036,4.494)--(4.063,4.538)--(4.089,4.573)--(4.115,4.600)%
  --(4.141,4.603)--(4.167,4.583)--(4.193,4.545)--(4.219,4.497)--(4.245,4.443)--(4.271,4.378)%
  --(4.298,4.323)--(4.324,4.268)--(4.350,4.227)--(4.376,4.203)--(4.402,4.193)--(4.428,4.207)%
  --(4.454,4.237)--(4.480,4.275)--(4.506,4.326)--(4.532,4.391)--(4.559,4.449)--(4.585,4.497)%
  --(4.611,4.542)--(4.637,4.579)--(4.663,4.590)--(4.689,4.583)--(4.715,4.562)--(4.741,4.525)%
  --(4.767,4.477)--(4.794,4.415)--(4.820,4.361)--(4.846,4.306)--(4.872,4.258)--(4.898,4.224)%
  --(4.924,4.210)--(4.950,4.207)--(4.976,4.220)--(5.002,4.248)--(5.029,4.299)--(5.055,4.354)%
  --(5.081,4.408)--(5.107,4.463)--(5.133,4.511)--(5.159,4.552)--(5.185,4.573)--(5.211,4.579)%
  --(5.237,4.573)--(5.263,4.549)--(5.290,4.508)--(5.316,4.456)--(5.342,4.402)--(5.368,4.347)%
  --(5.394,4.292)--(5.420,4.255)--(5.446,4.227)--(5.472,4.217)--(5.498,4.217)--(5.525,4.234)%
  --(5.551,4.275)--(5.577,4.320)--(5.603,4.371)--(5.629,4.426)--(5.655,4.480)--(5.681,4.525)%
  --(5.707,4.552)--(5.733,4.569)--(5.760,4.573)--(5.786,4.559)--(5.812,4.528)--(5.838,4.484)%
  --(5.864,4.436)--(5.890,4.381)--(5.916,4.330)--(5.942,4.285)--(5.968,4.251)--(5.994,4.227)%
  --(6.021,4.220)--(6.047,4.231)--(6.073,4.258)--(6.099,4.292)--(6.125,4.340)--(6.151,4.391)%
  --(6.177,4.443)--(6.203,4.491)--(6.229,4.528)--(6.256,4.552)--(6.282,4.566)--(6.308,4.562)%
  --(6.334,4.542)--(6.360,4.508)--(6.386,4.467)--(6.412,4.415)--(6.438,4.367)--(6.464,4.323)%
  --(6.491,4.282)--(6.517,4.251)--(6.543,4.227)--(6.569,4.231)--(6.595,4.248)--(6.621,4.272)%
  --(6.647,4.309)--(6.673,4.361)--(6.699,4.408)--(6.726,4.456)--(6.752,4.497)--(6.778,4.528)%
  --(6.804,4.552)--(6.830,4.555)--(6.856,4.545)--(6.882,4.525)--(6.908,4.491)--(6.934,4.443)%
  --(6.960,4.402)--(6.987,4.354)--(7.013,4.309)--(7.039,4.275)--(7.065,4.248)--(7.091,4.237)%
  --(7.117,4.237)--(7.143,4.258)--(7.169,4.289)--(7.195,4.330)--(7.222,4.374)--(7.248,4.422)%
  --(7.274,4.463)--(7.300,4.508)--(7.326,4.532)--(7.352,4.549)--(7.378,4.545)--(7.404,4.535)%
  --(7.430,4.508)--(7.457,4.473)--(7.483,4.432)--(7.509,4.384)--(7.535,4.340)--(7.561,4.296)%
  --(7.587,4.268)--(7.613,4.248)--(7.639,4.244)--(7.665,4.251)--(7.691,4.278)--(7.718,4.306)%
  --(7.744,4.347)--(7.770,4.391)--(7.796,4.436)--(7.822,4.480)--(7.848,4.511)--(7.874,4.535)%
  --(7.900,4.542)--(7.926,4.538)--(7.953,4.521)--(7.979,4.494)--(8.005,4.453)--(8.031,4.412)%
  --(8.057,4.367)--(8.083,4.330)--(8.109,4.299)--(8.135,4.268)--(8.161,4.255)--(8.188,4.251)%
  --(8.214,4.268)--(8.240,4.292)--(8.266,4.323)--(8.292,4.364)--(8.318,4.408)--(8.344,4.449)%
  --(8.370,4.484)--(8.396,4.511)--(8.422,4.535)--(8.449,4.535)--(8.475,4.528)--(8.501,4.504)%
  --(8.527,4.477)--(8.553,4.439)--(8.579,4.398)--(8.605,4.357)--(8.631,4.320)--(8.657,4.289)%
  --(8.684,4.265)--(8.710,4.258)--(8.736,4.261)--(8.762,4.278)--(8.788,4.306)--(8.814,4.337)%
  --(8.840,4.381)--(8.866,4.422)--(8.892,4.460)--(8.919,4.491)--(8.945,4.514)--(8.971,4.528)%
  --(8.997,4.525)--(9.023,4.514)--(9.049,4.491)--(9.075,4.460)--(9.101,4.426)--(9.127,4.384)%
  --(9.153,4.343)--(9.180,4.313)--(9.206,4.278)--(9.232,4.272)--(9.258,4.261)--(9.284,4.268)%
  --(9.310,4.292)--(9.336,4.320)--(9.362,4.357)--(9.388,4.391)--(9.415,4.432)--(9.441,4.467)%
  --(9.467,4.494)--(9.493,4.514)--(9.519,4.518)--(9.545,4.518)--(9.571,4.501)--(9.597,4.484)%
  --(9.623,4.446)--(9.650,4.412)--(9.676,4.371)--(9.702,4.333)--(9.728,4.302)--(9.754,4.282)%
  --(9.780,4.272)--(9.806,4.275)--(9.832,4.282)--(9.858,4.302)--(9.885,4.333)--(9.911,4.367)%
  --(9.937,4.405)--(9.963,4.443)--(9.989,4.473)--(10.015,4.497)--(10.041,4.511)--(10.067,4.514)%
  --(10.093,4.508)--(10.119,4.497)--(10.146,4.470)--(10.172,4.429)--(10.198,4.398)--(10.224,4.361)%
  --(10.250,4.326)--(10.276,4.299)--(10.302,4.285)--(10.328,4.275)--(10.354,4.285)--(10.381,4.296)%
  --(10.407,4.320)--(10.433,4.343)--(10.459,4.384)--(10.485,4.412)--(10.511,4.453)--(10.537,4.480)%
  --(10.563,4.501)--(10.589,4.508)--(10.616,4.511)--(10.642,4.497)--(10.668,4.477)--(10.694,4.453)%
  --(10.720,4.422)--(10.746,4.384)--(10.772,4.347)--(10.798,4.323)--(10.824,4.299)--(10.850,4.285)%
  --(10.877,4.282)--(10.903,4.289)--(10.929,4.306)--(10.955,4.330)--(10.981,4.364)--(11.007,4.398)%
  --(11.033,4.426)--(11.059,4.456)--(11.085,4.480)--(11.112,4.497)--(11.138,4.504)--(11.164,4.508)%
  --(11.190,4.494)--(11.216,4.470)--(11.242,4.439)--(11.268,4.405)--(11.294,4.374)--(11.320,4.343)%
  --(11.347,4.320)--(11.373,4.299)--(11.399,4.285)--(11.425,4.292)--(11.451,4.302)--(11.477,4.316)%
  --(11.503,4.347)--(11.529,4.374)--(11.555,4.405)--(11.581,4.439)--(11.608,4.467)--(11.634,4.484)%
  --(11.660,4.494)--(11.686,4.497)--(11.712,4.494)--(11.738,4.480)--(11.764,4.456)--(11.790,4.426)%
  --(11.816,4.395)--(11.843,4.367)--(11.869,4.337)--(11.895,4.309)--(11.921,4.299);
\gpcolor{color=gp lt color border}
\node[gp node right] at (10.479,7.491) {Punti tangenza};
\gpcolor{rgb color={1.000,0.000,0.000}}
\gpsetpointsize{4.00}
\gppoint{gp mark 3}{(1.537,4.501)}
\gppoint{gp mark 3}{(1.700,4.133)}
\gppoint{gp mark 3}{(1.974,4.648)}
\gppoint{gp mark 3}{(2.248,4.142)}
\gppoint{gp mark 3}{(2.509,4.638)}
\gppoint{gp mark 3}{(2.783,4.166)}
\gppoint{gp mark 3}{(3.057,4.622)}
\gppoint{gp mark 3}{(3.318,4.172)}
\gppoint{gp mark 3}{(3.593,4.607)}
\gppoint{gp mark 3}{(3.867,4.178)}
\gppoint{gp mark 3}{(4.128,4.614)}
\gppoint{gp mark 3}{(4.402,4.193)}
\gppoint{gp mark 3}{(4.676,4.593)}
\gppoint{gp mark 3}{(4.937,4.200)}
\gppoint{gp mark 3}{(5.198,4.583)}
\gppoint{gp mark 3}{(5.472,4.217)}
\gppoint{gp mark 3}{(5.746,4.579)}
\gppoint{gp mark 3}{(6.021,4.220)}
\gppoint{gp mark 3}{(6.295,4.573)}
\gppoint{gp mark 3}{(6.556,4.222)}
\gppoint{gp mark 3}{(6.817,4.561)}
\gppoint{gp mark 3}{(7.091,4.237)}
\gppoint{gp mark 3}{(7.365,4.550)}
\gppoint{gp mark 3}{(7.639,4.244)}
\gppoint{gp mark 3}{(7.913,4.547)}
\gppoint{gp mark 3}{(8.174,4.243)}
\gppoint{gp mark 3}{(8.436,4.538)}
\gppoint{gp mark 3}{(8.710,4.258)}
\gppoint{gp mark 3}{(8.984,4.530)}
\gppoint{gp mark 3}{(9.258,4.261)}
\gppoint{gp mark 3}{(9.532,4.526)}
\gppoint{gp mark 3}{(9.806,4.275)}
\gppoint{gp mark 3}{(10.067,4.514)}
\gppoint{gp mark 3}{(10.328,4.275)}
\gppoint{gp mark 3}{(10.602,4.518)}
\gppoint{gp mark 3}{(10.877,4.282)}
\gppoint{gp mark 3}{(11.151,4.514)}
\gppoint{gp mark 3}{(11.412,4.287)}
\gppoint{gp mark 3}{(11.673,4.499)}
\gppoint{gp mark 3}{(11.121,7.491)}
\gpcolor{color=gp lt color border}
\node[gp node right] at (10.479,7.183) {fexp(x)};
\gpcolor{color=gp lt color 2}
\gpsetlinetype{gp lt plot 2}
\draw[gp path] (10.663,7.183)--(11.579,7.183);
\draw[gp path] (1.504,7.825)--(1.609,7.798)--(1.714,7.771)--(1.820,7.744)--(1.925,7.718)%
  --(2.030,7.691)--(2.135,7.665)--(2.241,7.639)--(2.346,7.614)--(2.451,7.588)--(2.556,7.563)%
  --(2.661,7.538)--(2.767,7.513)--(2.872,7.488)--(2.977,7.464)--(3.082,7.439)--(3.188,7.415)%
  --(3.293,7.391)--(3.398,7.368)--(3.503,7.344)--(3.608,7.321)--(3.714,7.298)--(3.819,7.275)%
  --(3.924,7.252)--(4.029,7.229)--(4.135,7.207)--(4.240,7.185)--(4.345,7.162)--(4.450,7.141)%
  --(4.555,7.119)--(4.661,7.097)--(4.766,7.076)--(4.871,7.055)--(4.976,7.034)--(5.082,7.013)%
  --(5.187,6.992)--(5.292,6.972)--(5.397,6.951)--(5.502,6.931)--(5.608,6.911)--(5.713,6.891)%
  --(5.818,6.871)--(5.923,6.852)--(6.029,6.832)--(6.134,6.813)--(6.239,6.794)--(6.344,6.775)%
  --(6.449,6.756)--(6.555,6.737)--(6.660,6.719)--(6.765,6.700)--(6.870,6.682)--(6.975,6.664)%
  --(7.081,6.646)--(7.186,6.628)--(7.291,6.611)--(7.396,6.593)--(7.502,6.576)--(7.607,6.559)%
  --(7.712,6.541)--(7.817,6.524)--(7.922,6.508)--(8.028,6.491)--(8.133,6.474)--(8.238,6.458)%
  --(8.343,6.442)--(8.449,6.425)--(8.554,6.409)--(8.659,6.393)--(8.764,6.378)--(8.869,6.362)%
  --(8.975,6.346)--(9.080,6.331)--(9.185,6.316)--(9.290,6.301)--(9.396,6.285)--(9.501,6.271)%
  --(9.606,6.256)--(9.711,6.241)--(9.816,6.226)--(9.922,6.212)--(10.027,6.198)--(10.132,6.183)%
  --(10.237,6.169)--(10.343,6.155)--(10.448,6.141)--(10.553,6.128)--(10.658,6.114)--(10.763,6.100)%
  --(10.869,6.087)--(10.974,6.073)--(11.079,6.060)--(11.184,6.047)--(11.290,6.034)--(11.395,6.021)%
  --(11.500,6.008)--(11.605,5.996)--(11.710,5.983)--(11.816,5.970)--(11.921,5.958);
\gpcolor{color=gp lt color border}
\node[gp node right] at (10.479,6.875) {fexpneg(x)};
\gpcolor{color=gp lt color 3}
\gpsetlinetype{gp lt plot 3}
\draw[gp path] (10.663,6.875)--(11.579,6.875);
\draw[gp path] (1.504,0.985)--(1.609,1.012)--(1.714,1.039)--(1.820,1.066)--(1.925,1.092)%
  --(2.030,1.119)--(2.135,1.145)--(2.241,1.171)--(2.346,1.196)--(2.451,1.222)--(2.556,1.247)%
  --(2.661,1.272)--(2.767,1.297)--(2.872,1.322)--(2.977,1.346)--(3.082,1.371)--(3.188,1.395)%
  --(3.293,1.419)--(3.398,1.442)--(3.503,1.466)--(3.608,1.489)--(3.714,1.512)--(3.819,1.535)%
  --(3.924,1.558)--(4.029,1.581)--(4.135,1.603)--(4.240,1.625)--(4.345,1.648)--(4.450,1.669)%
  --(4.555,1.691)--(4.661,1.713)--(4.766,1.734)--(4.871,1.755)--(4.976,1.776)--(5.082,1.797)%
  --(5.187,1.818)--(5.292,1.838)--(5.397,1.859)--(5.502,1.879)--(5.608,1.899)--(5.713,1.919)%
  --(5.818,1.939)--(5.923,1.958)--(6.029,1.978)--(6.134,1.997)--(6.239,2.016)--(6.344,2.035)%
  --(6.449,2.054)--(6.555,2.073)--(6.660,2.091)--(6.765,2.110)--(6.870,2.128)--(6.975,2.146)%
  --(7.081,2.164)--(7.186,2.182)--(7.291,2.199)--(7.396,2.217)--(7.502,2.234)--(7.607,2.251)%
  --(7.712,2.269)--(7.817,2.286)--(7.922,2.302)--(8.028,2.319)--(8.133,2.336)--(8.238,2.352)%
  --(8.343,2.368)--(8.449,2.385)--(8.554,2.401)--(8.659,2.417)--(8.764,2.432)--(8.869,2.448)%
  --(8.975,2.464)--(9.080,2.479)--(9.185,2.494)--(9.290,2.509)--(9.396,2.525)--(9.501,2.539)%
  --(9.606,2.554)--(9.711,2.569)--(9.816,2.584)--(9.922,2.598)--(10.027,2.612)--(10.132,2.627)%
  --(10.237,2.641)--(10.343,2.655)--(10.448,2.669)--(10.553,2.682)--(10.658,2.696)--(10.763,2.710)%
  --(10.869,2.723)--(10.974,2.737)--(11.079,2.750)--(11.184,2.763)--(11.290,2.776)--(11.395,2.789)%
  --(11.500,2.802)--(11.605,2.814)--(11.710,2.827)--(11.816,2.840)--(11.921,2.852);
\gpcolor{color=gp lt color border}
\gpsetlinetype{gp lt border}
\draw[gp path] (1.504,7.825)--(1.504,0.985)--(11.921,0.985);
%% coordinates of the plot area
\gpdefrectangularnode{gp plot 1}{\pgfpoint{1.504cm}{0.985cm}}{\pgfpoint{11.947cm}{7.825cm}}
\end{tikzpicture}
%% gnuplot variables
 \caption{Grafico 0.940dgdecad.tex} \label{gr:0.940dgdecad.tex} \end{grafico}
\begin{grafico} \centering \begin{tikzpicture}[gnuplot]
%% generated with GNUPLOT 4.6p0 (Lua 5.1; terminal rev. 99, script rev. 100)
%% Tue 10 Jun 2014 07:57:01 PM CEST
\path (0.000,0.000) rectangle (12.500,8.750);
\gpcolor{color=gp lt color border}
\gpsetlinetype{gp lt border}
\gpsetlinewidth{1.00}
\draw[gp path] (1.504,0.985)--(1.684,0.985);
\node[gp node right] at (1.320,0.985) {-1};
\draw[gp path] (1.504,2.695)--(1.684,2.695);
\node[gp node right] at (1.320,2.695) {-0.5};
\draw[gp path] (1.504,4.405)--(1.684,4.405);
\node[gp node right] at (1.320,4.405) { 0};
\draw[gp path] (1.504,6.115)--(1.684,6.115);
\node[gp node right] at (1.320,6.115) { 0.5};
\draw[gp path] (1.504,7.825)--(1.684,7.825);
\node[gp node right] at (1.320,7.825) { 1};
\draw[gp path] (1.504,0.985)--(1.504,1.165);
\node[gp node center] at (1.504,0.677) { 0};
\draw[gp path] (3.593,0.985)--(3.593,1.165);
\node[gp node center] at (3.593,0.677) { 5};
\draw[gp path] (5.681,0.985)--(5.681,1.165);
\node[gp node center] at (5.681,0.677) { 10};
\draw[gp path] (7.770,0.985)--(7.770,1.165);
\node[gp node center] at (7.770,0.677) { 15};
\draw[gp path] (9.858,0.985)--(9.858,1.165);
\node[gp node center] at (9.858,0.677) { 20};
\draw[gp path] (1.504,7.825)--(1.504,0.985)--(9.984,0.985);
\node[gp node center,rotate=-270] at (0.246,4.405) {Ampiezza [???]};
\node[gp node center] at (6.725,0.215) {Tempo $[s]$};
\node[gp node center] at (6.725,8.287) {Dati decadimento 0.960d};
\gpcolor{rgb color={0.000,0.000,1.000}}
\gpsetlinetype{gp lt plot 0}
\draw[gp path] (1.504,3.902)--(1.525,3.769)--(1.546,3.690)--(1.567,3.677)--(1.588,3.728)%
  --(1.608,3.841)--(1.629,4.001)--(1.650,4.200)--(1.671,4.412)--(1.692,4.624)--(1.713,4.815)%
  --(1.734,4.966)--(1.755,5.058)--(1.776,5.096)--(1.796,5.068)--(1.817,4.976)--(1.838,4.833)%
  --(1.859,4.648)--(1.880,4.443)--(1.901,4.234)--(1.922,4.042)--(1.943,3.882)--(1.963,3.772)%
  --(1.984,3.714)--(2.005,3.724)--(2.026,3.789)--(2.047,3.916)--(2.068,4.080)--(2.089,4.275)%
  --(2.110,4.480)--(2.131,4.675)--(2.151,4.843)--(2.172,4.966)--(2.193,5.041)--(2.214,5.055)%
  --(2.235,5.007)--(2.256,4.908)--(2.277,4.761)--(2.298,4.573)--(2.319,4.374)--(2.339,4.179)%
  --(2.360,4.008)--(2.381,3.868)--(2.402,3.776)--(2.423,3.745)--(2.444,3.772)--(2.465,3.851)%
  --(2.486,3.984)--(2.507,4.152)--(2.527,4.343)--(2.548,4.535)--(2.569,4.713)--(2.590,4.863)%
  --(2.611,4.969)--(2.632,5.021)--(2.653,5.010)--(2.674,4.952)--(2.695,4.843)--(2.715,4.685)%
  --(2.736,4.504)--(2.757,4.316)--(2.778,4.131)--(2.799,3.978)--(2.820,3.861)--(2.841,3.789)%
  --(2.862,3.779)--(2.882,3.820)--(2.903,3.916)--(2.924,4.053)--(2.945,4.220)--(2.966,4.402)%
  --(2.987,4.583)--(3.008,4.747)--(3.029,4.877)--(3.050,4.962)--(3.070,4.993)--(3.091,4.973)%
  --(3.112,4.897)--(3.133,4.774)--(3.154,4.614)--(3.175,4.443)--(3.196,4.265)--(3.217,4.094)%
  --(3.238,3.957)--(3.258,3.861)--(3.279,3.810)--(3.300,3.813)--(3.321,3.875)--(3.342,3.978)%
  --(3.363,4.118)--(3.384,4.285)--(3.405,4.456)--(3.426,4.631)--(3.446,4.774)--(3.467,4.884)%
  --(3.488,4.949)--(3.509,4.962)--(3.530,4.925)--(3.551,4.839)--(3.572,4.713)--(3.593,4.555)%
  --(3.613,4.384)--(3.634,4.217)--(3.655,4.063)--(3.676,3.950)--(3.697,3.865)--(3.718,3.837)%
  --(3.739,3.858)--(3.760,3.930)--(3.781,4.036)--(3.801,4.179)--(3.822,4.347)--(3.843,4.511)%
  --(3.864,4.665)--(3.885,4.791)--(3.906,4.884)--(3.927,4.932)--(3.948,4.925)--(3.969,4.874)%
  --(3.989,4.781)--(4.010,4.644)--(4.031,4.494)--(4.052,4.333)--(4.073,4.176)--(4.094,4.039)%
  --(4.115,3.940)--(4.136,3.882)--(4.157,3.868)--(4.177,3.902)--(4.198,3.981)--(4.219,4.097)%
  --(4.240,4.241)--(4.261,4.398)--(4.282,4.555)--(4.303,4.692)--(4.324,4.805)--(4.344,4.877)%
  --(4.365,4.908)--(4.386,4.887)--(4.407,4.826)--(4.428,4.723)--(4.449,4.590)--(4.470,4.439)%
  --(4.491,4.285)--(4.512,4.138)--(4.532,4.022)--(4.553,3.936)--(4.574,3.895)--(4.595,3.895)%
  --(4.616,3.947)--(4.637,4.036)--(4.658,4.152)--(4.679,4.296)--(4.700,4.446)--(4.720,4.590)%
  --(4.741,4.716)--(4.762,4.812)--(4.783,4.870)--(4.804,4.884)--(4.825,4.850)--(4.846,4.771)%
  --(4.867,4.668)--(4.888,4.535)--(4.908,4.388)--(4.929,4.244)--(4.950,4.111)--(4.971,4.012)%
  --(4.992,3.940)--(5.013,3.916)--(5.034,3.933)--(5.055,3.995)--(5.076,4.087)--(5.096,4.210)%
  --(5.117,4.350)--(5.138,4.491)--(5.159,4.620)--(5.180,4.737)--(5.201,4.815)--(5.222,4.856)%
  --(5.243,4.853)--(5.263,4.809)--(5.284,4.723)--(5.305,4.614)--(5.326,4.484)--(5.347,4.343)%
  --(5.368,4.203)--(5.389,4.094)--(5.410,4.005)--(5.431,3.950)--(5.451,3.940)--(5.472,3.971)%
  --(5.493,4.039)--(5.514,4.138)--(5.535,4.261)--(5.556,4.398)--(5.577,4.525)--(5.598,4.651)%
  --(5.619,4.750)--(5.639,4.812)--(5.660,4.833)--(5.681,4.819)--(5.702,4.764)--(5.723,4.675)%
  --(5.744,4.562)--(5.765,4.436)--(5.786,4.302)--(5.807,4.179)--(5.827,4.073)--(5.848,4.005)%
  --(5.869,3.967)--(5.890,3.967)--(5.911,4.008)--(5.932,4.080)--(5.953,4.186)--(5.974,4.306)%
  --(5.994,4.439)--(6.015,4.559)--(6.036,4.668)--(6.057,4.750)--(6.078,4.798)--(6.099,4.812)%
  --(6.120,4.785)--(6.141,4.723)--(6.162,4.627)--(6.182,4.514)--(6.203,4.388)--(6.224,4.268)%
  --(6.245,4.152)--(6.266,4.063)--(6.287,4.005)--(6.308,3.984)--(6.329,3.995)--(6.350,4.046)%
  --(6.370,4.128)--(6.391,4.234)--(6.412,4.354)--(6.433,4.473)--(6.454,4.590)--(6.475,4.685)%
  --(6.496,4.754)--(6.517,4.791)--(6.538,4.785)--(6.558,4.750)--(6.579,4.679)--(6.600,4.583)%
  --(6.621,4.470)--(6.642,4.350)--(6.663,4.234)--(6.684,4.135)--(6.705,4.056)--(6.726,4.012)%
  --(6.746,4.005)--(6.767,4.029)--(6.788,4.087)--(6.809,4.169)--(6.830,4.278)--(6.851,4.391)%
  --(6.872,4.511)--(6.893,4.614)--(6.913,4.692)--(6.934,4.750)--(6.955,4.771)--(6.976,4.761)%
  --(6.997,4.713)--(7.018,4.638)--(7.039,4.538)--(7.060,4.432)--(7.081,4.316)--(7.101,4.210)%
  --(7.122,4.125)--(7.143,4.056)--(7.164,4.025)--(7.185,4.029)--(7.206,4.060)--(7.227,4.125)%
  --(7.248,4.213)--(7.269,4.323)--(7.289,4.432)--(7.310,4.538)--(7.331,4.631)--(7.352,4.699)%
  --(7.373,4.744)--(7.394,4.754)--(7.415,4.730)--(7.436,4.679)--(7.457,4.593)--(7.477,4.497)%
  --(7.498,4.395)--(7.519,4.285)--(7.540,4.186)--(7.561,4.114)--(7.582,4.060)--(7.603,4.039)%
  --(7.624,4.053)--(7.644,4.094)--(7.665,4.162)--(7.686,4.258)--(7.707,4.361)--(7.728,4.463)%
  --(7.749,4.559)--(7.770,4.641)--(7.791,4.699)--(7.812,4.730)--(7.832,4.730)--(7.853,4.699)%
  --(7.874,4.641)--(7.895,4.555)--(7.916,4.463)--(7.937,4.357)--(7.958,4.258)--(7.979,4.172)%
  --(8.000,4.104)--(8.020,4.066)--(8.041,4.053)--(8.062,4.080)--(8.083,4.128)--(8.104,4.203)%
  --(8.125,4.289)--(8.146,4.388)--(8.167,4.491)--(8.188,4.576)--(8.208,4.651)--(8.229,4.699)%
  --(8.250,4.720)--(8.271,4.706)--(8.292,4.665)--(8.313,4.600)--(8.334,4.518)--(8.355,4.422)%
  --(8.375,4.323)--(8.396,4.234)--(8.417,4.159)--(8.438,4.104)--(8.459,4.073)--(8.480,4.077)%
  --(8.501,4.104)--(8.522,4.159)--(8.543,4.237)--(8.563,4.330)--(8.584,4.426)--(8.605,4.518)%
  --(8.626,4.597)--(8.647,4.655)--(8.668,4.692)--(8.689,4.703)--(8.710,4.682)--(8.731,4.638)%
  --(8.751,4.566)--(8.772,4.484)--(8.793,4.395)--(8.814,4.302)--(8.835,4.217)--(8.856,4.152)%
  --(8.877,4.104)--(8.898,4.090)--(8.919,4.101)--(8.939,4.135)--(8.960,4.200)--(8.981,4.275)%
  --(9.002,4.361)--(9.023,4.453)--(9.044,4.538)--(9.065,4.607)--(9.086,4.658)--(9.107,4.685)%
  --(9.127,4.682)--(9.148,4.658)--(9.169,4.603)--(9.190,4.535)--(9.211,4.453)--(9.232,4.361)%
  --(9.253,4.275)--(9.274,4.200)--(9.294,4.142)--(9.315,4.107)--(9.336,4.097)--(9.357,4.118)%
  --(9.378,4.159)--(9.399,4.227)--(9.420,4.302)--(9.441,4.384)--(9.462,4.470)--(9.482,4.552)%
  --(9.503,4.607)--(9.524,4.655)--(9.545,4.668)--(9.566,4.662)--(9.587,4.624)--(9.608,4.573)%
  --(9.629,4.501)--(9.650,4.419)--(9.670,4.333)--(9.691,4.255)--(9.712,4.190)--(9.733,4.142)%
  --(9.754,4.114)--(9.775,4.118)--(9.796,4.145)--(9.817,4.190)--(9.838,4.255)--(9.858,4.333)%
  --(9.879,4.415)--(9.900,4.494)--(9.921,4.566)--(9.942,4.617)--(9.963,4.644)--(9.984,4.655);
\gpcolor{color=gp lt color border}
\node[gp node right] at (10.479,7.491) {Punti tangenza};
\gpcolor{rgb color={1.000,0.000,0.000}}
\gpsetpointsize{4.00}
\gppoint{gp mark 3}{(1.582,3.700)}
\gppoint{gp mark 3}{(1.776,5.096)}
\gppoint{gp mark 3}{(1.995,3.692)}
\gppoint{gp mark 3}{(2.204,5.079)}
\gppoint{gp mark 3}{(2.423,3.745)}
\gppoint{gp mark 3}{(2.642,5.039)}
\gppoint{gp mark 3}{(2.862,3.779)}
\gppoint{gp mark 3}{(3.081,5.010)}
\gppoint{gp mark 3}{(3.290,3.783)}
\gppoint{gp mark 3}{(3.499,4.981)}
\gppoint{gp mark 3}{(3.718,3.837)}
\gppoint{gp mark 3}{(3.937,4.950)}
\gppoint{gp mark 3}{(4.157,3.868)}
\gppoint{gp mark 3}{(4.376,4.918)}
\gppoint{gp mark 3}{(4.585,3.870)}
\gppoint{gp mark 3}{(4.794,4.901)}
\gppoint{gp mark 3}{(5.013,3.916)}
\gppoint{gp mark 3}{(5.232,4.875)}
\gppoint{gp mark 3}{(5.451,3.940)}
\gppoint{gp mark 3}{(5.671,4.848)}
\gppoint{gp mark 3}{(5.880,3.947)}
\gppoint{gp mark 3}{(6.088,4.826)}
\gppoint{gp mark 3}{(6.308,3.984)}
\gppoint{gp mark 3}{(6.527,4.802)}
\gppoint{gp mark 3}{(6.746,4.005)}
\gppoint{gp mark 3}{(6.966,4.785)}
\gppoint{gp mark 3}{(7.175,4.013)}
\gppoint{gp mark 3}{(7.383,4.766)}
\gppoint{gp mark 3}{(7.603,4.039)}
\gppoint{gp mark 3}{(7.822,4.745)}
\gppoint{gp mark 3}{(8.041,4.053)}
\gppoint{gp mark 3}{(8.261,4.726)}
\gppoint{gp mark 3}{(8.469,4.063)}
\gppoint{gp mark 3}{(8.678,4.713)}
\gppoint{gp mark 3}{(8.898,4.090)}
\gppoint{gp mark 3}{(9.117,4.694)}
\gppoint{gp mark 3}{(9.336,4.097)}
\gppoint{gp mark 3}{(9.556,4.680)}
\gppoint{gp mark 3}{(9.764,4.104)}
\gppoint{gp mark 3}{(11.121,7.491)}
\gpcolor{color=gp lt color border}
\node[gp node right] at (10.479,7.183) {fexp(x)};
\gpcolor{color=gp lt color 2}
\gpsetlinetype{gp lt plot 2}
\draw[gp path] (10.663,7.183)--(11.579,7.183);
\draw[gp path] (1.504,7.825)--(1.590,7.791)--(1.675,7.758)--(1.761,7.725)--(1.847,7.692)%
  --(1.932,7.660)--(2.018,7.628)--(2.104,7.596)--(2.189,7.565)--(2.275,7.534)--(2.361,7.503)%
  --(2.446,7.472)--(2.532,7.442)--(2.617,7.412)--(2.703,7.383)--(2.789,7.353)--(2.874,7.324)%
  --(2.960,7.296)--(3.046,7.267)--(3.131,7.239)--(3.217,7.211)--(3.303,7.184)--(3.388,7.156)%
  --(3.474,7.129)--(3.560,7.102)--(3.645,7.076)--(3.731,7.049)--(3.817,7.023)--(3.902,6.998)%
  --(3.988,6.972)--(4.074,6.947)--(4.159,6.922)--(4.245,6.897)--(4.331,6.873)--(4.416,6.848)%
  --(4.502,6.824)--(4.588,6.800)--(4.673,6.777)--(4.759,6.753)--(4.844,6.730)--(4.930,6.707)%
  --(5.016,6.685)--(5.101,6.662)--(5.187,6.640)--(5.273,6.618)--(5.358,6.596)--(5.444,6.575)%
  --(5.530,6.553)--(5.615,6.532)--(5.701,6.511)--(5.787,6.491)--(5.872,6.470)--(5.958,6.450)%
  --(6.044,6.430)--(6.129,6.410)--(6.215,6.390)--(6.301,6.370)--(6.386,6.351)--(6.472,6.332)%
  --(6.558,6.313)--(6.643,6.294)--(6.729,6.276)--(6.815,6.257)--(6.900,6.239)--(6.986,6.221)%
  --(7.071,6.203)--(7.157,6.185)--(7.243,6.168)--(7.328,6.150)--(7.414,6.133)--(7.500,6.116)%
  --(7.585,6.099)--(7.671,6.083)--(7.757,6.066)--(7.842,6.050)--(7.928,6.034)--(8.014,6.018)%
  --(8.099,6.002)--(8.185,5.986)--(8.271,5.971)--(8.356,5.955)--(8.442,5.940)--(8.528,5.925)%
  --(8.613,5.910)--(8.699,5.895)--(8.785,5.880)--(8.870,5.866)--(8.956,5.851)--(9.042,5.837)%
  --(9.127,5.823)--(9.213,5.809)--(9.298,5.795)--(9.384,5.782)--(9.470,5.768)--(9.555,5.755)%
  --(9.641,5.741)--(9.727,5.728)--(9.812,5.715)--(9.898,5.702)--(9.984,5.690);
\gpcolor{color=gp lt color border}
\node[gp node right] at (10.479,6.875) {fexpneg(x)};
\gpcolor{color=gp lt color 3}
\gpsetlinetype{gp lt plot 3}
\draw[gp path] (10.663,6.875)--(11.579,6.875);
\draw[gp path] (1.504,0.985)--(1.590,1.019)--(1.675,1.052)--(1.761,1.085)--(1.847,1.118)%
  --(1.932,1.150)--(2.018,1.182)--(2.104,1.214)--(2.189,1.245)--(2.275,1.276)--(2.361,1.307)%
  --(2.446,1.338)--(2.532,1.368)--(2.617,1.398)--(2.703,1.427)--(2.789,1.457)--(2.874,1.486)%
  --(2.960,1.514)--(3.046,1.543)--(3.131,1.571)--(3.217,1.599)--(3.303,1.626)--(3.388,1.654)%
  --(3.474,1.681)--(3.560,1.708)--(3.645,1.734)--(3.731,1.761)--(3.817,1.787)--(3.902,1.812)%
  --(3.988,1.838)--(4.074,1.863)--(4.159,1.888)--(4.245,1.913)--(4.331,1.937)--(4.416,1.962)%
  --(4.502,1.986)--(4.588,2.010)--(4.673,2.033)--(4.759,2.057)--(4.844,2.080)--(4.930,2.103)%
  --(5.016,2.125)--(5.101,2.148)--(5.187,2.170)--(5.273,2.192)--(5.358,2.214)--(5.444,2.235)%
  --(5.530,2.257)--(5.615,2.278)--(5.701,2.299)--(5.787,2.319)--(5.872,2.340)--(5.958,2.360)%
  --(6.044,2.380)--(6.129,2.400)--(6.215,2.420)--(6.301,2.440)--(6.386,2.459)--(6.472,2.478)%
  --(6.558,2.497)--(6.643,2.516)--(6.729,2.534)--(6.815,2.553)--(6.900,2.571)--(6.986,2.589)%
  --(7.071,2.607)--(7.157,2.625)--(7.243,2.642)--(7.328,2.660)--(7.414,2.677)--(7.500,2.694)%
  --(7.585,2.711)--(7.671,2.727)--(7.757,2.744)--(7.842,2.760)--(7.928,2.776)--(8.014,2.792)%
  --(8.099,2.808)--(8.185,2.824)--(8.271,2.839)--(8.356,2.855)--(8.442,2.870)--(8.528,2.885)%
  --(8.613,2.900)--(8.699,2.915)--(8.785,2.930)--(8.870,2.944)--(8.956,2.959)--(9.042,2.973)%
  --(9.127,2.987)--(9.213,3.001)--(9.298,3.015)--(9.384,3.028)--(9.470,3.042)--(9.555,3.055)%
  --(9.641,3.069)--(9.727,3.082)--(9.812,3.095)--(9.898,3.108)--(9.984,3.120);
\gpcolor{color=gp lt color border}
\gpsetlinetype{gp lt border}
\draw[gp path] (1.504,7.825)--(1.504,0.985)--(9.984,0.985);
%% coordinates of the plot area
\gpdefrectangularnode{gp plot 1}{\pgfpoint{1.504cm}{0.985cm}}{\pgfpoint{11.947cm}{7.825cm}}
\end{tikzpicture}
%% gnuplot variables
 \caption{Grafico 0.960dgdecad.tex} \label{gr:0.960dgdecad.tex} \end{grafico}
\begin{grafico} \centering \begin{tikzpicture}[gnuplot]
%% generated with GNUPLOT 4.6p4 (Lua 5.1; terminal rev. 99, script rev. 100)
%% gio 04 dic 2014 15:20:07 CET
\path (0.000,0.000) rectangle (12.500,8.750);
\gpcolor{color=gp lt color border}
\gpsetlinetype{gp lt border}
\gpsetlinewidth{1.00}
\draw[gp path] (1.504,2.218)--(1.684,2.218);
\node[gp node right] at (1.320,2.218) {-0.2};
\draw[gp path] (1.504,3.450)--(1.684,3.450);
\node[gp node right] at (1.320,3.450) {-0.1};
\draw[gp path] (1.504,4.683)--(1.684,4.683);
\node[gp node right] at (1.320,4.683) { 0};
\draw[gp path] (1.504,5.916)--(1.684,5.916);
\node[gp node right] at (1.320,5.916) { 0.1};
\draw[gp path] (1.504,7.148)--(1.684,7.148);
\node[gp node right] at (1.320,7.148) { 0.2};
\draw[gp path] (1.504,0.985)--(1.504,1.165);
\node[gp node center] at (1.504,0.677) { 0};
\draw[gp path] (2.664,0.985)--(2.664,1.165);
\node[gp node center] at (2.664,0.677) { 5};
\draw[gp path] (3.825,0.985)--(3.825,1.165);
\node[gp node center] at (3.825,0.677) { 10};
\draw[gp path] (4.985,0.985)--(4.985,1.165);
\node[gp node center] at (4.985,0.677) { 15};
\draw[gp path] (6.145,0.985)--(6.145,1.165);
\node[gp node center] at (6.145,0.677) { 20};
\draw[gp path] (7.306,0.985)--(7.306,1.165);
\node[gp node center] at (7.306,0.677) { 25};
\draw[gp path] (8.466,0.985)--(8.466,1.165);
\node[gp node center] at (8.466,0.677) { 30};
\draw[gp path] (9.626,0.985)--(9.626,1.165);
\node[gp node center] at (9.626,0.677) { 35};
\draw[gp path] (10.787,0.985)--(10.787,1.165);
\node[gp node center] at (10.787,0.677) { 40};
\draw[gp path] (1.504,7.789)--(1.504,1.611);
\draw[gp path] (1.504,0.985)--(10.845,0.985);
\node[gp node center,rotate=-270] at (0.246,4.683) {Ampiezza [giri]};
\node[gp node center] at (6.725,0.215) {Tempo $[s]$};
\gpcolor{rgb color={0.000,0.000,1.000}}
\gpsetlinetype{gp lt plot 0}
\draw[gp path] (1.504,3.956)--(1.516,3.093)--(1.527,2.366)--(1.539,1.873)--(1.550,1.651)%
  --(1.562,1.688)--(1.574,1.996)--(1.585,2.550)--(1.597,3.327)--(1.608,4.190)--(1.620,5.102)%
  --(1.632,5.977)--(1.643,6.742)--(1.655,7.296)--(1.666,7.604)--(1.678,7.678)--(1.690,7.456)%
  --(1.701,6.976)--(1.713,6.298)--(1.724,5.472)--(1.736,4.584)--(1.748,3.709)--(1.759,2.933)%
  --(1.771,2.329)--(1.782,1.910)--(1.794,1.786)--(1.806,1.897)--(1.817,2.279)--(1.829,2.883)%
  --(1.840,3.635)--(1.852,4.486)--(1.864,5.361)--(1.875,6.162)--(1.887,6.828)--(1.899,7.284)%
  --(1.910,7.506)--(1.922,7.481)--(1.933,7.198)--(1.945,6.668)--(1.957,5.965)--(1.968,5.164)%
  --(1.980,4.338)--(1.991,3.524)--(2.003,2.809)--(2.015,2.267)--(2.026,1.996)--(2.038,1.934)%
  --(2.049,2.131)--(2.061,2.550)--(2.073,3.191)--(2.084,3.943)--(2.096,4.757)--(2.107,5.571)%
  --(2.119,6.298)--(2.131,6.877)--(2.142,7.259)--(2.154,7.395)--(2.165,7.284)--(2.177,6.914)%
  --(2.189,6.372)--(2.200,5.681)--(2.212,4.893)--(2.223,4.079)--(2.235,3.339)--(2.247,2.723)%
  --(2.258,2.304)--(2.270,2.070)--(2.281,2.082)--(2.293,2.304)--(2.305,2.797)--(2.316,3.426)%
  --(2.328,4.165)--(2.339,4.991)--(2.351,5.743)--(2.363,6.396)--(2.374,6.926)--(2.386,7.210)%
  --(2.397,7.235)--(2.409,7.074)--(2.421,6.729)--(2.432,6.150)--(2.444,5.398)--(2.455,4.646)%
  --(2.467,3.943)--(2.479,3.265)--(2.490,2.686)--(2.502,2.292)--(2.513,2.193)--(2.525,2.292)%
  --(2.537,2.563)--(2.548,3.068)--(2.560,3.746)--(2.572,4.473)--(2.583,5.176)--(2.595,5.866)%
  --(2.606,6.532)--(2.618,6.914)--(2.630,7.050)--(2.641,7.062)--(2.653,6.877)--(2.664,6.421)%
  --(2.676,5.780)--(2.688,5.114)--(2.699,4.436)--(2.711,3.709)--(2.722,3.081)--(2.734,2.649)%
  --(2.746,2.427)--(2.757,2.341)--(2.769,2.452)--(2.780,2.846)--(2.792,3.413)--(2.804,4.030)%
  --(2.815,4.695)--(2.827,5.410)--(2.838,6.088)--(2.850,6.544)--(2.862,6.828)--(2.873,6.976)%
  --(2.885,6.939)--(2.896,6.618)--(2.908,6.113)--(2.920,5.558)--(2.931,4.917)--(2.943,4.190)%
  --(2.954,3.537)--(2.966,3.031)--(2.978,2.674)--(2.989,2.452)--(3.001,2.452)--(3.012,2.686)%
  --(3.024,3.105)--(3.036,3.623)--(3.047,4.227)--(3.059,4.942)--(3.070,5.608)--(3.082,6.138)%
  --(3.094,6.544)--(3.105,6.816)--(3.117,6.889)--(3.128,6.717)--(3.140,6.396)--(3.152,5.916)%
  --(3.163,5.312)--(3.175,4.646)--(3.186,4.005)--(3.198,3.438)--(3.210,2.994)--(3.221,2.661)%
  --(3.233,2.538)--(3.245,2.637)--(3.256,2.896)--(3.268,3.302)--(3.279,3.857)--(3.291,4.498)%
  --(3.303,5.139)--(3.314,5.706)--(3.326,6.199)--(3.337,6.569)--(3.349,6.742)--(3.361,6.729)%
  --(3.372,6.532)--(3.384,6.175)--(3.395,5.681)--(3.407,5.077)--(3.419,4.473)--(3.430,3.869)%
  --(3.442,3.339)--(3.453,2.957)--(3.465,2.723)--(3.477,2.674)--(3.488,2.797)--(3.500,3.093)%
  --(3.511,3.549)--(3.523,4.091)--(3.535,4.695)--(3.546,5.299)--(3.558,5.829)--(3.569,6.261)%
  --(3.581,6.544)--(3.593,6.655)--(3.604,6.594)--(3.616,6.347)--(3.627,5.953)--(3.639,5.447)%
  --(3.651,4.880)--(3.662,4.289)--(3.674,3.722)--(3.685,3.278)--(3.697,2.957)--(3.709,2.785)%
  --(3.720,2.772)--(3.732,2.957)--(3.743,3.302)--(3.755,3.759)--(3.767,4.313)--(3.778,4.880)%
  --(3.790,5.423)--(3.801,5.903)--(3.813,6.273)--(3.825,6.507)--(3.836,6.569)--(3.848,6.446)%
  --(3.859,6.150)--(3.871,5.731)--(3.883,5.250)--(3.894,4.683)--(3.906,4.128)--(3.917,3.623)%
  --(3.929,3.228)--(3.941,2.970)--(3.952,2.846)--(3.964,2.908)--(3.976,3.142)--(3.987,3.487)%
  --(3.999,3.943)--(4.010,4.498)--(4.022,5.040)--(4.034,5.546)--(4.045,5.965)--(4.057,6.285)%
  --(4.068,6.458)--(4.080,6.433)--(4.092,6.273)--(4.103,5.965)--(4.115,5.558)--(4.126,5.040)%
  --(4.138,4.510)--(4.150,3.993)--(4.161,3.549)--(4.173,3.191)--(4.184,2.994)--(4.196,2.945)%
  --(4.208,3.056)--(4.219,3.302)--(4.231,3.685)--(4.242,4.165)--(4.254,4.671)--(4.266,5.176)%
  --(4.277,5.644)--(4.289,6.027)--(4.300,6.298)--(4.312,6.384)--(4.324,6.310)--(4.335,6.113)%
  --(4.347,5.780)--(4.358,5.349)--(4.370,4.856)--(4.382,4.350)--(4.393,3.882)--(4.405,3.487)%
  --(4.416,3.191)--(4.428,3.044)--(4.440,3.044)--(4.451,3.191)--(4.463,3.487)--(4.474,3.882)%
  --(4.486,4.350)--(4.498,4.819)--(4.509,5.299)--(4.521,5.718)--(4.532,6.051)--(4.544,6.236)%
  --(4.556,6.285)--(4.567,6.187)--(4.579,5.953)--(4.590,5.595)--(4.602,5.176)--(4.614,4.695)%
  --(4.625,4.215)--(4.637,3.783)--(4.649,3.438)--(4.660,3.204)--(4.672,3.118)--(4.683,3.154)%
  --(4.695,3.339)--(4.707,3.635)--(4.718,4.054)--(4.730,4.510)--(4.741,4.991)--(4.753,5.410)%
  --(4.765,5.780)--(4.776,6.064)--(4.788,6.187)--(4.799,6.175)--(4.811,6.051)--(4.823,5.805)%
  --(4.834,5.447)--(4.846,4.991)--(4.857,4.523)--(4.869,4.091)--(4.881,3.722)--(4.892,3.413)%
  --(4.904,3.216)--(4.915,3.179)--(4.927,3.302)--(4.939,3.500)--(4.950,3.808)--(4.962,4.202)%
  --(4.973,4.671)--(4.985,5.090)--(4.997,5.497)--(5.008,5.829)--(5.020,6.039)--(5.031,6.101)%
  --(5.043,6.076)--(5.055,5.940)--(5.066,5.644)--(5.078,5.238)--(5.089,4.819)--(5.101,4.424)%
  --(5.113,4.017)--(5.124,3.635)--(5.136,3.389)--(5.147,3.278)--(5.159,3.290)--(5.171,3.389)%
  --(5.182,3.635)--(5.194,3.980)--(5.205,4.375)--(5.217,4.782)--(5.229,5.201)--(5.240,5.583)%
  --(5.252,5.842)--(5.263,5.990)--(5.275,6.051)--(5.287,6.014)--(5.298,5.780)--(5.310,5.447)%
  --(5.321,5.102)--(5.333,4.720)--(5.345,4.289)--(5.356,3.882)--(5.368,3.611)--(5.380,3.438)%
  --(5.391,3.327)--(5.403,3.352)--(5.414,3.512)--(5.426,3.795)--(5.438,4.128)--(5.449,4.498)%
  --(5.461,4.942)--(5.472,5.336)--(5.484,5.620)--(5.496,5.817)--(5.507,5.990)--(5.519,6.002)%
  --(5.530,5.854)--(5.542,5.620)--(5.554,5.336)--(5.565,4.979)--(5.577,4.560)--(5.588,4.165)%
  --(5.600,3.845)--(5.612,3.611)--(5.623,3.438)--(5.635,3.401)--(5.646,3.487)--(5.658,3.660)%
  --(5.670,3.931)--(5.681,4.264)--(5.693,4.671)--(5.704,5.040)--(5.716,5.361)--(5.728,5.644)%
  --(5.739,5.866)--(5.751,5.928)--(5.762,5.879)--(5.774,5.743)--(5.786,5.521)--(5.797,5.201)%
  --(5.809,4.819)--(5.820,4.449)--(5.832,4.116)--(5.844,3.808)--(5.855,3.574)--(5.867,3.475)%
  --(5.878,3.487)--(5.890,3.586)--(5.902,3.783)--(5.913,4.079)--(5.925,4.424)--(5.936,4.782)%
  --(5.948,5.127)--(5.960,5.435)--(5.971,5.706)--(5.983,5.842)--(5.994,5.854)--(6.006,5.780)%
  --(6.018,5.632)--(6.029,5.373)--(6.041,5.053)--(6.053,4.695)--(6.064,4.350)--(6.076,4.030)%
  --(6.087,3.771)--(6.099,3.598)--(6.111,3.524)--(6.122,3.549)--(6.134,3.672)--(6.145,3.906)%
  --(6.157,4.215)--(6.169,4.547)--(6.180,4.880)--(6.192,5.201)--(6.203,5.497)--(6.215,5.681)%
  --(6.227,5.780)--(6.238,5.780)--(6.250,5.694)--(6.261,5.509)--(6.273,5.238)--(6.285,4.905)%
  --(6.296,4.597)--(6.308,4.264)--(6.319,3.968)--(6.331,3.746)--(6.343,3.598)--(6.354,3.574)%
  --(6.366,3.635)--(6.377,3.795)--(6.389,4.042)--(6.401,4.326)--(6.412,4.658)--(6.424,4.979)%
  --(6.435,5.287)--(6.447,5.534)--(6.459,5.681)--(6.470,5.743)--(6.482,5.731)--(6.493,5.620)%
  --(6.505,5.386)--(6.517,5.114)--(6.528,4.806)--(6.540,4.498)--(6.551,4.178)--(6.563,3.906)%
  --(6.575,3.746)--(6.586,3.635)--(6.598,3.623)--(6.609,3.709)--(6.621,3.906)--(6.633,4.178)%
  --(6.644,4.461)--(6.656,4.745)--(6.667,5.077)--(6.679,5.336)--(6.691,5.534)--(6.702,5.657)%
  --(6.714,5.706)--(6.726,5.657)--(6.737,5.497)--(6.749,5.262)--(6.760,5.003)--(6.772,4.708)%
  --(6.784,4.387)--(6.795,4.116)--(6.807,3.894)--(6.818,3.746)--(6.830,3.672)--(6.842,3.697)%
  --(6.853,3.820)--(6.865,4.017)--(6.876,4.276)--(6.888,4.547)--(6.900,4.868)--(6.911,5.139)%
  --(6.923,5.361)--(6.934,5.546)--(6.946,5.657)--(6.958,5.644)--(6.969,5.546)--(6.981,5.373)%
  --(6.992,5.164)--(7.004,4.905)--(7.016,4.609)--(7.027,4.313)--(7.039,4.079)--(7.050,3.869)%
  --(7.062,3.746)--(7.074,3.709)--(7.085,3.783)--(7.097,3.906)--(7.108,4.116)--(7.120,4.375)%
  --(7.132,4.671)--(7.143,4.942)--(7.155,5.176)--(7.166,5.398)--(7.178,5.558)--(7.190,5.595)%
  --(7.201,5.571)--(7.213,5.472)--(7.224,5.299)--(7.236,5.053)--(7.248,4.782)--(7.259,4.510)%
  --(7.271,4.252)--(7.282,4.017)--(7.294,3.845)--(7.306,3.771)--(7.317,3.771)--(7.329,3.845)%
  --(7.340,3.993)--(7.352,4.227)--(7.364,4.486)--(7.375,4.732)--(7.387,5.003)--(7.398,5.250)%
  --(7.410,5.423)--(7.422,5.521)--(7.433,5.546)--(7.445,5.497)--(7.457,5.386)--(7.468,5.188)%
  --(7.480,4.942)--(7.491,4.683)--(7.503,4.424)--(7.515,4.190)--(7.526,3.993)--(7.538,3.857)%
  --(7.549,3.808)--(7.561,3.845)--(7.573,3.943)--(7.584,4.091)--(7.596,4.326)--(7.607,4.560)%
  --(7.619,4.819)--(7.631,5.065)--(7.642,5.275)--(7.654,5.410)--(7.665,5.484)--(7.677,5.497)%
  --(7.689,5.435)--(7.700,5.287)--(7.712,5.077)--(7.723,4.843)--(7.735,4.621)--(7.747,4.363)%
  --(7.758,4.141)--(7.770,3.968)--(7.781,3.869)--(7.793,3.857)--(7.805,3.906)--(7.816,4.005)%
  --(7.828,4.190)--(7.839,4.424)--(7.851,4.658)--(7.863,4.893)--(7.874,5.127)--(7.886,5.287)%
  --(7.897,5.410)--(7.909,5.472)--(7.921,5.460)--(7.932,5.349)--(7.944,5.188)--(7.955,5.003)%
  --(7.967,4.757)--(7.979,4.523)--(7.990,4.289)--(8.002,4.104)--(8.013,3.968)--(8.025,3.894)%
  --(8.037,3.882)--(8.048,3.956)--(8.060,4.104)--(8.071,4.276)--(8.083,4.498)--(8.095,4.745)%
  --(8.106,4.967)--(8.118,5.151)--(8.130,5.299)--(8.141,5.423)--(8.153,5.447)--(8.164,5.373)%
  --(8.176,5.275)--(8.188,5.127)--(8.199,4.930)--(8.211,4.683)--(8.222,4.436)--(8.234,4.252)%
  --(8.246,4.091)--(8.257,3.968)--(8.269,3.919)--(8.280,3.943)--(8.292,4.030)--(8.304,4.178)%
  --(8.315,4.363)--(8.327,4.584)--(8.338,4.819)--(8.350,5.003)--(8.362,5.188)--(8.373,5.324)%
  --(8.385,5.398)--(8.396,5.373)--(8.408,5.324)--(8.420,5.225)--(8.431,5.053)--(8.443,4.819)%
  --(8.454,4.597)--(8.466,4.399)--(8.478,4.215)--(8.489,4.067)--(8.501,3.968)--(8.512,3.968)%
  --(8.524,4.005)--(8.536,4.091)--(8.547,4.264)--(8.559,4.449)--(8.570,4.658)--(8.582,4.856)%
  --(8.594,5.065)--(8.605,5.225)--(8.617,5.324)--(8.628,5.349)--(8.640,5.349)--(8.652,5.275)%
  --(8.663,5.139)--(8.675,4.942)--(8.686,4.745)--(8.698,4.547)--(8.710,4.350)--(8.721,4.178)%
  --(8.733,4.067)--(8.744,4.005)--(8.756,4.005)--(8.768,4.054)--(8.779,4.165)--(8.791,4.338)%
  --(8.802,4.535)--(8.814,4.732)--(8.826,4.930)--(8.837,5.090)--(8.849,5.225)--(8.861,5.287)%
  --(8.872,5.324)--(8.884,5.299)--(8.895,5.188)--(8.907,5.040)--(8.919,4.868)--(8.930,4.695)%
  --(8.942,4.498)--(8.953,4.289)--(8.965,4.178)--(8.977,4.067)--(8.988,4.017)--(9.000,4.042)%
  --(9.011,4.116)--(9.023,4.252)--(9.035,4.412)--(9.046,4.597)--(9.058,4.806)--(9.069,4.979)%
  --(9.081,5.114)--(9.093,5.225)--(9.104,5.324)--(9.116,5.312)--(9.127,5.225)--(9.139,5.114)%
  --(9.151,4.991)--(9.162,4.806)--(9.174,4.609)--(9.185,4.424)--(9.197,4.289)--(9.209,4.153)%
  --(9.220,4.067)--(9.232,4.042)--(9.243,4.104)--(9.255,4.190)--(9.267,4.313)--(9.278,4.473)%
  --(9.290,4.671)--(9.301,4.843)--(9.313,4.991)--(9.325,5.139)--(9.336,5.238)--(9.348,5.262)%
  --(9.359,5.238)--(9.371,5.188)--(9.383,5.077)--(9.394,4.905)--(9.406,4.720)--(9.417,4.560)%
  --(9.429,4.412)--(9.441,4.239)--(9.452,4.128)--(9.464,4.079)--(9.475,4.104)--(9.487,4.153)%
  --(9.499,4.239)--(9.510,4.387)--(9.522,4.560)--(9.534,4.708)--(9.545,4.893)--(9.557,5.040)%
  --(9.568,5.164)--(9.580,5.225)--(9.592,5.238)--(9.603,5.225)--(9.615,5.127)--(9.626,4.979)%
  --(9.638,4.843)--(9.650,4.683)--(9.661,4.510)--(9.673,4.338)--(9.684,4.227)--(9.696,4.153)%
  --(9.708,4.116)--(9.719,4.116)--(9.731,4.190)--(9.742,4.313)--(9.754,4.449)--(9.766,4.609)%
  --(9.777,4.782)--(9.789,4.930)--(9.800,5.077)--(9.812,5.164)--(9.824,5.213)--(9.835,5.225)%
  --(9.847,5.151)--(9.858,5.053)--(9.870,4.930)--(9.882,4.782)--(9.893,4.621)--(9.905,4.449)%
  --(9.916,4.326)--(9.928,4.227)--(9.940,4.141)--(9.951,4.116)--(9.963,4.165)--(9.974,4.252)%
  --(9.986,4.363)--(9.998,4.498)--(10.009,4.683)--(10.021,4.843)--(10.032,4.967)--(10.044,5.065)%
  --(10.056,5.164)--(10.067,5.201)--(10.079,5.151)--(10.090,5.090)--(10.102,5.003)--(10.114,4.880)%
  --(10.125,4.732)--(10.137,4.560)--(10.148,4.424)--(10.160,4.313)--(10.172,4.202)--(10.183,4.153)%
  --(10.195,4.165)--(10.207,4.215)--(10.218,4.289)--(10.230,4.424)--(10.241,4.572)--(10.253,4.720)%
  --(10.265,4.868)--(10.276,5.003)--(10.288,5.102)--(10.299,5.151)--(10.311,5.151)--(10.323,5.139)%
  --(10.334,5.065)--(10.346,4.954)--(10.357,4.819)--(10.369,4.658)--(10.381,4.523)--(10.392,4.399)%
  --(10.404,4.276)--(10.415,4.215)--(10.427,4.178)--(10.439,4.202)--(10.450,4.252)--(10.462,4.363)%
  --(10.473,4.486)--(10.485,4.621)--(10.497,4.757)--(10.508,4.893)--(10.520,5.016)--(10.531,5.090)%
  --(10.543,5.139)--(10.555,5.139)--(10.566,5.090)--(10.578,5.016)--(10.589,4.893)--(10.601,4.757)%
  --(10.613,4.621)--(10.624,4.486)--(10.636,4.363)--(10.647,4.276)--(10.659,4.227)--(10.671,4.202)%
  --(10.682,4.239)--(10.694,4.301)--(10.705,4.412)--(10.717,4.523)--(10.729,4.671)--(10.740,4.819)%
  --(10.752,4.942)--(10.763,5.016)--(10.775,5.077)--(10.787,5.114)--(10.798,5.114)--(10.810,5.053)%
  --(10.821,4.942)--(10.833,4.843)--(10.845,4.720);
\gpcolor{color=gp lt color border}
\node[gp node right] at (10.479,8.047) {Punti tangenza};
\gpcolor{rgb color={1.000,0.000,0.000}}
\gpsetpointsize{4.00}
\gppoint{gp mark 3}{(1.559,1.611)}
\gppoint{gp mark 3}{(1.672,7.789)}
\gppoint{gp mark 3}{(1.794,1.786)}
\gppoint{gp mark 3}{(1.916,7.623)}
\gppoint{gp mark 3}{(2.032,1.836)}
\gppoint{gp mark 3}{(2.154,7.395)}
\gppoint{gp mark 3}{(2.276,1.971)}
\gppoint{gp mark 3}{(2.392,7.315)}
\gppoint{gp mark 3}{(2.513,2.193)}
\gppoint{gp mark 3}{(2.635,7.154)}
\gppoint{gp mark 3}{(2.751,2.285)}
\gppoint{gp mark 3}{(2.873,6.976)}
\gppoint{gp mark 3}{(2.995,2.335)}
\gppoint{gp mark 3}{(3.111,6.976)}
\gppoint{gp mark 3}{(3.233,2.538)}
\gppoint{gp mark 3}{(3.355,6.828)}
\gppoint{gp mark 3}{(3.471,2.612)}
\gppoint{gp mark 3}{(3.593,6.655)}
\gppoint{gp mark 3}{(3.714,2.680)}
\gppoint{gp mark 3}{(3.836,6.569)}
\gppoint{gp mark 3}{(3.958,2.791)}
\gppoint{gp mark 3}{(4.074,6.514)}
\gppoint{gp mark 3}{(4.196,2.945)}
\gppoint{gp mark 3}{(4.318,6.409)}
\gppoint{gp mark 3}{(4.434,2.970)}
\gppoint{gp mark 3}{(4.556,6.285)}
\gppoint{gp mark 3}{(4.678,3.062)}
\gppoint{gp mark 3}{(4.794,6.236)}
\gppoint{gp mark 3}{(4.915,3.179)}
\gppoint{gp mark 3}{(5.037,6.144)}
\gppoint{gp mark 3}{(5.153,3.241)}
\gppoint{gp mark 3}{(5.275,6.051)}
\gppoint{gp mark 3}{(5.397,3.272)}
\gppoint{gp mark 3}{(5.513,6.076)}
\gppoint{gp mark 3}{(5.635,3.401)}
\gppoint{gp mark 3}{(5.757,5.946)}
\gppoint{gp mark 3}{(5.873,3.438)}
\gppoint{gp mark 3}{(5.994,5.854)}
\gppoint{gp mark 3}{(6.116,3.487)}
\gppoint{gp mark 3}{(6.232,5.823)}
\gppoint{gp mark 3}{(6.354,3.574)}
\gppoint{gp mark 3}{(6.476,5.786)}
\gppoint{gp mark 3}{(6.592,3.580)}
\gppoint{gp mark 3}{(6.714,5.706)}
\gppoint{gp mark 3}{(6.836,3.635)}
\gppoint{gp mark 3}{(6.952,5.694)}
\gppoint{gp mark 3}{(7.074,3.709)}
\gppoint{gp mark 3}{(7.195,5.620)}
\gppoint{gp mark 3}{(7.311,3.734)}
\gppoint{gp mark 3}{(7.433,5.546)}
\gppoint{gp mark 3}{(7.555,3.795)}
\gppoint{gp mark 3}{(7.671,5.527)}
\gppoint{gp mark 3}{(7.793,3.857)}
\gppoint{gp mark 3}{(7.915,5.515)}
\gppoint{gp mark 3}{(8.031,3.845)}
\gppoint{gp mark 3}{(8.153,5.447)}
\gppoint{gp mark 3}{(8.275,3.900)}
\gppoint{gp mark 3}{(8.391,5.398)}
\gppoint{gp mark 3}{(8.512,3.968)}
\gppoint{gp mark 3}{(8.634,5.386)}
\gppoint{gp mark 3}{(8.750,3.980)}
\gppoint{gp mark 3}{(8.872,5.324)}
\gppoint{gp mark 3}{(8.994,4.005)}
\gppoint{gp mark 3}{(9.110,5.355)}
\gppoint{gp mark 3}{(9.232,4.042)}
\gppoint{gp mark 3}{(9.354,5.262)}
\gppoint{gp mark 3}{(9.470,4.079)}
\gppoint{gp mark 3}{(9.592,5.238)}
\gppoint{gp mark 3}{(9.713,4.079)}
\gppoint{gp mark 3}{(9.829,5.262)}
\gppoint{gp mark 3}{(9.951,4.116)}
\gppoint{gp mark 3}{(10.073,5.182)}
\gppoint{gp mark 3}{(10.189,4.141)}
\gppoint{gp mark 3}{(10.305,5.158)}
\gppoint{gp mark 3}{(10.427,4.178)}
\gppoint{gp mark 3}{(10.549,5.164)}
\gppoint{gp mark 3}{(10.671,4.202)}
\gppoint{gp mark 3}{(11.121,8.047)}
\gpcolor{color=gp lt color border}
\gpsetlinetype{gp lt border}
\draw[gp path] (1.504,7.789)--(1.504,1.611);
\draw[gp path] (1.504,0.985)--(10.845,0.985);
%% coordinates of the plot area
\gpdefrectangularnode{gp plot 1}{\pgfpoint{1.504cm}{0.985cm}}{\pgfpoint{11.947cm}{8.381cm}}
\end{tikzpicture}
%% gnuplot variables
 \caption{Grafico 0.965dgdecad.tex} \label{gr:0.965dgdecad.tex} \end{grafico}
\begin{grafico} \centering \begin{tikzpicture}[gnuplot]
%% generated with GNUPLOT 4.6p0 (Lua 5.1; terminal rev. 99, script rev. 100)
%% Mon 09 Jun 2014 04:19:46 PM CEST
\path (0.000,0.000) rectangle (12.500,8.750);
\gpcolor{color=gp lt color border}
\gpsetlinetype{gp lt border}
\gpsetlinewidth{1.00}
\draw[gp path] (1.688,1.607)--(1.868,1.607);
\node[gp node right] at (1.504,1.607) {-0.25};
\draw[gp path] (1.688,2.229)--(1.868,2.229);
\node[gp node right] at (1.504,2.229) {-0.2};
\draw[gp path] (1.688,2.850)--(1.868,2.850);
\node[gp node right] at (1.504,2.850) {-0.15};
\draw[gp path] (1.688,3.472)--(1.868,3.472);
\node[gp node right] at (1.504,3.472) {-0.1};
\draw[gp path] (1.688,4.094)--(1.868,4.094);
\node[gp node right] at (1.504,4.094) {-0.05};
\draw[gp path] (1.688,4.716)--(1.868,4.716);
\node[gp node right] at (1.504,4.716) { 0};
\draw[gp path] (1.688,5.338)--(1.868,5.338);
\node[gp node right] at (1.504,5.338) { 0.05};
\draw[gp path] (1.688,5.960)--(1.868,5.960);
\node[gp node right] at (1.504,5.960) { 0.1};
\draw[gp path] (1.688,6.581)--(1.868,6.581);
\node[gp node right] at (1.504,6.581) { 0.15};
\draw[gp path] (1.688,7.203)--(1.868,7.203);
\node[gp node right] at (1.504,7.203) { 0.2};
\draw[gp path] (1.688,0.985)--(1.688,1.165);
\node[gp node center] at (1.688,0.677) { 0};
\draw[gp path] (2.828,0.985)--(2.828,1.165);
\node[gp node center] at (2.828,0.677) { 5};
\draw[gp path] (3.968,0.985)--(3.968,1.165);
\node[gp node center] at (3.968,0.677) { 10};
\draw[gp path] (5.108,0.985)--(5.108,1.165);
\node[gp node center] at (5.108,0.677) { 15};
\draw[gp path] (6.248,0.985)--(6.248,1.165);
\node[gp node center] at (6.248,0.677) { 20};
\draw[gp path] (7.387,0.985)--(7.387,1.165);
\node[gp node center] at (7.387,0.677) { 25};
\draw[gp path] (8.527,0.985)--(8.527,1.165);
\node[gp node center] at (8.527,0.677) { 30};
\draw[gp path] (9.667,0.985)--(9.667,1.165);
\node[gp node center] at (9.667,0.677) { 35};
\draw[gp path] (10.807,0.985)--(10.807,1.165);
\node[gp node center] at (10.807,0.677) { 40};
\draw[gp path] (1.688,7.751)--(1.688,1.548);
\draw[gp path] (1.688,0.985)--(10.910,0.985);
\node[gp node center,rotate=-270] at (0.246,4.405) {Ampiezza [???]};
\node[gp node center] at (6.817,0.215) {Tempo $[s]$};
\node[gp node center] at (6.817,8.287) {Dati decadimento 0.966d};
\gpcolor{rgb color={0.000,0.000,1.000}}
\gpsetlinetype{gp lt plot 0}
\draw[gp path] (1.688,2.614)--(1.699,2.017)--(1.711,1.632)--(1.722,1.557)--(1.734,1.781)%
  --(1.745,2.278)--(1.756,2.975)--(1.768,3.820)--(1.779,4.741)--(1.791,5.686)--(1.802,6.519)%
  --(1.813,7.141)--(1.825,7.576)--(1.836,7.750)--(1.848,7.601)--(1.859,7.216)--(1.870,6.619)%
  --(1.882,5.835)--(1.893,4.940)--(1.905,4.044)--(1.916,3.224)--(1.927,2.527)--(1.939,1.980)%
  --(1.950,1.731)--(1.962,1.768)--(1.973,2.067)--(1.984,2.589)--(1.996,3.298)--(2.007,4.169)%
  --(2.019,5.052)--(2.030,5.860)--(2.041,6.606)--(2.053,7.203)--(2.064,7.527)--(2.076,7.564)%
  --(2.087,7.377)--(2.098,6.942)--(2.110,6.308)--(2.121,5.499)--(2.133,4.666)--(2.144,3.833)%
  --(2.155,3.062)--(2.167,2.440)--(2.178,2.067)--(2.190,1.918)--(2.201,2.017)--(2.212,2.365)%
  --(2.224,2.938)--(2.235,3.684)--(2.247,4.480)--(2.258,5.300)--(2.269,6.084)--(2.281,6.743)%
  --(2.292,7.191)--(2.304,7.402)--(2.315,7.390)--(2.326,7.129)--(2.338,6.631)--(2.349,5.984)%
  --(2.361,5.213)--(2.372,4.405)--(2.383,3.609)--(2.395,2.925)--(2.406,2.428)--(2.418,2.129)%
  --(2.429,2.055)--(2.440,2.241)--(2.452,2.639)--(2.463,3.248)--(2.475,3.957)--(2.486,4.741)%
  --(2.497,5.524)--(2.509,6.221)--(2.520,6.780)--(2.532,7.141)--(2.543,7.303)--(2.554,7.203)%
  --(2.566,6.867)--(2.577,6.358)--(2.589,5.686)--(2.600,4.940)--(2.611,4.156)--(2.623,3.447)%
  --(2.634,2.863)--(2.646,2.428)--(2.657,2.204)--(2.668,2.216)--(2.680,2.465)--(2.691,2.900)%
  --(2.703,3.510)--(2.714,4.231)--(2.725,4.977)--(2.737,5.698)--(2.748,6.333)--(2.759,6.830)%
  --(2.771,7.104)--(2.782,7.166)--(2.794,6.992)--(2.805,6.631)--(2.816,6.084)--(2.828,5.425)%
  --(2.839,4.679)--(2.851,3.970)--(2.862,3.311)--(2.873,2.788)--(2.885,2.452)--(2.896,2.316)%
  --(2.908,2.378)--(2.919,2.701)--(2.930,3.161)--(2.942,3.783)--(2.953,4.467)--(2.965,5.176)%
  --(2.976,5.848)--(2.987,6.407)--(2.999,6.793)--(3.010,7.004)--(3.022,6.979)--(3.033,6.793)%
  --(3.044,6.382)--(3.056,5.823)--(3.067,5.151)--(3.079,4.455)--(3.090,3.771)--(3.101,3.199)%
  --(3.113,2.763)--(3.124,2.490)--(3.136,2.428)--(3.147,2.577)--(3.158,2.913)--(3.170,3.398)%
  --(3.181,4.019)--(3.193,4.691)--(3.204,5.375)--(3.215,5.972)--(3.227,6.457)--(3.238,6.780)%
  --(3.250,6.905)--(3.261,6.843)--(3.272,6.569)--(3.284,6.134)--(3.295,5.562)--(3.307,4.940)%
  --(3.318,4.256)--(3.329,3.634)--(3.341,3.124)--(3.352,2.763)--(3.364,2.564)--(3.375,2.577)%
  --(3.386,2.776)--(3.398,3.149)--(3.409,3.646)--(3.421,4.268)--(3.432,4.915)--(3.443,5.537)%
  --(3.455,6.071)--(3.466,6.482)--(3.478,6.743)--(3.489,6.805)--(3.500,6.668)--(3.512,6.358)%
  --(3.523,5.897)--(3.535,5.338)--(3.546,4.703)--(3.557,4.094)--(3.569,3.522)--(3.580,3.074)%
  --(3.592,2.776)--(3.603,2.651)--(3.614,2.739)--(3.626,2.962)--(3.637,3.360)--(3.649,3.870)%
  --(3.660,4.480)--(3.671,5.089)--(3.683,5.674)--(3.694,6.146)--(3.706,6.494)--(3.717,6.681)%
  --(3.728,6.668)--(3.740,6.494)--(3.751,6.159)--(3.763,5.674)--(3.774,5.114)--(3.785,4.517)%
  --(3.797,3.932)--(3.808,3.435)--(3.820,3.037)--(3.831,2.813)--(3.842,2.751)--(3.854,2.888)%
  --(3.865,3.161)--(3.877,3.572)--(3.888,4.107)--(3.899,4.679)--(3.911,5.263)--(3.922,5.785)%
  --(3.934,6.196)--(3.945,6.482)--(3.956,6.606)--(3.968,6.532)--(3.979,6.308)--(3.991,5.947)%
  --(4.002,5.450)--(4.013,4.915)--(4.025,4.355)--(4.036,3.808)--(4.048,3.348)--(4.059,3.062)%
  --(4.070,2.888)--(4.082,2.875)--(4.093,3.025)--(4.105,3.348)--(4.116,3.796)--(4.127,4.306)%
  --(4.139,4.853)--(4.150,5.400)--(4.162,5.872)--(4.173,6.233)--(4.184,6.445)--(4.196,6.507)%
  --(4.207,6.407)--(4.219,6.134)--(4.230,5.748)--(4.241,5.263)--(4.253,4.728)--(4.264,4.181)%
  --(4.276,3.709)--(4.287,3.311)--(4.298,3.049)--(4.310,2.938)--(4.321,2.987)--(4.333,3.199)%
  --(4.344,3.534)--(4.355,3.982)--(4.367,4.504)--(4.378,5.027)--(4.390,5.512)--(4.401,5.935)%
  --(4.412,6.233)--(4.424,6.395)--(4.435,6.407)--(4.447,6.246)--(4.458,5.960)--(4.469,5.562)%
  --(4.481,5.064)--(4.492,4.554)--(4.504,4.069)--(4.515,3.622)--(4.526,3.298)--(4.538,3.087)%
  --(4.549,3.025)--(4.561,3.124)--(4.572,3.360)--(4.583,3.721)--(4.595,4.169)--(4.606,4.654)%
  --(4.618,5.151)--(4.629,5.611)--(4.640,5.972)--(4.652,6.221)--(4.663,6.320)--(4.675,6.283)%
  --(4.686,6.096)--(4.697,5.773)--(4.709,5.363)--(4.720,4.890)--(4.732,4.417)--(4.743,3.957)%
  --(4.754,3.572)--(4.766,3.273)--(4.777,3.149)--(4.788,3.124)--(4.800,3.261)--(4.811,3.522)%
  --(4.823,3.908)--(4.834,4.355)--(4.845,4.815)--(4.857,5.288)--(4.868,5.686)--(4.880,6.009)%
  --(4.891,6.196)--(4.902,6.258)--(4.914,6.171)--(4.925,5.922)--(4.937,5.611)--(4.948,5.188)%
  --(4.959,4.741)--(4.971,4.281)--(4.982,3.858)--(4.994,3.510)--(5.005,3.298)--(5.016,3.186)%
  --(5.028,3.224)--(5.039,3.410)--(5.051,3.696)--(5.062,4.069)--(5.073,4.504)--(5.085,4.952)%
  --(5.096,5.400)--(5.108,5.748)--(5.119,6.009)--(5.130,6.159)--(5.142,6.171)--(5.153,6.022)%
  --(5.165,5.773)--(5.176,5.437)--(5.187,5.027)--(5.199,4.592)--(5.210,4.169)--(5.222,3.783)%
  --(5.233,3.485)--(5.244,3.298)--(5.256,3.261)--(5.267,3.335)--(5.279,3.547)--(5.290,3.845)%
  --(5.301,4.231)--(5.313,4.666)--(5.324,5.089)--(5.336,5.475)--(5.347,5.785)--(5.358,6.009)%
  --(5.370,6.096)--(5.381,6.059)--(5.393,5.910)--(5.404,5.624)--(5.415,5.276)--(5.427,4.853)%
  --(5.438,4.430)--(5.450,4.057)--(5.461,3.709)--(5.472,3.472)--(5.484,3.360)--(5.495,3.348)%
  --(5.507,3.460)--(5.518,3.684)--(5.529,4.007)--(5.541,4.393)--(5.552,4.803)--(5.564,5.201)%
  --(5.575,5.549)--(5.586,5.823)--(5.598,5.984)--(5.609,6.034)--(5.621,5.960)--(5.632,5.773)%
  --(5.643,5.487)--(5.655,5.126)--(5.666,4.741)--(5.678,4.343)--(5.689,3.982)--(5.700,3.684)%
  --(5.712,3.485)--(5.723,3.410)--(5.735,3.435)--(5.746,3.572)--(5.757,3.820)--(5.769,4.156)%
  --(5.780,4.529)--(5.792,4.915)--(5.803,5.288)--(5.814,5.599)--(5.826,5.835)--(5.837,5.947)%
  --(5.849,5.947)--(5.860,5.860)--(5.871,5.624)--(5.883,5.350)--(5.894,4.990)--(5.906,4.616)%
  --(5.917,4.243)--(5.928,3.908)--(5.940,3.646)--(5.951,3.497)--(5.963,3.460)--(5.974,3.534)%
  --(5.985,3.696)--(5.997,3.970)--(6.008,4.293)--(6.020,4.666)--(6.031,5.014)--(6.042,5.375)%
  --(6.054,5.636)--(6.065,5.823)--(6.077,5.910)--(6.088,5.872)--(6.099,5.736)--(6.111,5.524)%
  --(6.122,5.213)--(6.134,4.865)--(6.145,4.492)--(6.156,4.144)--(6.168,3.870)--(6.179,3.646)%
  --(6.191,3.534)--(6.202,3.534)--(6.213,3.622)--(6.225,3.833)--(6.236,4.107)--(6.248,4.430)%
  --(6.259,4.778)--(6.270,5.114)--(6.282,5.425)--(6.293,5.649)--(6.305,5.810)--(6.316,5.835)%
  --(6.327,5.785)--(6.339,5.636)--(6.350,5.387)--(6.362,5.077)--(6.373,4.741)--(6.384,4.393)%
  --(6.396,4.069)--(6.407,3.820)--(6.419,3.646)--(6.430,3.584)--(6.441,3.597)--(6.453,3.733)%
  --(6.464,3.945)--(6.476,4.231)--(6.487,4.554)--(6.498,4.890)--(6.510,5.201)--(6.521,5.462)%
  --(6.533,5.661)--(6.544,5.773)--(6.555,5.785)--(6.567,5.686)--(6.578,5.512)--(6.590,5.251)%
  --(6.601,4.927)--(6.612,4.629)--(6.624,4.306)--(6.635,4.019)--(6.647,3.808)--(6.658,3.671)%
  --(6.669,3.634)--(6.681,3.696)--(6.692,3.845)--(6.704,4.069)--(6.715,4.355)--(6.726,4.679)%
  --(6.738,4.990)--(6.749,5.263)--(6.761,5.499)--(6.772,5.649)--(6.783,5.736)--(6.795,5.711)%
  --(6.806,5.611)--(6.818,5.412)--(6.829,5.139)--(6.840,4.840)--(6.852,4.529)--(6.863,4.231)%
  --(6.874,3.970)--(6.886,3.796)--(6.897,3.696)--(6.909,3.696)--(6.920,3.771)--(6.931,3.945)%
  --(6.943,4.194)--(6.954,4.467)--(6.966,4.766)--(6.977,5.064)--(6.988,5.325)--(7.000,5.524)%
  --(7.011,5.649)--(7.023,5.698)--(7.034,5.636)--(7.045,5.499)--(7.057,5.288)--(7.068,5.014)%
  --(7.080,4.716)--(7.091,4.430)--(7.102,4.156)--(7.114,3.945)--(7.125,3.783)--(7.137,3.746)%
  --(7.148,3.758)--(7.159,3.870)--(7.171,4.057)--(7.182,4.293)--(7.194,4.579)--(7.205,4.865)%
  --(7.216,5.139)--(7.228,5.375)--(7.239,5.537)--(7.251,5.649)--(7.262,5.636)--(7.273,5.562)%
  --(7.285,5.400)--(7.296,5.176)--(7.308,4.902)--(7.319,4.629)--(7.330,4.368)--(7.342,4.119)%
  --(7.353,3.920)--(7.365,3.808)--(7.376,3.771)--(7.387,3.820)--(7.399,3.957)--(7.410,4.156)%
  --(7.422,4.405)--(7.433,4.679)--(7.444,4.940)--(7.456,5.188)--(7.467,5.387)--(7.479,5.549)%
  --(7.490,5.599)--(7.501,5.562)--(7.513,5.462)--(7.524,5.288)--(7.536,5.077)--(7.547,4.815)%
  --(7.558,4.542)--(7.570,4.281)--(7.581,4.069)--(7.593,3.920)--(7.604,3.820)--(7.615,3.833)%
  --(7.627,3.895)--(7.638,4.044)--(7.650,4.268)--(7.661,4.504)--(7.672,4.766)--(7.684,5.014)%
  --(7.695,5.238)--(7.707,5.425)--(7.718,5.537)--(7.729,5.549)--(7.741,5.512)--(7.752,5.375)%
  --(7.764,5.201)--(7.775,4.977)--(7.786,4.716)--(7.798,4.480)--(7.809,4.231)--(7.821,4.057)%
  --(7.832,3.920)--(7.843,3.870)--(7.855,3.883)--(7.866,3.982)--(7.878,4.131)--(7.889,4.355)%
  --(7.900,4.592)--(7.912,4.840)--(7.923,5.064)--(7.935,5.288)--(7.946,5.425)--(7.957,5.487)%
  --(7.969,5.512)--(7.980,5.437)--(7.992,5.288)--(8.003,5.114)--(8.014,4.890)--(8.026,4.641)%
  --(8.037,4.393)--(8.049,4.194)--(8.060,4.019)--(8.071,3.932)--(8.083,3.908)--(8.094,3.945)%
  --(8.106,4.044)--(8.117,4.231)--(8.128,4.442)--(8.140,4.679)--(8.151,4.915)--(8.163,5.139)%
  --(8.174,5.313)--(8.185,5.412)--(8.197,5.462)--(8.208,5.450)--(8.220,5.350)--(8.231,5.201)%
  --(8.242,5.014)--(8.254,4.791)--(8.265,4.567)--(8.277,4.355)--(8.288,4.156)--(8.299,4.032)%
  --(8.311,3.945)--(8.322,3.945)--(8.334,4.007)--(8.345,4.131)--(8.356,4.318)--(8.368,4.529)%
  --(8.379,4.753)--(8.391,4.990)--(8.402,5.176)--(8.413,5.313)--(8.425,5.412)--(8.436,5.437)%
  --(8.448,5.387)--(8.459,5.288)--(8.470,5.139)--(8.482,4.940)--(8.493,4.703)--(8.505,4.492)%
  --(8.516,4.293)--(8.527,4.131)--(8.539,4.019)--(8.550,3.970)--(8.562,3.982)--(8.573,4.082)%
  --(8.584,4.218)--(8.596,4.405)--(8.607,4.604)--(8.619,4.840)--(8.630,5.027)--(8.641,5.188)%
  --(8.653,5.325)--(8.664,5.400)--(8.676,5.387)--(8.687,5.325)--(8.698,5.201)--(8.710,5.052)%
  --(8.721,4.853)--(8.733,4.641)--(8.744,4.442)--(8.755,4.256)--(8.767,4.131)--(8.778,4.032)%
  --(8.790,4.007)--(8.801,4.057)--(8.812,4.156)--(8.824,4.306)--(8.835,4.480)--(8.847,4.679)%
  --(8.858,4.878)--(8.869,5.077)--(8.881,5.238)--(8.892,5.325)--(8.903,5.350)--(8.915,5.350)%
  --(8.926,5.276)--(8.938,5.139)--(8.949,4.965)--(8.960,4.778)--(8.972,4.579)--(8.983,4.393)%
  --(8.995,4.231)--(9.006,4.119)--(9.017,4.057)--(9.029,4.044)--(9.040,4.107)--(9.052,4.231)%
  --(9.063,4.380)--(9.074,4.579)--(9.086,4.766)--(9.097,4.940)--(9.109,5.101)--(9.120,5.238)%
  --(9.131,5.313)--(9.143,5.338)--(9.154,5.300)--(9.166,5.213)--(9.177,5.064)--(9.188,4.902)%
  --(9.200,4.703)--(9.211,4.517)--(9.223,4.343)--(9.234,4.206)--(9.245,4.107)--(9.257,4.069)%
  --(9.268,4.082)--(9.280,4.156)--(9.291,4.281)--(9.302,4.455)--(9.314,4.641)--(9.325,4.803)%
  --(9.337,4.977)--(9.348,5.126)--(9.359,5.238)--(9.371,5.288)--(9.382,5.288)--(9.394,5.251)%
  --(9.405,5.126)--(9.416,5.002)--(9.428,4.828)--(9.439,4.641)--(9.451,4.467)--(9.462,4.318)%
  --(9.473,4.194)--(9.485,4.119)--(9.496,4.094)--(9.508,4.144)--(9.519,4.218)--(9.530,4.368)%
  --(9.542,4.529)--(9.553,4.703)--(9.565,4.865)--(9.576,5.014)--(9.587,5.151)--(9.599,5.238)%
  --(9.610,5.276)--(9.622,5.251)--(9.633,5.188)--(9.644,5.077)--(9.656,4.927)--(9.667,4.766)%
  --(9.679,4.592)--(9.690,4.417)--(9.701,4.293)--(9.713,4.181)--(9.724,4.131)--(9.736,4.131)%
  --(9.747,4.181)--(9.758,4.293)--(9.770,4.430)--(9.781,4.592)--(9.793,4.753)--(9.804,4.902)%
  --(9.815,5.052)--(9.827,5.164)--(9.838,5.226)--(9.850,5.251)--(9.861,5.226)--(9.872,5.139)%
  --(9.884,5.014)--(9.895,4.865)--(9.907,4.703)--(9.918,4.529)--(9.929,4.393)--(9.941,4.281)%
  --(9.952,4.194)--(9.964,4.144)--(9.975,4.181)--(9.986,4.243)--(9.998,4.355)--(10.009,4.492)%
  --(10.021,4.641)--(10.032,4.791)--(10.043,4.927)--(10.055,5.077)--(10.066,5.164)--(10.078,5.213)%
  --(10.089,5.213)--(10.100,5.164)--(10.112,5.077)--(10.123,4.952)--(10.135,4.803)--(10.146,4.641)%
  --(10.157,4.504)--(10.169,4.355)--(10.180,4.256)--(10.192,4.194)--(10.203,4.181)--(10.214,4.218)%
  --(10.226,4.293)--(10.237,4.405)--(10.249,4.542)--(10.260,4.691)--(10.271,4.840)--(10.283,4.977)%
  --(10.294,5.101)--(10.306,5.151)--(10.317,5.201)--(10.328,5.176)--(10.340,5.114)--(10.351,5.027)%
  --(10.363,4.890)--(10.374,4.741)--(10.385,4.592)--(10.397,4.442)--(10.408,4.343)--(10.420,4.256)%
  --(10.431,4.206)--(10.442,4.206)--(10.454,4.243)--(10.465,4.330)--(10.477,4.455)--(10.488,4.592)%
  --(10.499,4.741)--(10.511,4.878)--(10.522,5.014)--(10.534,5.101)--(10.545,5.151)--(10.556,5.176)%
  --(10.568,5.139)--(10.579,5.077)--(10.591,4.965)--(10.602,4.840)--(10.613,4.691)--(10.625,4.567)%
  --(10.636,4.430)--(10.648,4.330)--(10.659,4.268)--(10.670,4.243)--(10.682,4.243)--(10.693,4.293)%
  --(10.705,4.393)--(10.716,4.504)--(10.727,4.641)--(10.739,4.778)--(10.750,4.902)--(10.762,5.014)%
  --(10.773,5.089)--(10.784,5.139)--(10.796,5.151)--(10.807,5.101)--(10.819,5.014)--(10.830,4.890)%
  --(10.841,4.791)--(10.853,4.666)--(10.864,4.504)--(10.876,4.393)--(10.887,4.306)--(10.898,4.281)%
  --(10.910,4.256);
\gpcolor{color=gp lt color border}
\node[gp node right] at (10.479,7.491) {Massimi e minimi};
\gpcolor{rgb color={1.000,0.000,0.000}}
\gpsetpointsize{4.00}
\gppoint{gp mark 3}{(1.848,7.751)}
\gppoint{gp mark 3}{(2.083,7.576)}
\gppoint{gp mark 3}{(2.320,7.424)}
\gppoint{gp mark 3}{(2.556,7.305)}
\gppoint{gp mark 3}{(2.791,7.173)}
\gppoint{gp mark 3}{(3.026,7.023)}
\gppoint{gp mark 3}{(3.263,6.907)}
\gppoint{gp mark 3}{(3.498,6.809)}
\gppoint{gp mark 3}{(3.733,6.700)}
\gppoint{gp mark 3}{(3.969,6.608)}
\gppoint{gp mark 3}{(4.206,6.508)}
\gppoint{gp mark 3}{(4.442,6.423)}
\gppoint{gp mark 3}{(4.677,6.324)}
\gppoint{gp mark 3}{(4.913,6.259)}
\gppoint{gp mark 3}{(5.148,6.185)}
\gppoint{gp mark 3}{(5.384,6.099)}
\gppoint{gp mark 3}{(5.619,6.035)}
\gppoint{gp mark 3}{(5.854,5.961)}
\gppoint{gp mark 3}{(6.090,5.912)}
\gppoint{gp mark 3}{(6.325,5.836)}
\gppoint{gp mark 3}{(6.562,5.794)}
\gppoint{gp mark 3}{(6.798,5.740)}
\gppoint{gp mark 3}{(7.033,5.699)}
\gppoint{gp mark 3}{(7.267,5.659)}
\gppoint{gp mark 3}{(7.502,5.599)}
\gppoint{gp mark 3}{(7.738,5.551)}
\gppoint{gp mark 3}{(7.977,5.515)}
\gppoint{gp mark 3}{(8.212,5.465)}
\gppoint{gp mark 3}{(8.446,5.438)}
\gppoint{gp mark 3}{(8.680,5.405)}
\gppoint{gp mark 3}{(8.921,5.353)}
\gppoint{gp mark 3}{(9.153,5.338)}
\gppoint{gp mark 3}{(9.388,5.294)}
\gppoint{gp mark 3}{(9.623,5.276)}
\gppoint{gp mark 3}{(9.861,5.251)}
\gppoint{gp mark 3}{(10.095,5.220)}
\gppoint{gp mark 3}{(10.330,5.202)}
\gppoint{gp mark 3}{(10.567,5.176)}
\gppoint{gp mark 3}{(10.804,5.154)}
\gppoint{gp mark 3}{(1.731,1.548)}
\gppoint{gp mark 3}{(1.966,1.712)}
\gppoint{gp mark 3}{(2.202,1.916)}
\gppoint{gp mark 3}{(2.438,2.049)}
\gppoint{gp mark 3}{(2.673,2.180)}
\gppoint{gp mark 3}{(2.910,2.312)}
\gppoint{gp mark 3}{(3.145,2.423)}
\gppoint{gp mark 3}{(3.380,2.544)}
\gppoint{gp mark 3}{(3.615,2.651)}
\gppoint{gp mark 3}{(3.852,2.747)}
\gppoint{gp mark 3}{(4.088,2.861)}
\gppoint{gp mark 3}{(4.323,2.935)}
\gppoint{gp mark 3}{(4.559,3.023)}
\gppoint{gp mark 3}{(4.796,3.114)}
\gppoint{gp mark 3}{(5.031,3.182)}
\gppoint{gp mark 3}{(5.265,3.259)}
\gppoint{gp mark 3}{(5.502,3.338)}
\gppoint{gp mark 3}{(5.737,3.407)}
\gppoint{gp mark 3}{(5.972,3.458)}
\gppoint{gp mark 3}{(6.208,3.520)}
\gppoint{gp mark 3}{(6.445,3.580)}
\gppoint{gp mark 3}{(6.679,3.633)}
\gppoint{gp mark 3}{(6.914,3.684)}
\gppoint{gp mark 3}{(7.151,3.744)}
\gppoint{gp mark 3}{(7.387,3.771)}
\gppoint{gp mark 3}{(7.620,3.812)}
\gppoint{gp mark 3}{(7.858,3.867)}
\gppoint{gp mark 3}{(8.093,3.907)}
\gppoint{gp mark 3}{(8.328,3.934)}
\gppoint{gp mark 3}{(8.565,3.967)}
\gppoint{gp mark 3}{(8.799,4.006)}
\gppoint{gp mark 3}{(9.036,4.040)}
\gppoint{gp mark 3}{(9.271,4.068)}
\gppoint{gp mark 3}{(9.506,4.093)}
\gppoint{gp mark 3}{(9.741,4.125)}
\gppoint{gp mark 3}{(9.976,4.144)}
\gppoint{gp mark 3}{(10.212,4.180)}
\gppoint{gp mark 3}{(10.448,4.200)}
\gppoint{gp mark 3}{(10.687,4.240)}
\gppoint{gp mark 3}{(11.121,7.491)}
\gpcolor{color=gp lt color border}
\gpsetlinetype{gp lt border}
\draw[gp path] (1.688,7.751)--(1.688,1.548);
\draw[gp path] (1.688,0.985)--(10.910,0.985);
%% coordinates of the plot area
\gpdefrectangularnode{gp plot 1}{\pgfpoint{1.688cm}{0.985cm}}{\pgfpoint{11.947cm}{7.825cm}}
\end{tikzpicture}
%% gnuplot variables
 \caption{Grafico 0.966dgdecad.tex} \label{gr:0.966dgdecad.tex} \end{grafico}
\begin{grafico} \centering \begin{tikzpicture}[gnuplot]
%% generated with GNUPLOT 4.6p0 (Lua 5.1; terminal rev. 99, script rev. 100)
%% Tue 10 Jun 2014 07:45:02 PM CEST
\path (0.000,0.000) rectangle (12.500,8.750);
\gpcolor{color=gp lt color border}
\gpsetlinetype{gp lt border}
\gpsetlinewidth{1.00}
\draw[gp path] (1.504,0.985)--(1.684,0.985);
\node[gp node right] at (1.320,0.985) {-1};
\draw[gp path] (1.504,2.695)--(1.684,2.695);
\node[gp node right] at (1.320,2.695) {-0.5};
\draw[gp path] (1.504,4.405)--(1.684,4.405);
\node[gp node right] at (1.320,4.405) { 0};
\draw[gp path] (1.504,6.115)--(1.684,6.115);
\node[gp node right] at (1.320,6.115) { 0.5};
\draw[gp path] (1.504,7.825)--(1.684,7.825);
\node[gp node right] at (1.320,7.825) { 1};
\draw[gp path] (1.504,0.985)--(1.504,1.165);
\node[gp node center] at (1.504,0.677) { 0};
\draw[gp path] (2.664,0.985)--(2.664,1.165);
\node[gp node center] at (2.664,0.677) { 5};
\draw[gp path] (3.825,0.985)--(3.825,1.165);
\node[gp node center] at (3.825,0.677) { 10};
\draw[gp path] (4.985,0.985)--(4.985,1.165);
\node[gp node center] at (4.985,0.677) { 15};
\draw[gp path] (6.145,0.985)--(6.145,1.165);
\node[gp node center] at (6.145,0.677) { 20};
\draw[gp path] (7.306,0.985)--(7.306,1.165);
\node[gp node center] at (7.306,0.677) { 25};
\draw[gp path] (8.466,0.985)--(8.466,1.165);
\node[gp node center] at (8.466,0.677) { 30};
\draw[gp path] (9.626,0.985)--(9.626,1.165);
\node[gp node center] at (9.626,0.677) { 35};
\draw[gp path] (10.787,0.985)--(10.787,1.165);
\node[gp node center] at (10.787,0.677) { 40};
\draw[gp path] (1.504,7.825)--(1.504,0.985)--(10.845,0.985);
\node[gp node center,rotate=-270] at (0.246,4.405) {Ampiezza [???]};
\node[gp node center] at (6.725,0.215) {Tempo $[s]$};
\node[gp node center] at (6.725,8.287) {Dati decadimento 0.967d};
\gpcolor{rgb color={0.000,0.000,1.000}}
\gpsetlinetype{gp lt plot 0}
\draw[gp path] (1.504,5.130)--(1.516,4.915)--(1.527,4.651)--(1.539,4.367)--(1.550,4.077)%
  --(1.562,3.837)--(1.574,3.653)--(1.585,3.533)--(1.597,3.492)--(1.608,3.540)--(1.620,3.666)%
  --(1.632,3.858)--(1.643,4.097)--(1.655,4.367)--(1.666,4.641)--(1.678,4.891)--(1.690,5.096)%
  --(1.701,5.236)--(1.713,5.298)--(1.724,5.281)--(1.736,5.188)--(1.748,5.021)--(1.759,4.795)%
  --(1.771,4.538)--(1.782,4.272)--(1.794,4.022)--(1.806,3.803)--(1.817,3.642)--(1.829,3.557)%
  --(1.840,3.550)--(1.852,3.615)--(1.864,3.752)--(1.875,3.954)--(1.887,4.196)--(1.899,4.453)%
  --(1.910,4.706)--(1.922,4.935)--(1.933,5.110)--(1.945,5.219)--(1.957,5.260)--(1.968,5.219)%
  --(1.980,5.103)--(1.991,4.921)--(2.003,4.703)--(2.015,4.456)--(2.026,4.200)--(2.038,3.967)%
  --(2.049,3.779)--(2.061,3.653)--(2.073,3.588)--(2.084,3.601)--(2.096,3.690)--(2.107,3.844)%
  --(2.119,4.046)--(2.131,4.278)--(2.142,4.535)--(2.154,4.771)--(2.165,4.966)--(2.177,5.110)%
  --(2.189,5.198)--(2.200,5.212)--(2.212,5.147)--(2.223,5.014)--(2.235,4.836)--(2.247,4.610)%
  --(2.258,4.371)--(2.270,4.135)--(2.281,3.933)--(2.293,3.765)--(2.305,3.659)--(2.316,3.625)%
  --(2.328,3.663)--(2.339,3.765)--(2.351,3.926)--(2.363,4.128)--(2.374,4.367)--(2.386,4.603)%
  --(2.397,4.809)--(2.409,4.986)--(2.421,5.113)--(2.432,5.168)--(2.444,5.157)--(2.455,5.079)%
  --(2.467,4.939)--(2.479,4.750)--(2.490,4.532)--(2.502,4.306)--(2.513,4.080)--(2.525,3.895)%
  --(2.537,3.755)--(2.548,3.680)--(2.560,3.666)--(2.572,3.721)--(2.583,3.837)--(2.595,4.005)%
  --(2.606,4.210)--(2.618,4.432)--(2.630,4.648)--(2.641,4.850)--(2.653,5.004)--(2.664,5.099)%
  --(2.676,5.133)--(2.688,5.099)--(2.699,5.004)--(2.711,4.853)--(2.722,4.665)--(2.734,4.453)%
  --(2.746,4.237)--(2.757,4.039)--(2.769,3.875)--(2.780,3.762)--(2.792,3.704)--(2.804,3.711)%
  --(2.815,3.783)--(2.827,3.916)--(2.838,4.087)--(2.850,4.292)--(2.862,4.497)--(2.873,4.703)%
  --(2.885,4.877)--(2.896,5.004)--(2.908,5.079)--(2.920,5.096)--(2.931,5.045)--(2.943,4.932)%
  --(2.954,4.774)--(2.966,4.593)--(2.978,4.388)--(2.989,4.183)--(3.001,4.001)--(3.012,3.865)%
  --(3.024,3.765)--(3.036,3.735)--(3.047,3.762)--(3.059,3.851)--(3.070,3.988)--(3.082,4.159)%
  --(3.094,4.357)--(3.105,4.559)--(3.117,4.744)--(3.128,4.891)--(3.140,5.000)--(3.152,5.058)%
  --(3.163,5.051)--(3.175,4.983)--(3.186,4.863)--(3.198,4.709)--(3.210,4.521)--(3.221,4.323)%
  --(3.233,4.135)--(3.245,3.981)--(3.256,3.854)--(3.268,3.783)--(3.279,3.772)--(3.291,3.817)%
  --(3.303,3.916)--(3.314,4.060)--(3.326,4.234)--(3.337,4.426)--(3.349,4.610)--(3.361,4.774)%
  --(3.372,4.908)--(3.384,4.997)--(3.395,5.031)--(3.407,5.000)--(3.419,4.925)--(3.430,4.798)%
  --(3.442,4.638)--(3.453,4.456)--(3.465,4.272)--(3.477,4.101)--(3.488,3.957)--(3.500,3.854)%
  --(3.511,3.807)--(3.523,3.813)--(3.535,3.871)--(3.546,3.981)--(3.558,4.128)--(3.569,4.302)%
  --(3.581,4.480)--(3.593,4.651)--(3.604,4.802)--(3.616,4.918)--(3.627,4.983)--(3.639,4.990)%
  --(3.651,4.956)--(3.662,4.867)--(3.674,4.733)--(3.685,4.569)--(3.697,4.398)--(3.709,4.224)%
  --(3.720,4.066)--(3.732,3.943)--(3.743,3.865)--(3.755,3.830)--(3.767,3.854)--(3.778,3.923)%
  --(3.790,4.042)--(3.801,4.193)--(3.813,4.357)--(3.825,4.528)--(3.836,4.689)--(3.848,4.822)%
  --(3.859,4.915)--(3.871,4.962)--(3.883,4.959)--(3.894,4.904)--(3.906,4.805)--(3.917,4.668)%
  --(3.929,4.511)--(3.941,4.343)--(3.952,4.183)--(3.964,4.042)--(3.976,3.940)--(3.987,3.875)%
  --(3.999,3.865)--(4.010,3.906)--(4.022,3.984)--(4.034,4.104)--(4.045,4.251)--(4.057,4.412)%
  --(4.068,4.573)--(4.080,4.716)--(4.092,4.836)--(4.103,4.911)--(4.115,4.939)--(4.126,4.918)%
  --(4.138,4.853)--(4.150,4.750)--(4.161,4.610)--(4.173,4.453)--(4.184,4.296)--(4.196,4.152)%
  --(4.208,4.025)--(4.219,3.936)--(4.231,3.889)--(4.242,3.899)--(4.254,3.950)--(4.266,4.039)%
  --(4.277,4.159)--(4.289,4.306)--(4.300,4.463)--(4.312,4.610)--(4.324,4.740)--(4.335,4.839)%
  --(4.347,4.897)--(4.358,4.908)--(4.370,4.880)--(4.382,4.805)--(4.393,4.689)--(4.405,4.549)%
  --(4.416,4.402)--(4.428,4.255)--(4.440,4.118)--(4.451,4.008)--(4.463,3.947)--(4.474,3.923)%
  --(4.486,3.930)--(4.498,3.991)--(4.509,4.090)--(4.521,4.217)--(4.532,4.361)--(4.544,4.508)%
  --(4.556,4.644)--(4.567,4.757)--(4.579,4.836)--(4.590,4.880)--(4.602,4.880)--(4.614,4.836)%
  --(4.625,4.747)--(4.637,4.634)--(4.649,4.504)--(4.660,4.354)--(4.672,4.213)--(4.683,4.097)%
  --(4.695,4.008)--(4.707,3.950)--(4.718,3.936)--(4.730,3.967)--(4.741,4.036)--(4.753,4.138)%
  --(4.765,4.265)--(4.776,4.405)--(4.788,4.549)--(4.799,4.668)--(4.811,4.768)--(4.823,4.836)%
  --(4.834,4.860)--(4.846,4.846)--(4.857,4.788)--(4.869,4.699)--(4.881,4.586)--(4.892,4.449)%
  --(4.904,4.309)--(4.915,4.186)--(4.927,4.084)--(4.939,4.001)--(4.950,3.960)--(4.962,3.967)%
  --(4.973,4.012)--(4.985,4.080)--(4.997,4.190)--(5.008,4.320)--(5.020,4.453)--(5.031,4.576)%
  --(5.043,4.689)--(5.055,4.774)--(5.066,4.829)--(5.078,4.833)--(5.089,4.805)--(5.101,4.750)%
  --(5.113,4.655)--(5.124,4.532)--(5.136,4.402)--(5.147,4.278)--(5.159,4.169)--(5.171,4.070)%
  --(5.182,4.005)--(5.194,3.984)--(5.205,3.998)--(5.217,4.046)--(5.229,4.125)--(5.240,4.237)%
  --(5.252,4.361)--(5.263,4.487)--(5.275,4.603)--(5.287,4.706)--(5.298,4.778)--(5.310,4.809)%
  --(5.321,4.809)--(5.333,4.778)--(5.345,4.703)--(5.356,4.600)--(5.368,4.487)--(5.380,4.371)%
  --(5.391,4.244)--(5.403,4.138)--(5.414,4.066)--(5.426,4.022)--(5.438,4.005)--(5.449,4.025)%
  --(5.461,4.087)--(5.472,4.176)--(5.484,4.285)--(5.496,4.402)--(5.507,4.525)--(5.519,4.631)%
  --(5.530,4.716)--(5.542,4.771)--(5.554,4.798)--(5.565,4.788)--(5.577,4.737)--(5.588,4.658)%
  --(5.600,4.562)--(5.612,4.449)--(5.623,4.330)--(5.635,4.220)--(5.646,4.128)--(5.658,4.063)%
  --(5.670,4.025)--(5.681,4.025)--(5.693,4.063)--(5.704,4.131)--(5.716,4.217)--(5.728,4.323)%
  --(5.739,4.443)--(5.751,4.552)--(5.762,4.644)--(5.774,4.723)--(5.786,4.768)--(5.797,4.781)%
  --(5.809,4.750)--(5.820,4.699)--(5.832,4.620)--(5.844,4.514)--(5.855,4.408)--(5.867,4.299)%
  --(5.878,4.196)--(5.890,4.114)--(5.902,4.060)--(5.913,4.039)--(5.925,4.053)--(5.936,4.094)%
  --(5.948,4.162)--(5.960,4.261)--(5.971,4.367)--(5.983,4.473)--(5.994,4.573)--(6.006,4.662)%
  --(6.018,4.723)--(6.029,4.754)--(6.041,4.754)--(6.053,4.726)--(6.064,4.662)--(6.076,4.579)%
  --(6.087,4.477)--(6.099,4.374)--(6.111,4.272)--(6.122,4.183)--(6.134,4.111)--(6.145,4.070)%
  --(6.157,4.063)--(6.169,4.077)--(6.180,4.128)--(6.192,4.207)--(6.203,4.302)--(6.215,4.402)%
  --(6.227,4.501)--(6.238,4.593)--(6.250,4.672)--(6.261,4.720)--(6.273,4.744)--(6.285,4.733)%
  --(6.296,4.692)--(6.308,4.624)--(6.319,4.542)--(6.331,4.449)--(6.343,4.340)--(6.354,4.241)%
  --(6.366,4.166)--(6.377,4.111)--(6.389,4.080)--(6.401,4.077)--(6.412,4.104)--(6.424,4.166)%
  --(6.435,4.244)--(6.447,4.337)--(6.459,4.432)--(6.470,4.528)--(6.482,4.610)--(6.493,4.675)%
  --(6.505,4.716)--(6.517,4.726)--(6.528,4.703)--(6.540,4.658)--(6.551,4.590)--(6.563,4.504)%
  --(6.575,4.405)--(6.586,4.313)--(6.598,4.234)--(6.609,4.159)--(6.621,4.111)--(6.633,4.090)%
  --(6.644,4.101)--(6.656,4.135)--(6.667,4.196)--(6.679,4.278)--(6.691,4.374)--(6.702,4.463)%
  --(6.714,4.549)--(6.726,4.624)--(6.737,4.682)--(6.749,4.706)--(6.760,4.703)--(6.772,4.679)%
  --(6.784,4.631)--(6.795,4.559)--(6.807,4.467)--(6.818,4.378)--(6.830,4.289)--(6.842,4.210)%
  --(6.853,4.149)--(6.865,4.118)--(6.876,4.104)--(6.888,4.125)--(6.900,4.166)--(6.911,4.234)%
  --(6.923,4.316)--(6.934,4.402)--(6.946,4.487)--(6.958,4.569)--(6.969,4.634)--(6.981,4.675)%
  --(6.992,4.692)--(7.004,4.689)--(7.016,4.655)--(7.027,4.597)--(7.039,4.518)--(7.050,4.439)%
  --(7.062,4.354)--(7.074,4.265)--(7.085,4.200)--(7.097,4.152)--(7.108,4.125)--(7.120,4.121)%
  --(7.132,4.149)--(7.143,4.196)--(7.155,4.265)--(7.166,4.347)--(7.178,4.429)--(7.190,4.514)%
  --(7.201,4.586)--(7.213,4.638)--(7.224,4.672)--(7.236,4.682)--(7.248,4.668)--(7.259,4.627)%
  --(7.271,4.569)--(7.282,4.491)--(7.294,4.408)--(7.306,4.326)--(7.317,4.251)--(7.329,4.190)%
  --(7.340,4.149)--(7.352,4.131)--(7.364,4.142)--(7.375,4.176)--(7.387,4.227)--(7.398,4.296)%
  --(7.410,4.374)--(7.422,4.456)--(7.433,4.528)--(7.445,4.593)--(7.457,4.641)--(7.468,4.665)%
  --(7.480,4.665)--(7.491,4.641)--(7.503,4.597)--(7.515,4.535)--(7.526,4.460)--(7.538,4.378)%
  --(7.549,4.302)--(7.561,4.241)--(7.573,4.183)--(7.584,4.152)--(7.596,4.145)--(7.607,4.166)%
  --(7.619,4.196)--(7.631,4.258)--(7.642,4.330)--(7.654,4.402)--(7.665,4.473)--(7.677,4.545)%
  --(7.689,4.603)--(7.700,4.638)--(7.712,4.651)--(7.723,4.648)--(7.735,4.620)--(7.747,4.573)%
  --(7.758,4.504)--(7.770,4.432)--(7.781,4.361)--(7.793,4.289)--(7.805,4.227)--(7.816,4.183)%
  --(7.828,4.162)--(7.839,4.162)--(7.851,4.183)--(7.863,4.227)--(7.874,4.285)--(7.886,4.350)%
  --(7.897,4.426)--(7.909,4.497)--(7.921,4.559)--(7.932,4.607)--(7.944,4.634)--(7.955,4.644)%
  --(7.967,4.631)--(7.979,4.593)--(7.990,4.538)--(8.002,4.477)--(8.013,4.408)--(8.025,4.337)%
  --(8.037,4.272)--(8.048,4.217)--(8.060,4.186)--(8.071,4.172)--(8.083,4.176)--(8.095,4.207)%
  --(8.106,4.251)--(8.118,4.309)--(8.130,4.378)--(8.141,4.449)--(8.153,4.514)--(8.164,4.562)%
  --(8.176,4.610)--(8.188,4.634)--(8.199,4.631)--(8.211,4.607)--(8.222,4.566)--(8.234,4.514)%
  --(8.246,4.453)--(8.257,4.384)--(8.269,4.316)--(8.280,4.261)--(8.292,4.217)--(8.304,4.186)%
  --(8.315,4.183)--(8.327,4.196)--(8.338,4.227)--(8.350,4.275)--(8.362,4.337)--(8.373,4.405)%
  --(8.385,4.467)--(8.396,4.525)--(8.408,4.579)--(8.420,4.610)--(8.431,4.624)--(8.443,4.610)%
  --(8.454,4.586)--(8.466,4.549)--(8.478,4.494)--(8.489,4.429)--(8.501,4.361)--(8.512,4.302)%
  --(8.524,4.255)--(8.536,4.213)--(8.547,4.193)--(8.559,4.193)--(8.570,4.213)--(8.582,4.251)%
  --(8.594,4.302)--(8.605,4.364)--(8.617,4.422)--(8.628,4.484)--(8.640,4.542)--(8.652,4.586)%
  --(8.663,4.603)--(8.675,4.607)--(8.686,4.597)--(8.698,4.569)--(8.710,4.525)--(8.721,4.463)%
  --(8.733,4.405)--(8.744,4.343)--(8.756,4.289)--(8.768,4.248)--(8.779,4.217)--(8.791,4.200)%
  --(8.802,4.210)--(8.814,4.234)--(8.826,4.268)--(8.837,4.326)--(8.849,4.381)--(8.861,4.439)%
  --(8.872,4.497)--(8.884,4.545)--(8.895,4.579)--(8.907,4.593)--(8.919,4.597)--(8.930,4.583)%
  --(8.942,4.545)--(8.953,4.501)--(8.965,4.443)--(8.977,4.384)--(8.988,4.330)--(9.000,4.278)%
  --(9.011,4.241)--(9.023,4.217)--(9.035,4.210)--(9.046,4.220)--(9.058,4.251)--(9.069,4.296)%
  --(9.081,4.347)--(9.093,4.402)--(9.104,4.467)--(9.116,4.514)--(9.127,4.552)--(9.139,4.579)%
  --(9.151,4.593)--(9.162,4.586)--(9.174,4.562)--(9.185,4.525)--(9.197,4.480)--(9.209,4.422)%
  --(9.220,4.367)--(9.232,4.316)--(9.243,4.275)--(9.255,4.241)--(9.267,4.220)--(9.278,4.224)%
  --(9.290,4.241)--(9.301,4.272)--(9.313,4.316)--(9.325,4.367)--(9.336,4.426)--(9.348,4.477)%
  --(9.359,4.521)--(9.371,4.555)--(9.383,4.576)--(9.394,4.583)--(9.406,4.569)--(9.417,4.545)%
  --(9.429,4.508)--(9.441,4.456)--(9.452,4.405)--(9.464,4.354)--(9.475,4.306)--(9.487,4.265)%
  --(9.499,4.241)--(9.510,4.231)--(9.522,4.237)--(9.534,4.258)--(9.545,4.292)--(9.557,4.337)%
  --(9.568,4.391)--(9.580,4.443)--(9.592,4.487)--(9.603,4.528)--(9.615,4.559)--(9.626,4.573)%
  --(9.638,4.573)--(9.650,4.559)--(9.661,4.528)--(9.673,4.487)--(9.684,4.443)--(9.696,4.391)%
  --(9.708,4.340)--(9.719,4.296)--(9.731,4.268)--(9.742,4.244)--(9.754,4.237)--(9.766,4.248)%
  --(9.777,4.275)--(9.789,4.313)--(9.800,4.361)--(9.812,4.405)--(9.824,4.453)--(9.835,4.497)%
  --(9.847,4.535)--(9.858,4.559)--(9.870,4.566)--(9.882,4.559)--(9.893,4.542)--(9.905,4.511)%
  --(9.916,4.467)--(9.928,4.422)--(9.940,4.374)--(9.951,4.326)--(9.963,4.289)--(9.974,4.261)%
  --(9.986,4.248)--(9.998,4.248)--(10.009,4.265)--(10.021,4.292)--(10.032,4.330)--(10.044,4.374)%
  --(10.056,4.419)--(10.067,4.467)--(10.079,4.508)--(10.090,4.535)--(10.102,4.555)--(10.114,4.559)%
  --(10.125,4.549)--(10.137,4.528)--(10.148,4.494)--(10.160,4.449)--(10.172,4.405)--(10.183,4.361)%
  --(10.195,4.316)--(10.207,4.289)--(10.218,4.265)--(10.230,4.255)--(10.241,4.258)--(10.253,4.278)%
  --(10.265,4.316)--(10.276,4.347)--(10.288,4.388)--(10.299,4.432)--(10.311,4.480)--(10.323,4.511)%
  --(10.334,4.538)--(10.346,4.552)--(10.357,4.552)--(10.369,4.535)--(10.381,4.508)--(10.392,4.477)%
  --(10.404,4.436)--(10.415,4.391)--(10.427,4.347)--(10.439,4.309)--(10.450,4.285)--(10.462,4.265)%
  --(10.473,4.261)--(10.485,4.272)--(10.497,4.296)--(10.508,4.326)--(10.520,4.361)--(10.531,4.405)%
  --(10.543,4.449)--(10.555,4.484)--(10.566,4.518)--(10.578,4.538)--(10.589,4.545)--(10.601,4.538)%
  --(10.613,4.521)--(10.624,4.497)--(10.636,4.460)--(10.647,4.415)--(10.659,4.374)--(10.671,4.340)%
  --(10.682,4.306)--(10.694,4.278)--(10.705,4.268)--(10.717,4.272)--(10.729,4.285)--(10.740,4.306)%
  --(10.752,4.337)--(10.763,4.378)--(10.775,4.422)--(10.787,4.456)--(10.798,4.494)--(10.810,4.518)%
  --(10.821,4.535)--(10.833,4.538)--(10.845,4.528);
\gpcolor{color=gp lt color border}
\node[gp node right] at (10.479,7.491) {Punti tangenza};
\gpcolor{rgb color={1.000,0.000,0.000}}
\gpsetpointsize{4.00}
\gppoint{gp mark 3}{(1.519,4.864)}
\gppoint{gp mark 3}{(1.597,3.492)}
\gppoint{gp mark 3}{(1.719,5.327)}
\gppoint{gp mark 3}{(1.835,3.518)}
\gppoint{gp mark 3}{(1.957,5.260)}
\gppoint{gp mark 3}{(2.078,3.557)}
\gppoint{gp mark 3}{(2.194,5.245)}
\gppoint{gp mark 3}{(2.316,3.625)}
\gppoint{gp mark 3}{(2.438,5.197)}
\gppoint{gp mark 3}{(2.554,3.639)}
\gppoint{gp mark 3}{(2.676,5.133)}
\gppoint{gp mark 3}{(2.798,3.675)}
\gppoint{gp mark 3}{(2.914,5.121)}
\gppoint{gp mark 3}{(3.036,3.735)}
\gppoint{gp mark 3}{(3.157,5.086)}
\gppoint{gp mark 3}{(3.274,3.750)}
\gppoint{gp mark 3}{(3.395,5.031)}
\gppoint{gp mark 3}{(3.517,3.784)}
\gppoint{gp mark 3}{(3.633,5.007)}
\gppoint{gp mark 3}{(3.755,3.786)}
\gppoint{gp mark 3}{(3.877,4.986)}
\gppoint{gp mark 3}{(3.993,3.844)}
\gppoint{gp mark 3}{(4.115,4.939)}
\gppoint{gp mark 3}{(4.237,3.873)}
\gppoint{gp mark 3}{(4.353,4.921)}
\gppoint{gp mark 3}{(4.474,3.923)}
\gppoint{gp mark 3}{(4.596,4.903)}
\gppoint{gp mark 3}{(4.718,3.936)}
\gppoint{gp mark 3}{(4.840,4.875)}
\gppoint{gp mark 3}{(4.956,3.945)}
\gppoint{gp mark 3}{(5.072,4.846)}
\gppoint{gp mark 3}{(5.194,3.984)}
\gppoint{gp mark 3}{(5.316,4.824)}
\gppoint{gp mark 3}{(5.438,4.005)}
\gppoint{gp mark 3}{(5.559,4.814)}
\gppoint{gp mark 3}{(5.675,4.007)}
\gppoint{gp mark 3}{(5.797,4.781)}
\gppoint{gp mark 3}{(5.919,4.032)}
\gppoint{gp mark 3}{(6.035,4.768)}
\gppoint{gp mark 3}{(6.157,4.063)}
\gppoint{gp mark 3}{(6.279,4.754)}
\gppoint{gp mark 3}{(6.395,4.063)}
\gppoint{gp mark 3}{(6.517,4.726)}
\gppoint{gp mark 3}{(6.638,4.084)}
\gppoint{gp mark 3}{(6.755,4.715)}
\gppoint{gp mark 3}{(6.876,4.104)}
\gppoint{gp mark 3}{(6.998,4.706)}
\gppoint{gp mark 3}{(7.114,4.107)}
\gppoint{gp mark 3}{(7.236,4.682)}
\gppoint{gp mark 3}{(7.358,4.125)}
\gppoint{gp mark 3}{(7.474,4.677)}
\gppoint{gp mark 3}{(7.596,4.145)}
\gppoint{gp mark 3}{(7.718,4.662)}
\gppoint{gp mark 3}{(7.834,4.152)}
\gppoint{gp mark 3}{(7.955,4.644)}
\gppoint{gp mark 3}{(8.077,4.160)}
\gppoint{gp mark 3}{(8.193,4.643)}
\gppoint{gp mark 3}{(8.315,4.183)}
\gppoint{gp mark 3}{(8.437,4.622)}
\gppoint{gp mark 3}{(8.553,4.183)}
\gppoint{gp mark 3}{(8.669,4.612)}
\gppoint{gp mark 3}{(8.791,4.200)}
\gppoint{gp mark 3}{(8.913,4.603)}
\gppoint{gp mark 3}{(9.035,4.210)}
\gppoint{gp mark 3}{(9.156,4.598)}
\gppoint{gp mark 3}{(9.272,4.215)}
\gppoint{gp mark 3}{(9.388,4.590)}
\gppoint{gp mark 3}{(9.510,4.231)}
\gppoint{gp mark 3}{(9.632,4.579)}
\gppoint{gp mark 3}{(9.748,4.232)}
\gppoint{gp mark 3}{(9.870,4.566)}
\gppoint{gp mark 3}{(9.992,4.239)}
\gppoint{gp mark 3}{(10.114,4.559)}
\gppoint{gp mark 3}{(10.236,4.248)}
\gppoint{gp mark 3}{(10.352,4.561)}
\gppoint{gp mark 3}{(10.468,4.256)}
\gppoint{gp mark 3}{(10.589,4.545)}
\gppoint{gp mark 3}{(10.711,4.265)}
\gppoint{gp mark 3}{(11.121,7.491)}
\gpcolor{color=gp lt color border}
\node[gp node right] at (10.479,7.183) {fexp(x)};
\gpcolor{color=gp lt color 2}
\gpsetlinetype{gp lt plot 2}
\draw[gp path] (10.663,7.183)--(11.579,7.183);
\draw[gp path] (1.504,7.825)--(1.598,7.760)--(1.693,7.697)--(1.787,7.635)--(1.881,7.573)%
  --(1.976,7.514)--(2.070,7.455)--(2.164,7.397)--(2.259,7.340)--(2.353,7.285)--(2.448,7.230)%
  --(2.542,7.177)--(2.636,7.125)--(2.731,7.073)--(2.825,7.023)--(2.919,6.973)--(3.014,6.925)%
  --(3.108,6.877)--(3.202,6.830)--(3.297,6.784)--(3.391,6.739)--(3.485,6.695)--(3.580,6.652)%
  --(3.674,6.609)--(3.768,6.568)--(3.863,6.527)--(3.957,6.487)--(4.051,6.447)--(4.146,6.409)%
  --(4.240,6.371)--(4.335,6.333)--(4.429,6.297)--(4.523,6.261)--(4.618,6.226)--(4.712,6.192)%
  --(4.806,6.158)--(4.901,6.125)--(4.995,6.092)--(5.089,6.060)--(5.184,6.029)--(5.278,5.998)%
  --(5.372,5.968)--(5.467,5.939)--(5.561,5.909)--(5.655,5.881)--(5.750,5.853)--(5.844,5.826)%
  --(5.938,5.799)--(6.033,5.772)--(6.127,5.747)--(6.222,5.721)--(6.316,5.696)--(6.410,5.672)%
  --(6.505,5.648)--(6.599,5.624)--(6.693,5.601)--(6.788,5.579)--(6.882,5.557)--(6.976,5.535)%
  --(7.071,5.513)--(7.165,5.492)--(7.259,5.472)--(7.354,5.452)--(7.448,5.432)--(7.542,5.412)%
  --(7.637,5.393)--(7.731,5.375)--(7.825,5.356)--(7.920,5.338)--(8.014,5.321)--(8.109,5.303)%
  --(8.203,5.286)--(8.297,5.270)--(8.392,5.253)--(8.486,5.237)--(8.580,5.222)--(8.675,5.206)%
  --(8.769,5.191)--(8.863,5.176)--(8.958,5.162)--(9.052,5.147)--(9.146,5.133)--(9.241,5.119)%
  --(9.335,5.106)--(9.429,5.093)--(9.524,5.080)--(9.618,5.067)--(9.712,5.054)--(9.807,5.042)%
  --(9.901,5.030)--(9.996,5.018)--(10.090,5.007)--(10.184,4.995)--(10.279,4.984)--(10.373,4.973)%
  --(10.467,4.962)--(10.562,4.952)--(10.656,4.941)--(10.750,4.931)--(10.845,4.921);
\gpcolor{color=gp lt color border}
\node[gp node right] at (10.479,6.875) {fexpneg(x)};
\gpcolor{color=gp lt color 3}
\gpsetlinetype{gp lt plot 3}
\draw[gp path] (10.663,6.875)--(11.579,6.875);
\draw[gp path] (1.504,0.985)--(1.598,1.050)--(1.693,1.113)--(1.787,1.175)--(1.881,1.237)%
  --(1.976,1.296)--(2.070,1.355)--(2.164,1.413)--(2.259,1.470)--(2.353,1.525)--(2.448,1.580)%
  --(2.542,1.633)--(2.636,1.685)--(2.731,1.737)--(2.825,1.787)--(2.919,1.837)--(3.014,1.885)%
  --(3.108,1.933)--(3.202,1.980)--(3.297,2.026)--(3.391,2.071)--(3.485,2.115)--(3.580,2.158)%
  --(3.674,2.201)--(3.768,2.242)--(3.863,2.283)--(3.957,2.323)--(4.051,2.363)--(4.146,2.401)%
  --(4.240,2.439)--(4.335,2.477)--(4.429,2.513)--(4.523,2.549)--(4.618,2.584)--(4.712,2.618)%
  --(4.806,2.652)--(4.901,2.685)--(4.995,2.718)--(5.089,2.750)--(5.184,2.781)--(5.278,2.812)%
  --(5.372,2.842)--(5.467,2.871)--(5.561,2.901)--(5.655,2.929)--(5.750,2.957)--(5.844,2.984)%
  --(5.938,3.011)--(6.033,3.038)--(6.127,3.063)--(6.222,3.089)--(6.316,3.114)--(6.410,3.138)%
  --(6.505,3.162)--(6.599,3.186)--(6.693,3.209)--(6.788,3.231)--(6.882,3.253)--(6.976,3.275)%
  --(7.071,3.297)--(7.165,3.318)--(7.259,3.338)--(7.354,3.358)--(7.448,3.378)--(7.542,3.398)%
  --(7.637,3.417)--(7.731,3.435)--(7.825,3.454)--(7.920,3.472)--(8.014,3.489)--(8.109,3.507)%
  --(8.203,3.524)--(8.297,3.540)--(8.392,3.557)--(8.486,3.573)--(8.580,3.588)--(8.675,3.604)%
  --(8.769,3.619)--(8.863,3.634)--(8.958,3.648)--(9.052,3.663)--(9.146,3.677)--(9.241,3.691)%
  --(9.335,3.704)--(9.429,3.717)--(9.524,3.730)--(9.618,3.743)--(9.712,3.756)--(9.807,3.768)%
  --(9.901,3.780)--(9.996,3.792)--(10.090,3.803)--(10.184,3.815)--(10.279,3.826)--(10.373,3.837)%
  --(10.467,3.848)--(10.562,3.858)--(10.656,3.869)--(10.750,3.879)--(10.845,3.889);
\gpcolor{color=gp lt color border}
\gpsetlinetype{gp lt border}
\draw[gp path] (1.504,7.825)--(1.504,0.985)--(10.845,0.985);
%% coordinates of the plot area
\gpdefrectangularnode{gp plot 1}{\pgfpoint{1.504cm}{0.985cm}}{\pgfpoint{11.947cm}{7.825cm}}
\end{tikzpicture}
%% gnuplot variables
 \caption{Grafico 0.967dgdecad.tex} \label{gr:0.967dgdecad.tex} \end{grafico}
\begin{grafico} \centering \begin{tikzpicture}[gnuplot]
%% generated with GNUPLOT 4.6p0 (Lua 5.1; terminal rev. 99, script rev. 100)
%% Tue 10 Jun 2014 05:52:58 PM CEST
\path (0.000,0.000) rectangle (12.500,8.750);
\gpcolor{color=gp lt color border}
\gpsetlinetype{gp lt border}
\gpsetlinewidth{1.00}
\draw[gp path] (1.504,2.125)--(1.684,2.125);
\node[gp node right] at (1.320,2.125) {-0.2};
\draw[gp path] (1.504,3.265)--(1.684,3.265);
\node[gp node right] at (1.320,3.265) {-0.1};
\draw[gp path] (1.504,4.405)--(1.684,4.405);
\node[gp node right] at (1.320,4.405) { 0};
\draw[gp path] (1.504,5.545)--(1.684,5.545);
\node[gp node right] at (1.320,5.545) { 0.1};
\draw[gp path] (1.504,6.685)--(1.684,6.685);
\node[gp node right] at (1.320,6.685) { 0.2};
\draw[gp path] (1.504,0.985)--(1.504,1.165);
\node[gp node center] at (1.504,0.677) { 0};
\draw[gp path] (3.245,0.985)--(3.245,1.165);
\node[gp node center] at (3.245,0.677) { 20};
\draw[gp path] (4.985,0.985)--(4.985,1.165);
\node[gp node center] at (4.985,0.677) { 40};
\draw[gp path] (6.726,0.985)--(6.726,1.165);
\node[gp node center] at (6.726,0.677) { 60};
\draw[gp path] (8.466,0.985)--(8.466,1.165);
\node[gp node center] at (8.466,0.677) { 80};
\draw[gp path] (10.207,0.985)--(10.207,1.165);
\node[gp node center] at (10.207,0.677) { 100};
\draw[gp path] (1.504,7.473)--(1.504,1.218);
\draw[gp path] (1.504,0.985)--(10.222,0.985);
\node[gp node center,rotate=-270] at (0.246,4.405) {Ampiezza [???]};
\node[gp node center] at (6.725,0.215) {Tempo $[s]$};
\node[gp node center] at (6.725,8.287) {Dati decadimento 0.968d};
\gpcolor{rgb color={0.000,0.000,1.000}}
\gpsetlinetype{gp lt plot 0}
\draw[gp path] (1.504,3.561)--(1.508,2.672)--(1.513,1.931)--(1.517,1.441)--(1.521,1.224)%
  --(1.526,1.316)--(1.530,1.680)--(1.534,2.285)--(1.539,3.048)--(1.543,3.949)--(1.548,4.907)%
  --(1.552,5.796)--(1.556,6.571)--(1.561,7.118)--(1.565,7.426)--(1.569,7.449)--(1.574,7.198)%
  --(1.578,6.696)--(1.582,5.990)--(1.587,5.123)--(1.591,4.223)--(1.595,3.322)--(1.600,2.535)%
  --(1.604,1.874)--(1.608,1.498)--(1.613,1.407)--(1.617,1.555)--(1.621,1.954)--(1.626,2.592)%
  --(1.630,3.379)--(1.635,4.268)--(1.639,5.157)--(1.643,5.990)--(1.648,6.639)--(1.652,7.095)%
  --(1.656,7.323)--(1.661,7.266)--(1.665,6.936)--(1.669,6.389)--(1.674,5.659)--(1.678,4.827)%
  --(1.682,3.960)--(1.687,3.140)--(1.691,2.421)--(1.695,1.874)--(1.700,1.578)--(1.704,1.555)%
  --(1.709,1.760)--(1.713,2.228)--(1.717,2.900)--(1.722,3.675)--(1.726,4.508)--(1.730,5.363)%
  --(1.735,6.115)--(1.739,6.708)--(1.743,7.073)--(1.748,7.175)--(1.752,7.050)--(1.756,6.674)%
  --(1.761,6.081)--(1.765,5.363)--(1.769,4.553)--(1.774,3.721)--(1.778,2.969)--(1.782,2.342)%
  --(1.787,1.897)--(1.791,1.703)--(1.796,1.737)--(1.800,2.022)--(1.804,2.535)--(1.809,3.197)%
  --(1.813,3.972)--(1.817,4.781)--(1.822,5.545)--(1.826,6.218)--(1.830,6.719)--(1.835,6.981)%
  --(1.839,7.004)--(1.843,6.822)--(1.848,6.389)--(1.852,5.796)--(1.856,5.055)--(1.861,4.291)%
  --(1.865,3.527)--(1.870,2.843)--(1.874,2.319)--(1.878,1.965)--(1.883,1.851)--(1.887,1.954)%
  --(1.891,2.307)--(1.896,2.832)--(1.900,3.516)--(1.904,4.268)--(1.909,5.009)--(1.913,5.716)%
  --(1.917,6.297)--(1.922,6.696)--(1.926,6.879)--(1.930,6.833)--(1.935,6.582)--(1.939,6.115)%
  --(1.943,5.511)--(1.948,4.804)--(1.952,4.052)--(1.957,3.356)--(1.961,2.752)--(1.965,2.307)%
  --(1.970,2.022)--(1.974,1.977)--(1.978,2.159)--(1.983,2.558)--(1.987,3.117)--(1.991,3.767)%
  --(1.996,4.496)--(2.000,5.214)--(2.004,5.841)--(2.009,6.332)--(2.013,6.651)--(2.017,6.765)%
  --(2.022,6.662)--(2.026,6.354)--(2.031,5.864)--(2.035,5.249)--(2.039,4.553)--(2.044,3.846)%
  --(2.048,3.197)--(2.052,2.684)--(2.057,2.296)--(2.061,2.102)--(2.065,2.136)--(2.070,2.376)%
  --(2.074,2.786)--(2.078,3.356)--(2.083,4.040)--(2.087,4.736)--(2.091,5.397)--(2.096,5.944)%
  --(2.100,6.354)--(2.104,6.594)--(2.109,6.651)--(2.113,6.468)--(2.118,6.115)--(2.122,5.613)%
  --(2.126,4.998)--(2.131,4.325)--(2.135,3.664)--(2.139,3.094)--(2.144,2.627)--(2.148,2.319)%
  --(2.152,2.216)--(2.157,2.307)--(2.161,2.581)--(2.165,3.026)--(2.170,3.618)--(2.174,4.245)%
  --(2.178,4.907)--(2.183,5.499)--(2.187,6.001)--(2.191,6.354)--(2.196,6.525)--(2.200,6.491)%
  --(2.205,6.286)--(2.209,5.898)--(2.213,5.374)--(2.218,4.781)--(2.222,4.154)--(2.226,3.539)%
  --(2.231,2.991)--(2.235,2.604)--(2.239,2.364)--(2.244,2.330)--(2.248,2.478)--(2.252,2.775)%
  --(2.257,3.254)--(2.261,3.812)--(2.265,4.428)--(2.270,5.055)--(2.274,5.591)--(2.279,6.035)%
  --(2.283,6.320)--(2.287,6.411)--(2.292,6.343)--(2.296,6.104)--(2.300,5.670)--(2.305,5.135)%
  --(2.309,4.565)--(2.313,3.972)--(2.318,3.402)--(2.322,2.923)--(2.326,2.604)--(2.331,2.444)%
  --(2.335,2.478)--(2.339,2.638)--(2.344,3.003)--(2.348,3.493)--(2.352,4.040)--(2.357,4.656)%
  --(2.361,5.226)--(2.366,5.693)--(2.370,6.069)--(2.374,6.263)--(2.379,6.332)--(2.383,6.206)%
  --(2.387,5.876)--(2.392,5.454)--(2.396,4.952)--(2.400,4.371)--(2.405,3.789)--(2.409,3.276)%
  --(2.413,2.889)--(2.418,2.627)--(2.422,2.524)--(2.426,2.592)--(2.431,2.832)--(2.435,3.219)%
  --(2.440,3.687)--(2.444,4.245)--(2.448,4.804)--(2.453,5.328)--(2.457,5.762)--(2.461,6.047)%
  --(2.466,6.218)--(2.470,6.218)--(2.474,6.024)--(2.479,5.693)--(2.483,5.271)--(2.487,4.747)%
  --(2.492,4.177)--(2.496,3.664)--(2.500,3.208)--(2.505,2.877)--(2.509,2.649)--(2.513,2.627)%
  --(2.518,2.752)--(2.522,3.014)--(2.527,3.402)--(2.531,3.881)--(2.535,4.428)--(2.540,4.952)%
  --(2.544,5.408)--(2.548,5.784)--(2.553,6.047)--(2.557,6.149)--(2.561,6.069)--(2.566,5.841)%
  --(2.570,5.511)--(2.574,5.066)--(2.579,4.553)--(2.583,4.040)--(2.587,3.550)--(2.592,3.140)%
  --(2.596,2.866)--(2.601,2.729)--(2.605,2.741)--(2.609,2.889)--(2.614,3.197)--(2.618,3.618)%
  --(2.622,4.097)--(2.627,4.587)--(2.631,5.055)--(2.635,5.488)--(2.640,5.819)--(2.644,6.024)%
  --(2.648,6.058)--(2.653,5.921)--(2.657,5.693)--(2.661,5.340)--(2.666,4.884)--(2.670,4.394)%
  --(2.674,3.881)--(2.679,3.459)--(2.683,3.105)--(2.688,2.877)--(2.692,2.786)--(2.696,2.843)%
  --(2.701,3.060)--(2.705,3.356)--(2.709,3.778)--(2.714,4.245)--(2.718,4.724)--(2.722,5.169)%
  --(2.727,5.556)--(2.731,5.819)--(2.735,5.955)--(2.740,5.944)--(2.744,5.796)--(2.748,5.534)%
  --(2.753,5.169)--(2.757,4.701)--(2.762,4.234)--(2.766,3.789)--(2.770,3.413)--(2.775,3.094)%
  --(2.779,2.923)--(2.783,2.877)--(2.788,2.980)--(2.792,3.208)--(2.796,3.539)--(2.801,3.972)%
  --(2.805,4.405)--(2.809,4.861)--(2.814,5.260)--(2.818,5.591)--(2.822,5.819)--(2.827,5.887)%
  --(2.831,5.841)--(2.835,5.693)--(2.840,5.363)--(2.844,4.975)--(2.849,4.542)--(2.853,4.120)%
  --(2.857,3.698)--(2.862,3.333)--(2.866,3.094)--(2.870,2.957)--(2.875,2.980)--(2.879,3.094)%
  --(2.883,3.356)--(2.888,3.710)--(2.892,4.109)--(2.896,4.542)--(2.901,4.964)--(2.905,5.328)%
  --(2.909,5.625)--(2.914,5.762)--(2.918,5.819)--(2.923,5.739)--(2.927,5.511)--(2.931,5.192)%
  --(2.936,4.827)--(2.940,4.416)--(2.944,3.972)--(2.949,3.584)--(2.953,3.311)--(2.957,3.105)%
  --(2.962,3.026)--(2.966,3.060)--(2.970,3.242)--(2.975,3.504)--(2.979,3.858)--(2.983,4.257)%
  --(2.988,4.679)--(2.992,5.078)--(2.996,5.385)--(3.001,5.613)--(3.005,5.750)--(3.010,5.762)%
  --(3.014,5.625)--(3.018,5.385)--(3.023,5.043)--(3.027,4.667)--(3.031,4.268)--(3.036,3.892)%
  --(3.040,3.539)--(3.044,3.276)--(3.049,3.140)--(3.053,3.094)--(3.057,3.174)--(3.062,3.368)%
  --(3.066,3.641)--(3.070,4.006)--(3.075,4.405)--(3.079,4.781)--(3.084,5.146)--(3.088,5.408)%
  --(3.092,5.613)--(3.097,5.705)--(3.101,5.636)--(3.105,5.488)--(3.110,5.226)--(3.114,4.907)%
  --(3.118,4.542)--(3.123,4.166)--(3.127,3.789)--(3.131,3.493)--(3.136,3.288)--(3.140,3.174)%
  --(3.144,3.174)--(3.149,3.288)--(3.153,3.493)--(3.157,3.801)--(3.162,4.154)--(3.166,4.519)%
  --(3.171,4.884)--(3.175,5.192)--(3.179,5.454)--(3.184,5.579)--(3.188,5.625)--(3.192,5.545)%
  --(3.197,5.363)--(3.201,5.100)--(3.205,4.770)--(3.210,4.405)--(3.214,4.052)--(3.218,3.721)%
  --(3.223,3.459)--(3.227,3.299)--(3.231,3.219)--(3.236,3.254)--(3.240,3.390)--(3.245,3.641)%
  --(3.249,3.938)--(3.253,4.268)--(3.258,4.633)--(3.262,4.964)--(3.266,5.249)--(3.271,5.442)%
  --(3.275,5.534)--(3.279,5.545)--(3.284,5.454)--(3.288,5.237)--(3.292,4.964)--(3.297,4.633)%
  --(3.301,4.302)--(3.305,3.960)--(3.310,3.653)--(3.314,3.436)--(3.318,3.322)--(3.323,3.276)%
  --(3.327,3.345)--(3.332,3.504)--(3.336,3.755)--(3.340,4.063)--(3.345,4.405)--(3.349,4.724)%
  --(3.353,5.055)--(3.358,5.283)--(3.362,5.420)--(3.366,5.499)--(3.371,5.477)--(3.375,5.328)%
  --(3.379,5.112)--(3.384,4.838)--(3.388,4.542)--(3.392,4.188)--(3.397,3.869)--(3.401,3.607)%
  --(3.405,3.436)--(3.410,3.333)--(3.414,3.333)--(3.419,3.425)--(3.423,3.618)--(3.427,3.869)%
  --(3.432,4.177)--(3.436,4.508)--(3.440,4.838)--(3.445,5.089)--(3.449,5.271)--(3.453,5.442)%
  --(3.458,5.465)--(3.462,5.385)--(3.466,5.214)--(3.471,4.998)--(3.475,4.724)--(3.479,4.405)%
  --(3.484,4.109)--(3.488,3.812)--(3.493,3.607)--(3.497,3.425)--(3.501,3.390)--(3.506,3.413)%
  --(3.510,3.527)--(3.514,3.732)--(3.519,3.995)--(3.523,4.291)--(3.527,4.610)--(3.532,4.884)%
  --(3.536,5.123)--(3.540,5.294)--(3.545,5.397)--(3.549,5.385)--(3.553,5.283)--(3.558,5.135)%
  --(3.562,4.907)--(3.566,4.599)--(3.571,4.325)--(3.575,4.017)--(3.580,3.778)--(3.584,3.561)%
  --(3.588,3.459)--(3.593,3.436)--(3.597,3.493)--(3.601,3.618)--(3.606,3.858)--(3.610,4.109)%
  --(3.614,4.394)--(3.619,4.679)--(3.623,4.941)--(3.627,5.146)--(3.632,5.283)--(3.636,5.340)%
  --(3.640,5.317)--(3.645,5.203)--(3.649,5.021)--(3.654,4.781)--(3.658,4.519)--(3.662,4.223)%
  --(3.667,3.949)--(3.671,3.732)--(3.675,3.573)--(3.680,3.482)--(3.684,3.482)--(3.688,3.573)%
  --(3.693,3.732)--(3.697,3.949)--(3.701,4.211)--(3.706,4.496)--(3.710,4.758)--(3.714,4.986)%
  --(3.719,5.169)--(3.723,5.271)--(3.727,5.306)--(3.732,5.249)--(3.736,5.123)--(3.741,4.918)%
  --(3.745,4.690)--(3.749,4.428)--(3.754,4.143)--(3.758,3.903)--(3.762,3.698)--(3.767,3.584)%
  --(3.771,3.516)--(3.775,3.539)--(3.780,3.664)--(3.784,3.812)--(3.788,4.074)--(3.793,4.314)%
  --(3.797,4.565)--(3.801,4.804)--(3.806,5.032)--(3.810,5.157)--(3.815,5.249)--(3.819,5.249)%
  --(3.823,5.180)--(3.828,5.032)--(3.832,4.815)--(3.836,4.587)--(3.841,4.325)--(3.845,4.086)%
  --(3.849,3.869)--(3.854,3.675)--(3.858,3.596)--(3.862,3.573)--(3.867,3.618)--(3.871,3.755)%
  --(3.875,3.926)--(3.880,4.166)--(3.884,4.405)--(3.888,4.644)--(3.893,4.872)--(3.897,5.032)%
  --(3.902,5.180)--(3.906,5.214)--(3.910,5.192)--(3.915,5.100)--(3.919,4.941)--(3.923,4.736)%
  --(3.928,4.496)--(3.932,4.245)--(3.936,4.029)--(3.941,3.835)--(3.945,3.675)--(3.949,3.618)%
  --(3.954,3.607)--(3.958,3.687)--(3.962,3.824)--(3.967,4.029)--(3.971,4.245)--(3.976,4.473)%
  --(3.980,4.713)--(3.984,4.895)--(3.989,5.066)--(3.993,5.157)--(3.997,5.180)--(4.002,5.123)%
  --(4.006,5.021)--(4.010,4.850)--(4.015,4.644)--(4.019,4.416)--(4.023,4.177)--(4.028,3.972)%
  --(4.032,3.801)--(4.036,3.687)--(4.041,3.641)--(4.045,3.664)--(4.049,3.767)--(4.054,3.915)%
  --(4.058,4.097)--(4.063,4.337)--(4.067,4.530)--(4.071,4.758)--(4.076,4.941)--(4.080,5.078)%
  --(4.084,5.135)--(4.089,5.135)--(4.093,5.055)--(4.097,4.964)--(4.102,4.770)--(4.106,4.542)%
  --(4.110,4.337)--(4.115,4.131)--(4.119,3.938)--(4.123,3.801)--(4.128,3.698)--(4.132,3.675)%
  --(4.137,3.732)--(4.141,3.824)--(4.145,3.983)--(4.150,4.177)--(4.154,4.394)--(4.158,4.610)%
  --(4.163,4.804)--(4.167,4.964)--(4.171,5.055)--(4.176,5.089)--(4.180,5.089)--(4.184,5.009)%
  --(4.189,4.861)--(4.193,4.679)--(4.197,4.485)--(4.202,4.280)--(4.206,4.074)--(4.210,3.892)%
  --(4.215,3.778)--(4.219,3.732)--(4.224,3.721)--(4.228,3.789)--(4.232,3.903)--(4.237,4.086)%
  --(4.241,4.257)--(4.245,4.462)--(4.250,4.667)--(4.254,4.838)--(4.258,4.964)--(4.263,5.055)%
  --(4.267,5.078)--(4.271,5.043)--(4.276,4.929)--(4.280,4.793)--(4.284,4.622)--(4.289,4.416)%
  --(4.293,4.200)--(4.298,4.029)--(4.302,3.892)--(4.306,3.789)--(4.311,3.755)--(4.315,3.755)%
  --(4.319,3.858)--(4.324,3.983)--(4.328,4.131)--(4.332,4.337)--(4.337,4.530)--(4.341,4.713)%
  --(4.345,4.850)--(4.350,4.975)--(4.354,5.032)--(4.358,5.021)--(4.363,4.964)--(4.367,4.861)%
  --(4.371,4.724)--(4.376,4.553)--(4.380,4.348)--(4.385,4.154)--(4.389,4.006)--(4.393,3.869)%
  --(4.398,3.812)--(4.402,3.778)--(4.406,3.812)--(4.411,3.881)--(4.415,4.040)--(4.419,4.211)%
  --(4.424,4.405)--(4.428,4.565)--(4.432,4.736)--(4.437,4.884)--(4.441,4.975)--(4.445,4.998)%
  --(4.450,4.998)--(4.454,4.929)--(4.458,4.804)--(4.463,4.656)--(4.467,4.473)--(4.472,4.302)%
  --(4.476,4.120)--(4.480,3.960)--(4.485,3.858)--(4.489,3.801)--(4.493,3.835)--(4.498,3.892)%
  --(4.502,3.972)--(4.506,4.120)--(4.511,4.291)--(4.515,4.451)--(4.519,4.622)--(4.524,4.781)%
  --(4.528,4.884)--(4.532,4.964)--(4.537,4.975)--(4.541,4.941)--(4.546,4.872)--(4.550,4.713)%
  --(4.554,4.576)--(4.559,4.405)--(4.563,4.223)--(4.567,4.063)--(4.572,3.960)--(4.576,3.881)%
  --(4.580,3.824)--(4.585,3.846)--(4.589,3.926)--(4.593,4.017)--(4.598,4.166)--(4.602,4.337)%
  --(4.606,4.519)--(4.611,4.656)--(4.615,4.804)--(4.619,4.895)--(4.624,4.952)--(4.628,4.941)%
  --(4.633,4.884)--(4.637,4.793)--(4.641,4.667)--(4.646,4.508)--(4.650,4.325)--(4.654,4.200)%
  --(4.659,4.052)--(4.663,3.938)--(4.667,3.858)--(4.672,3.869)--(4.676,3.892)--(4.680,3.972)%
  --(4.685,4.086)--(4.689,4.234)--(4.693,4.416)--(4.698,4.542)--(4.702,4.690)--(4.707,4.815)%
  --(4.711,4.895)--(4.715,4.929)--(4.720,4.918)--(4.724,4.838)--(4.728,4.747)--(4.733,4.599)%
  --(4.737,4.439)--(4.741,4.291)--(4.746,4.154)--(4.750,4.029)--(4.754,3.926)--(4.759,3.881)%
  --(4.763,3.881)--(4.767,3.926)--(4.772,4.017)--(4.776,4.143)--(4.780,4.291)--(4.785,4.439)%
  --(4.789,4.587)--(4.794,4.736)--(4.798,4.827)--(4.802,4.872)--(4.807,4.895)--(4.811,4.872)%
  --(4.815,4.793)--(4.820,4.679)--(4.824,4.530)--(4.828,4.382)--(4.833,4.257)--(4.837,4.120)%
  --(4.841,4.006)--(4.846,3.938)--(4.850,3.903)--(4.854,3.915)--(4.859,3.972)--(4.863,4.074)%
  --(4.868,4.200)--(4.872,4.348)--(4.876,4.508)--(4.881,4.656)--(4.885,4.736)--(4.889,4.827)%
  --(4.894,4.872)--(4.898,4.861)--(4.902,4.815)--(4.907,4.736)--(4.911,4.633)--(4.915,4.496)%
  --(4.920,4.371)--(4.924,4.211)--(4.928,4.086)--(4.933,3.983)--(4.937,3.926)--(4.941,3.915)%
  --(4.946,3.949)--(4.950,4.029)--(4.955,4.131)--(4.959,4.257)--(4.963,4.405)--(4.968,4.542)%
  --(4.972,4.667)--(4.976,4.758)--(4.981,4.850)--(4.985,4.861)--(4.989,4.827)--(4.994,4.770)%
  --(4.998,4.701)--(5.002,4.565)--(5.007,4.451)--(5.011,4.302)--(5.015,4.177)--(5.020,4.074)%
  --(5.024,3.983)--(5.029,3.949)--(5.033,3.960)--(5.037,4.017)--(5.042,4.074)--(5.046,4.188)%
  --(5.050,4.314)--(5.055,4.451)--(5.059,4.576)--(5.063,4.679)--(5.068,4.770)--(5.072,4.815)%
  --(5.076,4.838)--(5.081,4.804)--(5.085,4.747)--(5.089,4.656)--(5.094,4.519)--(5.098,4.405)%
  --(5.102,4.268)--(5.107,4.143)--(5.111,4.052)--(5.116,3.983)--(5.120,3.960)--(5.124,3.972)%
  --(5.129,4.029)--(5.133,4.131)--(5.137,4.223)--(5.142,4.348)--(5.146,4.485)--(5.150,4.599)%
  --(5.155,4.724)--(5.159,4.758)--(5.163,4.793)--(5.168,4.804)--(5.172,4.770)--(5.176,4.690)%
  --(5.181,4.587)--(5.185,4.485)--(5.190,4.359)--(5.194,4.223)--(5.198,4.131)--(5.203,4.052)%
  --(5.207,3.995)--(5.211,3.983)--(5.216,4.017)--(5.220,4.063)--(5.224,4.166)--(5.229,4.268)%
  --(5.233,4.394)--(5.237,4.542)--(5.242,4.622)--(5.246,4.701)--(5.250,4.758)--(5.255,4.781)%
  --(5.259,4.781)--(5.263,4.724)--(5.268,4.656)--(5.272,4.553)--(5.277,4.451)--(5.281,4.325)%
  --(5.285,4.188)--(5.290,4.109)--(5.294,4.040)--(5.298,4.006)--(5.303,4.029)--(5.307,4.063)%
  --(5.311,4.120)--(5.316,4.200)--(5.320,4.314)--(5.324,4.439)--(5.329,4.553)--(5.333,4.656)%
  --(5.337,4.724)--(5.342,4.758)--(5.346,4.770)--(5.351,4.747)--(5.355,4.679)--(5.359,4.610)%
  --(5.364,4.496)--(5.368,4.382)--(5.372,4.268)--(5.377,4.177)--(5.381,4.097)--(5.385,4.063)%
  --(5.390,4.017)--(5.394,4.029)--(5.398,4.074)--(5.403,4.166)--(5.407,4.245)--(5.411,4.359)%
  --(5.416,4.485)--(5.420,4.565)--(5.424,4.656)--(5.429,4.724)--(5.433,4.747)--(5.438,4.758)%
  --(5.442,4.724)--(5.446,4.656)--(5.451,4.565)--(5.455,4.485)--(5.459,4.348)--(5.464,4.257)%
  --(5.468,4.154)--(5.472,4.086)--(5.477,4.052)--(5.481,4.040)--(5.485,4.063)--(5.490,4.109)%
  --(5.494,4.200)--(5.498,4.302)--(5.503,4.394)--(5.507,4.496)--(5.512,4.599)--(5.516,4.667)%
  --(5.520,4.713)--(5.525,4.736)--(5.529,4.736)--(5.533,4.701)--(5.538,4.610)--(5.542,4.519)%
  --(5.546,4.439)--(5.551,4.314)--(5.555,4.211)--(5.559,4.143)--(5.564,4.086)--(5.568,4.052)%
  --(5.572,4.052)--(5.577,4.109)--(5.581,4.166)--(5.585,4.234)--(5.590,4.325)--(5.594,4.439)%
  --(5.599,4.519)--(5.603,4.610)--(5.607,4.679)--(5.612,4.713)--(5.616,4.736)--(5.620,4.713)%
  --(5.625,4.644)--(5.629,4.576)--(5.633,4.496)--(5.638,4.382)--(5.642,4.291)--(5.646,4.200)%
  --(5.651,4.131)--(5.655,4.097)--(5.659,4.063)--(5.664,4.074)--(5.668,4.131)--(5.672,4.200)%
  --(5.677,4.280)--(5.681,4.371)--(5.686,4.439)--(5.690,4.542)--(5.694,4.622)--(5.699,4.690)%
  --(5.703,4.724)--(5.707,4.713)--(5.712,4.679)--(5.716,4.610)--(5.720,4.530)--(5.725,4.451)%
  --(5.729,4.348)--(5.733,4.257)--(5.738,4.188)--(5.742,4.120)--(5.746,4.097)--(5.751,4.074)%
  --(5.755,4.120)--(5.760,4.143)--(5.764,4.223)--(5.768,4.314)--(5.773,4.394)--(5.777,4.496)%
  --(5.781,4.576)--(5.786,4.656)--(5.790,4.667)--(5.794,4.690)--(5.799,4.679)--(5.803,4.644)%
  --(5.807,4.576)--(5.812,4.496)--(5.816,4.416)--(5.820,4.325)--(5.825,4.234)--(5.829,4.166)%
  --(5.833,4.120)--(5.838,4.120)--(5.842,4.097)--(5.847,4.131)--(5.851,4.177)--(5.855,4.268)%
  --(5.860,4.337)--(5.864,4.428)--(5.868,4.508)--(5.873,4.587)--(5.877,4.656)--(5.881,4.690)%
  --(5.886,4.679)--(5.890,4.667)--(5.894,4.599)--(5.899,4.542)--(5.903,4.485)--(5.907,4.382)%
  --(5.912,4.291)--(5.916,4.223)--(5.921,4.166)--(5.925,4.154)--(5.929,4.109)--(5.934,4.120)%
  --(5.938,4.154)--(5.942,4.211)--(5.947,4.291)--(5.951,4.382)--(5.955,4.462)--(5.960,4.519)%
  --(5.964,4.599)--(5.968,4.656)--(5.973,4.667)--(5.977,4.656)--(5.981,4.633)--(5.986,4.587)%
  --(5.990,4.519)--(5.994,4.439)--(5.999,4.359)--(6.003,4.268)--(6.008,4.223)--(6.012,4.166)%
  --(6.016,4.131)--(6.021,4.131)--(6.025,4.143)--(6.029,4.188)--(6.034,4.257)--(6.038,4.325)%
  --(6.042,4.405)--(6.047,4.496)--(6.051,4.565)--(6.055,4.610)--(6.060,4.644)--(6.064,4.656)%
  --(6.068,4.644)--(6.073,4.610)--(6.077,4.553)--(6.082,4.485)--(6.086,4.416)--(6.090,4.348)%
  --(6.095,4.257)--(6.099,4.211)--(6.103,4.154)--(6.108,4.143)--(6.112,4.143)--(6.116,4.177)%
  --(6.121,4.211)--(6.125,4.291)--(6.129,4.348)--(6.134,4.428)--(6.138,4.508)--(6.142,4.576)%
  --(6.147,4.610)--(6.151,4.633)--(6.155,4.633)--(6.160,4.633)--(6.164,4.576)--(6.169,4.530)%
  --(6.173,4.451)--(6.177,4.382)--(6.182,4.314)--(6.186,4.245)--(6.190,4.188)--(6.195,4.177)%
  --(6.199,4.143)--(6.203,4.166)--(6.208,4.188)--(6.212,4.257)--(6.216,4.302)--(6.221,4.382)%
  --(6.225,4.462)--(6.229,4.530)--(6.234,4.576)--(6.238,4.610)--(6.243,4.644)--(6.247,4.633)%
  --(6.251,4.599)--(6.256,4.565)--(6.260,4.496)--(6.264,4.439)--(6.269,4.359)--(6.273,4.291)%
  --(6.277,4.234)--(6.282,4.188)--(6.286,4.166)--(6.290,4.166)--(6.295,4.177)--(6.299,4.223)%
  --(6.303,4.268)--(6.308,4.337)--(6.312,4.428)--(6.316,4.485)--(6.321,4.530)--(6.325,4.576)%
  --(6.330,4.622)--(6.334,4.622)--(6.338,4.610)--(6.343,4.576)--(6.347,4.530)--(6.351,4.473)%
  --(6.356,4.405)--(6.360,4.337)--(6.364,4.280)--(6.369,4.234)--(6.373,4.188)--(6.377,4.177)%
  --(6.382,4.177)--(6.386,4.200)--(6.390,4.245)--(6.395,4.302)--(6.399,4.371)--(6.404,4.439)%
  --(6.408,4.496)--(6.412,4.542)--(6.417,4.587)--(6.421,4.610)--(6.425,4.610)--(6.430,4.576)%
  --(6.434,4.553)--(6.438,4.508)--(6.443,4.451)--(6.447,4.382)--(6.451,4.325)--(6.456,4.268)%
  --(6.460,4.211)--(6.464,4.200)--(6.469,4.188)--(6.473,4.200)--(6.477,4.211)--(6.482,4.268)%
  --(6.486,4.337)--(6.491,4.394)--(6.495,4.451)--(6.499,4.530)--(6.504,4.542)--(6.508,4.576)%
  --(6.512,4.622)--(6.517,4.599)--(6.521,4.576)--(6.525,4.530)--(6.530,4.473)--(6.534,4.439)%
  --(6.538,4.371)--(6.543,4.302)--(6.547,4.245)--(6.551,4.234)--(6.556,4.188)--(6.560,4.200)%
  --(6.565,4.211)--(6.569,4.245)--(6.573,4.302)--(6.578,4.337)--(6.582,4.416)--(6.586,4.462)%
  --(6.591,4.519)--(6.595,4.553)--(6.599,4.587)--(6.604,4.587)--(6.608,4.587)--(6.612,4.553)%
  --(6.617,4.519)--(6.621,4.462)--(6.625,4.394)--(6.630,4.337)--(6.634,4.302)--(6.638,4.245)%
  --(6.643,4.211)--(6.647,4.200)--(6.652,4.211)--(6.656,4.234)--(6.660,4.257)--(6.665,4.314)%
  --(6.669,4.359)--(6.673,4.428)--(6.678,4.485)--(6.682,4.542)--(6.686,4.553)--(6.691,4.587)%
  --(6.695,4.587)--(6.699,4.565)--(6.704,4.542)--(6.708,4.496)--(6.712,4.451)--(6.717,4.382)%
  --(6.721,4.337)--(6.726,4.280)--(6.730,4.234)--(6.734,4.223)--(6.739,4.211)--(6.743,4.223)%
  --(6.747,4.245)--(6.752,4.268)--(6.756,4.337)--(6.760,4.382)--(6.765,4.439)--(6.769,4.496)%
  --(6.773,4.519)--(6.778,4.553)--(6.782,4.576)--(6.786,4.565)--(6.791,4.565)--(6.795,4.519)%
  --(6.799,4.462)--(6.804,4.428)--(6.808,4.382)--(6.813,4.325)--(6.817,4.268)--(6.821,4.245)%
  --(6.826,4.234)--(6.830,4.223)--(6.834,4.257)--(6.839,4.291)--(6.843,4.302)--(6.847,4.359)%
  --(6.852,4.394)--(6.856,4.462)--(6.860,4.496)--(6.865,4.530)--(6.869,4.553)--(6.873,4.565)%
  --(6.878,4.576)--(6.882,4.530)--(6.886,4.508)--(6.891,4.451)--(6.895,4.405)--(6.900,4.337)%
  --(6.904,4.302)--(6.908,4.268)--(6.913,4.234)--(6.917,4.223)--(6.921,4.245)--(6.926,4.268)%
  --(6.930,4.280)--(6.934,4.325)--(6.939,4.371)--(6.943,4.416)--(6.947,4.485)--(6.952,4.496)%
  --(6.956,4.542)--(6.960,4.542)--(6.965,4.565)--(6.969,4.542)--(6.974,4.519)--(6.978,4.485)%
  --(6.982,4.428)--(6.987,4.382)--(6.991,4.337)--(6.995,4.280)--(7.000,4.268)--(7.004,4.234)%
  --(7.008,4.245)--(7.013,4.257)--(7.017,4.268)--(7.021,4.302)--(7.026,4.337)--(7.030,4.382)%
  --(7.034,4.428)--(7.039,4.485)--(7.043,4.508)--(7.047,4.542)--(7.052,4.553)--(7.056,4.542)%
  --(7.061,4.530)--(7.065,4.508)--(7.069,4.451)--(7.074,4.416)--(7.078,4.382)--(7.082,4.325)%
  --(7.087,4.291)--(7.091,4.257)--(7.095,4.257)--(7.100,4.234)--(7.104,4.268)--(7.108,4.291)%
  --(7.113,4.337)--(7.117,4.371)--(7.121,4.405)--(7.126,4.451)--(7.130,4.485)--(7.135,4.519)%
  --(7.139,4.530)--(7.143,4.542)--(7.148,4.530)--(7.152,4.519)--(7.156,4.485)--(7.161,4.439)%
  --(7.165,4.394)--(7.169,4.359)--(7.174,4.325)--(7.178,4.280)--(7.182,4.280)--(7.187,4.268)%
  --(7.191,4.257)--(7.195,4.268)--(7.200,4.302)--(7.204,4.337)--(7.208,4.371)--(7.213,4.416)%
  --(7.217,4.462)--(7.222,4.496)--(7.226,4.519)--(7.230,4.553)--(7.235,4.530)--(7.239,4.530)%
  --(7.243,4.508)--(7.248,4.462)--(7.252,4.428)--(7.256,4.394)--(7.261,4.337)--(7.265,4.302)%
  --(7.269,4.291)--(7.274,4.268)--(7.278,4.280)--(7.282,4.257)--(7.287,4.291)--(7.291,4.314)%
  --(7.296,4.359)--(7.300,4.394)--(7.304,4.428)--(7.309,4.462)--(7.313,4.508)--(7.317,4.519)%
  --(7.322,4.530)--(7.326,4.530)--(7.330,4.508)--(7.335,4.473)--(7.339,4.439)--(7.343,4.405)%
  --(7.348,4.382)--(7.352,4.325)--(7.356,4.302)--(7.361,4.291)--(7.365,4.268)--(7.369,4.257)%
  --(7.374,4.268)--(7.378,4.302)--(7.383,4.325)--(7.387,4.359)--(7.391,4.394)--(7.396,4.439)%
  --(7.400,4.485)--(7.404,4.496)--(7.409,4.508)--(7.413,4.519)--(7.417,4.519)--(7.422,4.508)%
  --(7.426,4.473)--(7.430,4.439)--(7.435,4.382)--(7.439,4.359)--(7.443,4.337)--(7.448,4.291)%
  --(7.452,4.268)--(7.457,4.291)--(7.461,4.268)--(7.465,4.280)--(7.470,4.314)--(7.474,4.337)%
  --(7.478,4.394)--(7.483,4.416)--(7.487,4.451)--(7.491,4.473)--(7.496,4.508)--(7.500,4.508)%
  --(7.504,4.508)--(7.509,4.508)--(7.513,4.485)--(7.517,4.462)--(7.522,4.428)--(7.526,4.382)%
  --(7.530,4.337)--(7.535,4.314)--(7.539,4.314)--(7.544,4.291)--(7.548,4.268)--(7.552,4.280)%
  --(7.557,4.291)--(7.561,4.337)--(7.565,4.348)--(7.570,4.382)--(7.574,4.428)--(7.578,4.462)%
  --(7.583,4.485)--(7.587,4.508)--(7.591,4.508)--(7.596,4.519)--(7.600,4.496)--(7.604,4.462)%
  --(7.609,4.439)--(7.613,4.405)--(7.618,4.382)--(7.622,4.337)--(7.626,4.314)--(7.631,4.291)%
  --(7.635,4.280)--(7.639,4.280)--(7.644,4.291)--(7.648,4.302)--(7.652,4.337)--(7.657,4.359)%
  --(7.661,4.405)--(7.665,4.439)--(7.670,4.473)--(7.674,4.485)--(7.678,4.508)--(7.683,4.508)%
  --(7.687,4.485)--(7.691,4.485)--(7.696,4.451)--(7.700,4.428)--(7.705,4.405)--(7.709,4.371)%
  --(7.713,4.325)--(7.718,4.302)--(7.722,4.291)--(7.726,4.291)--(7.731,4.291)--(7.735,4.302)%
  --(7.739,4.314)--(7.744,4.348)--(7.748,4.382)--(7.752,4.405)--(7.757,4.451)--(7.761,4.473)%
  --(7.765,4.485)--(7.770,4.485)--(7.774,4.496)--(7.779,4.496)--(7.783,4.485)--(7.787,4.451)%
  --(7.792,4.428)--(7.796,4.394)--(7.800,4.359)--(7.805,4.325)--(7.809,4.314)--(7.813,4.291)%
  --(7.818,4.291)--(7.822,4.291)--(7.826,4.325)--(7.831,4.337)--(7.835,4.371)--(7.839,4.394)%
  --(7.844,4.428)--(7.848,4.451)--(7.852,4.473)--(7.857,4.485)--(7.861,4.496)--(7.866,4.496)%
  --(7.870,4.496)--(7.874,4.462)--(7.879,4.462)--(7.883,4.405)--(7.887,4.371)--(7.892,4.359)%
  --(7.896,4.325)--(7.900,4.314)--(7.905,4.291)--(7.909,4.302)--(7.913,4.302)--(7.918,4.325)%
  --(7.922,4.348)--(7.926,4.371)--(7.931,4.416)--(7.935,4.428)--(7.939,4.462)--(7.944,4.473)%
  --(7.948,4.485)--(7.953,4.485)--(7.957,4.485)--(7.961,4.473)--(7.966,4.462)--(7.970,4.428)%
  --(7.974,4.394)--(7.979,4.371)--(7.983,4.337)--(7.987,4.314)--(7.992,4.302)--(7.996,4.302)%
  --(8.000,4.302)--(8.005,4.314)--(8.009,4.337)--(8.013,4.359)--(8.018,4.382)--(8.022,4.416)%
  --(8.027,4.439)--(8.031,4.462)--(8.035,4.473)--(8.040,4.496)--(8.044,4.485)--(8.048,4.485)%
  --(8.053,4.462)--(8.057,4.451)--(8.061,4.405)--(8.066,4.382)--(8.070,4.359)--(8.074,4.325)%
  --(8.079,4.325)--(8.083,4.302)--(8.087,4.302)--(8.092,4.302)--(8.096,4.325)--(8.100,4.337)%
  --(8.105,4.359)--(8.109,4.394)--(8.114,4.428)--(8.118,4.439)--(8.122,4.462)--(8.127,4.473)%
  --(8.131,4.485)--(8.135,4.485)--(8.140,4.462)--(8.144,4.451)--(8.148,4.416)--(8.153,4.405)%
  --(8.157,4.382)--(8.161,4.348)--(8.166,4.337)--(8.170,4.314)--(8.174,4.314)--(8.179,4.314)%
  --(8.183,4.314)--(8.188,4.325)--(8.192,4.348)--(8.196,4.382)--(8.201,4.405)--(8.205,4.428)%
  --(8.209,4.451)--(8.214,4.462)--(8.218,4.473)--(8.222,4.473)--(8.227,4.496)--(8.231,4.462)%
  --(8.235,4.439)--(8.240,4.416)--(8.244,4.394)--(8.248,4.371)--(8.253,4.348)--(8.257,4.325)%
  --(8.261,4.314)--(8.266,4.291)--(8.270,4.314)--(8.275,4.337)--(8.279,4.337)--(8.283,4.359)%
  --(8.288,4.382)--(8.292,4.416)--(8.296,4.428)--(8.301,4.451)--(8.305,4.451)--(8.309,4.473)%
  --(8.314,4.485)--(8.318,4.473)--(8.322,4.462)--(8.327,4.428)--(8.331,4.405)--(8.335,4.382)%
  --(8.340,4.371)--(8.344,4.337)--(8.349,4.337)--(8.353,4.302)--(8.357,4.314)--(8.362,4.314)%
  --(8.366,4.325)--(8.370,4.348)--(8.375,4.359)--(8.379,4.394)--(8.383,4.416)--(8.388,4.439)%
  --(8.392,4.451)--(8.396,4.462)--(8.401,4.485)--(8.405,4.473)--(8.409,4.439)--(8.414,4.428)%
  --(8.418,4.428)--(8.422,4.394)--(8.427,4.371)--(8.431,4.359)--(8.436,4.337)--(8.440,4.314)%
  --(8.444,4.325)--(8.449,4.325)--(8.453,4.337)--(8.457,4.337)--(8.462,4.359)--(8.466,4.371)%
  --(8.470,4.405)--(8.475,4.428)--(8.479,4.451)--(8.483,4.451)--(8.488,4.462)--(8.492,4.485)%
  --(8.496,4.473)--(8.501,4.451)--(8.505,4.428)--(8.510,4.416)--(8.514,4.394)--(8.518,4.382)%
  --(8.523,4.348)--(8.527,4.325)--(8.531,4.325)--(8.536,4.302)--(8.540,4.325)--(8.544,4.325)%
  --(8.549,4.348)--(8.553,4.371)--(8.557,4.382)--(8.562,4.405)--(8.566,4.428)--(8.570,4.439)%
  --(8.575,4.462)--(8.579,4.462)--(8.583,4.451)--(8.588,4.462)--(8.592,4.451)--(8.597,4.416)%
  --(8.601,4.405)--(8.605,4.394)--(8.610,4.371)--(8.614,4.348)--(8.618,4.325)--(8.623,4.337)%
  --(8.627,4.325)--(8.631,4.325)--(8.636,4.337)--(8.640,4.359)--(8.644,4.371)--(8.649,4.394)%
  --(8.653,4.416)--(8.657,4.428)--(8.662,4.462)--(8.666,4.462)--(8.671,4.462)--(8.675,4.462)%
  --(8.679,4.451)--(8.684,4.428)--(8.688,4.416)--(8.692,4.416)--(8.697,4.382)--(8.701,4.359)%
  --(8.705,4.348)--(8.710,4.337)--(8.714,4.325)--(8.718,4.337)--(8.723,4.325)--(8.727,4.337)%
  --(8.731,4.371)--(8.736,4.382)--(8.740,4.405)--(8.744,4.428)--(8.749,4.451)--(8.753,4.462)%
  --(8.758,4.462)--(8.762,4.462)--(8.766,4.439)--(8.771,4.439)--(8.775,4.428)--(8.779,4.416)%
  --(8.784,4.382)--(8.788,4.382)--(8.792,4.359)--(8.797,4.348)--(8.801,4.348)--(8.805,4.337)%
  --(8.810,4.325)--(8.814,4.337)--(8.818,4.348)--(8.823,4.371)--(8.827,4.394)--(8.832,4.405)%
  --(8.836,4.416)--(8.840,4.439)--(8.845,4.451)--(8.849,4.473)--(8.853,4.451)--(8.858,4.451)%
  --(8.862,4.439)--(8.866,4.428)--(8.871,4.405)--(8.875,4.382)--(8.879,4.359)--(8.884,4.359)%
  --(8.888,4.337)--(8.892,4.337)--(8.897,4.348)--(8.901,4.325)--(8.905,4.348)--(8.910,4.359)%
  --(8.914,4.371)--(8.919,4.394)--(8.923,4.416)--(8.927,4.439)--(8.932,4.451)--(8.936,4.451)%
  --(8.940,4.451)--(8.945,4.451)--(8.949,4.439)--(8.953,4.428)--(8.958,4.405)--(8.962,4.394)%
  --(8.966,4.382)--(8.971,4.371)--(8.975,4.348)--(8.979,4.337)--(8.984,4.337)--(8.988,4.337)%
  --(8.993,4.337)--(8.997,4.359)--(9.001,4.359)--(9.006,4.382)--(9.010,4.405)--(9.014,4.416)%
  --(9.019,4.428)--(9.023,4.439)--(9.027,4.451)--(9.032,4.439)--(9.036,4.428)--(9.040,4.439)%
  --(9.045,4.405)--(9.049,4.416)--(9.053,4.394)--(9.058,4.371)--(9.062,4.359)--(9.066,4.337)%
  --(9.071,4.337)--(9.075,4.348)--(9.080,4.337)--(9.084,4.348)--(9.088,4.359)--(9.093,4.371)%
  --(9.097,4.394)--(9.101,4.416)--(9.106,4.405)--(9.110,4.439)--(9.114,4.439)--(9.119,4.451)%
  --(9.123,4.451)--(9.127,4.439)--(9.132,4.428)--(9.136,4.416)--(9.140,4.405)--(9.145,4.394)%
  --(9.149,4.359)--(9.153,4.371)--(9.158,4.348)--(9.162,4.348)--(9.167,4.337)--(9.171,4.348)%
  --(9.175,4.359)--(9.180,4.359)--(9.184,4.394)--(9.188,4.405)--(9.193,4.405)--(9.197,4.428)%
  --(9.201,4.439)--(9.206,4.439)--(9.210,4.439)--(9.214,4.439)--(9.219,4.439)--(9.223,4.428)%
  --(9.227,4.405)--(9.232,4.405)--(9.236,4.394)--(9.241,4.359)--(9.245,4.348)--(9.249,4.348)%
  --(9.254,4.337)--(9.258,4.337)--(9.262,4.348)--(9.267,4.359)--(9.271,4.382)--(9.275,4.394)%
  --(9.280,4.394)--(9.284,4.405)--(9.288,4.428)--(9.293,4.439)--(9.297,4.439)--(9.301,4.439)%
  --(9.306,4.439)--(9.310,4.416)--(9.314,4.416)--(9.319,4.405)--(9.323,4.394)--(9.328,4.382)%
  --(9.332,4.371)--(9.336,4.348)--(9.341,4.348)--(9.345,4.348)--(9.349,4.348)--(9.354,4.348)%
  --(9.358,4.359)--(9.362,4.382)--(9.367,4.394)--(9.371,4.405)--(9.375,4.428)--(9.380,4.416)%
  --(9.384,4.439)--(9.388,4.439)--(9.393,4.428)--(9.397,4.428)--(9.402,4.428)--(9.406,4.428)%
  --(9.410,4.394)--(9.415,4.394)--(9.419,4.371)--(9.423,4.359)--(9.428,4.348)--(9.432,4.348)%
  --(9.436,4.348)--(9.441,4.348)--(9.445,4.348)--(9.449,4.371)--(9.454,4.382)--(9.458,4.394)%
  --(9.462,4.416)--(9.467,4.428)--(9.471,4.428)--(9.475,4.439)--(9.480,4.439)--(9.484,4.439)%
  --(9.489,4.428)--(9.493,4.428)--(9.497,4.394)--(9.502,4.405)--(9.506,4.394)--(9.510,4.371)%
  --(9.515,4.359)--(9.519,4.348)--(9.523,4.348)--(9.528,4.348)--(9.532,4.371)--(9.536,4.382)%
  --(9.541,4.382)--(9.545,4.394)--(9.549,4.405)--(9.554,4.416)--(9.558,4.416)--(9.563,4.439)%
  --(9.567,4.439)--(9.571,4.428)--(9.576,4.428)--(9.580,4.416)--(9.584,4.416)--(9.589,4.405)%
  --(9.593,4.394)--(9.597,4.382)--(9.602,4.359)--(9.606,4.359)--(9.610,4.359)--(9.615,4.348)%
  --(9.619,4.348)--(9.623,4.371)--(9.628,4.371)--(9.632,4.382)--(9.636,4.394)--(9.641,4.394)%
  --(9.645,4.416)--(9.650,4.428)--(9.654,4.428)--(9.658,4.428)--(9.663,4.439)--(9.667,4.428)%
  --(9.671,4.416)--(9.676,4.405)--(9.680,4.405)--(9.684,4.394)--(9.689,4.371)--(9.693,4.371)%
  --(9.697,4.348)--(9.702,4.359)--(9.706,4.348)--(9.710,4.359)--(9.715,4.371)--(9.719,4.371)%
  --(9.724,4.394)--(9.728,4.405)--(9.732,4.405)--(9.737,4.428)--(9.741,4.428)--(9.745,4.428)%
  --(9.750,4.428)--(9.754,4.428)--(9.758,4.428)--(9.763,4.416)--(9.767,4.405)--(9.771,4.394)%
  --(9.776,4.382)--(9.780,4.382)--(9.784,4.359)--(9.789,4.359)--(9.793,4.359)--(9.797,4.382)%
  --(9.802,4.371)--(9.806,4.371)--(9.811,4.382)--(9.815,4.382)--(9.819,4.405)--(9.824,4.416)%
  --(9.828,4.416)--(9.832,4.416)--(9.837,4.439)--(9.841,4.428)--(9.845,4.439)--(9.850,4.428)%
  --(9.854,4.416)--(9.858,4.394)--(9.863,4.394)--(9.867,4.394)--(9.871,4.371)--(9.876,4.359)%
  --(9.880,4.371)--(9.885,4.371)--(9.889,4.359)--(9.893,4.359)--(9.898,4.371)--(9.902,4.371)%
  --(9.906,4.394)--(9.911,4.405)--(9.915,4.416)--(9.919,4.416)--(9.924,4.428)--(9.928,4.439)%
  --(9.932,4.428)--(9.937,4.428)--(9.941,4.416)--(9.945,4.405)--(9.950,4.405)--(9.954,4.394)%
  --(9.958,4.371)--(9.963,4.371)--(9.967,4.382)--(9.972,4.359)--(9.976,4.371)--(9.980,4.359)%
  --(9.985,4.371)--(9.989,4.382)--(9.993,4.394)--(9.998,4.394)--(10.002,4.416)--(10.006,4.416)%
  --(10.011,4.428)--(10.015,4.428)--(10.019,4.428)--(10.024,4.428)--(10.028,4.416)--(10.032,4.416)%
  --(10.037,4.405)--(10.041,4.394)--(10.046,4.394)--(10.050,4.371)--(10.054,4.382)--(10.059,4.359)%
  --(10.063,4.371)--(10.067,4.359)--(10.072,4.371)--(10.076,4.382)--(10.080,4.382)--(10.085,4.394)%
  --(10.089,4.405)--(10.093,4.405)--(10.098,4.416)--(10.102,4.428)--(10.106,4.428)--(10.111,4.439)%
  --(10.115,4.416)--(10.119,4.416)--(10.124,4.416)--(10.128,4.394)--(10.133,4.394)--(10.137,4.382)%
  --(10.141,4.382)--(10.146,4.371)--(10.150,4.371)--(10.154,4.359)--(10.159,4.359)--(10.163,4.371)%
  --(10.167,4.382)--(10.172,4.382)--(10.176,4.394)--(10.180,4.405)--(10.185,4.416)--(10.189,4.416)%
  --(10.193,4.416)--(10.198,4.416)--(10.202,4.428)--(10.207,4.416)--(10.211,4.416)--(10.215,4.405)%
  --(10.220,4.405);
\gpcolor{color=gp lt color border}
\node[gp node right] at (10.479,7.491) {Massimi e minimi};
\gpcolor{rgb color={1.000,0.000,0.000}}
\gpsetpointsize{4.00}
\gppoint{gp mark 3}{(1.572,7.473)}
\gppoint{gp mark 3}{(1.662,7.336)}
\gppoint{gp mark 3}{(1.752,7.175)}
\gppoint{gp mark 3}{(1.842,7.020)}
\gppoint{gp mark 3}{(1.932,6.889)}
\gppoint{gp mark 3}{(2.022,6.765)}
\gppoint{gp mark 3}{(2.112,6.659)}
\gppoint{gp mark 3}{(2.202,6.537)}
\gppoint{gp mark 3}{(2.292,6.412)}
\gppoint{gp mark 3}{(2.382,6.334)}
\gppoint{gp mark 3}{(2.472,6.239)}
\gppoint{gp mark 3}{(2.562,6.150)}
\gppoint{gp mark 3}{(2.651,6.066)}
\gppoint{gp mark 3}{(2.742,5.969)}
\gppoint{gp mark 3}{(2.832,5.888)}
\gppoint{gp mark 3}{(2.922,5.819)}
\gppoint{gp mark 3}{(3.012,5.775)}
\gppoint{gp mark 3}{(3.101,5.705)}
\gppoint{gp mark 3}{(3.192,5.626)}
\gppoint{gp mark 3}{(3.282,5.553)}
\gppoint{gp mark 3}{(3.372,5.503)}
\gppoint{gp mark 3}{(3.461,5.469)}
\gppoint{gp mark 3}{(3.551,5.406)}
\gppoint{gp mark 3}{(3.641,5.342)}
\gppoint{gp mark 3}{(3.731,5.306)}
\gppoint{gp mark 3}{(3.821,5.260)}
\gppoint{gp mark 3}{(3.911,5.215)}
\gppoint{gp mark 3}{(4.001,5.182)}
\gppoint{gp mark 3}{(4.091,5.142)}
\gppoint{gp mark 3}{(4.182,5.093)}
\gppoint{gp mark 3}{(4.271,5.078)}
\gppoint{gp mark 3}{(4.360,5.036)}
\gppoint{gp mark 3}{(4.452,5.001)}
\gppoint{gp mark 3}{(4.540,4.976)}
\gppoint{gp mark 3}{(4.630,4.956)}
\gppoint{gp mark 3}{(4.721,4.931)}
\gppoint{gp mark 3}{(4.811,4.895)}
\gppoint{gp mark 3}{(4.899,4.875)}
\gppoint{gp mark 3}{(4.988,4.862)}
\gppoint{gp mark 3}{(5.080,4.838)}
\gppoint{gp mark 3}{(5.171,4.805)}
\gppoint{gp mark 3}{(5.261,4.784)}
\gppoint{gp mark 3}{(5.350,4.770)}
\gppoint{gp mark 3}{(5.441,4.760)}
\gppoint{gp mark 3}{(5.531,4.738)}
\gppoint{gp mark 3}{(5.620,4.736)}
\gppoint{gp mark 3}{(5.708,4.726)}
\gppoint{gp mark 3}{(5.799,4.690)}
\gppoint{gp mark 3}{(5.887,4.691)}
\gppoint{gp mark 3}{(5.977,4.667)}
\gppoint{gp mark 3}{(6.068,4.656)}
\gppoint{gp mark 3}{(6.158,4.636)}
\gppoint{gp mark 3}{(6.248,4.646)}
\gppoint{gp mark 3}{(6.336,4.627)}
\gppoint{gp mark 3}{(6.427,4.613)}
\gppoint{gp mark 3}{(6.517,4.623)}
\gppoint{gp mark 3}{(6.606,4.592)}
\gppoint{gp mark 3}{(6.697,4.592)}
\gppoint{gp mark 3}{(6.787,4.576)}
\gppoint{gp mark 3}{(6.881,4.579)}
\gppoint{gp mark 3}{(6.969,4.565)}
\gppoint{gp mark 3}{(7.056,4.553)}
\gppoint{gp mark 3}{(7.148,4.542)}
\gppoint{gp mark 3}{(7.235,4.553)}
\gppoint{gp mark 3}{(7.328,4.532)}
\gppoint{gp mark 3}{(7.420,4.520)}
\gppoint{gp mark 3}{(7.502,4.512)}
\gppoint{gp mark 3}{(7.599,4.519)}
\gppoint{gp mark 3}{(7.685,4.510)}
\gppoint{gp mark 3}{(7.781,4.498)}
\gppoint{gp mark 3}{(7.868,4.498)}
\gppoint{gp mark 3}{(7.955,4.486)}
\gppoint{gp mark 3}{(8.045,4.497)}
\gppoint{gp mark 3}{(8.137,4.486)}
\gppoint{gp mark 3}{(8.231,4.496)}
\gppoint{gp mark 3}{(8.318,4.485)}
\gppoint{gp mark 3}{(8.406,4.485)}
\gppoint{gp mark 3}{(8.497,4.485)}
\gppoint{gp mark 3}{(8.581,4.465)}
\gppoint{gp mark 3}{(8.668,4.466)}
\gppoint{gp mark 3}{(8.760,4.463)}
\gppoint{gp mark 3}{(8.853,4.473)}
\gppoint{gp mark 3}{(8.938,4.452)}
\gppoint{gp mark 3}{(9.032,4.451)}
\gppoint{gp mark 3}{(9.125,4.452)}
\gppoint{gp mark 3}{(9.208,4.441)}
\gppoint{gp mark 3}{(9.299,4.441)}
\gppoint{gp mark 3}{(9.391,4.442)}
\gppoint{gp mark 3}{(9.482,4.441)}
\gppoint{gp mark 3}{(9.569,4.442)}
\gppoint{gp mark 3}{(9.667,4.439)}
\gppoint{gp mark 3}{(9.743,4.431)}
\gppoint{gp mark 3}{(9.842,4.440)}
\gppoint{gp mark 3}{(9.932,4.439)}
\gppoint{gp mark 3}{(10.017,4.429)}
\gppoint{gp mark 3}{(10.114,4.440)}
\gppoint{gp mark 3}{(10.207,4.428)}
\gppoint{gp mark 3}{(1.527,1.218)}
\gppoint{gp mark 3}{(1.617,1.405)}
\gppoint{gp mark 3}{(1.707,1.537)}
\gppoint{gp mark 3}{(1.797,1.689)}
\gppoint{gp mark 3}{(1.887,1.851)}
\gppoint{gp mark 3}{(1.977,1.967)}
\gppoint{gp mark 3}{(2.067,2.088)}
\gppoint{gp mark 3}{(2.157,2.216)}
\gppoint{gp mark 3}{(2.247,2.321)}
\gppoint{gp mark 3}{(2.336,2.434)}
\gppoint{gp mark 3}{(2.427,2.523)}
\gppoint{gp mark 3}{(2.516,2.618)}
\gppoint{gp mark 3}{(2.607,2.716)}
\gppoint{gp mark 3}{(2.697,2.785)}
\gppoint{gp mark 3}{(2.787,2.875)}
\gppoint{gp mark 3}{(2.876,2.947)}
\gppoint{gp mark 3}{(2.967,3.023)}
\gppoint{gp mark 3}{(3.057,3.093)}
\gppoint{gp mark 3}{(3.147,3.160)}
\gppoint{gp mark 3}{(3.237,3.217)}
\gppoint{gp mark 3}{(3.327,3.276)}
\gppoint{gp mark 3}{(3.416,3.321)}
\gppoint{gp mark 3}{(3.506,3.390)}
\gppoint{gp mark 3}{(3.596,3.434)}
\gppoint{gp mark 3}{(3.686,3.470)}
\gppoint{gp mark 3}{(3.776,3.513)}
\gppoint{gp mark 3}{(3.866,3.572)}
\gppoint{gp mark 3}{(3.956,3.601)}
\gppoint{gp mark 3}{(4.046,3.640)}
\gppoint{gp mark 3}{(4.136,3.674)}
\gppoint{gp mark 3}{(4.226,3.716)}
\gppoint{gp mark 3}{(4.317,3.751)}
\gppoint{gp mark 3}{(4.406,3.778)}
\gppoint{gp mark 3}{(4.494,3.800)}
\gppoint{gp mark 3}{(4.586,3.822)}
\gppoint{gp mark 3}{(4.673,3.851)}
\gppoint{gp mark 3}{(4.765,3.875)}
\gppoint{gp mark 3}{(4.856,3.902)}
\gppoint{gp mark 3}{(4.945,3.913)}
\gppoint{gp mark 3}{(5.034,3.948)}
\gppoint{gp mark 3}{(5.125,3.960)}
\gppoint{gp mark 3}{(5.215,3.982)}
\gppoint{gp mark 3}{(5.303,4.006)}
\gppoint{gp mark 3}{(5.395,4.015)}
\gppoint{gp mark 3}{(5.485,4.040)}
\gppoint{gp mark 3}{(5.575,4.047)}
\gppoint{gp mark 3}{(5.665,4.062)}
\gppoint{gp mark 3}{(5.754,4.073)}
\gppoint{gp mark 3}{(5.846,4.097)}
\gppoint{gp mark 3}{(5.935,4.106)}
\gppoint{gp mark 3}{(6.023,4.127)}
\gppoint{gp mark 3}{(6.114,4.141)}
\gppoint{gp mark 3}{(6.204,4.143)}
\gppoint{gp mark 3}{(6.293,4.163)}
\gppoint{gp mark 3}{(6.384,4.176)}
\gppoint{gp mark 3}{(6.473,4.188)}
\gppoint{gp mark 3}{(6.561,4.186)}
\gppoint{gp mark 3}{(6.652,4.200)}
\gppoint{gp mark 3}{(6.833,4.221)}
\gppoint{gp mark 3}{(6.921,4.222)}
\gppoint{gp mark 3}{(7.009,4.233)}
\gppoint{gp mark 3}{(7.104,4.234)}
\gppoint{gp mark 3}{(7.286,4.257)}
\gppoint{gp mark 3}{(7.374,4.257)}
\gppoint{gp mark 3}{(7.466,4.268)}
\gppoint{gp mark 3}{(7.553,4.268)}
\gppoint{gp mark 3}{(7.728,4.290)}
\gppoint{gp mark 3}{(7.910,4.291)}
\gppoint{gp mark 3}{(8.090,4.300)}
\gppoint{gp mark 3}{(8.177,4.311)}
\gppoint{gp mark 3}{(8.270,4.291)}
\gppoint{gp mark 3}{(8.445,4.313)}
\gppoint{gp mark 3}{(8.718,4.325)}
\gppoint{gp mark 3}{(8.905,4.325)}
\gppoint{gp mark 3}{(9.073,4.334)}
\gppoint{gp mark 3}{(9.343,4.345)}
\gppoint{gp mark 3}{(9.702,4.348)}
\gppoint{gp mark 3}{(9.791,4.357)}
\gppoint{gp mark 3}{(9.977,4.359)}
\gppoint{gp mark 3}{(10.064,4.359)}
\gppoint{gp mark 3}{(10.222,4.404)}
\gppoint{gp mark 3}{(11.121,7.491)}
\gpcolor{color=gp lt color border}
\gpsetlinetype{gp lt border}
\draw[gp path] (1.504,7.473)--(1.504,1.218);
\draw[gp path] (1.504,0.985)--(10.222,0.985);
%% coordinates of the plot area
\gpdefrectangularnode{gp plot 1}{\pgfpoint{1.504cm}{0.985cm}}{\pgfpoint{11.947cm}{7.825cm}}
\end{tikzpicture}
%% gnuplot variables
 \caption{Grafico 0.968dgdecad.tex} \label{gr:0.968dgdecad.tex} \end{grafico}
\begin{grafico} \centering \begin{tikzpicture}[gnuplot]
%% generated with GNUPLOT 4.6p3 (Lua 5.1; terminal rev. 99, script rev. 100)
%% dom 08 giu 2014 21:22:03 CEST
\path (0.000,0.000) rectangle (12.500,8.750);
\gpcolor{color=gp lt color border}
\gpsetlinetype{gp lt border}
\gpsetlinewidth{1.00}
\draw[gp path] (1.504,2.125)--(1.684,2.125);
\node[gp node right] at (1.320,2.125) {-0.2};
\draw[gp path] (1.504,3.265)--(1.684,3.265);
\node[gp node right] at (1.320,3.265) {-0.1};
\draw[gp path] (1.504,4.405)--(1.684,4.405);
\node[gp node right] at (1.320,4.405) { 0};
\draw[gp path] (1.504,5.545)--(1.684,5.545);
\node[gp node right] at (1.320,5.545) { 0.1};
\draw[gp path] (1.504,6.685)--(1.684,6.685);
\node[gp node right] at (1.320,6.685) { 0.2};
\draw[gp path] (1.504,0.985)--(1.504,1.165);
\node[gp node center] at (1.504,0.677) { 0};
\draw[gp path] (3.245,0.985)--(3.245,1.165);
\node[gp node center] at (3.245,0.677) { 20};
\draw[gp path] (4.985,0.985)--(4.985,1.165);
\node[gp node center] at (4.985,0.677) { 40};
\draw[gp path] (6.726,0.985)--(6.726,1.165);
\node[gp node center] at (6.726,0.677) { 60};
\draw[gp path] (8.466,0.985)--(8.466,1.165);
\node[gp node center] at (8.466,0.677) { 80};
\draw[gp path] (10.207,0.985)--(10.207,1.165);
\node[gp node center] at (10.207,0.677) { 100};
\draw[gp path] (1.504,7.506)--(1.504,1.293);
\draw[gp path] (1.504,0.985)--(10.233,0.985);
\node[gp node center,rotate=-270] at (0.246,4.405) {Ampiezza [???]};
\node[gp node center] at (6.725,0.215) {Tempo $[s]$};
\node[gp node center] at (6.725,8.287) {Dati decadimento 0.969d};
\gpcolor{rgb color={0.000,0.000,1.000}}
\gpsetlinetype{gp lt plot 0}
\draw[gp path] (1.504,1.293)--(1.508,1.544)--(1.513,2.057)--(1.517,2.786)--(1.521,3.710)%
  --(1.526,4.633)--(1.530,5.556)--(1.534,6.389)--(1.539,7.050)--(1.543,7.426)--(1.548,7.506)%
  --(1.552,7.346)--(1.556,6.947)--(1.561,6.263)--(1.565,5.431)--(1.569,4.553)--(1.574,3.641)%
  --(1.578,2.775)--(1.582,2.079)--(1.587,1.635)--(1.591,1.441)--(1.595,1.509)--(1.600,1.829)%
  --(1.604,2.410)--(1.608,3.162)--(1.613,4.029)--(1.617,4.907)--(1.621,5.750)--(1.626,6.503)%
  --(1.630,7.016)--(1.635,7.289)--(1.639,7.312)--(1.643,7.084)--(1.648,6.605)--(1.652,5.921)%
  --(1.656,5.089)--(1.661,4.234)--(1.665,3.390)--(1.669,2.638)--(1.674,2.057)--(1.678,1.646)%
  --(1.682,1.532)--(1.687,1.680)--(1.691,2.079)--(1.695,2.684)--(1.700,3.436)--(1.704,4.280)%
  --(1.709,5.123)--(1.713,5.933)--(1.717,6.560)--(1.722,6.981)--(1.726,7.175)--(1.730,7.130)%
  --(1.735,6.822)--(1.739,6.297)--(1.743,5.602)--(1.748,4.815)--(1.752,3.995)--(1.756,3.208)%
  --(1.761,2.535)--(1.765,2.045)--(1.769,1.794)--(1.774,1.749)--(1.778,1.977)--(1.782,2.410)%
  --(1.787,3.037)--(1.791,3.789)--(1.796,4.587)--(1.800,5.363)--(1.804,6.069)--(1.809,6.617)%
  --(1.813,6.936)--(1.817,7.038)--(1.822,6.924)--(1.826,6.560)--(1.830,6.001)--(1.835,5.328)%
  --(1.839,4.553)--(1.843,3.755)--(1.848,3.037)--(1.852,2.467)--(1.856,2.079)--(1.861,1.874)%
  --(1.865,1.931)--(1.870,2.205)--(1.874,2.695)--(1.878,3.322)--(1.883,4.040)--(1.887,4.815)%
  --(1.891,5.545)--(1.896,6.149)--(1.900,6.605)--(1.904,6.867)--(1.909,6.902)--(1.913,6.708)%
  --(1.917,6.297)--(1.922,5.727)--(1.926,5.032)--(1.930,4.291)--(1.935,3.573)--(1.939,2.923)%
  --(1.943,2.421)--(1.948,2.102)--(1.952,1.988)--(1.957,2.114)--(1.961,2.433)--(1.965,2.946)%
  --(1.970,3.584)--(1.974,4.302)--(1.978,5.009)--(1.983,5.682)--(1.987,6.206)--(1.991,6.605)%
  --(1.996,6.776)--(2.000,6.731)--(2.004,6.480)--(2.009,6.035)--(2.013,5.454)--(2.017,4.781)%
  --(2.022,4.074)--(2.026,3.402)--(2.031,2.832)--(2.035,2.410)--(2.039,2.159)--(2.044,2.136)%
  --(2.048,2.296)--(2.052,2.672)--(2.057,3.185)--(2.061,3.824)--(2.065,4.508)--(2.070,5.169)%
  --(2.074,5.773)--(2.078,6.252)--(2.083,6.548)--(2.087,6.674)--(2.091,6.560)--(2.096,6.252)%
  --(2.100,5.784)--(2.104,5.203)--(2.109,4.542)--(2.113,3.869)--(2.118,3.265)--(2.122,2.763)%
  --(2.126,2.410)--(2.131,2.239)--(2.135,2.273)--(2.139,2.513)--(2.144,2.912)--(2.148,3.436)%
  --(2.152,4.063)--(2.157,4.713)--(2.161,5.340)--(2.165,5.864)--(2.170,6.275)--(2.174,6.514)%
  --(2.178,6.525)--(2.183,6.377)--(2.187,6.047)--(2.191,5.556)--(2.196,4.964)--(2.200,4.325)%
  --(2.205,3.710)--(2.209,3.151)--(2.213,2.718)--(2.218,2.433)--(2.222,2.342)--(2.226,2.456)%
  --(2.231,2.706)--(2.235,3.117)--(2.239,3.675)--(2.244,4.280)--(2.248,4.895)--(2.252,5.465)%
  --(2.257,5.933)--(2.261,6.275)--(2.265,6.423)--(2.270,6.400)--(2.274,6.206)--(2.279,5.830)%
  --(2.283,5.317)--(2.287,4.758)--(2.292,4.154)--(2.296,3.584)--(2.300,3.060)--(2.305,2.695)%
  --(2.309,2.501)--(2.313,2.456)--(2.318,2.592)--(2.322,2.900)--(2.326,3.356)--(2.331,3.892)%
  --(2.335,4.473)--(2.339,5.055)--(2.344,5.568)--(2.348,5.978)--(2.352,6.229)--(2.357,6.343)%
  --(2.361,6.263)--(2.366,6.012)--(2.370,5.613)--(2.374,5.123)--(2.379,4.565)--(2.383,3.983)%
  --(2.387,3.436)--(2.392,3.026)--(2.396,2.718)--(2.400,2.570)--(2.405,2.581)--(2.409,2.775)%
  --(2.413,3.117)--(2.418,3.561)--(2.422,4.086)--(2.426,4.656)--(2.431,5.203)--(2.435,5.648)%
  --(2.440,5.990)--(2.444,6.206)--(2.448,6.252)--(2.453,6.104)--(2.457,5.819)--(2.461,5.420)%
  --(2.466,4.918)--(2.470,4.382)--(2.474,3.824)--(2.479,3.356)--(2.483,2.980)--(2.487,2.741)%
  --(2.492,2.638)--(2.496,2.729)--(2.500,2.934)--(2.505,3.299)--(2.509,3.755)--(2.513,4.291)%
  --(2.518,4.815)--(2.522,5.317)--(2.527,5.705)--(2.531,6.001)--(2.535,6.149)--(2.540,6.115)%
  --(2.544,5.955)--(2.548,5.648)--(2.553,5.226)--(2.557,4.736)--(2.561,4.211)--(2.566,3.721)%
  --(2.570,3.276)--(2.574,2.980)--(2.579,2.775)--(2.583,2.729)--(2.587,2.843)--(2.592,3.105)%
  --(2.596,3.482)--(2.601,3.949)--(2.605,4.451)--(2.609,4.929)--(2.614,5.385)--(2.618,5.750)%
  --(2.622,5.967)--(2.627,6.058)--(2.631,6.001)--(2.635,5.796)--(2.640,5.465)--(2.644,5.032)%
  --(2.648,4.553)--(2.653,4.063)--(2.657,3.607)--(2.661,3.231)--(2.666,2.969)--(2.670,2.820)%
  --(2.674,2.843)--(2.679,3.003)--(2.683,3.276)--(2.688,3.664)--(2.692,4.120)--(2.696,4.599)%
  --(2.701,5.043)--(2.705,5.465)--(2.709,5.773)--(2.714,5.933)--(2.718,5.978)--(2.722,5.876)%
  --(2.727,5.636)--(2.731,5.283)--(2.735,4.861)--(2.740,4.405)--(2.744,3.938)--(2.748,3.516)%
  --(2.753,3.197)--(2.757,2.969)--(2.762,2.889)--(2.766,2.946)--(2.770,3.140)--(2.775,3.447)%
  --(2.779,3.835)--(2.783,4.291)--(2.788,4.736)--(2.792,5.157)--(2.796,5.511)--(2.801,5.762)%
  --(2.805,5.898)--(2.809,5.876)--(2.814,5.750)--(2.818,5.477)--(2.822,5.123)--(2.827,4.701)%
  --(2.831,4.257)--(2.835,3.835)--(2.840,3.459)--(2.844,3.185)--(2.849,3.003)--(2.853,2.969)%
  --(2.857,3.060)--(2.862,3.288)--(2.866,3.607)--(2.870,3.995)--(2.875,4.428)--(2.879,4.850)%
  --(2.883,5.237)--(2.888,5.545)--(2.892,5.739)--(2.896,5.830)--(2.901,5.784)--(2.905,5.602)%
  --(2.909,5.317)--(2.914,4.952)--(2.918,4.542)--(2.923,4.131)--(2.927,3.721)--(2.931,3.402)%
  --(2.936,3.174)--(2.940,3.048)--(2.944,3.060)--(2.949,3.185)--(2.953,3.425)--(2.957,3.755)%
  --(2.962,4.143)--(2.966,4.553)--(2.970,4.964)--(2.975,5.294)--(2.979,5.556)--(2.983,5.716)%
  --(2.988,5.750)--(2.992,5.659)--(2.996,5.465)--(3.001,5.180)--(3.005,4.804)--(3.010,4.416)%
  --(3.014,4.017)--(3.018,3.653)--(3.023,3.356)--(3.027,3.174)--(3.031,3.105)--(3.036,3.140)%
  --(3.040,3.311)--(3.044,3.561)--(3.049,3.892)--(3.053,4.280)--(3.057,4.679)--(3.062,5.043)%
  --(3.066,5.340)--(3.070,5.568)--(3.075,5.670)--(3.079,5.682)--(3.084,5.556)--(3.088,5.328)%
  --(3.092,5.021)--(3.097,4.679)--(3.101,4.291)--(3.105,3.926)--(3.110,3.584)--(3.114,3.345)%
  --(3.118,3.208)--(3.123,3.174)--(3.127,3.254)--(3.131,3.436)--(3.136,3.698)--(3.140,4.052)%
  --(3.144,4.416)--(3.149,4.781)--(3.153,5.112)--(3.157,5.385)--(3.162,5.556)--(3.166,5.613)%
  --(3.171,5.579)--(3.175,5.442)--(3.179,5.192)--(3.184,4.895)--(3.188,4.542)--(3.192,4.166)%
  --(3.197,3.835)--(3.201,3.539)--(3.205,3.333)--(3.210,3.231)--(3.214,3.242)--(3.218,3.345)%
  --(3.223,3.550)--(3.227,3.846)--(3.231,4.166)--(3.236,4.519)--(3.240,4.861)--(3.245,5.157)%
  --(3.249,5.397)--(3.253,5.522)--(3.258,5.556)--(3.262,5.488)--(3.266,5.317)--(3.271,5.066)%
  --(3.275,4.758)--(3.279,4.416)--(3.284,4.063)--(3.288,3.755)--(3.292,3.504)--(3.297,3.356)%
  --(3.301,3.288)--(3.305,3.322)--(3.310,3.447)--(3.314,3.675)--(3.318,3.972)--(3.323,4.291)%
  --(3.327,4.622)--(3.332,4.941)--(3.336,5.214)--(3.340,5.397)--(3.345,5.499)--(3.349,5.488)%
  --(3.353,5.397)--(3.358,5.203)--(3.362,4.929)--(3.366,4.622)--(3.371,4.302)--(3.375,3.983)%
  --(3.379,3.710)--(3.384,3.504)--(3.388,3.368)--(3.392,3.333)--(3.397,3.413)--(3.401,3.561)%
  --(3.405,3.801)--(3.410,4.086)--(3.414,4.405)--(3.419,4.724)--(3.423,5.009)--(3.427,5.237)%
  --(3.432,5.385)--(3.436,5.454)--(3.440,5.420)--(3.445,5.294)--(3.449,5.100)--(3.453,4.827)%
  --(3.458,4.519)--(3.462,4.200)--(3.466,3.915)--(3.471,3.664)--(3.475,3.493)--(3.479,3.390)%
  --(3.484,3.402)--(3.488,3.493)--(3.493,3.675)--(3.497,3.903)--(3.501,4.200)--(3.506,4.508)%
  --(3.510,4.804)--(3.514,5.043)--(3.519,5.260)--(3.523,5.363)--(3.527,5.385)--(3.532,5.351)%
  --(3.536,5.203)--(3.540,4.964)--(3.545,4.690)--(3.549,4.416)--(3.553,4.120)--(3.558,3.846)%
  --(3.562,3.630)--(3.566,3.493)--(3.571,3.436)--(3.575,3.459)--(3.580,3.573)--(3.584,3.767)%
  --(3.588,4.029)--(3.593,4.302)--(3.597,4.599)--(3.601,4.861)--(3.606,5.100)--(3.610,5.249)%
  --(3.614,5.351)--(3.619,5.351)--(3.623,5.249)--(3.627,5.078)--(3.632,4.872)--(3.636,4.599)%
  --(3.640,4.314)--(3.645,4.040)--(3.649,3.812)--(3.654,3.630)--(3.658,3.516)--(3.662,3.482)%
  --(3.667,3.539)--(3.671,3.687)--(3.675,3.881)--(3.680,4.131)--(3.684,4.405)--(3.688,4.679)%
  --(3.693,4.907)--(3.697,5.112)--(3.701,5.260)--(3.706,5.317)--(3.710,5.283)--(3.714,5.157)%
  --(3.719,4.986)--(3.723,4.770)--(3.727,4.496)--(3.732,4.234)--(3.736,3.983)--(3.741,3.778)%
  --(3.745,3.607)--(3.749,3.539)--(3.754,3.527)--(3.758,3.618)--(3.762,3.767)--(3.767,3.983)%
  --(3.771,4.234)--(3.775,4.496)--(3.780,4.724)--(3.784,4.964)--(3.788,5.135)--(3.793,5.237)%
  --(3.797,5.249)--(3.801,5.214)--(3.806,5.100)--(3.810,4.895)--(3.815,4.656)--(3.819,4.416)%
  --(3.823,4.177)--(3.828,3.926)--(3.832,3.732)--(3.836,3.607)--(3.841,3.584)--(3.845,3.596)%
  --(3.849,3.687)--(3.854,3.846)--(3.858,4.086)--(3.862,4.314)--(3.867,4.553)--(3.871,4.804)%
  --(3.875,5.009)--(3.880,5.135)--(3.884,5.203)--(3.888,5.214)--(3.893,5.146)--(3.897,4.986)%
  --(3.902,4.793)--(3.906,4.576)--(3.910,4.337)--(3.915,4.086)--(3.919,3.881)--(3.923,3.732)%
  --(3.928,3.641)--(3.932,3.607)--(3.936,3.653)--(3.941,3.789)--(3.945,3.949)--(3.949,4.166)%
  --(3.954,4.405)--(3.958,4.633)--(3.962,4.850)--(3.967,5.021)--(3.971,5.135)--(3.976,5.180)%
  --(3.980,5.157)--(3.984,5.055)--(3.989,4.895)--(3.993,4.713)--(3.997,4.485)--(4.002,4.257)%
  --(4.006,4.040)--(4.010,3.858)--(4.015,3.732)--(4.019,3.641)--(4.023,3.653)--(4.028,3.732)%
  --(4.032,3.858)--(4.036,4.040)--(4.041,4.245)--(4.045,4.485)--(4.049,4.679)--(4.054,4.884)%
  --(4.058,5.032)--(4.063,5.135)--(4.067,5.146)--(4.071,5.078)--(4.076,4.986)--(4.080,4.827)%
  --(4.084,4.633)--(4.089,4.394)--(4.093,4.188)--(4.097,3.995)--(4.102,3.835)--(4.106,3.721)%
  --(4.110,3.687)--(4.115,3.710)--(4.119,3.789)--(4.123,3.926)--(4.128,4.120)--(4.132,4.325)%
  --(4.137,4.542)--(4.141,4.758)--(4.145,4.918)--(4.150,5.055)--(4.154,5.078)--(4.158,5.100)%
  --(4.163,5.043)--(4.167,4.918)--(4.171,4.747)--(4.176,4.553)--(4.180,4.337)--(4.184,4.131)%
  --(4.189,3.949)--(4.193,3.812)--(4.197,3.744)--(4.202,3.710)--(4.206,3.767)--(4.210,3.869)%
  --(4.215,4.017)--(4.219,4.211)--(4.224,4.405)--(4.228,4.599)--(4.232,4.793)--(4.237,4.929)%
  --(4.241,5.032)--(4.245,5.078)--(4.250,5.055)--(4.254,4.964)--(4.258,4.838)--(4.263,4.656)%
  --(4.267,4.473)--(4.271,4.280)--(4.276,4.086)--(4.280,3.926)--(4.284,3.812)--(4.289,3.755)%
  --(4.293,3.755)--(4.298,3.824)--(4.302,3.938)--(4.306,4.086)--(4.311,4.268)--(4.315,4.485)%
  --(4.319,4.667)--(4.324,4.815)--(4.328,4.941)--(4.332,5.021)--(4.337,5.043)--(4.341,4.998)%
  --(4.345,4.907)--(4.350,4.770)--(4.354,4.599)--(4.358,4.405)--(4.363,4.211)--(4.367,4.052)%
  --(4.371,3.903)--(4.376,3.824)--(4.380,3.778)--(4.385,3.801)--(4.389,3.869)--(4.393,4.006)%
  --(4.398,4.154)--(4.402,4.359)--(4.406,4.542)--(4.411,4.701)--(4.415,4.850)--(4.419,4.964)%
  --(4.424,5.009)--(4.428,5.009)--(4.432,4.941)--(4.437,4.850)--(4.441,4.690)--(4.445,4.530)%
  --(4.450,4.348)--(4.454,4.154)--(4.458,4.017)--(4.463,3.892)--(4.467,3.835)--(4.472,3.812)%
  --(4.476,3.858)--(4.480,3.938)--(4.485,4.074)--(4.489,4.245)--(4.493,4.416)--(4.498,4.576)%
  --(4.502,4.736)--(4.506,4.861)--(4.511,4.952)--(4.515,4.975)--(4.519,4.964)--(4.524,4.884)%
  --(4.528,4.781)--(4.532,4.633)--(4.537,4.462)--(4.541,4.302)--(4.546,4.131)--(4.550,4.006)%
  --(4.554,3.903)--(4.559,3.846)--(4.563,3.846)--(4.567,3.915)--(4.572,4.006)--(4.576,4.143)%
  --(4.580,4.302)--(4.585,4.473)--(4.589,4.622)--(4.593,4.770)--(4.598,4.872)--(4.602,4.941)%
  --(4.606,4.952)--(4.611,4.929)--(4.615,4.838)--(4.619,4.724)--(4.624,4.565)--(4.628,4.405)%
  --(4.633,4.245)--(4.637,4.097)--(4.641,3.972)--(4.646,3.903)--(4.650,3.869)--(4.654,3.881)%
  --(4.659,3.949)--(4.663,4.074)--(4.667,4.177)--(4.672,4.359)--(4.676,4.519)--(4.680,4.667)%
  --(4.685,4.793)--(4.689,4.884)--(4.693,4.929)--(4.698,4.941)--(4.702,4.872)--(4.707,4.781)%
  --(4.711,4.644)--(4.715,4.508)--(4.720,4.359)--(4.724,4.188)--(4.728,4.074)--(4.733,3.972)%
  --(4.737,3.903)--(4.741,3.892)--(4.746,3.926)--(4.750,4.017)--(4.754,4.131)--(4.759,4.257)%
  --(4.763,4.416)--(4.767,4.565)--(4.772,4.701)--(4.776,4.815)--(4.780,4.872)--(4.785,4.907)%
  --(4.789,4.895)--(4.794,4.827)--(4.798,4.736)--(4.802,4.587)--(4.807,4.462)--(4.811,4.314)%
  --(4.815,4.177)--(4.820,4.052)--(4.824,3.972)--(4.828,3.926)--(4.833,3.915)--(4.837,3.972)%
  --(4.841,4.063)--(4.846,4.188)--(4.850,4.314)--(4.854,4.451)--(4.859,4.610)--(4.863,4.724)%
  --(4.868,4.815)--(4.872,4.872)--(4.876,4.872)--(4.881,4.850)--(4.885,4.781)--(4.889,4.667)%
  --(4.894,4.542)--(4.898,4.405)--(4.902,4.257)--(4.907,4.131)--(4.911,4.040)--(4.915,3.972)%
  --(4.920,3.938)--(4.924,3.960)--(4.928,4.017)--(4.933,4.109)--(4.937,4.245)--(4.941,4.371)%
  --(4.946,4.508)--(4.950,4.633)--(4.955,4.736)--(4.959,4.827)--(4.963,4.861)--(4.968,4.861)%
  --(4.972,4.804)--(4.976,4.724)--(4.981,4.633)--(4.985,4.496)--(4.989,4.359)--(4.994,4.223)%
  --(4.998,4.109)--(5.002,4.029)--(5.007,3.972)--(5.011,3.983)--(5.015,3.995)--(5.020,4.063)%
  --(5.024,4.166)--(5.029,4.280)--(5.033,4.416)--(5.037,4.553)--(5.042,4.667)--(5.046,4.758)%
  --(5.050,4.815)--(5.055,4.850)--(5.059,4.827)--(5.063,4.770)--(5.068,4.690)--(5.072,4.576)%
  --(5.076,4.451)--(5.081,4.302)--(5.085,4.188)--(5.089,4.097)--(5.094,4.017)--(5.098,4.006)%
  --(5.102,4.006)--(5.107,4.029)--(5.111,4.109)--(5.116,4.223)--(5.120,4.337)--(5.124,4.451)%
  --(5.129,4.576)--(5.133,4.690)--(5.137,4.747)--(5.142,4.815)--(5.146,4.827)--(5.150,4.793)%
  --(5.155,4.747)--(5.159,4.644)--(5.163,4.519)--(5.168,4.405)--(5.172,4.280)--(5.176,4.166)%
  --(5.181,4.086)--(5.185,4.040)--(5.190,4.029)--(5.194,4.029)--(5.198,4.063)--(5.203,4.143)%
  --(5.207,4.268)--(5.211,4.371)--(5.216,4.496)--(5.220,4.610)--(5.224,4.701)--(5.229,4.758)%
  --(5.233,4.804)--(5.237,4.804)--(5.242,4.770)--(5.246,4.679)--(5.250,4.599)--(5.255,4.473)%
  --(5.259,4.371)--(5.263,4.257)--(5.268,4.143)--(5.272,4.086)--(5.277,4.052)--(5.281,4.029)%
  --(5.285,4.052)--(5.290,4.109)--(5.294,4.200)--(5.298,4.302)--(5.303,4.416)--(5.307,4.530)%
  --(5.311,4.633)--(5.316,4.713)--(5.320,4.770)--(5.324,4.781)--(5.329,4.770)--(5.333,4.713)%
  --(5.337,4.644)--(5.342,4.553)--(5.346,4.451)--(5.351,4.325)--(5.355,4.223)--(5.359,4.143)%
  --(5.364,4.074)--(5.368,4.040)--(5.372,4.052)--(5.377,4.086)--(5.381,4.143)--(5.385,4.234)%
  --(5.390,4.337)--(5.394,4.462)--(5.398,4.553)--(5.403,4.644)--(5.407,4.713)--(5.411,4.770)%
  --(5.416,4.770)--(5.420,4.736)--(5.424,4.679)--(5.429,4.610)--(5.433,4.508)--(5.438,4.405)%
  --(5.442,4.280)--(5.446,4.223)--(5.451,4.143)--(5.455,4.074)--(5.459,4.063)--(5.464,4.086)%
  --(5.468,4.120)--(5.472,4.188)--(5.477,4.291)--(5.481,4.382)--(5.485,4.473)--(5.490,4.576)%
  --(5.494,4.667)--(5.498,4.736)--(5.503,4.747)--(5.507,4.747)--(5.512,4.713)--(5.516,4.667)%
  --(5.520,4.576)--(5.525,4.462)--(5.529,4.371)--(5.533,4.280)--(5.538,4.200)--(5.542,4.109)%
  --(5.546,4.086)--(5.551,4.074)--(5.555,4.097)--(5.559,4.154)--(5.564,4.234)--(5.568,4.314)%
  --(5.572,4.405)--(5.577,4.519)--(5.581,4.610)--(5.585,4.679)--(5.590,4.713)--(5.594,4.736)%
  --(5.599,4.713)--(5.603,4.679)--(5.607,4.622)--(5.612,4.519)--(5.616,4.428)--(5.620,4.348)%
  --(5.625,4.257)--(5.629,4.188)--(5.633,4.120)--(5.638,4.097)--(5.642,4.097)--(5.646,4.120)%
  --(5.651,4.177)--(5.655,4.257)--(5.659,4.359)--(5.664,4.451)--(5.668,4.553)--(5.672,4.622)%
  --(5.677,4.679)--(5.681,4.713)--(5.686,4.724)--(5.690,4.701)--(5.694,4.644)--(5.699,4.587)%
  --(5.703,4.496)--(5.707,4.416)--(5.712,4.314)--(5.716,4.234)--(5.720,4.166)--(5.725,4.120)%
  --(5.729,4.097)--(5.733,4.131)--(5.738,4.143)--(5.742,4.223)--(5.746,4.291)--(5.751,4.394)%
  --(5.755,4.485)--(5.760,4.565)--(5.764,4.633)--(5.768,4.679)--(5.773,4.713)--(5.777,4.701)%
  --(5.781,4.667)--(5.786,4.622)--(5.790,4.542)--(5.794,4.462)--(5.799,4.382)--(5.803,4.302)%
  --(5.807,4.211)--(5.812,4.154)--(5.816,4.131)--(5.820,4.120)--(5.825,4.154)--(5.829,4.188)%
  --(5.833,4.257)--(5.838,4.337)--(5.842,4.428)--(5.847,4.496)--(5.851,4.576)--(5.855,4.644)%
  --(5.860,4.690)--(5.864,4.690)--(5.868,4.679)--(5.873,4.656)--(5.877,4.587)--(5.881,4.530)%
  --(5.886,4.439)--(5.890,4.348)--(5.894,4.268)--(5.899,4.200)--(5.903,4.154)--(5.907,4.131)%
  --(5.912,4.143)--(5.916,4.166)--(5.921,4.223)--(5.925,4.291)--(5.929,4.359)--(5.934,4.439)%
  --(5.938,4.542)--(5.942,4.599)--(5.947,4.644)--(5.951,4.690)--(5.955,4.679)--(5.960,4.656)%
  --(5.964,4.622)--(5.968,4.565)--(5.973,4.485)--(5.977,4.405)--(5.981,4.314)--(5.986,4.245)%
  --(5.990,4.200)--(5.994,4.166)--(5.999,4.143)--(6.003,4.143)--(6.008,4.188)--(6.012,4.245)%
  --(6.016,4.314)--(6.021,4.394)--(6.025,4.473)--(6.029,4.553)--(6.034,4.599)--(6.038,4.644)%
  --(6.042,4.679)--(6.047,4.667)--(6.051,4.633)--(6.055,4.599)--(6.060,4.542)--(6.064,4.473)%
  --(6.068,4.371)--(6.073,4.291)--(6.077,4.234)--(6.082,4.188)--(6.086,4.166)--(6.090,4.154)%
  --(6.095,4.177)--(6.099,4.211)--(6.103,4.280)--(6.108,4.348)--(6.112,4.416)--(6.116,4.485)%
  --(6.121,4.553)--(6.125,4.610)--(6.129,4.644)--(6.134,4.656)--(6.138,4.633)--(6.142,4.622)%
  --(6.147,4.565)--(6.151,4.508)--(6.155,4.416)--(6.160,4.348)--(6.164,4.280)--(6.169,4.234)%
  --(6.173,4.188)--(6.177,4.166)--(6.182,4.177)--(6.186,4.200)--(6.190,4.245)--(6.195,4.302)%
  --(6.199,4.382)--(6.203,4.439)--(6.208,4.519)--(6.212,4.576)--(6.216,4.610)--(6.221,4.644)%
  --(6.225,4.644)--(6.229,4.644)--(6.234,4.599)--(6.238,4.542)--(6.243,4.462)--(6.247,4.394)%
  --(6.251,4.325)--(6.256,4.280)--(6.260,4.223)--(6.264,4.188)--(6.269,4.188)--(6.273,4.200)%
  --(6.277,4.234)--(6.282,4.280)--(6.286,4.337)--(6.290,4.394)--(6.295,4.462)--(6.299,4.519)%
  --(6.303,4.587)--(6.308,4.610)--(6.312,4.633)--(6.316,4.633)--(6.321,4.610)--(6.325,4.565)%
  --(6.330,4.508)--(6.334,4.451)--(6.338,4.371)--(6.343,4.314)--(6.347,4.257)--(6.351,4.211)%
  --(6.356,4.200)--(6.360,4.188)--(6.364,4.200)--(6.369,4.245)--(6.373,4.302)--(6.377,4.348)%
  --(6.382,4.428)--(6.386,4.473)--(6.390,4.542)--(6.395,4.576)--(6.399,4.610)--(6.404,4.633)%
  --(6.408,4.622)--(6.412,4.576)--(6.417,4.542)--(6.421,4.485)--(6.425,4.416)--(6.430,4.359)%
  --(6.434,4.280)--(6.438,4.257)--(6.443,4.223)--(6.447,4.211)--(6.451,4.200)--(6.456,4.223)%
  --(6.460,4.257)--(6.464,4.314)--(6.469,4.371)--(6.473,4.439)--(6.477,4.508)--(6.482,4.553)%
  --(6.486,4.599)--(6.491,4.622)--(6.495,4.622)--(6.499,4.599)--(6.504,4.553)--(6.508,4.519)%
  --(6.512,4.462)--(6.517,4.405)--(6.521,4.325)--(6.525,4.302)--(6.530,4.234)--(6.534,4.211)%
  --(6.538,4.211)--(6.543,4.211)--(6.547,4.234)--(6.551,4.280)--(6.556,4.348)--(6.560,4.405)%
  --(6.565,4.462)--(6.569,4.519)--(6.573,4.565)--(6.578,4.599)--(6.582,4.610)--(6.586,4.599)%
  --(6.591,4.565)--(6.595,4.542)--(6.599,4.485)--(6.604,4.439)--(6.608,4.382)--(6.612,4.325)%
  --(6.617,4.280)--(6.621,4.234)--(6.625,4.223)--(6.630,4.211)--(6.634,4.234)--(6.638,4.268)%
  --(6.643,4.314)--(6.647,4.348)--(6.652,4.428)--(6.656,4.473)--(6.660,4.519)--(6.665,4.565)%
  --(6.669,4.599)--(6.673,4.599)--(6.678,4.587)--(6.682,4.553)--(6.686,4.519)--(6.691,4.473)%
  --(6.695,4.416)--(6.699,4.359)--(6.704,4.302)--(6.708,4.268)--(6.712,4.234)--(6.717,4.234)%
  --(6.721,4.234)--(6.726,4.245)--(6.730,4.291)--(6.734,4.337)--(6.739,4.382)--(6.743,4.428)%
  --(6.747,4.496)--(6.752,4.542)--(6.756,4.576)--(6.760,4.576)--(6.765,4.587)--(6.769,4.565)%
  --(6.773,4.542)--(6.778,4.496)--(6.782,4.451)--(6.786,4.405)--(6.791,4.348)--(6.795,4.302)%
  --(6.799,4.257)--(6.804,4.245)--(6.808,4.234)--(6.813,4.234)--(6.817,4.268)--(6.821,4.314)%
  --(6.826,4.348)--(6.830,4.394)--(6.834,4.451)--(6.839,4.508)--(6.843,4.542)--(6.847,4.553)%
  --(6.852,4.587)--(6.856,4.576)--(6.860,4.553)--(6.865,4.530)--(6.869,4.496)--(6.873,4.428)%
  --(6.878,4.382)--(6.882,4.325)--(6.886,4.302)--(6.891,4.268)--(6.895,4.245)--(6.900,4.234)%
  --(6.904,4.268)--(6.908,4.291)--(6.913,4.325)--(6.917,4.371)--(6.921,4.439)--(6.926,4.473)%
  --(6.930,4.519)--(6.934,4.553)--(6.939,4.565)--(6.943,4.565)--(6.947,4.565)--(6.952,4.542)%
  --(6.956,4.519)--(6.960,4.451)--(6.965,4.405)--(6.969,4.371)--(6.974,4.337)--(6.978,4.280)%
  --(6.982,4.268)--(6.987,4.245)--(6.991,4.257)--(6.995,4.268)--(7.000,4.302)--(7.004,4.337)%
  --(7.008,4.394)--(7.013,4.439)--(7.017,4.496)--(7.021,4.530)--(7.026,4.542)--(7.030,4.542)%
  --(7.034,4.565)--(7.039,4.553)--(7.043,4.530)--(7.047,4.473)--(7.052,4.439)--(7.056,4.416)%
  --(7.061,4.359)--(7.065,4.314)--(7.069,4.280)--(7.074,4.268)--(7.078,4.268)--(7.082,4.257)%
  --(7.087,4.291)--(7.091,4.325)--(7.095,4.348)--(7.100,4.394)--(7.104,4.451)--(7.108,4.496)%
  --(7.113,4.519)--(7.117,4.530)--(7.121,4.565)--(7.126,4.553)--(7.130,4.530)--(7.135,4.519)%
  --(7.139,4.473)--(7.143,4.428)--(7.148,4.382)--(7.152,4.337)--(7.156,4.314)--(7.161,4.291)%
  --(7.165,4.268)--(7.169,4.268)--(7.174,4.280)--(7.178,4.302)--(7.182,4.325)--(7.187,4.382)%
  --(7.191,4.439)--(7.195,4.462)--(7.200,4.508)--(7.204,4.508)--(7.208,4.553)--(7.213,4.553)%
  --(7.217,4.519)--(7.222,4.519)--(7.226,4.485)--(7.230,4.451)--(7.235,4.405)--(7.239,4.371)%
  --(7.243,4.337)--(7.248,4.302)--(7.252,4.280)--(7.256,4.268)--(7.261,4.280)--(7.265,4.291)%
  --(7.269,4.325)--(7.274,4.371)--(7.278,4.416)--(7.282,4.428)--(7.287,4.462)--(7.291,4.508)%
  --(7.296,4.530)--(7.300,4.542)--(7.304,4.542)--(7.309,4.519)--(7.313,4.508)--(7.317,4.485)%
  --(7.322,4.416)--(7.326,4.405)--(7.330,4.359)--(7.335,4.325)--(7.339,4.280)--(7.343,4.280)%
  --(7.348,4.280)--(7.352,4.280)--(7.356,4.302)--(7.361,4.348)--(7.365,4.382)--(7.369,4.394)%
  --(7.374,4.439)--(7.378,4.485)--(7.383,4.508)--(7.387,4.519)--(7.391,4.530)--(7.396,4.530)%
  --(7.400,4.530)--(7.404,4.485)--(7.409,4.462)--(7.413,4.439)--(7.417,4.394)--(7.422,4.348)%
  --(7.426,4.325)--(7.430,4.302)--(7.435,4.291)--(7.439,4.280)--(7.443,4.291)--(7.448,4.337)%
  --(7.452,4.359)--(7.457,4.371)--(7.461,4.428)--(7.465,4.462)--(7.470,4.473)--(7.474,4.508)%
  --(7.478,4.530)--(7.483,4.519)--(7.487,4.519)--(7.491,4.485)--(7.496,4.485)--(7.500,4.451)%
  --(7.504,4.405)--(7.509,4.359)--(7.513,4.337)--(7.517,4.325)--(7.522,4.291)--(7.526,4.280)%
  --(7.530,4.302)--(7.535,4.314)--(7.539,4.314)--(7.544,4.348)--(7.548,4.394)--(7.552,4.416)%
  --(7.557,4.462)--(7.561,4.496)--(7.565,4.519)--(7.570,4.508)--(7.574,4.508)--(7.578,4.508)%
  --(7.583,4.508)--(7.587,4.462)--(7.591,4.405)--(7.596,4.394)--(7.600,4.359)--(7.604,4.314)%
  --(7.609,4.302)--(7.613,4.302)--(7.618,4.291)--(7.622,4.302)--(7.626,4.314)--(7.631,4.337)%
  --(7.635,4.359)--(7.639,4.405)--(7.644,4.439)--(7.648,4.462)--(7.652,4.496)--(7.657,4.508)%
  --(7.661,4.508)--(7.665,4.508)--(7.670,4.496)--(7.674,4.473)--(7.678,4.451)--(7.683,4.428)%
  --(7.687,4.382)--(7.691,4.348)--(7.696,4.325)--(7.700,4.325)--(7.705,4.302)--(7.709,4.302)%
  --(7.713,4.314)--(7.718,4.325)--(7.722,4.359)--(7.726,4.371)--(7.731,4.428)--(7.735,4.439)%
  --(7.739,4.462)--(7.744,4.496)--(7.748,4.519)--(7.752,4.508)--(7.757,4.496)--(7.761,4.473)%
  --(7.765,4.473)--(7.770,4.439)--(7.774,4.394)--(7.779,4.359)--(7.783,4.348)--(7.787,4.337)%
  --(7.792,4.302)--(7.796,4.302)--(7.800,4.302)--(7.805,4.314)--(7.809,4.337)--(7.813,4.371)%
  --(7.818,4.394)--(7.822,4.416)--(7.826,4.439)--(7.831,4.485)--(7.835,4.496)--(7.839,4.496)%
  --(7.844,4.496)--(7.848,4.485)--(7.852,4.462)--(7.857,4.462)--(7.861,4.439)--(7.866,4.382)%
  --(7.870,4.325)--(7.874,4.337)--(7.879,4.348)--(7.883,4.302)--(7.887,4.280)--(7.892,4.314)%
  --(7.896,4.348)--(7.900,4.359)--(7.905,4.359)--(7.909,4.382)--(7.913,4.439)--(7.918,4.462)%
  --(7.922,4.462)--(7.926,4.496)--(7.931,4.496)--(7.935,4.473)--(7.939,4.485)--(7.944,4.473)%
  --(7.948,4.451)--(7.953,4.382)--(7.957,4.371)--(7.961,4.371)--(7.966,4.348)--(7.970,4.302)%
  --(7.974,4.291)--(7.979,4.314)--(7.983,4.337)--(7.987,4.337)--(7.992,4.348)--(7.996,4.382)%
  --(8.000,4.416)--(8.005,4.451)--(8.009,4.451)--(8.013,4.485)--(8.018,4.473)--(8.022,4.485)%
  --(8.027,4.496)--(8.031,4.496)--(8.035,4.439)--(8.040,4.416)--(8.044,4.394)--(8.048,4.382)%
  --(8.053,4.348)--(8.057,4.325)--(8.061,4.314)--(8.066,4.325)--(8.070,4.314)--(8.074,4.314)%
  --(8.079,4.337)--(8.083,4.371)--(8.087,4.394)--(8.092,4.405)--(8.096,4.451)--(8.100,4.462)%
  --(8.105,4.473)--(8.109,4.496)--(8.114,4.496)--(8.118,4.485)--(8.122,4.451)--(8.127,4.439)%
  --(8.131,4.428)--(8.135,4.405)--(8.140,4.337)--(8.144,4.348)--(8.148,4.337)--(8.153,4.325)%
  --(8.157,4.302)--(8.161,4.302)--(8.166,4.337)--(8.170,4.359)--(8.174,4.371)--(8.179,4.405)%
  --(8.183,4.416)--(8.188,4.451)--(8.192,4.462)--(8.196,4.496)--(8.201,4.485)--(8.205,4.473)%
  --(8.209,4.451)--(8.214,4.451)--(8.218,4.451)--(8.222,4.405)--(8.227,4.359)--(8.231,4.371)%
  --(8.235,4.359)--(8.240,4.337)--(8.244,4.302)--(8.248,4.337)--(8.253,4.325)--(8.257,4.348)%
  --(8.261,4.371)--(8.266,4.394)--(8.270,4.394)--(8.275,4.416)--(8.279,4.451)--(8.283,4.485)%
  --(8.288,4.485)--(8.292,4.462)--(8.296,4.462)--(8.301,4.485)--(8.305,4.451)--(8.309,4.405)%
  --(8.314,4.382)--(8.318,4.394)--(8.322,4.382)--(8.327,4.337)--(8.331,4.325)--(8.335,4.325)%
  --(8.340,4.337)--(8.344,4.337)--(8.349,4.359)--(8.353,4.371)--(8.357,4.382)--(8.362,4.416)%
  --(8.366,4.439)--(8.370,4.462)--(8.375,4.473)--(8.379,4.439)--(8.383,4.473)--(8.388,4.485)%
  --(8.392,4.439)--(8.396,4.405)--(8.401,4.405)--(8.405,4.405)--(8.409,4.371)--(8.414,4.337)%
  --(8.418,4.337)--(8.422,4.337)--(8.427,4.314)--(8.431,4.325)--(8.436,4.348)--(8.440,4.359)%
  --(8.444,4.371)--(8.449,4.394)--(8.453,4.428)--(8.457,4.451)--(8.462,4.451)--(8.466,4.462)%
  --(8.470,4.485)--(8.475,4.485)--(8.479,4.451)--(8.483,4.428)--(8.488,4.428)--(8.492,4.416)%
  --(8.496,4.382)--(8.501,4.371)--(8.505,4.348)--(8.510,4.337)--(8.514,4.325)--(8.518,4.337)%
  --(8.523,4.337)--(8.527,4.348)--(8.531,4.348)--(8.536,4.394)--(8.540,4.416)--(8.544,4.416)%
  --(8.549,4.428)--(8.553,4.462)--(8.557,4.496)--(8.562,4.462)--(8.566,4.451)--(8.570,4.462)%
  --(8.575,4.439)--(8.579,4.428)--(8.583,4.405)--(8.588,4.371)--(8.592,4.348)--(8.597,4.325)%
  --(8.601,4.337)--(8.605,4.337)--(8.610,4.337)--(8.614,4.337)--(8.618,4.348)--(8.623,4.371)%
  --(8.627,4.405)--(8.631,4.416)--(8.636,4.439)--(8.640,4.462)--(8.644,4.462)--(8.649,4.462)%
  --(8.653,4.462)--(8.657,4.462)--(8.662,4.462)--(8.666,4.439)--(8.671,4.405)--(8.675,4.416)%
  --(8.679,4.359)--(8.684,4.359)--(8.688,4.348)--(8.692,4.348)--(8.697,4.325)--(8.701,4.337)%
  --(8.705,4.337)--(8.710,4.371)--(8.714,4.371)--(8.718,4.405)--(8.723,4.416)--(8.727,4.439)%
  --(8.731,4.451)--(8.736,4.451)--(8.740,4.451)--(8.744,4.473)--(8.749,4.451)--(8.753,4.451)%
  --(8.758,4.428)--(8.762,4.405)--(8.766,4.382)--(8.771,4.359)--(8.775,4.359)--(8.779,4.337)%
  --(8.784,4.314)--(8.788,4.348)--(8.792,4.337)--(8.797,4.359)--(8.801,4.359)--(8.805,4.382)%
  --(8.810,4.416)--(8.814,4.428)--(8.818,4.439)--(8.823,4.451)--(8.827,4.462)--(8.832,4.451)%
  --(8.836,4.462)--(8.840,4.439)--(8.845,4.439)--(8.849,4.405)--(8.853,4.416)--(8.858,4.394)%
  --(8.862,4.371)--(8.866,4.337)--(8.871,4.337)--(8.875,4.348)--(8.879,4.337)--(8.884,4.337)%
  --(8.888,4.348)--(8.892,4.382)--(8.897,4.416)--(8.901,4.405)--(8.905,4.428)--(8.910,4.451)%
  --(8.914,4.451)--(8.919,4.462)--(8.923,4.451)--(8.927,4.473)--(8.932,4.451)--(8.936,4.428)%
  --(8.940,4.416)--(8.945,4.405)--(8.949,4.371)--(8.953,4.359)--(8.958,4.359)--(8.962,4.359)%
  --(8.966,4.337)--(8.971,4.348)--(8.975,4.359)--(8.979,4.371)--(8.984,4.405)--(8.988,4.405)%
  --(8.993,4.416)--(8.997,4.439)--(9.001,4.462)--(9.006,4.462)--(9.010,4.462)--(9.014,4.473)%
  --(9.019,4.462)--(9.023,4.439)--(9.027,4.416)--(9.032,4.405)--(9.036,4.382)--(9.040,4.359)%
  --(9.045,4.359)--(9.049,4.359)--(9.053,4.325)--(9.058,4.348)--(9.062,4.348)--(9.066,4.371)%
  --(9.071,4.359)--(9.075,4.382)--(9.080,4.394)--(9.084,4.416)--(9.088,4.439)--(9.093,4.451)%
  --(9.097,4.439)--(9.101,4.462)--(9.106,4.462)--(9.110,4.439)--(9.114,4.428)--(9.119,4.405)%
  --(9.123,4.394)--(9.127,4.394)--(9.132,4.371)--(9.136,4.359)--(9.140,4.337)--(9.145,4.348)%
  --(9.149,4.348)--(9.153,4.359)--(9.158,4.359)--(9.162,4.371)--(9.167,4.405)--(9.171,4.405)%
  --(9.175,4.428)--(9.180,4.439)--(9.184,4.439)--(9.188,4.451)--(9.193,4.462)--(9.197,4.451)%
  --(9.201,4.428)--(9.206,4.405)--(9.210,4.416)--(9.214,4.394)--(9.219,4.382)--(9.223,4.371)%
  --(9.227,4.348)--(9.232,4.348)--(9.236,4.359)--(9.241,4.348)--(9.245,4.359)--(9.249,4.371)%
  --(9.254,4.394)--(9.258,4.394)--(9.262,4.405)--(9.267,4.428)--(9.271,4.439)--(9.275,4.462)%
  --(9.280,4.462)--(9.284,4.451)--(9.288,4.416)--(9.293,4.428)--(9.297,4.416)--(9.301,4.416)%
  --(9.306,4.394)--(9.310,4.359)--(9.314,4.359)--(9.319,4.348)--(9.323,4.359)--(9.328,4.348)%
  --(9.332,4.371)--(9.336,4.371)--(9.341,4.382)--(9.345,4.382)--(9.349,4.394)--(9.354,4.416)%
  --(9.358,4.428)--(9.362,4.451)--(9.367,4.462)--(9.371,4.462)--(9.375,4.428)--(9.380,4.439)%
  --(9.384,4.428)--(9.388,4.405)--(9.393,4.394)--(9.397,4.371)--(9.402,4.371)--(9.406,4.359)%
  --(9.410,4.359)--(9.415,4.348)--(9.419,4.359)--(9.423,4.371)--(9.428,4.371)--(9.432,4.382)%
  --(9.436,4.382)--(9.441,4.405)--(9.445,4.416)--(9.449,4.439)--(9.454,4.462)--(9.458,4.439)%
  --(9.462,4.439)--(9.467,4.439)--(9.471,4.439)--(9.475,4.416)--(9.480,4.405)--(9.484,4.382)%
  --(9.489,4.394)--(9.493,4.371)--(9.497,4.359)--(9.502,4.371)--(9.506,4.359)--(9.510,4.359)%
  --(9.515,4.359)--(9.519,4.382)--(9.523,4.382)--(9.528,4.405)--(9.532,4.416)--(9.536,4.439)%
  --(9.541,4.428)--(9.545,4.451)--(9.549,4.451)--(9.554,4.439)--(9.558,4.439)--(9.563,4.416)%
  --(9.567,4.416)--(9.571,4.405)--(9.576,4.382)--(9.580,4.359)--(9.584,4.359)--(9.589,4.359)%
  --(9.593,4.359)--(9.597,4.359)--(9.602,4.359)--(9.606,4.371)--(9.610,4.371)--(9.615,4.382)%
  --(9.619,4.405)--(9.623,4.428)--(9.628,4.428)--(9.632,4.439)--(9.636,4.439)--(9.641,4.428)%
  --(9.645,4.451)--(9.650,4.439)--(9.654,4.416)--(9.658,4.416)--(9.663,4.405)--(9.667,4.382)%
  --(9.671,4.382)--(9.676,4.371)--(9.680,4.371)--(9.684,4.348)--(9.689,4.359)--(9.693,4.371)%
  --(9.697,4.371)--(9.702,4.394)--(9.706,4.405)--(9.710,4.416)--(9.715,4.416)--(9.719,4.416)%
  --(9.724,4.439)--(9.728,4.451)--(9.732,4.439)--(9.737,4.428)--(9.741,4.428)--(9.745,4.405)%
  --(9.750,4.394)--(9.754,4.394)--(9.758,4.382)--(9.763,4.382)--(9.767,4.371)--(9.771,4.348)%
  --(9.776,4.371)--(9.780,4.359)--(9.784,4.382)--(9.789,4.371)--(9.793,4.382)--(9.797,4.416)%
  --(9.802,4.416)--(9.806,4.416)--(9.811,4.428)--(9.815,4.428)--(9.819,4.439)--(9.824,4.439)%
  --(9.828,4.439)--(9.832,4.416)--(9.837,4.405)--(9.841,4.405)--(9.845,4.382)--(9.850,4.371)%
  --(9.854,4.371)--(9.858,4.371)--(9.863,4.382)--(9.867,4.359)--(9.871,4.371)--(9.876,4.382)%
  --(9.880,4.382)--(9.885,4.405)--(9.889,4.416)--(9.893,4.416)--(9.898,4.416)--(9.902,4.428)%
  --(9.906,4.428)--(9.911,4.439)--(9.915,4.439)--(9.919,4.428)--(9.924,4.416)--(9.928,4.405)%
  --(9.932,4.405)--(9.937,4.382)--(9.941,4.382)--(9.945,4.382)--(9.950,4.371)--(9.954,4.371)%
  --(9.958,4.371)--(9.963,4.371)--(9.967,4.371)--(9.972,4.394)--(9.976,4.405)--(9.980,4.416)%
  --(9.985,4.416)--(9.989,4.416)--(9.993,4.428)--(9.998,4.439)--(10.002,4.439)--(10.006,4.428)%
  --(10.011,4.405)--(10.015,4.405)--(10.019,4.405)--(10.024,4.382)--(10.028,4.394)--(10.032,4.359)%
  --(10.037,4.371)--(10.041,4.371)--(10.046,4.382)--(10.050,4.382)--(10.054,4.371)--(10.059,4.405)%
  --(10.063,4.394)--(10.067,4.405)--(10.072,4.416)--(10.076,4.416)--(10.080,4.416)--(10.085,4.416)%
  --(10.089,4.439)--(10.093,4.439)--(10.098,4.416)--(10.102,4.416)--(10.106,4.405)--(10.111,4.394)%
  --(10.115,4.394)--(10.119,4.382)--(10.124,4.382)--(10.128,4.371)--(10.133,4.359)--(10.137,4.371)%
  --(10.141,4.382)--(10.146,4.394)--(10.150,4.394)--(10.154,4.382)--(10.159,4.416)--(10.163,4.428)%
  --(10.167,4.428)--(10.172,4.439)--(10.176,4.428)--(10.180,4.428)--(10.185,4.428)--(10.189,4.428)%
  --(10.193,4.416)--(10.198,4.405)--(10.202,4.405)--(10.207,4.394)--(10.211,4.394)--(10.215,4.371)%
  --(10.220,4.371)--(10.224,4.382)--(10.228,4.371)--(10.233,4.394);
\gpcolor{color=gp lt color border}
\gpsetlinetype{gp lt border}
\draw[gp path] (1.504,7.506)--(1.504,1.293);
\draw[gp path] (1.504,0.985)--(10.233,0.985);
%% coordinates of the plot area
\gpdefrectangularnode{gp plot 1}{\pgfpoint{1.504cm}{0.985cm}}{\pgfpoint{11.947cm}{7.825cm}}
\end{tikzpicture}
%% gnuplot variables
 \caption{Grafico 0.969dgdecad.tex} \label{gr:0.969dgdecad.tex} \end{grafico}
\begin{grafico} \centering \begin{tikzpicture}[gnuplot]
%% generated with GNUPLOT 4.6p0 (Lua 5.1; terminal rev. 99, script rev. 100)
%% Mon 09 Jun 2014 04:19:46 PM CEST
\path (0.000,0.000) rectangle (12.500,8.750);
\gpcolor{color=gp lt color border}
\gpsetlinetype{gp lt border}
\gpsetlinewidth{1.00}
\draw[gp path] (1.504,2.125)--(1.684,2.125);
\node[gp node right] at (1.320,2.125) {-0.2};
\draw[gp path] (1.504,3.265)--(1.684,3.265);
\node[gp node right] at (1.320,3.265) {-0.1};
\draw[gp path] (1.504,4.405)--(1.684,4.405);
\node[gp node right] at (1.320,4.405) { 0};
\draw[gp path] (1.504,5.545)--(1.684,5.545);
\node[gp node right] at (1.320,5.545) { 0.1};
\draw[gp path] (1.504,6.685)--(1.684,6.685);
\node[gp node right] at (1.320,6.685) { 0.2};
\draw[gp path] (1.504,0.985)--(1.504,1.165);
\node[gp node center] at (1.504,0.677) { 0};
\draw[gp path] (2.664,0.985)--(2.664,1.165);
\node[gp node center] at (2.664,0.677) { 5};
\draw[gp path] (3.825,0.985)--(3.825,1.165);
\node[gp node center] at (3.825,0.677) { 10};
\draw[gp path] (4.985,0.985)--(4.985,1.165);
\node[gp node center] at (4.985,0.677) { 15};
\draw[gp path] (6.145,0.985)--(6.145,1.165);
\node[gp node center] at (6.145,0.677) { 20};
\draw[gp path] (7.306,0.985)--(7.306,1.165);
\node[gp node center] at (7.306,0.677) { 25};
\draw[gp path] (8.466,0.985)--(8.466,1.165);
\node[gp node center] at (8.466,0.677) { 30};
\draw[gp path] (9.626,0.985)--(9.626,1.165);
\node[gp node center] at (9.626,0.677) { 35};
\draw[gp path] (10.787,0.985)--(10.787,1.165);
\node[gp node center] at (10.787,0.677) { 40};
\draw[gp path] (1.504,7.298)--(1.504,1.498);
\draw[gp path] (1.504,0.985)--(10.845,0.985);
\node[gp node center,rotate=-270] at (0.246,4.405) {Ampiezza [???]};
\node[gp node center] at (6.725,0.215) {Tempo $[s]$};
\node[gp node center] at (6.725,8.287) {Dati decadimento 0.970d};
\gpcolor{rgb color={0.000,0.000,1.000}}
\gpsetlinetype{gp lt plot 0}
\draw[gp path] (1.504,5.237)--(1.516,4.325)--(1.527,3.436)--(1.539,2.649)--(1.550,2.022)%
  --(1.562,1.623)--(1.574,1.498)--(1.585,1.612)--(1.597,2.000)--(1.608,2.592)--(1.620,3.333)%
  --(1.632,4.211)--(1.643,5.089)--(1.655,5.898)--(1.666,6.560)--(1.678,7.027)--(1.690,7.278)%
  --(1.701,7.244)--(1.713,6.947)--(1.724,6.434)--(1.736,5.739)--(1.748,4.918)--(1.759,4.052)%
  --(1.771,3.231)--(1.782,2.535)--(1.794,1.988)--(1.806,1.703)--(1.817,1.646)--(1.829,1.829)%
  --(1.840,2.262)--(1.852,2.889)--(1.864,3.641)--(1.875,4.485)--(1.887,5.294)--(1.899,6.035)%
  --(1.910,6.628)--(1.922,6.993)--(1.933,7.130)--(1.945,7.027)--(1.957,6.685)--(1.968,6.138)%
  --(1.980,5.420)--(1.991,4.633)--(2.003,3.824)--(2.015,3.071)--(2.026,2.456)--(2.038,2.011)%
  --(2.049,1.794)--(2.061,1.817)--(2.073,2.068)--(2.084,2.547)--(2.096,3.185)--(2.107,3.938)%
  --(2.119,4.713)--(2.131,5.499)--(2.142,6.149)--(2.154,6.639)--(2.165,6.924)--(2.177,6.993)%
  --(2.189,6.810)--(2.200,6.411)--(2.212,5.830)--(2.223,5.135)--(2.235,4.371)--(2.247,3.618)%
  --(2.258,2.923)--(2.270,2.399)--(2.281,2.034)--(2.293,1.897)--(2.305,2.000)--(2.316,2.307)%
  --(2.328,2.820)--(2.339,3.470)--(2.351,4.200)--(2.363,4.964)--(2.374,5.636)--(2.386,6.240)%
  --(2.397,6.628)--(2.409,6.867)--(2.421,6.833)--(2.432,6.594)--(2.444,6.161)--(2.455,5.556)%
  --(2.467,4.884)--(2.479,4.143)--(2.490,3.436)--(2.502,2.843)--(2.513,2.376)--(2.525,2.102)%
  --(2.537,2.022)--(2.548,2.193)--(2.560,2.547)--(2.572,3.071)--(2.583,3.721)--(2.595,4.439)%
  --(2.606,5.135)--(2.618,5.773)--(2.630,6.263)--(2.641,6.617)--(2.653,6.742)--(2.664,6.651)%
  --(2.676,6.366)--(2.688,5.910)--(2.699,5.306)--(2.711,4.622)--(2.722,3.926)--(2.734,3.288)%
  --(2.746,2.752)--(2.757,2.364)--(2.769,2.171)--(2.780,2.182)--(2.792,2.387)--(2.804,2.775)%
  --(2.815,3.333)--(2.827,3.972)--(2.838,4.644)--(2.850,5.306)--(2.862,5.876)--(2.873,6.309)%
  --(2.885,6.560)--(2.896,6.617)--(2.908,6.480)--(2.920,6.149)--(2.931,5.659)--(2.943,5.055)%
  --(2.954,4.405)--(2.966,3.767)--(2.978,3.174)--(2.989,2.684)--(3.001,2.387)--(3.012,2.250)%
  --(3.024,2.330)--(3.036,2.592)--(3.047,3.014)--(3.059,3.573)--(3.070,4.200)--(3.082,4.838)%
  --(3.094,5.442)--(3.105,5.944)--(3.117,6.309)--(3.128,6.491)--(3.140,6.491)--(3.152,6.286)%
  --(3.163,5.921)--(3.175,5.408)--(3.186,4.827)--(3.198,4.223)--(3.210,3.607)--(3.221,3.060)%
  --(3.233,2.661)--(3.245,2.433)--(3.256,2.364)--(3.268,2.501)--(3.279,2.798)--(3.291,3.242)%
  --(3.303,3.789)--(3.314,4.405)--(3.326,5.009)--(3.337,5.556)--(3.349,6.001)--(3.361,6.297)%
  --(3.372,6.400)--(3.384,6.343)--(3.395,6.092)--(3.407,5.705)--(3.419,5.203)--(3.430,4.633)%
  --(3.442,4.017)--(3.453,3.459)--(3.465,3.003)--(3.477,2.661)--(3.488,2.501)--(3.500,2.501)%
  --(3.511,2.672)--(3.523,3.003)--(3.535,3.447)--(3.546,4.006)--(3.558,4.587)--(3.569,5.157)%
  --(3.581,5.636)--(3.593,6.012)--(3.604,6.240)--(3.616,6.309)--(3.627,6.183)--(3.639,5.910)%
  --(3.651,5.477)--(3.662,4.998)--(3.674,4.428)--(3.685,3.869)--(3.697,3.345)--(3.709,2.946)%
  --(3.720,2.672)--(3.732,2.570)--(3.743,2.627)--(3.755,2.843)--(3.767,3.197)--(3.778,3.664)%
  --(3.790,4.211)--(3.801,4.758)--(3.813,5.260)--(3.825,5.705)--(3.836,6.024)--(3.848,6.183)%
  --(3.859,6.183)--(3.871,6.024)--(3.883,5.716)--(3.894,5.294)--(3.906,4.793)--(3.917,4.257)%
  --(3.929,3.721)--(3.941,3.265)--(3.952,2.923)--(3.964,2.718)--(3.976,2.649)--(3.987,2.752)%
  --(3.999,3.014)--(4.010,3.390)--(4.022,3.869)--(4.034,4.394)--(4.045,4.907)--(4.057,5.374)%
  --(4.068,5.750)--(4.080,6.001)--(4.092,6.104)--(4.103,6.058)--(4.115,5.864)--(4.126,5.534)%
  --(4.138,5.089)--(4.150,4.587)--(4.161,4.086)--(4.173,3.607)--(4.184,3.208)--(4.196,2.923)%
  --(4.208,2.763)--(4.219,2.763)--(4.231,2.900)--(4.242,3.185)--(4.254,3.584)--(4.266,4.052)%
  --(4.277,4.553)--(4.289,5.032)--(4.300,5.465)--(4.312,5.784)--(4.324,5.978)--(4.335,6.035)%
  --(4.347,5.933)--(4.358,5.727)--(4.370,5.363)--(4.382,4.918)--(4.393,4.428)--(4.405,3.960)%
  --(4.416,3.504)--(4.428,3.162)--(4.440,2.934)--(4.451,2.820)--(4.463,2.866)--(4.474,3.060)%
  --(4.486,3.368)--(4.498,3.767)--(4.509,4.223)--(4.521,4.701)--(4.532,5.135)--(4.544,5.511)%
  --(4.556,5.784)--(4.567,5.944)--(4.579,5.944)--(4.590,5.807)--(4.602,5.545)--(4.614,5.192)%
  --(4.625,4.747)--(4.637,4.291)--(4.649,3.835)--(4.660,3.447)--(4.672,3.151)--(4.683,2.957)%
  --(4.695,2.900)--(4.707,3.014)--(4.718,3.219)--(4.730,3.539)--(4.741,3.938)--(4.753,4.382)%
  --(4.765,4.815)--(4.776,5.226)--(4.788,5.545)--(4.799,5.773)--(4.811,5.876)--(4.823,5.830)%
  --(4.834,5.670)--(4.846,5.374)--(4.857,5.009)--(4.869,4.587)--(4.881,4.154)--(4.892,3.732)%
  --(4.904,3.390)--(4.915,3.140)--(4.927,3.003)--(4.939,3.003)--(4.950,3.105)--(4.962,3.356)%
  --(4.973,3.687)--(4.985,4.097)--(4.997,4.519)--(5.008,4.929)--(5.020,5.306)--(5.031,5.579)%
  --(5.043,5.750)--(5.055,5.807)--(5.066,5.727)--(5.078,5.522)--(5.089,5.214)--(5.101,4.850)%
  --(5.113,4.428)--(5.124,4.029)--(5.136,3.653)--(5.147,3.333)--(5.159,3.140)--(5.171,3.048)%
  --(5.182,3.083)--(5.194,3.242)--(5.205,3.504)--(5.217,3.858)--(5.229,4.234)--(5.240,4.656)%
  --(5.252,5.032)--(5.263,5.351)--(5.275,5.579)--(5.287,5.716)--(5.298,5.727)--(5.310,5.602)%
  --(5.321,5.374)--(5.333,5.066)--(5.345,4.701)--(5.356,4.302)--(5.368,3.915)--(5.380,3.573)%
  --(5.391,3.322)--(5.403,3.162)--(5.414,3.117)--(5.426,3.185)--(5.438,3.368)--(5.449,3.630)%
  --(5.461,3.983)--(5.472,4.371)--(5.484,4.747)--(5.496,5.100)--(5.507,5.385)--(5.519,5.579)%
  --(5.530,5.670)--(5.542,5.636)--(5.554,5.477)--(5.565,5.237)--(5.577,4.929)--(5.588,4.565)%
  --(5.600,4.188)--(5.612,3.824)--(5.623,3.527)--(5.635,3.311)--(5.646,3.197)--(5.658,3.185)%
  --(5.670,3.299)--(5.681,3.493)--(5.693,3.778)--(5.704,4.131)--(5.716,4.496)--(5.728,4.850)%
  --(5.739,5.157)--(5.751,5.420)--(5.762,5.568)--(5.774,5.602)--(5.786,5.534)--(5.797,5.363)%
  --(5.809,5.123)--(5.820,4.793)--(5.832,4.439)--(5.844,4.086)--(5.855,3.755)--(5.867,3.504)%
  --(5.878,3.311)--(5.890,3.242)--(5.902,3.265)--(5.913,3.402)--(5.925,3.618)--(5.936,3.915)%
  --(5.948,4.257)--(5.960,4.610)--(5.971,4.929)--(5.983,5.226)--(5.994,5.431)--(6.006,5.522)%
  --(6.018,5.534)--(6.029,5.431)--(6.041,5.237)--(6.053,4.975)--(6.064,4.656)--(6.076,4.314)%
  --(6.087,3.983)--(6.099,3.698)--(6.111,3.459)--(6.122,3.333)--(6.134,3.299)--(6.145,3.345)%
  --(6.157,3.504)--(6.169,3.744)--(6.180,4.040)--(6.192,4.382)--(6.203,4.690)--(6.215,5.009)%
  --(6.227,5.249)--(6.238,5.420)--(6.250,5.488)--(6.261,5.454)--(6.273,5.328)--(6.285,5.135)%
  --(6.296,4.850)--(6.308,4.542)--(6.319,4.211)--(6.331,3.903)--(6.343,3.641)--(6.354,3.459)%
  --(6.366,3.356)--(6.377,3.345)--(6.389,3.436)--(6.401,3.607)--(6.412,3.869)--(6.424,4.154)%
  --(6.435,4.473)--(6.447,4.781)--(6.459,5.055)--(6.470,5.271)--(6.482,5.397)--(6.493,5.431)%
  --(6.505,5.374)--(6.517,5.237)--(6.528,5.009)--(6.540,4.736)--(6.551,4.428)--(6.563,4.120)%
  --(6.575,3.846)--(6.586,3.618)--(6.598,3.459)--(6.609,3.402)--(6.621,3.413)--(6.633,3.539)%
  --(6.644,3.721)--(6.656,3.983)--(6.667,4.268)--(6.679,4.565)--(6.691,4.872)--(6.702,5.112)%
  --(6.714,5.271)--(6.726,5.363)--(6.737,5.385)--(6.749,5.294)--(6.760,5.123)--(6.772,4.884)%
  --(6.784,4.610)--(6.795,4.337)--(6.807,4.040)--(6.818,3.789)--(6.830,3.596)--(6.842,3.482)%
  --(6.853,3.447)--(6.865,3.493)--(6.876,3.630)--(6.888,3.835)--(6.900,4.086)--(6.911,4.382)%
  --(6.923,4.679)--(6.934,4.929)--(6.946,5.135)--(6.958,5.260)--(6.969,5.340)--(6.981,5.306)%
  --(6.992,5.203)--(7.004,5.021)--(7.016,4.793)--(7.027,4.530)--(7.039,4.234)--(7.050,3.972)%
  --(7.062,3.755)--(7.074,3.584)--(7.085,3.504)--(7.097,3.493)--(7.108,3.573)--(7.120,3.721)%
  --(7.132,3.938)--(7.143,4.211)--(7.155,4.473)--(7.166,4.736)--(7.178,4.952)--(7.190,5.146)%
  --(7.201,5.271)--(7.213,5.283)--(7.224,5.237)--(7.236,5.112)--(7.248,4.918)--(7.259,4.679)%
  --(7.271,4.428)--(7.282,4.166)--(7.294,3.926)--(7.306,3.721)--(7.317,3.584)--(7.329,3.527)%
  --(7.340,3.561)--(7.352,3.641)--(7.364,3.812)--(7.375,4.040)--(7.387,4.280)--(7.398,4.553)%
  --(7.410,4.781)--(7.422,4.998)--(7.433,5.157)--(7.445,5.237)--(7.457,5.237)--(7.468,5.180)%
  --(7.480,5.032)--(7.491,4.827)--(7.503,4.587)--(7.515,4.337)--(7.526,4.097)--(7.538,3.858)%
  --(7.549,3.698)--(7.561,3.607)--(7.573,3.573)--(7.584,3.607)--(7.596,3.732)--(7.607,3.915)%
  --(7.619,4.131)--(7.631,4.371)--(7.642,4.622)--(7.654,4.838)--(7.665,5.021)--(7.677,5.135)%
  --(7.689,5.192)--(7.700,5.180)--(7.712,5.089)--(7.723,4.929)--(7.735,4.724)--(7.747,4.508)%
  --(7.758,4.257)--(7.770,4.029)--(7.781,3.835)--(7.793,3.698)--(7.805,3.618)--(7.816,3.618)%
  --(7.828,3.675)--(7.839,3.824)--(7.851,4.017)--(7.863,4.234)--(7.874,4.462)--(7.886,4.690)%
  --(7.897,4.895)--(7.909,5.043)--(7.921,5.123)--(7.932,5.157)--(7.944,5.123)--(7.955,5.009)%
  --(7.967,4.850)--(7.979,4.656)--(7.990,4.428)--(8.002,4.200)--(8.013,3.983)--(8.025,3.812)%
  --(8.037,3.687)--(8.048,3.641)--(8.060,3.664)--(8.071,3.755)--(8.083,3.915)--(8.095,4.086)%
  --(8.106,4.314)--(8.118,4.530)--(8.130,4.747)--(8.141,4.918)--(8.153,5.055)--(8.164,5.123)%
  --(8.176,5.123)--(8.188,5.066)--(8.199,4.952)--(8.211,4.758)--(8.222,4.565)--(8.234,4.337)%
  --(8.246,4.143)--(8.257,3.938)--(8.269,3.789)--(8.280,3.710)--(8.292,3.687)--(8.304,3.721)%
  --(8.315,3.824)--(8.327,3.972)--(8.338,4.177)--(8.350,4.382)--(8.362,4.587)--(8.373,4.781)%
  --(8.385,4.952)--(8.396,5.055)--(8.408,5.078)--(8.420,5.078)--(8.431,4.998)--(8.443,4.861)%
  --(8.454,4.679)--(8.466,4.485)--(8.478,4.280)--(8.489,4.074)--(8.501,3.903)--(8.512,3.789)%
  --(8.524,3.721)--(8.536,3.732)--(8.547,3.789)--(8.559,3.869)--(8.570,4.063)--(8.582,4.245)%
  --(8.594,4.439)--(8.605,4.656)--(8.617,4.827)--(8.628,4.952)--(8.640,5.032)--(8.652,5.055)%
  --(8.663,5.021)--(8.675,4.918)--(8.686,4.781)--(8.698,4.622)--(8.710,4.405)--(8.721,4.211)%
  --(8.733,4.040)--(8.744,3.881)--(8.756,3.789)--(8.768,3.744)--(8.779,3.767)--(8.791,3.835)%
  --(8.802,3.960)--(8.814,4.120)--(8.826,4.314)--(8.837,4.496)--(8.849,4.690)--(8.861,4.838)%
  --(8.872,4.964)--(8.884,5.009)--(8.895,5.021)--(8.907,4.964)--(8.919,4.850)--(8.930,4.713)%
  --(8.942,4.530)--(8.953,4.348)--(8.965,4.166)--(8.977,3.995)--(8.988,3.869)--(9.000,3.812)%
  --(9.011,3.789)--(9.023,3.812)--(9.035,3.892)--(9.046,4.029)--(9.058,4.200)--(9.069,4.371)%
  --(9.081,4.565)--(9.093,4.724)--(9.104,4.861)--(9.116,4.952)--(9.127,5.009)--(9.139,4.975)%
  --(9.151,4.918)--(9.162,4.793)--(9.174,4.633)--(9.185,4.473)--(9.197,4.291)--(9.209,4.109)%
  --(9.220,3.972)--(9.232,3.881)--(9.243,3.824)--(9.255,3.801)--(9.267,3.858)--(9.278,3.949)%
  --(9.290,4.097)--(9.301,4.268)--(9.313,4.439)--(9.325,4.610)--(9.336,4.758)--(9.348,4.872)%
  --(9.359,4.941)--(9.371,4.964)--(9.383,4.929)--(9.394,4.838)--(9.406,4.724)--(9.417,4.576)%
  --(9.429,4.394)--(9.441,4.223)--(9.452,4.074)--(9.464,3.972)--(9.475,3.869)--(9.487,3.835)%
  --(9.499,3.846)--(9.510,3.915)--(9.522,4.017)--(9.534,4.166)--(9.545,4.325)--(9.557,4.485)%
  --(9.568,4.644)--(9.580,4.781)--(9.592,4.895)--(9.603,4.929)--(9.615,4.918)--(9.626,4.895)%
  --(9.638,4.793)--(9.650,4.667)--(9.661,4.519)--(9.673,4.337)--(9.684,4.188)--(9.696,4.063)%
  --(9.708,3.949)--(9.719,3.869)--(9.731,3.858)--(9.742,3.881)--(9.754,3.972)--(9.766,4.086)%
  --(9.777,4.223)--(9.789,4.382)--(9.800,4.542)--(9.812,4.690)--(9.824,4.804)--(9.835,4.884)%
  --(9.847,4.907)--(9.858,4.907)--(9.870,4.838)--(9.882,4.747)--(9.893,4.599)--(9.905,4.473)%
  --(9.916,4.302)--(9.928,4.166)--(9.940,4.029)--(9.951,3.915)--(9.963,3.892)--(9.974,3.881)%
  --(9.986,3.926)--(9.998,4.029)--(10.009,4.131)--(10.021,4.280)--(10.032,4.428)--(10.044,4.599)%
  --(10.056,4.713)--(10.067,4.815)--(10.079,4.872)--(10.090,4.895)--(10.102,4.861)--(10.114,4.770)%
  --(10.125,4.679)--(10.137,4.542)--(10.148,4.405)--(10.160,4.257)--(10.172,4.109)--(10.183,4.017)%
  --(10.195,3.949)--(10.207,3.892)--(10.218,3.915)--(10.230,3.972)--(10.241,4.063)--(10.253,4.188)%
  --(10.265,4.337)--(10.276,4.485)--(10.288,4.610)--(10.299,4.736)--(10.311,4.815)--(10.323,4.850)%
  --(10.334,4.861)--(10.346,4.804)--(10.357,4.724)--(10.369,4.633)--(10.381,4.485)--(10.392,4.359)%
  --(10.404,4.223)--(10.415,4.097)--(10.427,3.995)--(10.439,3.938)--(10.450,3.926)--(10.462,3.960)%
  --(10.473,4.029)--(10.485,4.120)--(10.497,4.245)--(10.508,4.382)--(10.520,4.519)--(10.531,4.633)%
  --(10.543,4.747)--(10.555,4.815)--(10.566,4.838)--(10.578,4.815)--(10.589,4.781)--(10.601,4.679)%
  --(10.613,4.565)--(10.624,4.451)--(10.636,4.314)--(10.647,4.188)--(10.659,4.074)--(10.671,4.006)%
  --(10.682,3.949)--(10.694,3.949)--(10.705,3.995)--(10.717,4.074)--(10.729,4.166)--(10.740,4.302)%
  --(10.752,4.428)--(10.763,4.565)--(10.775,4.667)--(10.787,4.758)--(10.798,4.804)--(10.810,4.815)%
  --(10.821,4.793)--(10.833,4.736)--(10.845,4.633);
\gpcolor{color=gp lt color border}
\node[gp node right] at (10.479,7.491) {Massimi e minimi};
\gpcolor{rgb color={1.000,0.000,0.000}}
\gpsetpointsize{4.00}
\gppoint{gp mark 3}{(1.706,7.298)}
\gppoint{gp mark 3}{(1.946,7.130)}
\gppoint{gp mark 3}{(2.186,6.999)}
\gppoint{gp mark 3}{(2.425,6.887)}
\gppoint{gp mark 3}{(2.665,6.743)}
\gppoint{gp mark 3}{(2.906,6.621)}
\gppoint{gp mark 3}{(3.146,6.514)}
\gppoint{gp mark 3}{(3.385,6.402)}
\gppoint{gp mark 3}{(3.626,6.311)}
\gppoint{gp mark 3}{(3.865,6.203)}
\gppoint{gp mark 3}{(4.105,6.106)}
\gppoint{gp mark 3}{(4.345,6.037)}
\gppoint{gp mark 3}{(4.585,5.964)}
\gppoint{gp mark 3}{(4.825,5.878)}
\gppoint{gp mark 3}{(5.065,5.808)}
\gppoint{gp mark 3}{(5.305,5.739)}
\gppoint{gp mark 3}{(5.545,5.674)}
\gppoint{gp mark 3}{(5.784,5.603)}
\gppoint{gp mark 3}{(6.025,5.543)}
\gppoint{gp mark 3}{(6.263,5.489)}
\gppoint{gp mark 3}{(6.504,5.432)}
\gppoint{gp mark 3}{(6.745,5.391)}
\gppoint{gp mark 3}{(6.983,5.342)}
\gppoint{gp mark 3}{(7.221,5.285)}
\gppoint{gp mark 3}{(7.462,5.247)}
\gppoint{gp mark 3}{(7.704,5.195)}
\gppoint{gp mark 3}{(7.944,5.157)}
\gppoint{gp mark 3}{(8.182,5.132)}
\gppoint{gp mark 3}{(8.425,5.080)}
\gppoint{gp mark 3}{(8.662,5.055)}
\gppoint{gp mark 3}{(8.903,5.024)}
\gppoint{gp mark 3}{(9.140,5.010)}
\gppoint{gp mark 3}{(9.382,4.964)}
\gppoint{gp mark 3}{(9.618,4.931)}
\gppoint{gp mark 3}{(9.864,4.909)}
\gppoint{gp mark 3}{(10.101,4.895)}
\gppoint{gp mark 3}{(10.342,4.865)}
\gppoint{gp mark 3}{(10.578,4.838)}
\gppoint{gp mark 3}{(10.820,4.816)}
\gppoint{gp mark 3}{(1.585,1.498)}
\gppoint{gp mark 3}{(1.826,1.638)}
\gppoint{gp mark 3}{(2.066,1.775)}
\gppoint{gp mark 3}{(2.305,1.896)}
\gppoint{gp mark 3}{(2.546,2.018)}
\gppoint{gp mark 3}{(2.786,2.150)}
\gppoint{gp mark 3}{(3.026,2.249)}
\gppoint{gp mark 3}{(3.266,2.362)}
\gppoint{gp mark 3}{(3.506,2.481)}
\gppoint{gp mark 3}{(3.745,2.568)}
\gppoint{gp mark 3}{(3.986,2.649)}
\gppoint{gp mark 3}{(4.225,2.743)}
\gppoint{gp mark 3}{(4.465,2.817)}
\gppoint{gp mark 3}{(4.705,2.898)}
\gppoint{gp mark 3}{(4.944,2.986)}
\gppoint{gp mark 3}{(5.185,3.045)}
\gppoint{gp mark 3}{(5.425,3.116)}
\gppoint{gp mark 3}{(5.665,3.175)}
\gppoint{gp mark 3}{(5.905,3.239)}
\gppoint{gp mark 3}{(6.144,3.299)}
\gppoint{gp mark 3}{(6.385,3.337)}
\gppoint{gp mark 3}{(6.625,3.398)}
\gppoint{gp mark 3}{(6.864,3.447)}
\gppoint{gp mark 3}{(7.104,3.487)}
\gppoint{gp mark 3}{(7.342,3.526)}
\gppoint{gp mark 3}{(7.584,3.573)}
\gppoint{gp mark 3}{(7.822,3.608)}
\gppoint{gp mark 3}{(8.062,3.640)}
\gppoint{gp mark 3}{(8.302,3.687)}
\gppoint{gp mark 3}{(8.540,3.716)}
\gppoint{gp mark 3}{(8.781,3.743)}
\gppoint{gp mark 3}{(9.023,3.789)}
\gppoint{gp mark 3}{(9.264,3.799)}
\gppoint{gp mark 3}{(9.502,3.834)}
\gppoint{gp mark 3}{(9.740,3.857)}
\gppoint{gp mark 3}{(9.983,3.878)}
\gppoint{gp mark 3}{(10.221,3.890)}
\gppoint{gp mark 3}{(10.459,3.925)}
\gppoint{gp mark 3}{(10.700,3.942)}
\gppoint{gp mark 3}{(11.121,7.491)}
\gpcolor{color=gp lt color border}
\gpsetlinetype{gp lt border}
\draw[gp path] (1.504,7.298)--(1.504,1.498);
\draw[gp path] (1.504,0.985)--(10.845,0.985);
%% coordinates of the plot area
\gpdefrectangularnode{gp plot 1}{\pgfpoint{1.504cm}{0.985cm}}{\pgfpoint{11.947cm}{7.825cm}}
\end{tikzpicture}
%% gnuplot variables
 \caption{Grafico 0.970dgdecad.tex} \label{gr:0.970dgdecad.tex} \end{grafico}
\begin{grafico} \centering \begin{tikzpicture}[gnuplot]
%% generated with GNUPLOT 4.6p0 (Lua 5.1; terminal rev. 99, script rev. 100)
%% Tue 10 Jun 2014 07:57:01 PM CEST
\path (0.000,0.000) rectangle (12.500,8.750);
\gpcolor{color=gp lt color border}
\gpsetlinetype{gp lt border}
\gpsetlinewidth{1.00}
\draw[gp path] (1.504,0.985)--(1.684,0.985);
\node[gp node right] at (1.320,0.985) {-1};
\draw[gp path] (1.504,2.695)--(1.684,2.695);
\node[gp node right] at (1.320,2.695) {-0.5};
\draw[gp path] (1.504,4.405)--(1.684,4.405);
\node[gp node right] at (1.320,4.405) { 0};
\draw[gp path] (1.504,6.115)--(1.684,6.115);
\node[gp node right] at (1.320,6.115) { 0.5};
\draw[gp path] (1.504,7.825)--(1.684,7.825);
\node[gp node right] at (1.320,7.825) { 1};
\draw[gp path] (1.504,0.985)--(1.504,1.165);
\node[gp node center] at (1.504,0.677) { 0};
\draw[gp path] (2.664,0.985)--(2.664,1.165);
\node[gp node center] at (2.664,0.677) { 5};
\draw[gp path] (3.825,0.985)--(3.825,1.165);
\node[gp node center] at (3.825,0.677) { 10};
\draw[gp path] (4.985,0.985)--(4.985,1.165);
\node[gp node center] at (4.985,0.677) { 15};
\draw[gp path] (6.145,0.985)--(6.145,1.165);
\node[gp node center] at (6.145,0.677) { 20};
\draw[gp path] (7.306,0.985)--(7.306,1.165);
\node[gp node center] at (7.306,0.677) { 25};
\draw[gp path] (8.466,0.985)--(8.466,1.165);
\node[gp node center] at (8.466,0.677) { 30};
\draw[gp path] (9.626,0.985)--(9.626,1.165);
\node[gp node center] at (9.626,0.677) { 35};
\draw[gp path] (10.787,0.985)--(10.787,1.165);
\node[gp node center] at (10.787,0.677) { 40};
\draw[gp path] (1.504,7.825)--(1.504,0.985)--(10.810,0.985);
\node[gp node center,rotate=-270] at (0.246,4.405) {Ampiezza [???]};
\node[gp node center] at (6.725,0.215) {Tempo $[s]$};
\node[gp node center] at (6.725,8.287) {Dati decadimento 0.975d};
\gpcolor{rgb color={0.000,0.000,1.000}}
\gpsetlinetype{gp lt plot 0}
\draw[gp path] (1.504,3.830)--(1.516,3.981)--(1.527,4.155)--(1.539,4.374)--(1.550,4.593)%
  --(1.562,4.785)--(1.574,4.932)--(1.585,5.038)--(1.597,5.089)--(1.608,5.082)--(1.620,4.997)%
  --(1.632,4.860)--(1.643,4.685)--(1.655,4.497)--(1.666,4.285)--(1.678,4.087)--(1.690,3.930)%
  --(1.701,3.803)--(1.713,3.738)--(1.724,3.735)--(1.736,3.789)--(1.748,3.899)--(1.759,4.056)%
  --(1.771,4.244)--(1.782,4.449)--(1.794,4.648)--(1.806,4.809)--(1.817,4.945)--(1.829,5.034)%
  --(1.840,5.068)--(1.852,5.017)--(1.864,4.921)--(1.875,4.795)--(1.887,4.627)--(1.899,4.426)%
  --(1.910,4.220)--(1.922,4.049)--(1.933,3.913)--(1.945,3.807)--(1.957,3.755)--(1.968,3.779)%
  --(1.980,3.851)--(1.991,3.967)--(2.003,4.125)--(2.015,4.313)--(2.026,4.508)--(2.038,4.682)%
  --(2.049,4.833)--(2.061,4.952)--(2.073,5.021)--(2.084,5.014)--(2.096,4.959)--(2.107,4.874)%
  --(2.119,4.730)--(2.131,4.542)--(2.142,4.350)--(2.154,4.179)--(2.165,4.025)--(2.177,3.892)%
  --(2.189,3.810)--(2.200,3.793)--(2.212,3.827)--(2.223,3.906)--(2.235,4.029)--(2.247,4.196)%
  --(2.258,4.381)--(2.270,4.552)--(2.281,4.716)--(2.293,4.856)--(2.305,4.949)--(2.316,4.983)%
  --(2.328,4.976)--(2.339,4.915)--(2.351,4.802)--(2.363,4.644)--(2.374,4.477)--(2.386,4.309)%
  --(2.397,4.135)--(2.409,3.988)--(2.421,3.889)--(2.432,3.834)--(2.444,3.824)--(2.455,3.868)%
  --(2.467,3.964)--(2.479,4.101)--(2.490,4.261)--(2.502,4.426)--(2.513,4.597)--(2.525,4.754)%
  --(2.537,4.867)--(2.548,4.935)--(2.560,4.959)--(2.572,4.932)--(2.583,4.853)--(2.595,4.737)%
  --(2.606,4.600)--(2.618,4.432)--(2.630,4.248)--(2.641,4.097)--(2.653,3.981)--(2.664,3.889)%
  --(2.676,3.841)--(2.688,3.851)--(2.699,3.919)--(2.711,4.022)--(2.722,4.152)--(2.734,4.313)%
  --(2.746,4.480)--(2.757,4.638)--(2.769,4.764)--(2.780,4.867)--(2.792,4.925)--(2.804,4.925)%
  --(2.815,4.880)--(2.827,4.805)--(2.838,4.682)--(2.850,4.525)--(2.862,4.361)--(2.873,4.217)%
  --(2.885,4.080)--(2.896,3.964)--(2.908,3.892)--(2.920,3.875)--(2.931,3.906)--(2.943,3.967)%
  --(2.954,4.077)--(2.966,4.220)--(2.978,4.378)--(2.989,4.525)--(3.001,4.665)--(3.012,4.788)%
  --(3.024,4.867)--(3.036,4.901)--(3.047,4.894)--(3.059,4.843)--(3.070,4.747)--(3.082,4.610)%
  --(3.094,4.470)--(3.105,4.326)--(3.117,4.176)--(3.128,4.049)--(3.140,3.960)--(3.152,3.916)%
  --(3.163,3.906)--(3.175,3.940)--(3.186,4.025)--(3.198,4.142)--(3.210,4.275)--(3.221,4.415)%
  --(3.233,4.562)--(3.245,4.699)--(3.256,4.798)--(3.268,4.856)--(3.279,4.877)--(3.291,4.860)%
  --(3.303,4.791)--(3.314,4.685)--(3.326,4.566)--(3.337,4.426)--(3.349,4.278)--(3.361,4.138)%
  --(3.372,4.042)--(3.384,3.967)--(3.395,3.926)--(3.407,3.933)--(3.419,3.988)--(3.430,4.077)%
  --(3.442,4.186)--(3.453,4.323)--(3.465,4.467)--(3.477,4.603)--(3.488,4.716)--(3.500,4.798)%
  --(3.511,4.853)--(3.523,4.856)--(3.535,4.815)--(3.546,4.740)--(3.558,4.641)--(3.569,4.514)%
  --(3.581,4.374)--(3.593,4.241)--(3.604,4.121)--(3.616,4.032)--(3.627,3.967)--(3.639,3.950)%
  --(3.651,3.971)--(3.662,4.029)--(3.674,4.121)--(3.685,4.241)--(3.697,4.374)--(3.709,4.508)%
  --(3.720,4.624)--(3.732,4.730)--(3.743,4.802)--(3.755,4.833)--(3.767,4.826)--(3.778,4.778)%
  --(3.790,4.699)--(3.801,4.590)--(3.813,4.467)--(3.825,4.333)--(3.836,4.210)--(3.848,4.101)%
  --(3.859,4.025)--(3.871,3.984)--(3.883,3.974)--(3.894,4.005)--(3.906,4.073)--(3.917,4.172)%
  --(3.929,4.292)--(3.941,4.419)--(3.952,4.538)--(3.964,4.651)--(3.976,4.740)--(3.987,4.795)%
  --(3.999,4.815)--(4.010,4.791)--(4.022,4.740)--(4.034,4.655)--(4.045,4.545)--(4.057,4.422)%
  --(4.068,4.299)--(4.080,4.183)--(4.092,4.094)--(4.103,4.032)--(4.115,3.995)--(4.126,4.001)%
  --(4.138,4.042)--(4.150,4.121)--(4.161,4.217)--(4.173,4.333)--(4.184,4.453)--(4.196,4.573)%
  --(4.208,4.668)--(4.219,4.740)--(4.231,4.781)--(4.242,4.791)--(4.254,4.761)--(4.266,4.699)%
  --(4.277,4.610)--(4.289,4.497)--(4.300,4.378)--(4.312,4.268)--(4.324,4.162)--(4.335,4.080)%
  --(4.347,4.029)--(4.358,4.015)--(4.370,4.029)--(4.382,4.080)--(4.393,4.159)--(4.405,4.261)%
  --(4.416,4.374)--(4.428,4.491)--(4.440,4.597)--(4.451,4.682)--(4.463,4.744)--(4.474,4.768)%
  --(4.486,4.764)--(4.498,4.726)--(4.509,4.658)--(4.521,4.562)--(4.532,4.453)--(4.544,4.347)%
  --(4.556,4.241)--(4.567,4.145)--(4.579,4.077)--(4.590,4.039)--(4.602,4.032)--(4.614,4.056)%
  --(4.625,4.114)--(4.637,4.200)--(4.649,4.306)--(4.660,4.415)--(4.672,4.521)--(4.683,4.617)%
  --(4.695,4.689)--(4.707,4.737)--(4.718,4.754)--(4.730,4.740)--(4.741,4.692)--(4.753,4.617)%
  --(4.765,4.521)--(4.776,4.419)--(4.788,4.313)--(4.799,4.213)--(4.811,4.131)--(4.823,4.077)%
  --(4.834,4.056)--(4.846,4.056)--(4.857,4.090)--(4.869,4.155)--(4.881,4.244)--(4.892,4.340)%
  --(4.904,4.446)--(4.915,4.549)--(4.927,4.634)--(4.939,4.696)--(4.950,4.730)--(4.962,4.737)%
  --(4.973,4.713)--(4.985,4.658)--(4.997,4.576)--(5.008,4.487)--(5.020,4.391)--(5.031,4.285)%
  --(5.043,4.196)--(5.055,4.128)--(5.066,4.084)--(5.078,4.066)--(5.089,4.077)--(5.101,4.125)%
  --(5.113,4.196)--(5.124,4.275)--(5.136,4.374)--(5.147,4.477)--(5.159,4.566)--(5.171,4.641)%
  --(5.182,4.696)--(5.194,4.723)--(5.205,4.713)--(5.217,4.679)--(5.229,4.620)--(5.240,4.549)%
  --(5.252,4.453)--(5.263,4.354)--(5.275,4.261)--(5.287,4.186)--(5.298,4.125)--(5.310,4.090)%
  --(5.321,4.084)--(5.333,4.111)--(5.345,4.155)--(5.356,4.227)--(5.368,4.316)--(5.380,4.415)%
  --(5.391,4.504)--(5.403,4.579)--(5.414,4.648)--(5.426,4.692)--(5.438,4.706)--(5.449,4.689)%
  --(5.461,4.651)--(5.472,4.593)--(5.484,4.508)--(5.496,4.415)--(5.507,4.326)--(5.519,4.244)%
  --(5.530,4.169)--(5.542,4.121)--(5.554,4.101)--(5.565,4.107)--(5.577,4.131)--(5.588,4.190)%
  --(5.600,4.268)--(5.612,4.354)--(5.623,4.436)--(5.635,4.518)--(5.646,4.600)--(5.658,4.658)%
  --(5.670,4.682)--(5.681,4.685)--(5.693,4.668)--(5.704,4.624)--(5.716,4.552)--(5.728,4.470)%
  --(5.739,4.391)--(5.751,4.306)--(5.762,4.224)--(5.774,4.162)--(5.786,4.128)--(5.797,4.114)%
  --(5.809,4.121)--(5.820,4.162)--(5.832,4.231)--(5.844,4.299)--(5.855,4.374)--(5.867,4.467)%
  --(5.878,4.549)--(5.890,4.607)--(5.902,4.651)--(5.913,4.675)--(5.925,4.675)--(5.936,4.644)%
  --(5.948,4.590)--(5.960,4.525)--(5.971,4.446)--(5.983,4.364)--(5.994,4.278)--(6.006,4.217)%
  --(6.018,4.166)--(6.029,4.131)--(6.041,4.128)--(6.053,4.152)--(6.064,4.196)--(6.076,4.255)%
  --(6.087,4.326)--(6.099,4.408)--(6.111,4.494)--(6.122,4.559)--(6.134,4.617)--(6.145,4.655)%
  --(6.157,4.665)--(6.169,4.651)--(6.180,4.617)--(6.192,4.562)--(6.203,4.491)--(6.215,4.415)%
  --(6.227,4.340)--(6.238,4.261)--(6.250,4.203)--(6.261,4.159)--(6.273,4.138)--(6.285,4.152)%
  --(6.296,4.169)--(6.308,4.213)--(6.319,4.282)--(6.331,4.357)--(6.343,4.436)--(6.354,4.504)%
  --(6.366,4.569)--(6.377,4.617)--(6.389,4.648)--(6.401,4.651)--(6.412,4.631)--(6.424,4.590)%
  --(6.435,4.535)--(6.447,4.463)--(6.459,4.391)--(6.470,4.316)--(6.482,4.248)--(6.493,4.196)%
  --(6.505,4.162)--(6.517,4.152)--(6.528,4.162)--(6.540,4.193)--(6.551,4.244)--(6.563,4.309)%
  --(6.575,4.384)--(6.586,4.453)--(6.598,4.525)--(6.609,4.579)--(6.621,4.617)--(6.633,4.634)%
  --(6.644,4.631)--(6.656,4.607)--(6.667,4.566)--(6.679,4.508)--(6.691,4.439)--(6.702,4.364)%
  --(6.714,4.296)--(6.726,4.237)--(6.737,4.193)--(6.749,4.172)--(6.760,4.166)--(6.772,4.183)%
  --(6.784,4.220)--(6.795,4.272)--(6.807,4.337)--(6.818,4.408)--(6.830,4.477)--(6.842,4.535)%
  --(6.853,4.583)--(6.865,4.617)--(6.876,4.624)--(6.888,4.617)--(6.900,4.586)--(6.911,4.542)%
  --(6.923,4.480)--(6.934,4.412)--(6.946,4.347)--(6.958,4.282)--(6.969,4.231)--(6.981,4.196)%
  --(6.992,4.179)--(7.004,4.179)--(7.016,4.203)--(7.027,4.244)--(7.039,4.302)--(7.050,4.364)%
  --(7.062,4.432)--(7.074,4.494)--(7.085,4.545)--(7.097,4.590)--(7.108,4.610)--(7.120,4.614)%
  --(7.132,4.600)--(7.143,4.566)--(7.155,4.514)--(7.166,4.456)--(7.178,4.391)--(7.190,4.326)%
  --(7.201,4.265)--(7.213,4.224)--(7.224,4.196)--(7.236,4.186)--(7.248,4.196)--(7.259,4.224)%
  --(7.271,4.268)--(7.282,4.323)--(7.294,4.381)--(7.306,4.449)--(7.317,4.504)--(7.329,4.555)%
  --(7.340,4.590)--(7.352,4.603)--(7.364,4.600)--(7.375,4.583)--(7.387,4.542)--(7.398,4.487)%
  --(7.410,4.432)--(7.422,4.367)--(7.433,4.309)--(7.445,4.261)--(7.457,4.220)--(7.468,4.200)%
  --(7.480,4.193)--(7.491,4.210)--(7.503,4.244)--(7.515,4.292)--(7.526,4.350)--(7.538,4.405)%
  --(7.549,4.467)--(7.561,4.518)--(7.573,4.559)--(7.584,4.586)--(7.596,4.603)--(7.607,4.586)%
  --(7.619,4.559)--(7.631,4.521)--(7.642,4.473)--(7.654,4.408)--(7.665,4.347)--(7.677,4.302)%
  --(7.689,4.258)--(7.700,4.220)--(7.712,4.203)--(7.723,4.210)--(7.735,4.227)--(7.747,4.265)%
  --(7.758,4.309)--(7.770,4.371)--(7.781,4.426)--(7.793,4.477)--(7.805,4.525)--(7.816,4.562)%
  --(7.828,4.583)--(7.839,4.583)--(7.851,4.569)--(7.863,4.542)--(7.874,4.497)--(7.886,4.446)%
  --(7.897,4.388)--(7.909,4.340)--(7.921,4.289)--(7.932,4.241)--(7.944,4.224)--(7.955,4.213)%
  --(7.967,4.220)--(7.979,4.248)--(7.990,4.285)--(8.002,4.333)--(8.013,4.388)--(8.025,4.443)%
  --(8.037,4.494)--(8.048,4.532)--(8.060,4.566)--(8.071,4.573)--(8.083,4.576)--(8.095,4.559)%
  --(8.106,4.518)--(8.118,4.473)--(8.130,4.429)--(8.141,4.378)--(8.153,4.320)--(8.164,4.275)%
  --(8.176,4.241)--(8.188,4.227)--(8.199,4.220)--(8.211,4.234)--(8.222,4.265)--(8.234,4.306)%
  --(8.246,4.350)--(8.257,4.405)--(8.269,4.460)--(8.280,4.504)--(8.292,4.535)--(8.304,4.562)%
  --(8.315,4.569)--(8.327,4.562)--(8.338,4.535)--(8.350,4.501)--(8.362,4.456)--(8.373,4.408)%
  --(8.385,4.350)--(8.396,4.306)--(8.408,4.275)--(8.420,4.244)--(8.431,4.231)--(8.443,4.231)%
  --(8.454,4.251)--(8.466,4.282)--(8.478,4.323)--(8.489,4.374)--(8.501,4.426)--(8.512,4.470)%
  --(8.524,4.508)--(8.536,4.538)--(8.547,4.559)--(8.559,4.559)--(8.570,4.542)--(8.582,4.521)%
  --(8.594,4.484)--(8.605,4.443)--(8.617,4.391)--(8.628,4.340)--(8.640,4.299)--(8.652,4.265)%
  --(8.663,4.241)--(8.675,4.241)--(8.686,4.248)--(8.698,4.268)--(8.710,4.299)--(8.721,4.343)%
  --(8.733,4.395)--(8.744,4.436)--(8.756,4.480)--(8.768,4.518)--(8.779,4.538)--(8.791,4.555)%
  --(8.802,4.549)--(8.814,4.535)--(8.826,4.504)--(8.837,4.467)--(8.849,4.422)--(8.861,4.374)%
  --(8.872,4.330)--(8.884,4.292)--(8.895,4.265)--(8.907,4.248)--(8.919,4.248)--(8.930,4.258)%
  --(8.942,4.285)--(8.953,4.316)--(8.965,4.361)--(8.977,4.405)--(8.988,4.453)--(9.000,4.491)%
  --(9.011,4.518)--(9.023,4.535)--(9.035,4.545)--(9.046,4.538)--(9.058,4.518)--(9.069,4.491)%
  --(9.081,4.453)--(9.093,4.408)--(9.104,4.361)--(9.116,4.316)--(9.127,4.285)--(9.139,4.261)%
  --(9.151,4.248)--(9.162,4.255)--(9.174,4.268)--(9.185,4.296)--(9.197,4.333)--(9.209,4.374)%
  --(9.220,4.426)--(9.232,4.460)--(9.243,4.494)--(9.255,4.525)--(9.267,4.538)--(9.278,4.535)%
  --(9.290,4.528)--(9.301,4.508)--(9.313,4.473)--(9.325,4.432)--(9.336,4.388)--(9.348,4.350)%
  --(9.359,4.316)--(9.371,4.278)--(9.383,4.265)--(9.394,4.261)--(9.406,4.265)--(9.417,4.285)%
  --(9.429,4.313)--(9.441,4.350)--(9.452,4.391)--(9.464,4.432)--(9.475,4.473)--(9.487,4.501)%
  --(9.499,4.521)--(9.510,4.528)--(9.522,4.532)--(9.534,4.514)--(9.545,4.491)--(9.557,4.453)%
  --(9.568,4.422)--(9.580,4.381)--(9.592,4.340)--(9.603,4.306)--(9.615,4.278)--(9.626,4.272)%
  --(9.638,4.268)--(9.650,4.275)--(9.661,4.299)--(9.673,4.330)--(9.684,4.367)--(9.696,4.408)%
  --(9.708,4.446)--(9.719,4.480)--(9.731,4.504)--(9.742,4.518)--(9.754,4.525)--(9.766,4.525)%
  --(9.777,4.504)--(9.789,4.470)--(9.800,4.439)--(9.812,4.402)--(9.824,4.364)--(9.835,4.330)%
  --(9.847,4.299)--(9.858,4.282)--(9.870,4.272)--(9.882,4.275)--(9.893,4.292)--(9.905,4.313)%
  --(9.916,4.343)--(9.928,4.381)--(9.940,4.419)--(9.951,4.456)--(9.963,4.480)--(9.974,4.508)%
  --(9.986,4.518)--(9.998,4.525)--(10.009,4.511)--(10.021,4.487)--(10.032,4.463)--(10.044,4.429)%
  --(10.056,4.388)--(10.067,4.357)--(10.079,4.326)--(10.090,4.299)--(10.102,4.282)--(10.114,4.278)%
  --(10.125,4.289)--(10.137,4.302)--(10.148,4.326)--(10.160,4.361)--(10.172,4.395)--(10.183,4.429)%
  --(10.195,4.463)--(10.207,4.487)--(10.218,4.508)--(10.230,4.518)--(10.241,4.521)--(10.253,4.504)%
  --(10.265,4.484)--(10.276,4.446)--(10.288,4.415)--(10.299,4.378)--(10.311,4.347)--(10.323,4.316)%
  --(10.334,4.296)--(10.346,4.282)--(10.357,4.285)--(10.369,4.296)--(10.381,4.309)--(10.392,4.343)%
  --(10.404,4.378)--(10.415,4.402)--(10.427,4.439)--(10.439,4.467)--(10.450,4.491)--(10.462,4.511)%
  --(10.473,4.504)--(10.485,4.504)--(10.497,4.494)--(10.508,4.463)--(10.520,4.432)--(10.531,4.398)%
  --(10.543,4.371)--(10.555,4.340)--(10.566,4.313)--(10.578,4.299)--(10.589,4.285)--(10.601,4.289)%
  --(10.613,4.306)--(10.624,4.326)--(10.636,4.350)--(10.647,4.384)--(10.659,4.412)--(10.671,4.449)%
  --(10.682,4.477)--(10.694,4.487)--(10.705,4.504)--(10.717,4.504)--(10.729,4.501)--(10.740,4.480)%
  --(10.752,4.453)--(10.763,4.422)--(10.775,4.388)--(10.787,4.357)--(10.798,4.330)--(10.810,4.316);
\gpcolor{color=gp lt color border}
\node[gp node right] at (10.479,7.491) {Punti tangenza};
\gpcolor{rgb color={1.000,0.000,0.000}}
\gpsetpointsize{4.00}
\gppoint{gp mark 3}{(1.524,4.101)}
\gppoint{gp mark 3}{(1.603,5.125)}
\gppoint{gp mark 3}{(1.719,3.707)}
\gppoint{gp mark 3}{(1.840,5.068)}
\gppoint{gp mark 3}{(1.962,3.743)}
\gppoint{gp mark 3}{(2.078,5.041)}
\gppoint{gp mark 3}{(2.200,3.793)}
\gppoint{gp mark 3}{(2.322,5.007)}
\gppoint{gp mark 3}{(2.438,3.801)}
\gppoint{gp mark 3}{(2.560,4.959)}
\gppoint{gp mark 3}{(2.682,3.817)}
\gppoint{gp mark 3}{(2.798,4.947)}
\gppoint{gp mark 3}{(2.920,3.875)}
\gppoint{gp mark 3}{(3.041,4.920)}
\gppoint{gp mark 3}{(3.157,3.889)}
\gppoint{gp mark 3}{(3.279,4.877)}
\gppoint{gp mark 3}{(3.401,3.906)}
\gppoint{gp mark 3}{(3.517,4.877)}
\gppoint{gp mark 3}{(3.639,3.950)}
\gppoint{gp mark 3}{(3.761,4.849)}
\gppoint{gp mark 3}{(3.877,3.959)}
\gppoint{gp mark 3}{(3.999,4.815)}
\gppoint{gp mark 3}{(4.121,3.981)}
\gppoint{gp mark 3}{(4.237,4.807)}
\gppoint{gp mark 3}{(4.358,4.015)}
\gppoint{gp mark 3}{(4.480,4.783)}
\gppoint{gp mark 3}{(4.596,4.020)}
\gppoint{gp mark 3}{(4.718,4.754)}
\gppoint{gp mark 3}{(4.840,4.039)}
\gppoint{gp mark 3}{(4.956,4.749)}
\gppoint{gp mark 3}{(5.078,4.066)}
\gppoint{gp mark 3}{(5.200,4.730)}
\gppoint{gp mark 3}{(5.316,4.070)}
\gppoint{gp mark 3}{(5.438,4.706)}
\gppoint{gp mark 3}{(5.559,4.095)}
\gppoint{gp mark 3}{(5.675,4.694)}
\gppoint{gp mark 3}{(5.797,4.114)}
\gppoint{gp mark 3}{(5.919,4.691)}
\gppoint{gp mark 3}{(6.035,4.116)}
\gppoint{gp mark 3}{(6.157,4.665)}
\gppoint{gp mark 3}{(6.279,4.143)}
\gppoint{gp mark 3}{(6.395,4.662)}
\gppoint{gp mark 3}{(6.517,4.152)}
\gppoint{gp mark 3}{(6.638,4.643)}
\gppoint{gp mark 3}{(6.755,4.157)}
\gppoint{gp mark 3}{(6.876,4.624)}
\gppoint{gp mark 3}{(6.998,4.167)}
\gppoint{gp mark 3}{(7.114,4.620)}
\gppoint{gp mark 3}{(7.236,4.186)}
\gppoint{gp mark 3}{(7.358,4.608)}
\gppoint{gp mark 3}{(7.474,4.184)}
\gppoint{gp mark 3}{(7.596,4.603)}
\gppoint{gp mark 3}{(7.718,4.202)}
\gppoint{gp mark 3}{(7.834,4.590)}
\gppoint{gp mark 3}{(7.955,4.213)}
\gppoint{gp mark 3}{(8.077,4.585)}
\gppoint{gp mark 3}{(8.199,4.220)}
\gppoint{gp mark 3}{(8.321,4.576)}
\gppoint{gp mark 3}{(8.437,4.220)}
\gppoint{gp mark 3}{(8.553,4.567)}
\gppoint{gp mark 3}{(8.675,4.241)}
\gppoint{gp mark 3}{(8.797,4.555)}
\gppoint{gp mark 3}{(8.913,4.243)}
\gppoint{gp mark 3}{(9.035,4.545)}
\gppoint{gp mark 3}{(9.156,4.248)}
\gppoint{gp mark 3}{(9.272,4.538)}
\gppoint{gp mark 3}{(9.394,4.261)}
\gppoint{gp mark 3}{(9.516,4.540)}
\gppoint{gp mark 3}{(9.632,4.265)}
\gppoint{gp mark 3}{(9.748,4.525)}
\gppoint{gp mark 3}{(9.870,4.272)}
\gppoint{gp mark 3}{(9.992,4.532)}
\gppoint{gp mark 3}{(10.114,4.278)}
\gppoint{gp mark 3}{(10.236,4.530)}
\gppoint{gp mark 3}{(10.357,4.285)}
\gppoint{gp mark 3}{(10.473,4.504)}
\gppoint{gp mark 3}{(10.589,4.285)}
\gppoint{gp mark 3}{(10.711,4.506)}
\gppoint{gp mark 3}{(11.121,7.491)}
\gpcolor{color=gp lt color border}
\node[gp node right] at (10.479,7.183) {fexp(x)};
\gpcolor{color=gp lt color 2}
\gpsetlinetype{gp lt plot 2}
\draw[gp path] (10.663,7.183)--(11.579,7.183);
\draw[gp path] (1.504,7.825)--(1.598,7.761)--(1.692,7.699)--(1.786,7.638)--(1.880,7.578)%
  --(1.974,7.519)--(2.068,7.461)--(2.162,7.404)--(2.256,7.349)--(2.350,7.294)--(2.444,7.240)%
  --(2.538,7.188)--(2.632,7.136)--(2.726,7.085)--(2.820,7.035)--(2.914,6.987)--(3.008,6.939)%
  --(3.102,6.892)--(3.196,6.845)--(3.290,6.800)--(3.384,6.756)--(3.478,6.712)--(3.572,6.669)%
  --(3.666,6.627)--(3.760,6.586)--(3.854,6.545)--(3.948,6.505)--(4.042,6.466)--(4.136,6.428)%
  --(4.230,6.391)--(4.324,6.354)--(4.418,6.318)--(4.512,6.282)--(4.606,6.247)--(4.700,6.213)%
  --(4.794,6.179)--(4.888,6.146)--(4.982,6.114)--(5.076,6.082)--(5.170,6.051)--(5.264,6.021)%
  --(5.358,5.991)--(5.452,5.961)--(5.546,5.932)--(5.640,5.904)--(5.734,5.876)--(5.828,5.849)%
  --(5.922,5.822)--(6.016,5.796)--(6.110,5.770)--(6.204,5.744)--(6.298,5.719)--(6.392,5.695)%
  --(6.486,5.671)--(6.580,5.648)--(6.674,5.625)--(6.768,5.602)--(6.862,5.580)--(6.956,5.558)%
  --(7.050,5.536)--(7.144,5.515)--(7.238,5.495)--(7.332,5.475)--(7.426,5.455)--(7.520,5.435)%
  --(7.614,5.416)--(7.708,5.397)--(7.802,5.379)--(7.896,5.361)--(7.990,5.343)--(8.084,5.326)%
  --(8.178,5.308)--(8.272,5.292)--(8.366,5.275)--(8.460,5.259)--(8.554,5.243)--(8.648,5.228)%
  --(8.742,5.212)--(8.836,5.197)--(8.930,5.183)--(9.024,5.168)--(9.118,5.154)--(9.212,5.140)%
  --(9.306,5.126)--(9.400,5.113)--(9.494,5.100)--(9.588,5.087)--(9.682,5.074)--(9.776,5.062)%
  --(9.870,5.050)--(9.964,5.038)--(10.058,5.026)--(10.152,5.014)--(10.246,5.003)--(10.340,4.992)%
  --(10.434,4.981)--(10.528,4.970)--(10.622,4.960)--(10.716,4.950)--(10.810,4.939);
\gpcolor{color=gp lt color border}
\node[gp node right] at (10.479,6.875) {fexpneg(x)};
\gpcolor{color=gp lt color 3}
\gpsetlinetype{gp lt plot 3}
\draw[gp path] (10.663,6.875)--(11.579,6.875);
\draw[gp path] (1.504,0.985)--(1.598,1.049)--(1.692,1.111)--(1.786,1.172)--(1.880,1.232)%
  --(1.974,1.291)--(2.068,1.349)--(2.162,1.406)--(2.256,1.461)--(2.350,1.516)--(2.444,1.570)%
  --(2.538,1.622)--(2.632,1.674)--(2.726,1.725)--(2.820,1.775)--(2.914,1.823)--(3.008,1.871)%
  --(3.102,1.918)--(3.196,1.965)--(3.290,2.010)--(3.384,2.054)--(3.478,2.098)--(3.572,2.141)%
  --(3.666,2.183)--(3.760,2.224)--(3.854,2.265)--(3.948,2.305)--(4.042,2.344)--(4.136,2.382)%
  --(4.230,2.419)--(4.324,2.456)--(4.418,2.492)--(4.512,2.528)--(4.606,2.563)--(4.700,2.597)%
  --(4.794,2.631)--(4.888,2.664)--(4.982,2.696)--(5.076,2.728)--(5.170,2.759)--(5.264,2.789)%
  --(5.358,2.819)--(5.452,2.849)--(5.546,2.878)--(5.640,2.906)--(5.734,2.934)--(5.828,2.961)%
  --(5.922,2.988)--(6.016,3.014)--(6.110,3.040)--(6.204,3.066)--(6.298,3.091)--(6.392,3.115)%
  --(6.486,3.139)--(6.580,3.162)--(6.674,3.185)--(6.768,3.208)--(6.862,3.230)--(6.956,3.252)%
  --(7.050,3.274)--(7.144,3.295)--(7.238,3.315)--(7.332,3.335)--(7.426,3.355)--(7.520,3.375)%
  --(7.614,3.394)--(7.708,3.413)--(7.802,3.431)--(7.896,3.449)--(7.990,3.467)--(8.084,3.484)%
  --(8.178,3.502)--(8.272,3.518)--(8.366,3.535)--(8.460,3.551)--(8.554,3.567)--(8.648,3.582)%
  --(8.742,3.598)--(8.836,3.613)--(8.930,3.627)--(9.024,3.642)--(9.118,3.656)--(9.212,3.670)%
  --(9.306,3.684)--(9.400,3.697)--(9.494,3.710)--(9.588,3.723)--(9.682,3.736)--(9.776,3.748)%
  --(9.870,3.760)--(9.964,3.772)--(10.058,3.784)--(10.152,3.796)--(10.246,3.807)--(10.340,3.818)%
  --(10.434,3.829)--(10.528,3.840)--(10.622,3.850)--(10.716,3.860)--(10.810,3.871);
\gpcolor{color=gp lt color border}
\gpsetlinetype{gp lt border}
\draw[gp path] (1.504,7.825)--(1.504,0.985)--(10.810,0.985);
%% coordinates of the plot area
\gpdefrectangularnode{gp plot 1}{\pgfpoint{1.504cm}{0.985cm}}{\pgfpoint{11.947cm}{7.825cm}}
\end{tikzpicture}
%% gnuplot variables
 \caption{Grafico 0.975dgdecad.tex} \label{gr:0.975dgdecad.tex} \end{grafico}
\begin{grafico} \centering \begin{tikzpicture}[gnuplot]
%% generated with GNUPLOT 4.6p0 (Lua 5.1; terminal rev. 99, script rev. 100)
%% Tue 10 Jun 2014 10:23:51 PM CEST
\path (0.000,0.000) rectangle (12.500,8.750);
\gpcolor{color=gp lt color border}
\gpsetlinetype{gp lt border}
\gpsetlinewidth{1.00}
\draw[gp path] (1.688,1.840)--(1.868,1.840);
\node[gp node right] at (1.504,1.840) {-0.15};
\draw[gp path] (1.688,2.695)--(1.868,2.695);
\node[gp node right] at (1.504,2.695) {-0.1};
\draw[gp path] (1.688,3.550)--(1.868,3.550);
\node[gp node right] at (1.504,3.550) {-0.05};
\draw[gp path] (1.688,4.405)--(1.868,4.405);
\node[gp node right] at (1.504,4.405) { 0};
\draw[gp path] (1.688,5.260)--(1.868,5.260);
\node[gp node right] at (1.504,5.260) { 0.05};
\draw[gp path] (1.688,6.115)--(1.868,6.115);
\node[gp node right] at (1.504,6.115) { 0.1};
\draw[gp path] (1.688,6.970)--(1.868,6.970);
\node[gp node right] at (1.504,6.970) { 0.15};
\draw[gp path] (1.688,0.985)--(1.688,1.165);
\node[gp node center] at (1.688,0.677) { 0};
\draw[gp path] (2.714,0.985)--(2.714,1.165);
\node[gp node center] at (2.714,0.677) { 2};
\draw[gp path] (3.740,0.985)--(3.740,1.165);
\node[gp node center] at (3.740,0.677) { 4};
\draw[gp path] (4.766,0.985)--(4.766,1.165);
\node[gp node center] at (4.766,0.677) { 6};
\draw[gp path] (5.792,0.985)--(5.792,1.165);
\node[gp node center] at (5.792,0.677) { 8};
\draw[gp path] (6.818,0.985)--(6.818,1.165);
\node[gp node center] at (6.818,0.677) { 10};
\draw[gp path] (7.843,0.985)--(7.843,1.165);
\node[gp node center] at (7.843,0.677) { 12};
\draw[gp path] (8.869,0.985)--(8.869,1.165);
\node[gp node center] at (8.869,0.677) { 14};
\draw[gp path] (9.895,0.985)--(9.895,1.165);
\node[gp node center] at (9.895,0.677) { 16};
\draw[gp path] (10.921,0.985)--(10.921,1.165);
\node[gp node center] at (10.921,0.677) { 18};
\draw[gp path] (1.688,7.021)--(1.688,1.789);
\draw[gp path] (1.688,0.985)--(11.921,0.985);
\node[gp node center,rotate=-270] at (0.246,4.405) {Ampiezza [giri]};
\node[gp node center] at (6.817,0.215) {Tempo $[s]$};
\node[gp node center] at (6.817,8.287) {Dati decadimento 0.980d};
\gpcolor{rgb color={0.000,0.000,1.000}}
\gpsetlinetype{gp lt plot 0}
\draw[gp path] (1.688,2.678)--(1.714,3.362)--(1.739,4.166)--(1.765,4.918)--(1.791,5.636)%
  --(1.816,6.252)--(1.842,6.696)--(1.868,6.902)--(1.893,6.884)--(1.919,6.611)--(1.944,6.166)%
  --(1.970,5.551)--(1.996,4.798)--(2.021,4.029)--(2.047,3.294)--(2.073,2.678)--(2.098,2.182)%
  --(2.124,1.908)--(2.150,1.874)--(2.175,2.045)--(2.201,2.404)--(2.227,2.952)--(2.252,3.653)%
  --(2.278,4.405)--(2.304,5.123)--(2.329,5.790)--(2.355,6.303)--(2.380,6.645)--(2.406,6.765)%
  --(2.432,6.679)--(2.457,6.389)--(2.483,5.893)--(2.509,5.260)--(2.534,4.542)--(2.560,3.824)%
  --(2.586,3.140)--(2.611,2.592)--(2.637,2.199)--(2.663,2.011)--(2.688,2.011)--(2.714,2.233)%
  --(2.740,2.661)--(2.765,3.225)--(2.791,3.926)--(2.816,4.627)--(2.842,5.311)--(2.868,5.893)%
  --(2.893,6.354)--(2.919,6.594)--(2.945,6.662)--(2.970,6.491)--(2.996,6.149)--(3.022,5.636)%
  --(3.047,5.004)--(3.073,4.320)--(3.099,3.636)--(3.124,3.020)--(3.150,2.541)--(3.176,2.216)%
  --(3.201,2.097)--(3.227,2.182)--(3.252,2.456)--(3.278,2.900)--(3.304,3.499)--(3.329,4.149)%
  --(3.355,4.833)--(3.381,5.448)--(3.406,5.978)--(3.432,6.337)--(3.458,6.525)--(3.483,6.525)%
  --(3.509,6.320)--(3.535,5.893)--(3.560,5.380)--(3.586,4.764)--(3.612,4.097)--(3.637,3.465)%
  --(3.663,2.917)--(3.689,2.507)--(3.714,2.268)--(3.740,2.216)--(3.765,2.353)--(3.791,2.661)%
  --(3.817,3.157)--(3.842,3.755)--(3.868,4.388)--(3.894,5.004)--(3.919,5.585)--(3.945,6.030)%
  --(3.971,6.320)--(3.996,6.440)--(4.022,6.389)--(4.048,6.115)--(4.073,5.688)--(4.099,5.140)%
  --(4.125,4.559)--(4.150,3.926)--(4.176,3.328)--(4.201,2.832)--(4.227,2.524)--(4.253,2.336)%
  --(4.278,2.336)--(4.304,2.541)--(4.330,2.934)--(4.355,3.396)--(4.381,3.978)--(4.407,4.593)%
  --(4.432,5.192)--(4.458,5.670)--(4.484,6.030)--(4.509,6.286)--(4.535,6.354)--(4.561,6.183)%
  --(4.586,5.876)--(4.612,5.465)--(4.637,4.952)--(4.663,4.337)--(4.689,3.704)--(4.714,3.225)%
  --(4.740,2.798)--(4.766,2.524)--(4.791,2.404)--(4.817,2.473)--(4.843,2.746)--(4.868,3.123)%
  --(4.894,3.618)--(4.920,4.200)--(4.945,4.781)--(4.971,5.277)--(4.997,5.722)--(5.022,6.081)%
  --(5.048,6.252)--(5.073,6.201)--(5.099,6.012)--(5.125,5.722)--(5.150,5.277)--(5.176,4.713)%
  --(5.202,4.131)--(5.227,3.618)--(5.253,3.140)--(5.279,2.746)--(5.304,2.541)--(5.330,2.524)%
  --(5.356,2.661)--(5.381,2.900)--(5.407,3.311)--(5.433,3.841)--(5.458,4.371)--(5.484,4.901)%
  --(5.509,5.380)--(5.535,5.807)--(5.561,6.064)--(5.586,6.115)--(5.612,6.081)--(5.638,5.876)%
  --(5.663,5.517)--(5.689,5.021)--(5.715,4.508)--(5.740,4.012)--(5.766,3.482)--(5.792,3.037)%
  --(5.817,2.746)--(5.843,2.627)--(5.869,2.627)--(5.894,2.763)--(5.920,3.088)--(5.945,3.533)%
  --(5.971,4.012)--(5.997,4.525)--(6.022,5.038)--(6.048,5.499)--(6.074,5.807)--(6.099,5.995)%
  --(6.125,6.047)--(6.151,5.961)--(6.176,5.688)--(6.202,5.277)--(6.228,4.850)--(6.253,4.354)%
  --(6.279,3.824)--(6.305,3.362)--(6.330,3.003)--(6.356,2.781)--(6.381,2.678)--(6.407,2.746)%
  --(6.433,2.952)--(6.458,3.294)--(6.484,3.704)--(6.510,4.166)--(6.535,4.696)--(6.561,5.157)%
  --(6.587,5.534)--(6.612,5.790)--(6.638,5.978)--(6.664,5.961)--(6.689,5.790)--(6.715,5.499)%
  --(6.741,5.140)--(6.766,4.662)--(6.792,4.166)--(6.818,3.704)--(6.843,3.294)--(6.869,2.986)%
  --(6.894,2.798)--(6.920,2.763)--(6.946,2.900)--(6.971,3.105)--(6.997,3.465)--(7.023,3.875)%
  --(7.048,4.371)--(7.074,4.833)--(7.100,5.243)--(7.125,5.585)--(7.151,5.841)--(7.177,5.910)%
  --(7.202,5.841)--(7.228,5.670)--(7.254,5.346)--(7.279,4.952)--(7.305,4.508)--(7.330,4.046)%
  --(7.356,3.618)--(7.382,3.242)--(7.407,2.986)--(7.433,2.883)--(7.459,2.883)--(7.484,3.020)%
  --(7.510,3.259)--(7.536,3.636)--(7.561,4.080)--(7.587,4.525)--(7.613,4.935)--(7.638,5.328)%
  --(7.664,5.619)--(7.690,5.790)--(7.715,5.824)--(7.741,5.722)--(7.766,5.499)--(7.792,5.175)%
  --(7.818,4.781)--(7.843,4.354)--(7.869,3.892)--(7.895,3.499)--(7.920,3.225)--(7.946,3.003)%
  --(7.972,2.917)--(7.997,2.986)--(8.023,3.157)--(8.049,3.430)--(8.074,3.807)--(8.100,4.217)%
  --(8.126,4.662)--(8.151,5.055)--(8.177,5.397)--(8.202,5.619)--(8.228,5.756)--(8.254,5.739)%
  --(8.279,5.602)--(8.305,5.363)--(8.331,5.021)--(8.356,4.627)--(8.382,4.200)--(8.408,3.807)%
  --(8.433,3.465)--(8.459,3.174)--(8.485,3.003)--(8.510,2.969)--(8.536,3.088)--(8.562,3.294)%
  --(8.587,3.584)--(8.613,3.960)--(8.638,4.388)--(8.664,4.764)--(8.690,5.123)--(8.715,5.431)%
  --(8.741,5.619)--(8.767,5.688)--(8.792,5.636)--(8.818,5.482)--(8.844,5.209)--(8.869,4.867)%
  --(8.895,4.473)--(8.921,4.080)--(8.946,3.738)--(8.972,3.396)--(8.998,3.157)--(9.023,3.054)%
  --(9.049,3.071)--(9.074,3.191)--(9.100,3.413)--(9.126,3.738)--(9.151,4.114)--(9.177,4.491)%
  --(9.203,4.867)--(9.228,5.209)--(9.254,5.448)--(9.280,5.568)--(9.305,5.619)--(9.331,5.551)%
  --(9.357,5.346)--(9.382,5.038)--(9.408,4.730)--(9.434,4.354)--(9.459,3.995)--(9.485,3.636)%
  --(9.510,3.345)--(9.536,3.174)--(9.562,3.105)--(9.587,3.140)--(9.613,3.311)--(9.639,3.567)%
  --(9.664,3.875)--(9.690,4.217)--(9.716,4.610)--(9.741,4.969)--(9.767,5.243)--(9.793,5.431)%
  --(9.818,5.568)--(9.844,5.551)--(9.870,5.431)--(9.895,5.192)--(9.921,4.935)--(9.946,4.593)%
  --(9.972,4.234)--(9.998,3.875)--(10.023,3.567)--(10.049,3.345)--(10.075,3.208)--(10.100,3.157)%
  --(10.126,3.242)--(10.152,3.430)--(10.177,3.687)--(10.203,4.012)--(10.229,4.371)--(10.254,4.730)%
  --(10.280,5.021)--(10.306,5.277)--(10.331,5.448)--(10.357,5.517)--(10.383,5.448)--(10.408,5.311)%
  --(10.434,5.089)--(10.459,4.781)--(10.485,4.473)--(10.511,4.114)--(10.536,3.789)--(10.562,3.516)%
  --(10.588,3.311)--(10.613,3.225)--(10.639,3.242)--(10.665,3.328)--(10.690,3.533)--(10.716,3.824)%
  --(10.742,4.149)--(10.767,4.473)--(10.793,4.781)--(10.819,5.072)--(10.844,5.294)--(10.870,5.414)%
  --(10.895,5.431)--(10.921,5.346)--(10.947,5.209)--(10.972,4.952)--(10.998,4.644)--(11.024,4.337)%
  --(11.049,4.012)--(11.075,3.721)--(11.101,3.482)--(11.126,3.328)--(11.152,3.276)--(11.178,3.311)%
  --(11.203,3.430)--(11.229,3.653)--(11.255,3.960)--(11.280,4.251)--(11.306,4.576)--(11.331,4.867)%
  --(11.357,5.123)--(11.383,5.277)--(11.408,5.363)--(11.434,5.380)--(11.460,5.277)--(11.485,5.089)%
  --(11.511,4.833)--(11.537,4.542)--(11.562,4.234)--(11.588,3.926)--(11.614,3.670)--(11.639,3.465)%
  --(11.665,3.345)--(11.691,3.311)--(11.716,3.396)--(11.742,3.550)--(11.767,3.789)--(11.793,4.063)%
  --(11.819,4.354)--(11.844,4.679)--(11.870,4.935)--(11.896,5.157)--(11.921,5.294);
\gpcolor{color=gp lt color border}
\node[gp node right] at (10.479,7.491) {Punti tangenza};
\gpcolor{rgb color={1.000,0.000,0.000}}
\gpsetpointsize{4.00}
\gppoint{gp mark 3}{(1.720,3.601)}
\gppoint{gp mark 3}{(1.880,7.021)}
\gppoint{gp mark 3}{(2.137,1.789)}
\gppoint{gp mark 3}{(2.406,6.765)}
\gppoint{gp mark 3}{(2.675,1.900)}
\gppoint{gp mark 3}{(2.932,6.748)}
\gppoint{gp mark 3}{(3.201,2.097)}
\gppoint{gp mark 3}{(3.471,6.628)}
\gppoint{gp mark 3}{(3.740,2.216)}
\gppoint{gp mark 3}{(4.009,6.525)}
\gppoint{gp mark 3}{(4.266,2.233)}
\gppoint{gp mark 3}{(4.522,6.440)}
\gppoint{gp mark 3}{(4.791,2.404)}
\gppoint{gp mark 3}{(5.061,6.295)}
\gppoint{gp mark 3}{(5.330,2.524)}
\gppoint{gp mark 3}{(5.599,6.183)}
\gppoint{gp mark 3}{(5.856,2.558)}
\gppoint{gp mark 3}{(6.112,6.089)}
\gppoint{gp mark 3}{(6.381,2.678)}
\gppoint{gp mark 3}{(6.651,6.047)}
\gppoint{gp mark 3}{(6.920,2.763)}
\gppoint{gp mark 3}{(7.189,5.927)}
\gppoint{gp mark 3}{(7.446,2.815)}
\gppoint{gp mark 3}{(7.702,5.876)}
\gppoint{gp mark 3}{(7.972,2.917)}
\gppoint{gp mark 3}{(8.241,5.807)}
\gppoint{gp mark 3}{(8.510,2.969)}
\gppoint{gp mark 3}{(8.780,5.713)}
\gppoint{gp mark 3}{(9.036,3.011)}
\gppoint{gp mark 3}{(9.292,5.653)}
\gppoint{gp mark 3}{(9.562,3.105)}
\gppoint{gp mark 3}{(9.831,5.611)}
\gppoint{gp mark 3}{(10.100,3.157)}
\gppoint{gp mark 3}{(10.370,5.517)}
\gppoint{gp mark 3}{(10.626,3.199)}
\gppoint{gp mark 3}{(10.883,5.474)}
\gppoint{gp mark 3}{(11.152,3.276)}
\gppoint{gp mark 3}{(11.421,5.431)}
\gppoint{gp mark 3}{(11.691,3.311)}
\gppoint{gp mark 3}{(11.121,7.491)}
\gpcolor{color=gp lt color border}
\gpsetlinetype{gp lt border}
\draw[gp path] (1.688,7.021)--(1.688,1.789);
\draw[gp path] (1.688,0.985)--(11.921,0.985);
%% coordinates of the plot area
\gpdefrectangularnode{gp plot 1}{\pgfpoint{1.688cm}{0.985cm}}{\pgfpoint{11.947cm}{7.825cm}}
\end{tikzpicture}
%% gnuplot variables
 \caption{Grafico 0.980dgdecad.tex} \label{gr:0.980dgdecad.tex} \end{grafico}
\begin{grafico} \centering \begin{tikzpicture}[gnuplot]
%% generated with GNUPLOT 4.6p0 (Lua 5.1; terminal rev. 99, script rev. 100)
%% Tue 10 Jun 2014 10:23:51 PM CEST
\path (0.000,0.000) rectangle (12.500,8.750);
\gpcolor{color=gp lt color border}
\gpsetlinetype{gp lt border}
\gpsetlinewidth{1.00}
\draw[gp path] (1.688,1.840)--(1.868,1.840);
\node[gp node right] at (1.504,1.840) {-0.06};
\draw[gp path] (1.688,2.695)--(1.868,2.695);
\node[gp node right] at (1.504,2.695) {-0.04};
\draw[gp path] (1.688,3.550)--(1.868,3.550);
\node[gp node right] at (1.504,3.550) {-0.02};
\draw[gp path] (1.688,4.405)--(1.868,4.405);
\node[gp node right] at (1.504,4.405) { 0};
\draw[gp path] (1.688,5.260)--(1.868,5.260);
\node[gp node right] at (1.504,5.260) { 0.02};
\draw[gp path] (1.688,6.115)--(1.868,6.115);
\node[gp node right] at (1.504,6.115) { 0.04};
\draw[gp path] (1.688,6.970)--(1.868,6.970);
\node[gp node right] at (1.504,6.970) { 0.06};
\draw[gp path] (1.688,0.985)--(1.688,1.165);
\node[gp node center] at (1.688,0.677) { 0};
\draw[gp path] (3.740,0.985)--(3.740,1.165);
\node[gp node center] at (3.740,0.677) { 5};
\draw[gp path] (5.792,0.985)--(5.792,1.165);
\node[gp node center] at (5.792,0.677) { 10};
\draw[gp path] (7.843,0.985)--(7.843,1.165);
\node[gp node center] at (7.843,0.677) { 15};
\draw[gp path] (9.895,0.985)--(9.895,1.165);
\node[gp node center] at (9.895,0.677) { 20};
\draw[gp path] (1.688,7.697)--(1.688,0.985)--(9.957,0.985);
\node[gp node center,rotate=-270] at (0.246,4.405) {Ampiezza [giri]};
\node[gp node center] at (6.817,0.215) {Tempo $[s]$};
\node[gp node center] at (6.817,8.287) {Dati decadimento 1.000d};
\gpcolor{rgb color={0.000,0.000,1.000}}
\gpsetlinetype{gp lt plot 0}
\draw[gp path] (1.688,5.089)--(1.709,4.191)--(1.729,3.165)--(1.750,2.182)--(1.770,1.541)%
  --(1.791,1.199)--(1.811,0.985)--(1.832,1.071)--(1.852,1.541)--(1.873,2.353)--(1.893,3.123)%
  --(1.914,4.020)--(1.934,5.004)--(1.955,6.030)--(1.975,6.756)--(1.996,7.184)--(2.016,7.483)%
  --(2.037,7.483)--(2.057,7.056)--(2.078,6.414)--(2.098,5.688)--(2.119,4.833)--(2.139,3.807)%
  --(2.160,2.909)--(2.180,2.139)--(2.201,1.626)--(2.221,1.156)--(2.242,1.156)--(2.263,1.413)%
  --(2.283,1.968)--(2.304,2.610)--(2.324,3.422)--(2.345,4.491)--(2.365,5.346)--(2.386,6.158)%
  --(2.406,6.713)--(2.427,7.269)--(2.447,7.398)--(2.468,7.141)--(2.488,6.799)--(2.509,6.201)%
  --(2.529,5.431)--(2.550,4.405)--(2.570,3.550)--(2.591,2.695)--(2.611,2.054)--(2.632,1.498)%
  --(2.652,1.284)--(2.673,1.413)--(2.693,1.669)--(2.714,2.139)--(2.734,2.909)--(2.755,3.892)%
  --(2.775,4.747)--(2.796,5.517)--(2.816,6.286)--(2.837,6.842)--(2.858,7.098)--(2.878,7.098)%
  --(2.899,6.970)--(2.919,6.585)--(2.940,5.859)--(2.960,5.004)--(2.981,4.149)--(3.001,3.294)%
  --(3.022,2.524)--(3.042,1.883)--(3.063,1.584)--(3.083,1.498)--(3.104,1.541)--(3.124,1.883)%
  --(3.145,2.610)--(3.165,3.294)--(3.186,4.106)--(3.206,4.918)--(3.227,5.730)--(3.247,6.414)%
  --(3.268,6.842)--(3.288,6.970)--(3.309,7.056)--(3.329,6.671)--(3.350,6.201)--(3.370,5.474)%
  --(3.391,4.747)--(3.412,3.892)--(3.432,3.037)--(3.453,2.396)--(3.473,1.926)--(3.494,1.626)%
  --(3.514,1.541)--(3.535,1.797)--(3.555,2.225)--(3.576,2.823)--(3.596,3.593)--(3.617,4.448)%
  --(3.637,5.175)--(3.658,5.901)--(3.678,6.457)--(3.699,6.842)--(3.719,7.013)--(3.740,6.799)%
  --(3.760,6.371)--(3.781,5.859)--(3.801,5.175)--(3.822,4.448)--(3.842,3.678)--(3.863,2.909)%
  --(3.883,2.310)--(3.904,1.840)--(3.924,1.669)--(3.945,1.755)--(3.965,2.054)--(3.986,2.481)%
  --(4.007,3.165)--(4.027,3.935)--(4.048,4.619)--(4.068,5.346)--(4.089,6.030)--(4.109,6.500)%
  --(4.130,6.756)--(4.150,6.756)--(4.171,6.585)--(4.191,6.158)--(4.212,5.602)--(4.232,4.875)%
  --(4.253,4.149)--(4.273,3.465)--(4.294,2.781)--(4.314,2.225)--(4.335,1.968)--(4.355,1.797)%
  --(4.376,1.926)--(4.396,2.268)--(4.417,2.781)--(4.437,3.507)--(4.458,4.149)--(4.478,4.875)%
  --(4.499,5.602)--(4.519,6.158)--(4.540,6.500)--(4.561,6.671)--(4.581,6.671)--(4.602,6.371)%
  --(4.622,5.859)--(4.643,5.346)--(4.663,4.704)--(4.684,3.935)--(4.704,3.208)--(4.725,2.652)%
  --(4.745,2.310)--(4.766,1.926)--(4.786,1.926)--(4.807,2.182)--(4.827,2.567)--(4.848,3.080)%
  --(4.868,3.721)--(4.889,4.405)--(4.909,5.132)--(4.930,5.730)--(4.950,6.243)--(4.971,6.457)%
  --(4.991,6.628)--(5.012,6.457)--(5.032,6.072)--(5.053,5.645)--(5.073,5.089)--(5.094,4.362)%
  --(5.115,3.721)--(5.135,3.080)--(5.156,2.610)--(5.176,2.268)--(5.197,2.054)--(5.217,2.097)%
  --(5.238,2.353)--(5.258,2.738)--(5.279,3.294)--(5.299,4.020)--(5.320,4.662)--(5.340,5.260)%
  --(5.361,5.859)--(5.381,6.286)--(5.402,6.414)--(5.422,6.457)--(5.443,6.286)--(5.463,5.901)%
  --(5.484,5.474)--(5.504,4.790)--(5.525,4.149)--(5.545,3.593)--(5.566,2.994)--(5.586,2.524)%
  --(5.607,2.225)--(5.627,2.182)--(5.648,2.268)--(5.668,2.524)--(5.689,3.037)--(5.710,3.550)%
  --(5.730,4.234)--(5.751,4.833)--(5.771,5.474)--(5.792,5.944)--(5.812,6.201)--(5.833,6.371)%
  --(5.853,6.286)--(5.874,6.158)--(5.894,5.688)--(5.915,5.132)--(5.935,4.576)--(5.956,3.978)%
  --(5.976,3.336)--(5.997,2.866)--(6.017,2.524)--(6.038,2.310)--(6.058,2.268)--(6.079,2.481)%
  --(6.099,2.781)--(6.120,3.294)--(6.140,3.807)--(6.161,4.405)--(6.181,5.004)--(6.202,5.517)%
  --(6.222,5.901)--(6.243,6.201)--(6.264,6.286)--(6.284,6.201)--(6.305,5.859)--(6.325,5.474)%
  --(6.346,5.004)--(6.366,4.405)--(6.387,3.764)--(6.407,3.251)--(6.428,2.823)--(6.448,2.481)%
  --(6.469,2.353)--(6.489,2.396)--(6.510,2.610)--(6.530,2.994)--(6.551,3.507)--(6.571,4.020)%
  --(6.592,4.619)--(6.612,5.132)--(6.633,5.602)--(6.653,5.944)--(6.674,6.158)--(6.694,6.158)%
  --(6.715,5.987)--(6.735,5.688)--(6.756,5.346)--(6.776,4.747)--(6.797,4.149)--(6.818,3.636)%
  --(6.838,3.165)--(6.859,2.781)--(6.879,2.567)--(6.900,2.439)--(6.920,2.567)--(6.941,2.823)%
  --(6.961,3.165)--(6.982,3.636)--(7.002,4.277)--(7.023,4.790)--(7.043,5.303)--(7.064,5.602)%
  --(7.084,5.944)--(7.105,6.115)--(7.125,6.030)--(7.146,5.859)--(7.166,5.517)--(7.187,5.004)%
  --(7.207,4.491)--(7.228,3.978)--(7.248,3.465)--(7.269,3.080)--(7.289,2.695)--(7.310,2.567)%
  --(7.330,2.524)--(7.351,2.695)--(7.371,2.952)--(7.392,3.422)--(7.413,3.935)--(7.433,4.405)%
  --(7.454,4.875)--(7.474,5.388)--(7.495,5.773)--(7.515,5.944)--(7.536,5.987)--(7.556,5.987)%
  --(7.577,5.688)--(7.597,5.260)--(7.618,4.833)--(7.638,4.405)--(7.659,3.849)--(7.679,3.336)%
  --(7.700,2.994)--(7.720,2.738)--(7.741,2.652)--(7.761,2.652)--(7.782,2.823)--(7.802,3.208)%
  --(7.823,3.593)--(7.843,4.020)--(7.864,4.576)--(7.884,5.132)--(7.905,5.474)--(7.925,5.730)%
  --(7.946,5.901)--(7.967,5.944)--(7.987,5.773)--(8.008,5.474)--(8.028,5.132)--(8.049,4.662)%
  --(8.069,4.149)--(8.090,3.678)--(8.110,3.294)--(8.131,2.994)--(8.151,2.738)--(8.172,2.695)%
  --(8.192,2.781)--(8.213,2.994)--(8.233,3.294)--(8.254,3.764)--(8.274,4.234)--(8.295,4.747)%
  --(8.315,5.132)--(8.336,5.517)--(8.356,5.730)--(8.377,5.816)--(8.397,5.816)--(8.418,5.645)%
  --(8.438,5.346)--(8.459,4.918)--(8.479,4.448)--(8.500,4.063)--(8.520,3.593)--(8.541,3.208)%
  --(8.562,2.909)--(8.582,2.781)--(8.603,2.823)--(8.623,2.909)--(8.644,3.208)--(8.664,3.507)%
  --(8.685,3.978)--(8.705,4.405)--(8.726,4.875)--(8.746,5.260)--(8.767,5.559)--(8.787,5.730)%
  --(8.808,5.773)--(8.828,5.730)--(8.849,5.474)--(8.869,5.175)--(8.890,4.790)--(8.910,4.405)%
  --(8.931,3.892)--(8.951,3.422)--(8.972,3.165)--(8.992,2.952)--(9.013,2.866)--(9.033,2.866)%
  --(9.054,3.080)--(9.074,3.379)--(9.095,3.721)--(9.116,4.106)--(9.136,4.533)--(9.157,4.961)%
  --(9.177,5.303)--(9.198,5.517)--(9.218,5.688)--(9.239,5.688)--(9.259,5.517)--(9.280,5.303)%
  --(9.300,5.004)--(9.321,4.619)--(9.341,4.191)--(9.362,3.764)--(9.382,3.422)--(9.403,3.165)%
  --(9.423,2.952)--(9.444,2.909)--(9.464,3.037)--(9.485,3.251)--(9.505,3.507)--(9.526,3.849)%
  --(9.546,4.277)--(9.567,4.662)--(9.587,5.046)--(9.608,5.346)--(9.628,5.559)--(9.649,5.602)%
  --(9.670,5.602)--(9.690,5.474)--(9.711,5.260)--(9.731,4.875)--(9.752,4.448)--(9.772,4.063)%
  --(9.793,3.721)--(9.813,3.379)--(9.834,3.080)--(9.854,2.952)--(9.875,3.080)--(9.895,3.165)%
  --(9.916,3.336)--(9.936,3.636)--(9.957,4.063);
\gpcolor{color=gp lt color border}
\node[gp node right] at (10.479,7.491) {Punti tangenza};
\gpcolor{rgb color={1.000,0.000,0.000}}
\gpsetpointsize{4.00}
\gppoint{gp mark 3}{(1.693,4.961)}
\gppoint{gp mark 3}{(1.811,0.985)}
\gppoint{gp mark 3}{(2.027,7.697)}
\gppoint{gp mark 3}{(2.232,1.028)}
\gppoint{gp mark 3}{(2.447,7.398)}
\gppoint{gp mark 3}{(2.663,1.284)}
\gppoint{gp mark 3}{(2.868,7.162)}
\gppoint{gp mark 3}{(3.083,1.498)}
\gppoint{gp mark 3}{(3.299,7.248)}
\gppoint{gp mark 3}{(3.504,1.413)}
\gppoint{gp mark 3}{(3.719,7.013)}
\gppoint{gp mark 3}{(3.935,1.605)}
\gppoint{gp mark 3}{(4.140,6.842)}
\gppoint{gp mark 3}{(4.355,1.797)}
\gppoint{gp mark 3}{(4.571,6.820)}
\gppoint{gp mark 3}{(4.776,1.797)}
\gppoint{gp mark 3}{(4.981,6.713)}
\gppoint{gp mark 3}{(5.197,2.054)}
\gppoint{gp mark 3}{(5.412,6.542)}
\gppoint{gp mark 3}{(5.627,2.182)}
\gppoint{gp mark 3}{(5.843,6.350)}
\gppoint{gp mark 3}{(6.048,2.161)}
\gppoint{gp mark 3}{(6.253,6.329)}
\gppoint{gp mark 3}{(6.469,2.353)}
\gppoint{gp mark 3}{(6.684,6.243)}
\gppoint{gp mark 3}{(6.900,2.439)}
\gppoint{gp mark 3}{(7.115,6.115)}
\gppoint{gp mark 3}{(7.320,2.439)}
\gppoint{gp mark 3}{(7.525,5.987)}
\gppoint{gp mark 3}{(7.741,2.652)}
\gppoint{gp mark 3}{(7.956,6.030)}
\gppoint{gp mark 3}{(8.172,2.695)}
\gppoint{gp mark 3}{(8.387,5.901)}
\gppoint{gp mark 3}{(8.592,2.781)}
\gppoint{gp mark 3}{(8.797,5.794)}
\gppoint{gp mark 3}{(9.013,2.866)}
\gppoint{gp mark 3}{(9.228,5.773)}
\gppoint{gp mark 3}{(9.444,2.909)}
\gppoint{gp mark 3}{(9.659,5.666)}
\gppoint{gp mark 3}{(11.121,7.491)}
\gpcolor{color=gp lt color border}
\gpsetlinetype{gp lt border}
\draw[gp path] (1.688,7.697)--(1.688,0.985)--(9.957,0.985);
%% coordinates of the plot area
\gpdefrectangularnode{gp plot 1}{\pgfpoint{1.688cm}{0.985cm}}{\pgfpoint{11.947cm}{7.825cm}}
\end{tikzpicture}
%% gnuplot variables
 \caption{Grafico 1.000dgdecad.tex} \label{gr:1.000dgdecad.tex} \end{grafico}
\begin{grafico} \centering \begin{tikzpicture}[gnuplot]
%% generated with GNUPLOT 4.6p0 (Lua 5.1; terminal rev. 99, script rev. 100)
%% Tue 10 Jun 2014 07:57:01 PM CEST
\path (0.000,0.000) rectangle (12.500,8.750);
\gpcolor{color=gp lt color border}
\gpsetlinetype{gp lt border}
\gpsetlinewidth{1.00}
\draw[gp path] (1.504,0.985)--(1.684,0.985);
\node[gp node right] at (1.320,0.985) {-1};
\draw[gp path] (1.504,2.695)--(1.684,2.695);
\node[gp node right] at (1.320,2.695) {-0.5};
\draw[gp path] (1.504,4.405)--(1.684,4.405);
\node[gp node right] at (1.320,4.405) { 0};
\draw[gp path] (1.504,6.115)--(1.684,6.115);
\node[gp node right] at (1.320,6.115) { 0.5};
\draw[gp path] (1.504,7.825)--(1.684,7.825);
\node[gp node right] at (1.320,7.825) { 1};
\draw[gp path] (1.504,0.985)--(1.504,1.165);
\node[gp node center] at (1.504,0.677) { 0};
\draw[gp path] (3.593,0.985)--(3.593,1.165);
\node[gp node center] at (3.593,0.677) { 5};
\draw[gp path] (5.681,0.985)--(5.681,1.165);
\node[gp node center] at (5.681,0.677) { 10};
\draw[gp path] (7.770,0.985)--(7.770,1.165);
\node[gp node center] at (7.770,0.677) { 15};
\draw[gp path] (9.858,0.985)--(9.858,1.165);
\node[gp node center] at (9.858,0.677) { 20};
\draw[gp path] (1.504,7.825)--(1.504,0.985)--(9.921,0.985);
\node[gp node center,rotate=-270] at (0.246,4.405) {Ampiezza [???]};
\node[gp node center] at (6.725,0.215) {Tempo $[s]$};
\node[gp node center] at (6.725,8.287) {Dati decadimento 1.020d};
\gpcolor{rgb color={0.000,0.000,1.000}}
\gpsetlinetype{gp lt plot 0}
\draw[gp path] (1.504,4.289)--(1.525,4.265)--(1.546,4.248)--(1.567,4.248)--(1.588,4.265)%
  --(1.608,4.289)--(1.629,4.323)--(1.650,4.367)--(1.671,4.415)--(1.692,4.456)--(1.713,4.494)%
  --(1.734,4.525)--(1.755,4.545)--(1.776,4.549)--(1.796,4.535)--(1.817,4.518)--(1.838,4.487)%
  --(1.859,4.446)--(1.880,4.398)--(1.901,4.361)--(1.922,4.316)--(1.943,4.282)--(1.963,4.261)%
  --(1.984,4.255)--(2.005,4.261)--(2.026,4.272)--(2.047,4.306)--(2.068,4.340)--(2.089,4.384)%
  --(2.110,4.426)--(2.131,4.463)--(2.151,4.501)--(2.172,4.528)--(2.193,4.538)--(2.214,4.538)%
  --(2.235,4.528)--(2.256,4.501)--(2.277,4.467)--(2.298,4.426)--(2.319,4.388)--(2.339,4.347)%
  --(2.360,4.309)--(2.381,4.278)--(2.402,4.265)--(2.423,4.258)--(2.444,4.265)--(2.465,4.285)%
  --(2.486,4.323)--(2.507,4.354)--(2.527,4.395)--(2.548,4.436)--(2.569,4.480)--(2.590,4.508)%
  --(2.611,4.525)--(2.632,4.535)--(2.653,4.532)--(2.674,4.511)--(2.695,4.484)--(2.715,4.453)%
  --(2.736,4.419)--(2.757,4.374)--(2.778,4.333)--(2.799,4.306)--(2.820,4.282)--(2.841,4.268)%
  --(2.862,4.268)--(2.882,4.282)--(2.903,4.302)--(2.924,4.333)--(2.945,4.371)--(2.966,4.412)%
  --(2.987,4.449)--(3.008,4.480)--(3.029,4.508)--(3.050,4.521)--(3.070,4.528)--(3.091,4.518)%
  --(3.112,4.501)--(3.133,4.473)--(3.154,4.443)--(3.175,4.402)--(3.196,4.357)--(3.217,4.330)%
  --(3.238,4.299)--(3.258,4.282)--(3.279,4.272)--(3.300,4.278)--(3.321,4.292)--(3.342,4.316)%
  --(3.363,4.350)--(3.384,4.388)--(3.405,4.426)--(3.426,4.460)--(3.446,4.487)--(3.467,4.511)%
  --(3.488,4.518)--(3.509,4.518)--(3.530,4.511)--(3.551,4.491)--(3.572,4.460)--(3.593,4.422)%
  --(3.613,4.388)--(3.634,4.354)--(3.655,4.320)--(3.676,4.299)--(3.697,4.282)--(3.718,4.278)%
  --(3.739,4.285)--(3.760,4.302)--(3.781,4.333)--(3.801,4.367)--(3.822,4.398)--(3.843,4.436)%
  --(3.864,4.470)--(3.885,4.494)--(3.906,4.508)--(3.927,4.514)--(3.948,4.514)--(3.969,4.497)%
  --(3.989,4.473)--(4.010,4.446)--(4.031,4.412)--(4.052,4.381)--(4.073,4.343)--(4.094,4.316)%
  --(4.115,4.296)--(4.136,4.285)--(4.157,4.285)--(4.177,4.296)--(4.198,4.316)--(4.219,4.343)%
  --(4.240,4.374)--(4.261,4.408)--(4.282,4.443)--(4.303,4.473)--(4.324,4.494)--(4.344,4.508)%
  --(4.365,4.511)--(4.386,4.501)--(4.407,4.487)--(4.428,4.463)--(4.449,4.432)--(4.470,4.398)%
  --(4.491,4.364)--(4.512,4.333)--(4.532,4.313)--(4.553,4.296)--(4.574,4.289)--(4.595,4.296)%
  --(4.616,4.306)--(4.637,4.330)--(4.658,4.357)--(4.679,4.391)--(4.700,4.422)--(4.720,4.453)%
  --(4.741,4.477)--(4.762,4.494)--(4.783,4.504)--(4.804,4.504)--(4.825,4.497)--(4.846,4.480)%
  --(4.867,4.456)--(4.888,4.419)--(4.908,4.388)--(4.929,4.357)--(4.950,4.330)--(4.971,4.309)%
  --(4.992,4.296)--(5.013,4.296)--(5.034,4.302)--(5.055,4.320)--(5.076,4.340)--(5.096,4.371)%
  --(5.117,4.402)--(5.138,4.432)--(5.159,4.460)--(5.180,4.477)--(5.201,4.494)--(5.222,4.501)%
  --(5.243,4.501)--(5.263,4.491)--(5.284,4.470)--(5.305,4.439)--(5.326,4.408)--(5.347,4.381)%
  --(5.368,4.350)--(5.389,4.326)--(5.410,4.309)--(5.431,4.299)--(5.451,4.299)--(5.472,4.309)%
  --(5.493,4.326)--(5.514,4.354)--(5.535,4.381)--(5.556,4.412)--(5.577,4.439)--(5.598,4.463)%
  --(5.619,4.484)--(5.639,4.497)--(5.660,4.497)--(5.681,4.487)--(5.702,4.477)--(5.723,4.456)%
  --(5.744,4.429)--(5.765,4.398)--(5.786,4.371)--(5.807,4.343)--(5.827,4.323)--(5.848,4.309)%
  --(5.869,4.306)--(5.890,4.309)--(5.911,4.320)--(5.932,4.340)--(5.953,4.364)--(5.974,4.391)%
  --(5.994,4.419)--(6.015,4.446)--(6.036,4.467)--(6.057,4.484)--(6.078,4.494)--(6.099,4.494)%
  --(6.120,4.487)--(6.141,4.470)--(6.162,4.443)--(6.182,4.419)--(6.203,4.384)--(6.224,4.361)%
  --(6.245,4.340)--(6.266,4.320)--(6.287,4.313)--(6.308,4.309)--(6.329,4.316)--(6.350,4.330)%
  --(6.370,4.347)--(6.391,4.374)--(6.412,4.398)--(6.433,4.429)--(6.454,4.453)--(6.475,4.473)%
  --(6.496,4.487)--(6.517,4.484)--(6.538,4.484)--(6.558,4.477)--(6.579,4.456)--(6.600,4.429)%
  --(6.621,4.408)--(6.642,4.381)--(6.663,4.357)--(6.684,4.333)--(6.705,4.323)--(6.726,4.313)%
  --(6.746,4.316)--(6.767,4.323)--(6.788,4.337)--(6.809,4.361)--(6.830,4.384)--(6.851,4.408)%
  --(6.872,4.436)--(6.893,4.456)--(6.913,4.473)--(6.934,4.484)--(6.955,4.487)--(6.976,4.477)%
  --(6.997,4.463)--(7.018,4.446)--(7.039,4.422)--(7.060,4.398)--(7.081,4.371)--(7.101,4.347)%
  --(7.122,4.333)--(7.143,4.320)--(7.164,4.313)--(7.185,4.320)--(7.206,4.330)--(7.227,4.347)%
  --(7.248,4.367)--(7.269,4.395)--(7.289,4.419)--(7.310,4.439)--(7.331,4.460)--(7.352,4.477)%
  --(7.373,4.484)--(7.394,4.480)--(7.415,4.473)--(7.436,4.456)--(7.457,4.436)--(7.477,4.415)%
  --(7.498,4.391)--(7.519,4.367)--(7.540,4.350)--(7.561,4.330)--(7.582,4.323)--(7.603,4.320)%
  --(7.624,4.326)--(7.644,4.337)--(7.665,4.357)--(7.686,4.378)--(7.707,4.402)--(7.728,4.422)%
  --(7.749,4.446)--(7.770,4.460)--(7.791,4.470)--(7.812,4.473)--(7.832,4.473)--(7.853,4.463)%
  --(7.874,4.446)--(7.895,4.426)--(7.916,4.405)--(7.937,4.381)--(7.958,4.361)--(7.979,4.343)%
  --(8.000,4.330)--(8.020,4.326)--(8.041,4.326)--(8.062,4.333)--(8.083,4.347)--(8.104,4.367)%
  --(8.125,4.384)--(8.146,4.408)--(8.167,4.436)--(8.188,4.453)--(8.208,4.463)--(8.229,4.470)%
  --(8.250,4.470)--(8.271,4.467)--(8.292,4.456)--(8.313,4.439)--(8.334,4.422)--(8.355,4.398)%
  --(8.375,4.374)--(8.396,4.357)--(8.417,4.343)--(8.438,4.333)--(8.459,4.330)--(8.480,4.333)%
  --(8.501,4.340)--(8.522,4.354)--(8.543,4.374)--(8.563,4.391)--(8.584,4.419)--(8.605,4.439)%
  --(8.626,4.453)--(8.647,4.463)--(8.668,4.470)--(8.689,4.467)--(8.710,4.460)--(8.731,4.449)%
  --(8.751,4.432)--(8.772,4.412)--(8.793,4.391)--(8.814,4.371)--(8.835,4.354)--(8.856,4.340)%
  --(8.877,4.333)--(8.898,4.333)--(8.919,4.333)--(8.939,4.347)--(8.960,4.361)--(8.981,4.381)%
  --(9.002,4.402)--(9.023,4.422)--(9.044,4.439)--(9.065,4.453)--(9.086,4.463)--(9.107,4.467)%
  --(9.127,4.463)--(9.148,4.453)--(9.169,4.446)--(9.190,4.422)--(9.211,4.405)--(9.232,4.384)%
  --(9.253,4.367)--(9.274,4.350)--(9.294,4.340)--(9.315,4.333)--(9.336,4.333)--(9.357,4.343)%
  --(9.378,4.354)--(9.399,4.371)--(9.420,4.391)--(9.441,4.412)--(9.462,4.426)--(9.482,4.443)%
  --(9.503,4.456)--(9.524,4.460)--(9.545,4.460)--(9.566,4.456)--(9.587,4.446)--(9.608,4.432)%
  --(9.629,4.415)--(9.650,4.398)--(9.670,4.374)--(9.691,4.361)--(9.712,4.350)--(9.733,4.337)%
  --(9.754,4.337)--(9.775,4.337)--(9.796,4.347)--(9.817,4.361)--(9.838,4.378)--(9.858,4.391)%
  --(9.879,4.412)--(9.900,4.429)--(9.921,4.446);
\gpcolor{color=gp lt color border}
\node[gp node right] at (10.479,7.491) {Punti tangenza};
\gpcolor{rgb color={1.000,0.000,0.000}}
\gpsetpointsize{4.00}
\gppoint{gp mark 3}{(1.582,4.259)}
\gppoint{gp mark 3}{(1.765,4.555)}
\gppoint{gp mark 3}{(1.984,4.255)}
\gppoint{gp mark 3}{(2.204,4.544)}
\gppoint{gp mark 3}{(2.423,4.258)}
\gppoint{gp mark 3}{(2.642,4.542)}
\gppoint{gp mark 3}{(2.851,4.261)}
\gppoint{gp mark 3}{(3.060,4.533)}
\gppoint{gp mark 3}{(3.279,4.272)}
\gppoint{gp mark 3}{(3.499,4.521)}
\gppoint{gp mark 3}{(3.718,4.278)}
\gppoint{gp mark 3}{(3.937,4.523)}
\gppoint{gp mark 3}{(4.146,4.280)}
\gppoint{gp mark 3}{(4.355,4.516)}
\gppoint{gp mark 3}{(4.574,4.289)}
\gppoint{gp mark 3}{(4.794,4.508)}
\gppoint{gp mark 3}{(5.013,4.296)}
\gppoint{gp mark 3}{(5.232,4.506)}
\gppoint{gp mark 3}{(5.441,4.294)}
\gppoint{gp mark 3}{(5.650,4.502)}
\gppoint{gp mark 3}{(5.869,4.306)}
\gppoint{gp mark 3}{(6.088,4.497)}
\gppoint{gp mark 3}{(6.308,4.309)}
\gppoint{gp mark 3}{(6.527,4.487)}
\gppoint{gp mark 3}{(6.736,4.313)}
\gppoint{gp mark 3}{(6.945,4.492)}
\gppoint{gp mark 3}{(7.164,4.313)}
\gppoint{gp mark 3}{(7.383,4.484)}
\gppoint{gp mark 3}{(7.603,4.320)}
\gppoint{gp mark 3}{(7.822,4.479)}
\gppoint{gp mark 3}{(8.031,4.323)}
\gppoint{gp mark 3}{(8.240,4.472)}
\gppoint{gp mark 3}{(8.459,4.330)}
\gppoint{gp mark 3}{(8.678,4.470)}
\gppoint{gp mark 3}{(8.898,4.333)}
\gppoint{gp mark 3}{(9.117,4.468)}
\gppoint{gp mark 3}{(9.326,4.328)}
\gppoint{gp mark 3}{(9.535,4.461)}
\gppoint{gp mark 3}{(9.754,4.326)}
\gppoint{gp mark 3}{(11.121,7.491)}
\gpcolor{color=gp lt color border}
\node[gp node right] at (10.479,7.183) {fexp(x)};
\gpcolor{color=gp lt color 2}
\gpsetlinetype{gp lt plot 2}
\draw[gp path] (10.663,7.183)--(11.579,7.183);
\draw[gp path] (1.504,7.825)--(1.589,7.801)--(1.674,7.777)--(1.759,7.753)--(1.844,7.730)%
  --(1.929,7.706)--(2.014,7.683)--(2.099,7.660)--(2.184,7.637)--(2.269,7.614)--(2.354,7.591)%
  --(2.439,7.569)--(2.524,7.546)--(2.609,7.524)--(2.694,7.502)--(2.779,7.480)--(2.864,7.459)%
  --(2.949,7.437)--(3.034,7.416)--(3.119,7.395)--(3.204,7.374)--(3.289,7.353)--(3.374,7.332)%
  --(3.459,7.311)--(3.544,7.291)--(3.630,7.270)--(3.715,7.250)--(3.800,7.230)--(3.885,7.210)%
  --(3.970,7.190)--(4.055,7.171)--(4.140,7.151)--(4.225,7.132)--(4.310,7.113)--(4.395,7.093)%
  --(4.480,7.074)--(4.565,7.056)--(4.650,7.037)--(4.735,7.018)--(4.820,7.000)--(4.905,6.982)%
  --(4.990,6.963)--(5.075,6.945)--(5.160,6.928)--(5.245,6.910)--(5.330,6.892)--(5.415,6.875)%
  --(5.500,6.857)--(5.585,6.840)--(5.670,6.823)--(5.755,6.806)--(5.840,6.789)--(5.925,6.772)%
  --(6.010,6.755)--(6.095,6.739)--(6.180,6.722)--(6.265,6.706)--(6.350,6.690)--(6.435,6.673)%
  --(6.520,6.657)--(6.605,6.642)--(6.690,6.626)--(6.775,6.610)--(6.860,6.595)--(6.945,6.579)%
  --(7.030,6.564)--(7.115,6.549)--(7.200,6.533)--(7.285,6.518)--(7.370,6.503)--(7.455,6.489)%
  --(7.540,6.474)--(7.625,6.459)--(7.711,6.445)--(7.796,6.431)--(7.881,6.416)--(7.966,6.402)%
  --(8.051,6.388)--(8.136,6.374)--(8.221,6.360)--(8.306,6.346)--(8.391,6.333)--(8.476,6.319)%
  --(8.561,6.305)--(8.646,6.292)--(8.731,6.279)--(8.816,6.266)--(8.901,6.252)--(8.986,6.239)%
  --(9.071,6.226)--(9.156,6.214)--(9.241,6.201)--(9.326,6.188)--(9.411,6.176)--(9.496,6.163)%
  --(9.581,6.151)--(9.666,6.138)--(9.751,6.126)--(9.836,6.114)--(9.921,6.102);
\gpcolor{color=gp lt color border}
\node[gp node right] at (10.479,6.875) {fexpneg(x)};
\gpcolor{color=gp lt color 3}
\gpsetlinetype{gp lt plot 3}
\draw[gp path] (10.663,6.875)--(11.579,6.875);
\draw[gp path] (1.504,0.985)--(1.589,1.009)--(1.674,1.033)--(1.759,1.057)--(1.844,1.080)%
  --(1.929,1.104)--(2.014,1.127)--(2.099,1.150)--(2.184,1.173)--(2.269,1.196)--(2.354,1.219)%
  --(2.439,1.241)--(2.524,1.264)--(2.609,1.286)--(2.694,1.308)--(2.779,1.330)--(2.864,1.351)%
  --(2.949,1.373)--(3.034,1.394)--(3.119,1.415)--(3.204,1.436)--(3.289,1.457)--(3.374,1.478)%
  --(3.459,1.499)--(3.544,1.519)--(3.630,1.540)--(3.715,1.560)--(3.800,1.580)--(3.885,1.600)%
  --(3.970,1.620)--(4.055,1.639)--(4.140,1.659)--(4.225,1.678)--(4.310,1.697)--(4.395,1.717)%
  --(4.480,1.736)--(4.565,1.754)--(4.650,1.773)--(4.735,1.792)--(4.820,1.810)--(4.905,1.828)%
  --(4.990,1.847)--(5.075,1.865)--(5.160,1.882)--(5.245,1.900)--(5.330,1.918)--(5.415,1.935)%
  --(5.500,1.953)--(5.585,1.970)--(5.670,1.987)--(5.755,2.004)--(5.840,2.021)--(5.925,2.038)%
  --(6.010,2.055)--(6.095,2.071)--(6.180,2.088)--(6.265,2.104)--(6.350,2.120)--(6.435,2.137)%
  --(6.520,2.153)--(6.605,2.168)--(6.690,2.184)--(6.775,2.200)--(6.860,2.215)--(6.945,2.231)%
  --(7.030,2.246)--(7.115,2.261)--(7.200,2.277)--(7.285,2.292)--(7.370,2.307)--(7.455,2.321)%
  --(7.540,2.336)--(7.625,2.351)--(7.711,2.365)--(7.796,2.379)--(7.881,2.394)--(7.966,2.408)%
  --(8.051,2.422)--(8.136,2.436)--(8.221,2.450)--(8.306,2.464)--(8.391,2.477)--(8.476,2.491)%
  --(8.561,2.505)--(8.646,2.518)--(8.731,2.531)--(8.816,2.544)--(8.901,2.558)--(8.986,2.571)%
  --(9.071,2.584)--(9.156,2.596)--(9.241,2.609)--(9.326,2.622)--(9.411,2.634)--(9.496,2.647)%
  --(9.581,2.659)--(9.666,2.672)--(9.751,2.684)--(9.836,2.696)--(9.921,2.708);
\gpcolor{color=gp lt color border}
\gpsetlinetype{gp lt border}
\draw[gp path] (1.504,7.825)--(1.504,0.985)--(9.921,0.985);
%% coordinates of the plot area
\gpdefrectangularnode{gp plot 1}{\pgfpoint{1.504cm}{0.985cm}}{\pgfpoint{11.947cm}{7.825cm}}
\end{tikzpicture}
%% gnuplot variables
 \caption{Grafico 1.020dgdecad.tex} \label{gr:1.020dgdecad.tex} \end{grafico}
\begin{grafico} \centering \begin{tikzpicture}[gnuplot]
%% generated with GNUPLOT 4.6p4 (Lua 5.1; terminal rev. 99, script rev. 100)
%% mar 18 nov 2014 19:31:22 CET
\path (0.000,0.000) rectangle (12.500,8.750);
\gpcolor{color=gp lt color border}
\gpsetlinetype{gp lt border}
\gpsetlinewidth{1.00}
\draw[gp path] (1.872,1.725)--(2.052,1.725);
\node[gp node right] at (1.688,1.725) {-0.02};
\draw[gp path] (1.872,2.464)--(2.052,2.464);
\node[gp node right] at (1.688,2.464) {-0.015};
\draw[gp path] (1.872,3.204)--(2.052,3.204);
\node[gp node right] at (1.688,3.204) {-0.01};
\draw[gp path] (1.872,3.943)--(2.052,3.943);
\node[gp node right] at (1.688,3.943) {-0.005};
\draw[gp path] (1.872,4.683)--(2.052,4.683);
\node[gp node right] at (1.688,4.683) { 0};
\draw[gp path] (1.872,5.423)--(2.052,5.423);
\node[gp node right] at (1.688,5.423) { 0.005};
\draw[gp path] (1.872,6.162)--(2.052,6.162);
\node[gp node right] at (1.688,6.162) { 0.01};
\draw[gp path] (1.872,6.902)--(2.052,6.902);
\node[gp node right] at (1.688,6.902) { 0.015};
\draw[gp path] (1.872,7.641)--(2.052,7.641);
\node[gp node right] at (1.688,7.641) { 0.02};
\draw[gp path] (1.872,0.985)--(1.872,1.165);
\node[gp node center] at (1.872,0.677) { 0};
\draw[gp path] (3.887,0.985)--(3.887,1.165);
\node[gp node center] at (3.887,0.677) { 5};
\draw[gp path] (5.902,0.985)--(5.902,1.165);
\node[gp node center] at (5.902,0.677) { 10};
\draw[gp path] (7.917,0.985)--(7.917,1.165);
\node[gp node center] at (7.917,0.677) { 15};
\draw[gp path] (9.932,0.985)--(9.932,1.165);
\node[gp node center] at (9.932,0.677) { 20};
\draw[gp path] (1.872,7.789)--(1.872,1.651);
\draw[gp path] (1.872,0.985)--(10.033,0.985);
\node[gp node center,rotate=-270] at (0.246,4.683) {Ampiezza [giri]};
\node[gp node center] at (6.909,0.215) {Tempo $[s]$};
\gpcolor{rgb color={0.000,0.000,1.000}}
\gpsetlinetype{gp lt plot 0}
\draw[gp path] (1.872,6.014)--(1.892,6.754)--(1.912,7.346)--(1.932,7.346)--(1.953,7.346)%
  --(1.973,7.050)--(1.993,6.310)--(2.013,5.718)--(2.033,4.979)--(2.053,3.943)--(2.074,3.056)%
  --(2.094,2.316)--(2.114,2.020)--(2.134,1.725)--(2.154,1.725)--(2.174,1.873)--(2.194,2.464)%
  --(2.215,3.056)--(2.235,3.795)--(2.255,4.831)--(2.275,5.718)--(2.295,6.310)--(2.315,7.050)%
  --(2.335,7.493)--(2.356,7.493)--(2.376,7.346)--(2.396,6.902)--(2.416,6.458)--(2.436,5.718)%
  --(2.456,4.831)--(2.477,3.943)--(2.497,3.352)--(2.517,2.760)--(2.537,2.168)--(2.557,2.020)%
  --(2.577,2.168)--(2.597,2.316)--(2.618,2.760)--(2.638,3.352)--(2.658,4.239)--(2.678,5.127)%
  --(2.698,5.718)--(2.718,6.606)--(2.738,7.198)--(2.759,7.346)--(2.779,7.346)--(2.799,7.050)%
  --(2.819,6.606)--(2.839,6.014)--(2.859,5.275)--(2.880,4.535)--(2.900,3.500)--(2.920,2.908)%
  --(2.940,2.612)--(2.960,2.316)--(2.980,2.168)--(3.000,2.020)--(3.021,2.464)--(3.041,3.204)%
  --(3.061,3.943)--(3.081,4.535)--(3.101,5.423)--(3.121,6.162)--(3.141,6.606)--(3.162,7.050)%
  --(3.182,7.198)--(3.202,7.198)--(3.222,6.902)--(3.242,6.458)--(3.262,5.866)--(3.283,5.127)%
  --(3.303,4.091)--(3.323,3.648)--(3.343,3.056)--(3.363,2.612)--(3.383,2.168)--(3.403,2.168)%
  --(3.424,2.464)--(3.444,3.056)--(3.464,3.500)--(3.484,4.239)--(3.504,4.831)--(3.524,5.718)%
  --(3.544,6.458)--(3.565,6.902)--(3.585,7.050)--(3.605,7.050)--(3.625,6.902)--(3.645,6.754)%
  --(3.665,6.162)--(3.686,5.275)--(3.706,4.683)--(3.726,4.091)--(3.746,3.500)--(3.766,2.908)%
  --(3.786,2.316)--(3.806,2.464)--(3.827,2.316)--(3.847,2.760)--(3.867,3.056)--(3.887,3.648)%
  --(3.907,4.387)--(3.927,5.127)--(3.947,5.866)--(3.968,6.458)--(3.988,6.754)--(4.008,7.050)%
  --(4.028,7.050)--(4.048,7.050)--(4.068,6.310)--(4.089,5.718)--(4.109,5.127)--(4.129,4.535)%
  --(4.149,3.795)--(4.169,3.204)--(4.189,2.760)--(4.209,2.612)--(4.230,2.464)--(4.250,2.612)%
  --(4.270,2.908)--(4.290,3.204)--(4.310,3.943)--(4.330,4.831)--(4.350,5.423)--(4.371,6.014)%
  --(4.391,6.310)--(4.411,6.902)--(4.431,7.050)--(4.451,6.902)--(4.471,6.458)--(4.492,6.014)%
  --(4.512,5.571)--(4.532,4.979)--(4.552,4.239)--(4.572,3.500)--(4.592,3.056)--(4.612,2.760)%
  --(4.633,2.464)--(4.653,2.612)--(4.673,2.760)--(4.693,3.056)--(4.713,3.500)--(4.733,4.239)%
  --(4.753,4.979)--(4.774,5.571)--(4.794,6.014)--(4.814,6.606)--(4.834,6.902)--(4.854,6.754)%
  --(4.874,6.458)--(4.895,6.310)--(4.915,5.866)--(4.935,5.423)--(4.955,4.683)--(4.975,3.943)%
  --(4.995,3.500)--(5.015,3.056)--(5.036,2.760)--(5.056,2.612)--(5.076,2.612)--(5.096,3.056)%
  --(5.116,3.352)--(5.136,3.943)--(5.156,4.683)--(5.177,5.127)--(5.197,5.571)--(5.217,6.162)%
  --(5.237,6.606)--(5.257,6.606)--(5.277,6.606)--(5.298,6.458)--(5.318,6.162)--(5.338,5.718)%
  --(5.358,4.979)--(5.378,4.387)--(5.398,3.943)--(5.418,3.352)--(5.439,2.908)--(5.459,2.760)%
  --(5.479,2.760)--(5.499,2.760)--(5.519,3.204)--(5.539,3.795)--(5.559,4.091)--(5.580,4.535)%
  --(5.600,5.275)--(5.620,5.718)--(5.640,6.162)--(5.660,6.458)--(5.680,6.606)--(5.701,6.606)%
  --(5.721,6.310)--(5.741,5.866)--(5.761,5.423)--(5.781,4.831)--(5.801,4.387)--(5.821,3.648)%
  --(5.842,3.204)--(5.862,2.908)--(5.882,2.908)--(5.902,2.908)--(5.922,3.056)--(5.942,3.352)%
  --(5.962,3.648)--(5.983,4.239)--(6.003,4.979)--(6.023,5.571)--(6.043,5.866)--(6.063,6.310)%
  --(6.083,6.458)--(6.104,6.458)--(6.124,6.310)--(6.144,6.014)--(6.164,5.571)--(6.184,5.275)%
  --(6.204,4.683)--(6.224,4.091)--(6.245,3.648)--(6.265,3.204)--(6.285,2.908)--(6.305,2.908)%
  --(6.325,2.908)--(6.345,3.352)--(6.365,3.500)--(6.386,4.091)--(6.406,4.535)--(6.426,5.127)%
  --(6.446,5.423)--(6.466,6.014)--(6.486,6.162)--(6.507,6.458)--(6.527,6.310)--(6.547,6.162)%
  --(6.567,6.014)--(6.587,5.571)--(6.607,4.979)--(6.627,4.535)--(6.648,3.943)--(6.668,3.500)%
  --(6.688,3.204)--(6.708,3.056)--(6.728,2.908)--(6.748,3.056)--(6.768,3.500)--(6.789,3.795)%
  --(6.809,4.239)--(6.829,4.535)--(6.849,5.127)--(6.869,5.571)--(6.889,6.014)--(6.910,6.162)%
  --(6.930,6.310)--(6.950,6.310)--(6.970,6.162)--(6.990,5.718)--(7.010,5.275)--(7.030,4.831)%
  --(7.051,4.387)--(7.071,3.943)--(7.091,3.500)--(7.111,3.204)--(7.131,3.056)--(7.151,3.204)%
  --(7.171,3.352)--(7.192,3.648)--(7.212,3.943)--(7.232,4.239)--(7.252,4.683)--(7.272,5.275)%
  --(7.292,5.718)--(7.313,6.014)--(7.333,6.014)--(7.353,6.162)--(7.373,6.162)--(7.393,5.866)%
  --(7.413,5.423)--(7.433,5.127)--(7.454,4.683)--(7.474,4.239)--(7.494,3.795)--(7.514,3.352)%
  --(7.534,3.204)--(7.554,3.204)--(7.574,3.204)--(7.595,3.352)--(7.615,3.648)--(7.635,3.943)%
  --(7.655,4.535)--(7.675,4.979)--(7.695,5.423)--(7.716,5.718)--(7.736,5.866)--(7.756,6.162)%
  --(7.776,6.162)--(7.796,5.866)--(7.816,5.718)--(7.836,5.423)--(7.857,4.979)--(7.877,4.387)%
  --(7.897,4.091)--(7.917,3.648)--(7.937,3.352)--(7.957,3.204)--(7.977,3.352)--(7.998,3.204)%
  --(8.018,3.500)--(8.038,3.943)--(8.058,4.387)--(8.078,4.683)--(8.098,4.979)--(8.119,5.423)%
  --(8.139,5.718)--(8.159,6.014)--(8.179,6.014)--(8.199,6.014)--(8.219,5.866)--(8.239,5.571)%
  --(8.260,5.127)--(8.280,4.831)--(8.300,4.239)--(8.320,3.943)--(8.340,3.500)--(8.360,3.500)%
  --(8.380,3.352)--(8.401,3.204)--(8.421,3.500)--(8.441,3.500)--(8.461,3.943)--(8.481,4.387)%
  --(8.501,4.831)--(8.522,5.423)--(8.542,5.571)--(8.562,5.718)--(8.582,5.866)--(8.602,6.014)%
  --(8.622,5.866)--(8.642,5.718)--(8.663,5.423)--(8.683,4.979)--(8.703,4.683)--(8.723,4.239)%
  --(8.743,3.943)--(8.763,3.500)--(8.783,3.352)--(8.804,3.352)--(8.824,3.352)--(8.844,3.500)%
  --(8.864,3.795)--(8.884,4.091)--(8.904,4.683)--(8.925,4.979)--(8.945,5.275)--(8.965,5.718)%
  --(8.985,5.866)--(9.005,5.866)--(9.025,6.014)--(9.045,5.866)--(9.066,5.571)--(9.086,5.127)%
  --(9.106,4.831)--(9.126,4.683)--(9.146,4.239)--(9.166,3.943)--(9.186,3.648)--(9.207,3.500)%
  --(9.227,3.500)--(9.247,3.500)--(9.267,3.795)--(9.287,3.943)--(9.307,4.239)--(9.328,4.683)%
  --(9.348,5.127)--(9.368,5.423)--(9.388,5.571)--(9.408,5.718)--(9.428,5.866)--(9.448,5.866)%
  --(9.469,5.571)--(9.489,5.423)--(9.509,4.979)--(9.529,4.831)--(9.549,4.239)--(9.569,4.091)%
  --(9.589,3.943)--(9.610,3.648)--(9.630,3.500)--(9.650,3.352)--(9.670,3.500)--(9.690,3.795)%
  --(9.710,4.091)--(9.731,4.535)--(9.751,4.831)--(9.771,5.127)--(9.791,5.571)--(9.811,5.718)%
  --(9.831,5.866)--(9.851,5.718)--(9.872,5.718)--(9.892,5.718)--(9.912,5.275)--(9.932,4.979)%
  --(9.952,4.535)--(9.972,4.239)--(9.992,3.943)--(10.013,3.648)--(10.033,3.500);
\gpcolor{color=gp lt color border}
\node[gp node right] at (10.479,8.047) {Punti tangenza};
\gpcolor{rgb color={1.000,0.000,0.000}}
\gpsetpointsize{4.00}
\gppoint{gp mark 3}{(1.958,7.604)}
\gppoint{gp mark 3}{(2.144,1.651)}
\gppoint{gp mark 3}{(2.356,7.789)}
\gppoint{gp mark 3}{(2.567,2.094)}
\gppoint{gp mark 3}{(2.769,7.493)}
\gppoint{gp mark 3}{(2.980,2.168)}
\gppoint{gp mark 3}{(3.192,7.346)}
\gppoint{gp mark 3}{(3.393,2.020)}
\gppoint{gp mark 3}{(3.595,7.124)}
\gppoint{gp mark 3}{(3.806,2.464)}
\gppoint{gp mark 3}{(4.018,7.050)}
\gppoint{gp mark 3}{(4.219,2.390)}
\gppoint{gp mark 3}{(4.431,7.050)}
\gppoint{gp mark 3}{(4.643,2.538)}
\gppoint{gp mark 3}{(4.844,6.902)}
\gppoint{gp mark 3}{(5.056,2.612)}
\gppoint{gp mark 3}{(5.267,6.680)}
\gppoint{gp mark 3}{(5.479,2.760)}
\gppoint{gp mark 3}{(5.690,6.754)}
\gppoint{gp mark 3}{(5.892,2.829)}
\gppoint{gp mark 3}{(6.093,6.532)}
\gppoint{gp mark 3}{(6.305,2.908)}
\gppoint{gp mark 3}{(6.517,6.384)}
\gppoint{gp mark 3}{(6.728,2.908)}
\gppoint{gp mark 3}{(6.940,6.384)}
\gppoint{gp mark 3}{(7.141,3.130)}
\gppoint{gp mark 3}{(7.343,6.162)}
\gppoint{gp mark 3}{(7.554,3.204)}
\gppoint{gp mark 3}{(7.766,6.310)}
\gppoint{gp mark 3}{(7.967,3.426)}
\gppoint{gp mark 3}{(8.179,6.014)}
\gppoint{gp mark 3}{(8.391,3.056)}
\gppoint{gp mark 3}{(8.592,6.088)}
\gppoint{gp mark 3}{(8.804,3.352)}
\gppoint{gp mark 3}{(9.015,6.088)}
\gppoint{gp mark 3}{(9.217,3.500)}
\gppoint{gp mark 3}{(9.428,5.866)}
\gppoint{gp mark 3}{(9.640,3.278)}
\gppoint{gp mark 3}{(9.841,5.718)}
\gppoint{gp mark 3}{(11.121,8.047)}
\gpcolor{color=gp lt color border}
\gpsetlinetype{gp lt border}
\draw[gp path] (1.872,7.789)--(1.872,1.651);
\draw[gp path] (1.872,0.985)--(10.033,0.985);
%% coordinates of the plot area
\gpdefrectangularnode{gp plot 1}{\pgfpoint{1.872cm}{0.985cm}}{\pgfpoint{11.947cm}{8.381cm}}
\end{tikzpicture}
%% gnuplot variables
 \caption{Grafico 1.060dgdecad.tex} \label{gr:1.060dgdecad.tex} \end{grafico}
\begin{grafico} \centering \begin{tikzpicture}[gnuplot]
%% generated with GNUPLOT 4.6p0 (Lua 5.1; terminal rev. 99, script rev. 100)
%% Tue 10 Jun 2014 11:14:32 PM CEST
\path (0.000,0.000) rectangle (12.500,8.750);
\gpcolor{color=gp lt color border}
\gpsetlinetype{gp lt border}
\gpsetlinewidth{1.00}
\draw[gp path] (1.872,1.725)--(2.052,1.725);
\node[gp node right] at (1.688,1.725) {-0.02};
\draw[gp path] (1.872,2.464)--(2.052,2.464);
\node[gp node right] at (1.688,2.464) {-0.015};
\draw[gp path] (1.872,3.204)--(2.052,3.204);
\node[gp node right] at (1.688,3.204) {-0.01};
\draw[gp path] (1.872,3.943)--(2.052,3.943);
\node[gp node right] at (1.688,3.943) {-0.005};
\draw[gp path] (1.872,4.683)--(2.052,4.683);
\node[gp node right] at (1.688,4.683) { 0};
\draw[gp path] (1.872,5.423)--(2.052,5.423);
\node[gp node right] at (1.688,5.423) { 0.005};
\draw[gp path] (1.872,6.162)--(2.052,6.162);
\node[gp node right] at (1.688,6.162) { 0.01};
\draw[gp path] (1.872,6.902)--(2.052,6.902);
\node[gp node right] at (1.688,6.902) { 0.015};
\draw[gp path] (1.872,7.641)--(2.052,7.641);
\node[gp node right] at (1.688,7.641) { 0.02};
\draw[gp path] (1.872,0.985)--(1.872,1.165);
\node[gp node center] at (1.872,0.677) { 0};
\draw[gp path] (3.887,0.985)--(3.887,1.165);
\node[gp node center] at (3.887,0.677) { 5};
\draw[gp path] (5.902,0.985)--(5.902,1.165);
\node[gp node center] at (5.902,0.677) { 10};
\draw[gp path] (7.917,0.985)--(7.917,1.165);
\node[gp node center] at (7.917,0.677) { 15};
\draw[gp path] (9.932,0.985)--(9.932,1.165);
\node[gp node center] at (9.932,0.677) { 20};
\draw[gp path] (1.872,7.937)--(1.872,1.133);
\draw[gp path] (1.872,0.985)--(10.013,0.985);
\node[gp node center,rotate=-270] at (0.246,4.683) {Ampiezza [giri]};
\node[gp node center] at (6.909,0.215) {Tempo $[s]$};
\gpcolor{rgb color={0.000,0.000,1.000}}
\gpsetlinetype{gp lt plot 0}
\draw[gp path] (1.872,4.683)--(1.892,5.571)--(1.912,6.458)--(1.932,7.198)--(1.953,7.641)%
  --(1.973,7.789)--(1.993,7.641)--(2.013,7.050)--(2.033,6.606)--(2.053,5.866)--(2.074,4.831)%
  --(2.094,3.795)--(2.114,2.908)--(2.134,2.168)--(2.154,1.429)--(2.174,1.281)--(2.194,1.281)%
  --(2.215,1.577)--(2.235,2.464)--(2.255,3.056)--(2.275,3.943)--(2.295,4.979)--(2.315,5.718)%
  --(2.335,6.606)--(2.356,7.198)--(2.376,7.641)--(2.396,7.493)--(2.416,7.346)--(2.436,6.902)%
  --(2.456,6.310)--(2.477,5.423)--(2.497,4.387)--(2.517,3.500)--(2.537,2.760)--(2.557,2.168)%
  --(2.577,1.577)--(2.597,1.429)--(2.618,1.577)--(2.638,2.168)--(2.658,2.612)--(2.678,3.500)%
  --(2.698,4.239)--(2.718,5.127)--(2.738,6.014)--(2.759,6.754)--(2.779,7.198)--(2.799,7.346)%
  --(2.819,7.346)--(2.839,7.198)--(2.859,6.606)--(2.880,5.866)--(2.900,4.979)--(2.920,4.091)%
  --(2.940,3.204)--(2.960,2.464)--(2.980,2.020)--(3.000,1.577)--(3.021,1.577)--(3.041,1.873)%
  --(3.061,2.464)--(3.081,2.908)--(3.101,3.648)--(3.121,4.535)--(3.141,5.423)--(3.162,6.310)%
  --(3.182,6.754)--(3.202,7.198)--(3.222,7.346)--(3.242,7.198)--(3.262,6.754)--(3.283,6.162)%
  --(3.303,5.571)--(3.323,4.683)--(3.343,3.795)--(3.363,3.056)--(3.383,2.316)--(3.403,1.873)%
  --(3.424,1.577)--(3.444,1.725)--(3.464,2.020)--(3.484,2.612)--(3.504,3.204)--(3.524,3.943)%
  --(3.544,4.831)--(3.565,5.718)--(3.585,6.310)--(3.605,6.754)--(3.625,7.198)--(3.645,7.198)%
  --(3.665,6.902)--(3.686,6.606)--(3.706,5.866)--(3.726,5.127)--(3.746,4.387)--(3.766,3.500)%
  --(3.786,2.908)--(3.806,2.316)--(3.827,1.873)--(3.847,2.020)--(3.867,2.020)--(3.887,2.168)%
  --(3.907,2.908)--(3.927,3.500)--(3.947,4.387)--(3.968,5.127)--(3.988,5.866)--(4.008,6.458)%
  --(4.028,6.754)--(4.048,7.050)--(4.068,7.050)--(4.089,6.754)--(4.109,6.310)--(4.129,5.571)%
  --(4.149,4.831)--(4.169,4.091)--(4.189,3.352)--(4.209,2.612)--(4.230,2.168)--(4.250,2.020)%
  --(4.270,2.020)--(4.290,2.168)--(4.310,2.464)--(4.330,3.204)--(4.350,3.795)--(4.371,4.535)%
  --(4.391,5.423)--(4.411,6.014)--(4.431,6.606)--(4.451,6.754)--(4.471,7.050)--(4.492,6.902)%
  --(4.512,6.606)--(4.532,6.014)--(4.552,5.423)--(4.572,4.683)--(4.592,3.943)--(4.612,3.204)%
  --(4.633,2.760)--(4.653,2.464)--(4.673,2.168)--(4.693,2.168)--(4.713,2.316)--(4.733,2.760)%
  --(4.753,3.500)--(4.774,4.091)--(4.794,4.683)--(4.814,5.571)--(4.834,6.162)--(4.854,6.606)%
  --(4.874,6.754)--(4.895,6.754)--(4.915,6.754)--(4.935,6.310)--(4.955,5.718)--(4.975,4.979)%
  --(4.995,4.387)--(5.015,3.795)--(5.036,3.056)--(5.056,2.612)--(5.076,2.316)--(5.096,2.316)%
  --(5.116,2.316)--(5.136,2.612)--(5.156,3.056)--(5.177,3.648)--(5.197,4.387)--(5.217,4.979)%
  --(5.237,5.718)--(5.257,6.310)--(5.277,6.458)--(5.298,6.606)--(5.318,6.754)--(5.338,6.606)%
  --(5.358,6.162)--(5.378,5.423)--(5.398,4.831)--(5.418,4.239)--(5.439,3.648)--(5.459,3.056)%
  --(5.479,2.612)--(5.499,2.316)--(5.519,2.316)--(5.539,2.464)--(5.559,2.760)--(5.580,3.352)%
  --(5.600,3.943)--(5.620,4.535)--(5.640,5.275)--(5.660,5.866)--(5.680,6.162)--(5.701,6.458)%
  --(5.721,6.606)--(5.741,6.606)--(5.761,6.310)--(5.781,5.718)--(5.801,5.275)--(5.821,4.683)%
  --(5.842,3.943)--(5.862,3.500)--(5.882,3.056)--(5.902,2.760)--(5.922,2.316)--(5.942,2.316)%
  --(5.962,2.612)--(5.983,2.908)--(6.003,3.648)--(6.023,4.239)--(6.043,4.831)--(6.063,5.423)%
  --(6.083,5.866)--(6.104,6.310)--(6.124,6.458)--(6.144,6.606)--(6.164,6.310)--(6.184,5.866)%
  --(6.204,5.571)--(6.224,5.127)--(6.245,4.239)--(6.265,3.795)--(6.285,3.500)--(6.305,2.908)%
  --(6.325,2.464)--(6.345,2.316)--(6.365,2.464)--(6.386,2.760)--(6.406,3.204)--(6.426,3.795)%
  --(6.446,4.535)--(6.466,4.831)--(6.486,5.423)--(6.507,6.014)--(6.527,6.458)--(6.547,6.458)%
  --(6.567,6.162)--(6.587,6.162)--(6.607,5.866)--(6.627,5.275)--(6.648,4.831)--(6.668,4.239)%
  --(6.688,3.648)--(6.708,3.056)--(6.728,2.760)--(6.748,2.612)--(6.768,2.464)--(6.789,2.908)%
  --(6.809,4.091)--(6.829,3.204)--(6.849,3.648)--(6.869,4.979)--(6.889,5.423)--(6.910,5.423)%
  --(6.930,5.866)--(6.950,6.902)--(6.970,6.606)--(6.990,5.866)--(7.010,5.571)--(7.030,6.014)%
  --(7.051,5.423)--(7.071,4.535)--(7.091,3.795)--(7.111,3.795)--(7.131,3.352)--(7.151,2.612)%
  --(7.171,2.612)--(7.192,2.908)--(7.212,2.760)--(7.232,2.908)--(7.252,3.795)--(7.272,4.387)%
  --(7.292,4.683)--(7.313,4.831)--(7.333,5.866)--(7.353,6.606)--(7.373,6.162)--(7.393,5.718)%
  --(7.413,6.014)--(7.433,6.310)--(7.454,5.571)--(7.474,4.683)--(7.494,4.239)--(7.514,4.091)%
  --(7.534,3.500)--(7.554,3.056)--(7.574,2.908)--(7.595,2.908)--(7.615,2.760)--(7.635,3.056)%
  --(7.655,3.648)--(7.675,4.091)--(7.695,4.091)--(7.716,4.831)--(7.736,5.866)--(7.756,6.014)%
  --(7.776,5.718)--(7.796,5.866)--(7.816,6.310)--(7.836,6.310)--(7.857,5.571)--(7.877,4.979)%
  --(7.897,4.683)--(7.917,4.387)--(7.937,3.795)--(7.957,3.204)--(7.977,2.908)--(7.998,2.908)%
  --(8.018,2.760)--(8.038,3.056)--(8.058,3.352)--(8.078,3.648)--(8.098,3.795)--(8.119,4.683)%
  --(8.139,5.423)--(8.159,5.571)--(8.179,5.423)--(8.199,5.866)--(8.219,6.458)--(8.239,6.014)%
  --(8.260,5.423)--(8.280,5.423)--(8.300,5.275)--(8.320,4.535)--(8.340,3.943)--(8.360,3.500)%
  --(8.380,3.204)--(8.401,3.056)--(8.421,2.908)--(8.441,3.056)--(8.461,3.056)--(8.481,3.204)%
  --(8.501,3.795)--(8.522,4.535)--(8.542,4.979)--(8.562,4.979)--(8.582,5.423)--(8.602,6.014)%
  --(8.622,6.310)--(8.642,6.014)--(8.663,5.423)--(8.683,5.718)--(8.703,5.718)--(8.723,4.979)%
  --(8.743,4.387)--(8.763,3.795)--(8.783,3.648)--(8.804,3.352)--(8.824,3.204)--(8.844,3.056)%
  --(8.864,2.908)--(8.884,3.056)--(8.904,3.648)--(8.925,4.239)--(8.945,4.387)--(8.965,4.535)%
  --(8.985,5.275)--(9.005,6.014)--(9.025,6.014)--(9.045,5.718)--(9.066,5.718)--(9.086,6.162)%
  --(9.106,5.866)--(9.126,4.979)--(9.146,4.535)--(9.166,4.239)--(9.186,3.943)--(9.207,3.352)%
  --(9.227,3.352)--(9.247,2.908)--(9.267,2.908)--(9.287,3.056)--(9.307,3.500)--(9.328,3.795)%
  --(9.348,3.943)--(9.368,4.387)--(9.388,5.127)--(9.408,5.571)--(9.428,5.423)--(9.448,5.571)%
  --(9.469,5.866)--(9.489,5.866)--(9.509,5.571)--(9.529,5.275)--(9.549,4.831)--(9.569,4.683)%
  --(9.589,4.239)--(9.610,3.648)--(9.630,3.352)--(9.650,3.056)--(9.670,2.908)--(9.690,3.204)%
  --(9.710,3.500)--(9.731,3.500)--(9.751,3.648)--(9.771,4.239)--(9.791,4.831)--(9.811,5.127)%
  --(9.831,5.275)--(9.851,5.571)--(9.872,5.866)--(9.892,6.014)--(9.912,5.718)--(9.932,5.275)%
  --(9.952,5.127)--(9.972,4.831)--(9.992,4.239)--(10.013,4.091);
\gpcolor{color=gp lt color border}
\node[gp node right] at (10.479,8.047) {Punti tangenza};
\gpcolor{rgb color={1.000,0.000,0.000}}
\gpsetpointsize{4.00}
\gppoint{gp mark 3}{(1.877,4.905)}
\gppoint{gp mark 3}{(1.983,7.937)}
\gppoint{gp mark 3}{(2.184,1.133)}
\gppoint{gp mark 3}{(2.386,7.567)}
\gppoint{gp mark 3}{(2.597,1.429)}
\gppoint{gp mark 3}{(2.809,7.420)}
\gppoint{gp mark 3}{(3.021,1.577)}
\gppoint{gp mark 3}{(3.222,7.346)}
\gppoint{gp mark 3}{(3.424,1.577)}
\gppoint{gp mark 3}{(3.635,7.346)}
\gppoint{gp mark 3}{(3.847,2.020)}
\gppoint{gp mark 3}{(4.058,7.198)}
\gppoint{gp mark 3}{(4.270,2.020)}
\gppoint{gp mark 3}{(4.471,7.050)}
\gppoint{gp mark 3}{(4.673,2.168)}
\gppoint{gp mark 3}{(4.884,6.754)}
\gppoint{gp mark 3}{(5.096,2.316)}
\gppoint{gp mark 3}{(5.308,6.828)}
\gppoint{gp mark 3}{(5.519,2.316)}
\gppoint{gp mark 3}{(5.721,6.606)}
\gppoint{gp mark 3}{(5.922,2.316)}
\gppoint{gp mark 3}{(6.134,6.754)}
\gppoint{gp mark 3}{(6.345,2.316)}
\gppoint{gp mark 3}{(6.557,6.162)}
\gppoint{gp mark 3}{(6.758,2.242)}
\gppoint{gp mark 3}{(6.960,6.976)}
\gppoint{gp mark 3}{(7.182,2.982)}
\gppoint{gp mark 3}{(7.383,5.571)}
\gppoint{gp mark 3}{(7.585,2.982)}
\gppoint{gp mark 3}{(7.806,6.310)}
\gppoint{gp mark 3}{(8.018,2.760)}
\gppoint{gp mark 3}{(8.219,6.458)}
\gppoint{gp mark 3}{(8.421,2.908)}
\gppoint{gp mark 3}{(8.632,6.310)}
\gppoint{gp mark 3}{(8.854,2.834)}
\gppoint{gp mark 3}{(9.055,5.497)}
\gppoint{gp mark 3}{(9.257,2.834)}
\gppoint{gp mark 3}{(9.479,6.014)}
\gppoint{gp mark 3}{(9.680,3.056)}
\gppoint{gp mark 3}{(9.882,6.162)}
\gppoint{gp mark 3}{(11.121,8.047)}
\gpcolor{color=gp lt color border}
\gpsetlinetype{gp lt border}
\draw[gp path] (1.872,7.937)--(1.872,1.133);
\draw[gp path] (1.872,0.985)--(10.013,0.985);
%% coordinates of the plot area
\gpdefrectangularnode{gp plot 1}{\pgfpoint{1.872cm}{0.985cm}}{\pgfpoint{11.947cm}{8.381cm}}
\end{tikzpicture}
%% gnuplot variables
 \caption{Grafico 1.080dgdecad.tex} \label{gr:1.080dgdecad.tex} \end{grafico}
\begin{grafico} \centering \begin{tikzpicture}[gnuplot]
%% generated with GNUPLOT 4.6p0 (Lua 5.1; terminal rev. 99, script rev. 100)
%% Mon 09 Jun 2014 04:19:46 PM CEST
\path (0.000,0.000) rectangle (12.500,8.750);
\gpcolor{color=gp lt color border}
\gpsetlinetype{gp lt border}
\gpsetlinewidth{1.00}
\draw[gp path] (1.872,1.745)--(2.052,1.745);
\node[gp node right] at (1.688,1.745) {-0.02};
\draw[gp path] (1.872,2.505)--(2.052,2.505);
\node[gp node right] at (1.688,2.505) {-0.015};
\draw[gp path] (1.872,3.265)--(2.052,3.265);
\node[gp node right] at (1.688,3.265) {-0.01};
\draw[gp path] (1.872,4.025)--(2.052,4.025);
\node[gp node right] at (1.688,4.025) {-0.005};
\draw[gp path] (1.872,4.785)--(2.052,4.785);
\node[gp node right] at (1.688,4.785) { 0};
\draw[gp path] (1.872,5.545)--(2.052,5.545);
\node[gp node right] at (1.688,5.545) { 0.005};
\draw[gp path] (1.872,6.305)--(2.052,6.305);
\node[gp node right] at (1.688,6.305) { 0.01};
\draw[gp path] (1.872,7.065)--(2.052,7.065);
\node[gp node right] at (1.688,7.065) { 0.015};
\draw[gp path] (1.872,0.985)--(1.872,1.165);
\node[gp node center] at (1.872,0.677) { 0};
\draw[gp path] (3.887,0.985)--(3.887,1.165);
\node[gp node center] at (3.887,0.677) { 5};
\draw[gp path] (5.902,0.985)--(5.902,1.165);
\node[gp node center] at (5.902,0.677) { 10};
\draw[gp path] (7.917,0.985)--(7.917,1.165);
\node[gp node center] at (7.917,0.677) { 15};
\draw[gp path] (9.932,0.985)--(9.932,1.165);
\node[gp node center] at (9.932,0.677) { 20};
\draw[gp path] (1.872,7.692)--(1.872,1.726);
\draw[gp path] (1.872,0.985)--(9.992,0.985);
\node[gp node center,rotate=-270] at (0.246,4.405) {Ampiezza [???]};
\node[gp node center] at (6.909,0.215) {Tempo $[s]$};
\node[gp node center] at (6.909,8.287) {Dati decadimento 1.100d};
\gpcolor{rgb color={0.000,0.000,1.000}}
\gpsetlinetype{gp lt plot 0}
\draw[gp path] (1.872,1.745)--(1.892,1.745)--(1.912,2.353)--(1.932,3.113)--(1.953,3.873)%
  --(1.973,4.633)--(1.993,5.697)--(2.013,6.457)--(2.033,7.065)--(2.053,7.217)--(2.074,7.673)%
  --(2.094,7.521)--(2.114,6.913)--(2.134,6.305)--(2.154,5.697)--(2.174,5.241)--(2.194,4.177)%
  --(2.215,3.113)--(2.235,2.353)--(2.255,1.897)--(2.275,1.745)--(2.295,1.745)--(2.315,2.049)%
  --(2.335,2.505)--(2.356,3.265)--(2.376,4.329)--(2.396,4.937)--(2.416,5.849)--(2.436,6.609)%
  --(2.456,7.065)--(2.477,7.521)--(2.497,7.369)--(2.517,7.065)--(2.537,6.609)--(2.557,6.153)%
  --(2.577,5.697)--(2.597,4.633)--(2.618,3.569)--(2.638,2.809)--(2.658,2.505)--(2.678,2.049)%
  --(2.698,1.897)--(2.718,1.745)--(2.738,2.201)--(2.759,2.809)--(2.779,3.569)--(2.799,4.481)%
  --(2.819,5.241)--(2.839,5.697)--(2.859,6.609)--(2.880,7.217)--(2.900,7.369)--(2.920,7.065)%
  --(2.940,7.065)--(2.960,6.609)--(2.980,6.001)--(3.000,5.089)--(3.021,4.177)--(3.041,3.265)%
  --(3.061,2.809)--(3.081,2.353)--(3.101,2.049)--(3.121,1.897)--(3.141,2.201)--(3.162,2.657)%
  --(3.182,3.113)--(3.202,4.025)--(3.222,4.633)--(3.242,5.393)--(3.262,6.153)--(3.283,6.761)%
  --(3.303,6.761)--(3.323,6.913)--(3.343,7.065)--(3.363,6.761)--(3.383,6.001)--(3.403,5.241)%
  --(3.424,4.937)--(3.444,3.873)--(3.464,3.265)--(3.484,2.809)--(3.504,2.505)--(3.524,2.201)%
  --(3.544,2.201)--(3.565,2.201)--(3.585,2.961)--(3.605,3.417)--(3.625,4.025)--(3.645,5.089)%
  --(3.665,5.697)--(3.686,6.305)--(3.706,6.761)--(3.726,7.065)--(3.746,7.217)--(3.766,6.913)%
  --(3.786,6.457)--(3.806,6.001)--(3.827,5.241)--(3.847,4.481)--(3.867,3.721)--(3.887,2.961)%
  --(3.907,2.657)--(3.927,2.353)--(3.947,2.201)--(3.968,2.353)--(3.988,2.657)--(4.008,3.113)%
  --(4.028,3.569)--(4.048,4.481)--(4.068,5.241)--(4.089,5.849)--(4.109,6.305)--(4.129,6.609)%
  --(4.149,6.913)--(4.169,6.913)--(4.189,6.609)--(4.209,6.001)--(4.230,5.697)--(4.250,4.937)%
  --(4.270,4.329)--(4.290,3.721)--(4.310,2.961)--(4.330,2.505)--(4.350,2.201)--(4.371,2.353)%
  --(4.391,2.505)--(4.411,2.657)--(4.431,3.417)--(4.451,4.329)--(4.471,4.937)--(4.492,5.545)%
  --(4.512,6.153)--(4.532,6.609)--(4.552,6.913)--(4.572,6.913)--(4.592,6.913)--(4.612,6.609)%
  --(4.633,5.849)--(4.653,5.393)--(4.673,4.785)--(4.693,4.025)--(4.713,3.417)--(4.733,3.113)%
  --(4.753,2.657)--(4.774,2.657)--(4.794,2.353)--(4.814,2.961)--(4.834,3.417)--(4.854,3.721)%
  --(4.874,4.329)--(4.895,4.785)--(4.915,5.545)--(4.935,6.305)--(4.955,6.609)--(4.975,6.761)%
  --(4.995,6.913)--(5.015,6.609)--(5.036,6.001)--(5.056,5.697)--(5.076,4.937)--(5.096,4.481)%
  --(5.116,3.873)--(5.136,3.113)--(5.156,2.961)--(5.177,2.505)--(5.197,2.505)--(5.217,2.657)%
  --(5.237,2.961)--(5.257,3.265)--(5.277,3.873)--(5.298,4.329)--(5.318,5.241)--(5.338,5.849)%
  --(5.358,6.153)--(5.378,6.457)--(5.398,6.761)--(5.418,6.609)--(5.439,6.305)--(5.459,6.001)%
  --(5.479,5.545)--(5.499,4.785)--(5.519,4.177)--(5.539,3.569)--(5.559,3.113)--(5.580,2.809)%
  --(5.600,2.505)--(5.620,2.809)--(5.640,2.961)--(5.660,3.265)--(5.680,3.721)--(5.701,4.025)%
  --(5.721,4.785)--(5.741,5.545)--(5.761,6.001)--(5.781,6.457)--(5.801,6.457)--(5.821,6.609)%
  --(5.842,6.305)--(5.862,6.153)--(5.882,5.697)--(5.902,5.241)--(5.922,4.785)--(5.942,4.025)%
  --(5.962,3.721)--(5.983,3.265)--(6.003,2.961)--(6.023,2.657)--(6.043,2.809)--(6.063,2.961)%
  --(6.083,3.417)--(6.104,3.873)--(6.124,4.481)--(6.144,5.089)--(6.164,5.545)--(6.184,6.153)%
  --(6.204,6.457)--(6.224,6.305)--(6.245,6.609)--(6.265,6.305)--(6.285,5.849)--(6.305,5.545)%
  --(6.325,5.089)--(6.345,4.633)--(6.365,4.025)--(6.386,3.417)--(6.406,3.113)--(6.426,2.961)%
  --(6.446,2.809)--(6.466,2.809)--(6.486,3.265)--(6.507,3.569)--(6.527,3.873)--(6.547,4.481)%
  --(6.567,5.089)--(6.587,5.849)--(6.607,6.153)--(6.627,6.305)--(6.648,6.305)--(6.668,6.305)%
  --(6.688,6.001)--(6.708,5.849)--(6.728,5.241)--(6.748,4.633)--(6.768,4.177)--(6.789,3.721)%
  --(6.809,3.417)--(6.829,3.113)--(6.849,2.809)--(6.869,2.809)--(6.889,3.113)--(6.910,3.417)%
  --(6.930,4.025)--(6.950,4.329)--(6.970,4.785)--(6.990,5.241)--(7.010,5.697)--(7.030,6.153)%
  --(7.051,6.305)--(7.071,6.305)--(7.091,6.153)--(7.111,6.001)--(7.131,5.545)--(7.151,5.241)%
  --(7.171,4.633)--(7.192,4.177)--(7.212,3.721)--(7.232,3.569)--(7.252,3.113)--(7.272,2.961)%
  --(7.292,3.113)--(7.313,3.265)--(7.333,3.569)--(7.353,3.873)--(7.373,4.481)--(7.393,5.089)%
  --(7.413,5.393)--(7.433,5.849)--(7.454,5.849)--(7.474,6.305)--(7.494,6.305)--(7.514,6.305)%
  --(7.534,5.849)--(7.554,5.393)--(7.574,4.937)--(7.595,4.481)--(7.615,3.873)--(7.635,3.721)%
  --(7.655,3.265)--(7.675,3.113)--(7.695,3.113)--(7.716,3.113)--(7.736,3.265)--(7.756,3.873)%
  --(7.776,4.177)--(7.796,4.481)--(7.816,4.937)--(7.836,5.393)--(7.857,6.001)--(7.877,6.153)%
  --(7.897,6.153)--(7.917,5.849)--(7.937,6.001)--(7.957,5.697)--(7.977,5.089)--(7.998,4.785)%
  --(8.018,4.481)--(8.038,3.873)--(8.058,3.721)--(8.078,3.265)--(8.098,2.961)--(8.119,3.113)%
  --(8.139,3.265)--(8.159,3.569)--(8.179,3.873)--(8.199,4.481)--(8.219,4.785)--(8.239,5.089)%
  --(8.260,5.697)--(8.280,5.697)--(8.300,6.001)--(8.320,6.153)--(8.340,6.153)--(8.360,6.001)%
  --(8.380,5.393)--(8.401,5.089)--(8.421,4.633)--(8.441,4.329)--(8.461,3.721)--(8.481,3.417)%
  --(8.501,3.265)--(8.522,3.417)--(8.542,3.417)--(8.562,3.265)--(8.582,3.721)--(8.602,4.177)%
  --(8.622,4.785)--(8.642,5.089)--(8.663,5.241)--(8.683,5.545)--(8.703,5.697)--(8.723,6.001)%
  --(8.743,6.153)--(8.763,5.849)--(8.783,5.393)--(8.804,5.241)--(8.824,4.937)--(8.844,4.633)%
  --(8.864,4.025)--(8.884,3.721)--(8.904,3.417)--(8.925,3.417)--(8.945,3.265)--(8.965,3.265)%
  --(8.985,3.417)--(9.005,3.721)--(9.025,4.177)--(9.045,4.937)--(9.066,5.089)--(9.086,5.241)%
  --(9.106,5.545)--(9.126,5.849)--(9.146,6.001)--(9.166,5.849)--(9.186,5.697)--(9.207,5.393)%
  --(9.227,5.241)--(9.247,4.937)--(9.267,4.329)--(9.287,3.873)--(9.307,3.721)--(9.328,3.569)%
  --(9.348,3.417)--(9.368,3.417)--(9.388,3.417)--(9.408,3.873)--(9.428,4.329)--(9.448,4.329)%
  --(9.469,4.785)--(9.489,5.089)--(9.509,5.393)--(9.529,5.697)--(9.549,5.849)--(9.569,5.849)%
  --(9.589,5.697)--(9.610,5.545)--(9.630,5.393)--(9.650,4.937)--(9.670,4.481)--(9.690,4.177)%
  --(9.710,4.025)--(9.731,3.721)--(9.751,3.569)--(9.771,3.417)--(9.791,3.417)--(9.811,3.873)%
  --(9.831,4.025)--(9.851,4.177)--(9.872,4.633)--(9.892,4.937)--(9.912,5.393)--(9.932,5.697)%
  --(9.952,5.849)--(9.972,5.849)--(9.992,5.697);
\gpcolor{color=gp lt color border}
\node[gp node right] at (10.479,7.491) {Massimi e minimi};
\gpcolor{rgb color={1.000,0.000,0.000}}
\gpsetpointsize{4.00}
\gppoint{gp mark 3}{(2.099,7.692)}
\gppoint{gp mark 3}{(2.502,7.540)}
\gppoint{gp mark 3}{(2.916,7.375)}
\gppoint{gp mark 3}{(3.360,7.071)}
\gppoint{gp mark 3}{(3.763,7.223)}
\gppoint{gp mark 3}{(4.179,6.951)}
\gppoint{gp mark 3}{(4.582,6.951)}
\gppoint{gp mark 3}{(5.012,6.919)}
\gppoint{gp mark 3}{(5.422,6.767)}
\gppoint{gp mark 3}{(5.838,6.615)}
\gppoint{gp mark 3}{(6.265,6.609)}
\gppoint{gp mark 3}{(6.658,6.324)}
\gppoint{gp mark 3}{(7.081,6.324)}
\gppoint{gp mark 3}{(7.504,6.362)}
\gppoint{gp mark 3}{(7.907,6.172)}
\gppoint{gp mark 3}{(8.350,6.172)}
\gppoint{gp mark 3}{(8.760,6.159)}
\gppoint{gp mark 3}{(9.166,6.001)}
\gppoint{gp mark 3}{(9.579,5.868)}
\gppoint{gp mark 3}{(9.982,5.868)}
\gppoint{gp mark 3}{(2.305,1.726)}
\gppoint{gp mark 3}{(3.138,1.891)}
\gppoint{gp mark 3}{(3.555,2.163)}
\gppoint{gp mark 3}{(3.968,2.201)}
\gppoint{gp mark 3}{(4.374,2.195)}
\gppoint{gp mark 3}{(4.811,2.340)}
\gppoint{gp mark 3}{(5.207,2.448)}
\gppoint{gp mark 3}{(5.620,2.505)}
\gppoint{gp mark 3}{(6.476,2.790)}
\gppoint{gp mark 3}{(6.879,2.771)}
\gppoint{gp mark 3}{(7.292,2.961)}
\gppoint{gp mark 3}{(8.122,2.955)}
\gppoint{gp mark 3}{(8.577,3.246)}
\gppoint{gp mark 3}{(8.975,3.246)}
\gppoint{gp mark 3}{(9.378,3.398)}
\gppoint{gp mark 3}{(9.801,3.398)}
\gppoint{gp mark 3}{(11.121,7.491)}
\gpcolor{color=gp lt color border}
\gpsetlinetype{gp lt border}
\draw[gp path] (1.872,7.692)--(1.872,1.726);
\draw[gp path] (1.872,0.985)--(9.992,0.985);
%% coordinates of the plot area
\gpdefrectangularnode{gp plot 1}{\pgfpoint{1.872cm}{0.985cm}}{\pgfpoint{11.947cm}{7.825cm}}
\end{tikzpicture}
%% gnuplot variables
 \caption{Grafico 1.100dgdecad.tex} \label{gr:1.100dgdecad.tex} \end{grafico}
\begin{grafico} \centering \begin{tikzpicture}[gnuplot]
%% generated with GNUPLOT 4.6p0 (Lua 5.1; terminal rev. 99, script rev. 100)
%% Tue 10 Jun 2014 07:45:02 PM CEST
\path (0.000,0.000) rectangle (12.500,8.750);
\gpcolor{color=gp lt color border}
\gpsetlinetype{gp lt border}
\gpsetlinewidth{1.00}
\draw[gp path] (1.688,2.125)--(1.868,2.125);
\node[gp node right] at (1.504,2.125) { 0.05};
\draw[gp path] (1.688,3.265)--(1.868,3.265);
\node[gp node right] at (1.504,3.265) { 0.1};
\draw[gp path] (1.688,4.405)--(1.868,4.405);
\node[gp node right] at (1.504,4.405) { 0.15};
\draw[gp path] (1.688,5.545)--(1.868,5.545);
\node[gp node right] at (1.504,5.545) { 0.2};
\draw[gp path] (1.688,6.685)--(1.868,6.685);
\node[gp node right] at (1.504,6.685) { 0.25};
\draw[gp path] (1.688,0.985)--(1.688,1.165);
\node[gp node center] at (1.688,0.677) { 0.9};
\draw[gp path] (4.253,0.985)--(4.253,1.165);
\node[gp node center] at (4.253,0.677) { 0.95};
\draw[gp path] (6.817,0.985)--(6.817,1.165);
\node[gp node center] at (6.817,0.677) { 1};
\draw[gp path] (9.382,0.985)--(9.382,1.165);
\node[gp node center] at (9.382,0.677) { 1.05};
\draw[gp path] (11.947,0.985)--(11.947,1.165);
\node[gp node center] at (11.947,0.677) { 1.1};
\draw[gp path] (1.688,7.740)--(1.688,1.488);
\draw[gp path] (1.688,0.985)--(11.947,0.985);
\node[gp node center,rotate=-270] at (0.246,4.405) {Ampiezza [???]};
\node[gp node center] at (6.817,0.215) {Frequenza: [???]};
\node[gp node center] at (6.817,8.287) {Ampiezze a regime};
\gpcolor{color=gp lt color 0}
\gpsetlinetype{gp lt plot 0}
\draw[gp path] (1.688,1.680)--(1.688,1.725);
\draw[gp path] (1.598,1.680)--(1.778,1.680);
\draw[gp path] (1.598,1.725)--(1.778,1.725);
\draw[gp path] (2.714,1.996)--(2.714,2.036);
\draw[gp path] (2.624,1.996)--(2.804,1.996);
\draw[gp path] (2.624,2.036)--(2.804,2.036);
\draw[gp path] (3.740,2.865)--(3.740,2.933);
\draw[gp path] (3.650,2.865)--(3.830,2.865);
\draw[gp path] (3.650,2.933)--(3.830,2.933);
\draw[gp path] (4.766,5.954)--(4.766,5.989);
\draw[gp path] (4.676,5.954)--(4.856,5.954);
\draw[gp path] (4.676,5.989)--(4.856,5.989);
\draw[gp path] (5.022,6.437)--(5.022,6.797);
\draw[gp path] (4.932,6.437)--(5.112,6.437);
\draw[gp path] (4.932,6.797)--(5.112,6.797);
\draw[gp path] (5.073,5.985)--(5.073,6.448);
\draw[gp path] (4.983,5.985)--(5.163,5.985);
\draw[gp path] (4.983,6.448)--(5.163,6.448);
\draw[gp path] (5.125,7.369)--(5.125,7.479);
\draw[gp path] (5.035,7.369)--(5.215,7.369);
\draw[gp path] (5.035,7.479)--(5.215,7.479);
\draw[gp path] (5.176,7.469)--(5.176,7.740);
\draw[gp path] (5.086,7.469)--(5.266,7.469);
\draw[gp path] (5.086,7.740)--(5.266,7.740);
\draw[gp path] (5.227,7.568)--(5.227,7.640);
\draw[gp path] (5.137,7.568)--(5.317,7.568);
\draw[gp path] (5.137,7.640)--(5.317,7.640);
\draw[gp path] (5.279,7.206)--(5.279,7.287);
\draw[gp path] (5.189,7.206)--(5.369,7.206);
\draw[gp path] (5.189,7.287)--(5.369,7.287);
\draw[gp path] (5.535,6.142)--(5.535,6.197);
\draw[gp path] (5.445,6.142)--(5.625,6.142);
\draw[gp path] (5.445,6.197)--(5.625,6.197);
\draw[gp path] (5.792,4.741)--(5.792,4.880);
\draw[gp path] (5.702,4.741)--(5.882,4.741);
\draw[gp path] (5.702,4.880)--(5.882,4.880);
\draw[gp path] (6.817,2.559)--(6.817,2.703);
\draw[gp path] (6.727,2.559)--(6.907,2.559);
\draw[gp path] (6.727,2.703)--(6.907,2.703);
\draw[gp path] (7.843,2.073)--(7.843,2.164);
\draw[gp path] (7.753,2.073)--(7.933,2.073);
\draw[gp path] (7.753,2.164)--(7.933,2.164);
\draw[gp path] (8.869,1.820)--(8.869,1.852);
\draw[gp path] (8.779,1.820)--(8.959,1.820);
\draw[gp path] (8.779,1.852)--(8.959,1.852);
\draw[gp path] (9.895,1.663)--(9.895,1.685);
\draw[gp path] (9.805,1.663)--(9.985,1.663);
\draw[gp path] (9.805,1.685)--(9.985,1.685);
\draw[gp path] (10.921,1.554)--(10.921,1.587);
\draw[gp path] (10.831,1.554)--(11.011,1.554);
\draw[gp path] (10.831,1.587)--(11.011,1.587);
\draw[gp path] (11.947,1.488)--(11.947,1.517);
\draw[gp path] (11.857,1.488)--(12.037,1.488);
\draw[gp path] (11.857,1.517)--(12.037,1.517);
\gpsetpointsize{4.00}
\gppoint{gp mark 1}{(1.688,1.702)}
\gppoint{gp mark 1}{(2.714,2.016)}
\gppoint{gp mark 1}{(3.740,2.899)}
\gppoint{gp mark 1}{(4.766,5.971)}
\gppoint{gp mark 1}{(5.022,6.617)}
\gppoint{gp mark 1}{(5.073,6.217)}
\gppoint{gp mark 1}{(5.125,7.424)}
\gppoint{gp mark 1}{(5.176,7.604)}
\gppoint{gp mark 1}{(5.227,7.604)}
\gppoint{gp mark 1}{(5.279,7.246)}
\gppoint{gp mark 1}{(5.535,6.170)}
\gppoint{gp mark 1}{(5.792,4.811)}
\gppoint{gp mark 1}{(6.817,2.631)}
\gppoint{gp mark 1}{(7.843,2.118)}
\gppoint{gp mark 1}{(8.869,1.836)}
\gppoint{gp mark 1}{(9.895,1.674)}
\gppoint{gp mark 1}{(10.921,1.571)}
\gppoint{gp mark 1}{(11.947,1.502)}
\gpcolor{rgb color={0.000,0.000,1.000}}
\gpsetlinetype{gp lt plot 1}
\draw[gp path] (1.688,1.702)--(2.714,2.016)--(3.740,2.899)--(4.766,5.971)--(5.022,6.617)%
  --(5.073,6.217)--(5.125,7.424)--(5.176,7.604)--(5.227,7.604)--(5.279,7.246)--(5.535,6.170)%
  --(5.792,4.811)--(6.817,2.631)--(7.843,2.118)--(8.869,1.836)--(9.895,1.674)--(10.921,1.571)%
  --(11.947,1.502);
\gpcolor{color=gp lt color border}
\gpsetlinetype{gp lt border}
\draw[gp path] (1.688,7.740)--(1.688,1.488);
\draw[gp path] (1.688,0.985)--(11.947,0.985);
%% coordinates of the plot area
\gpdefrectangularnode{gp plot 1}{\pgfpoint{1.688cm}{0.985cm}}{\pgfpoint{11.947cm}{7.825cm}}
\end{tikzpicture}
%% gnuplot variables
 \caption{Grafico frequenza.tex} \label{gr:frequenza.tex} \end{grafico}

	
\clearpage
\section{Analisi dei dati}
	Come detto nella descrizione dell'apparato strumentale, il tasso di rilevamento dei dati è di 20 al secondo. Questo corrisponde a una
frequenza di campionamento di 20 Hertz, di molto superiore al 
Nyquist rate necessario per il pendolo (il doppio della massima frequenza
necessaria), dato che come verificabile a vista ha una frequenza dell'ordine di 1 Hz. 
Non ci sono quindi problemi di aliasing e sottocampionamento.
Per quanto riguarda l'offset, è stato rifatta la calibrazione prima di ogni presa dati (inizio giornata) e si può vedere che era
 calibrato da un'evidente simmetria rispetto all'asse delle ascisse.
Per il calcolo dei massimi è stato utilizzato un programma che riconoscesse i punti di massimo e minimo approssimando
la funzione come una parabola in un intorno dei dati ``stazionari`` (dati massimi e minimi locali) usando il dato precedente e il successivo,
vincolando la parabola a passare per questi 3 punti e trovandone il vertice. L'errore legato all'utilizzo
 di questa approssimazione è, come noto dallo sviluppo di Taylor delle funzioni goniometriche, $o(x^3)$, che essendo lo step 0.05 è compreso tra $10^{-3}$ e $10^{-4}$,
  e trascurabile rispetto agli altri errori.
 Per una stima delle ampiezze legate alle frequenze di oscillazione sono stati presi i valori medi delle ordinate dei massimi (e dei valori assoluti dei minimi).

Una stima della pulsazione di risonanza è stata fatta con un processo di esplorazione iniziale che ha permesso, attraverso
il metodo di bisezione, di concentrarsi sull'area nella quale l'ampiezza era più alta. Il valore finale trovato risulta di...
Per stimare il coefficiente di smorzamento $\gamma$ legato al movimento dell'acqua è stato tentato un approccio diretto con gnuplot, ma i problemi del suo algoritmo 
(Levenberg–Marquardt, una forma di step gradient descent) nel caso di funzioni come questa, in cui il gradiente dei minimi quadrati è pieno di punti 
stazionari locali, ne hanno impedito l'applicabilità pratica. Quindi è stato scelto un'altro approccio che elimini questi problemi, in particolare limitando lo
studio a una semplice funzione esponenziale, che è stata ulteriormente semplificata in un fit lineare usando una scala logaritmica.
Di consequenza, sono stati cercati gli $x_i \mid f(x_i)=0$. Essendo la funzione
\begin{equation}
	\theta_0 e^{-\gamma t} sin(\omega_s t + \phi),
\end{equation}
poichè $e^{-\gamma t} > 0$ gli zeri della funzione sono solo gli zeri del seno. Quindi i punti medi 
$x_m = \frac{x_i + x_{i+1}}{2}$ fra gli zeri sono i punti in cui $sin(\omega_s t + \phi) = 1$.
Interpolando questi punti la funzione diventa dunque
\[
	\theta_0 e^{-\gamma t} \cdot 1 = \theta_0 e^{-\gamma t}.
\]
Interpolando questa funzione con i punti $(x_m,\log{f(x_m)})$ (le coordinate $f(x_i)$ sono state calcolate, nei casi in cui fossero un punto già dei dati,
interpolando linearmente tra i due punti più prossimi).

%L'ampiezza massima della forzante è stata regolata a 10 milligiri perché altrimenti veniva
La pulsazione di smorzamento è stata ottenuta attraverso una media pesata delle pulsazioni ottenute dallo studio dei periodi dei
grafici durante la fase di smorzamento (vedasi tabella...)
La pulsazione propria è stata trovata attraverso la formula $\omega_0 =\sqrt{{\omega_s} ^ 2 + \gamma ^ 2 }$.
Gli errori sono stati stimati a partire da una stima diretta con la sommatoria degli scarti al quadrato diviso $N-1$.
I grafici rivelano che, entro gli errori casuali...



\section{Conclusioni}
	L'esperimento ha creato dei risultati che bene si accordano con le rpevisioni sperimentali (Per esempio, si può vedere dal fatto
 che i coefficienti di smorzamento sono molto simili per tutte le prove effettuate). La curva di risonanza si rivela un buono
 strumento per lo studio della frequenza di risonanza, e il grafico da essa disegnato non si discosta molto da quello atteso.
 L'analisi dati è stata fatta in modo da minimizzare gli errori, che risultano comunque molto buoni per tutti i risultati presentati.
 

	
\section{Codice}
	\input{./sezioni/codice.tex}
	
%\subsection{Esempio immagini}
%\begin{figure}[p]
% \centering
% \includegraphics[width=0.8\textwidth]{spazio1}
% \caption{Spazio!}
% \label{fig:spazio1}
%\end{figure}

%\end{multicols}

\end{document}
