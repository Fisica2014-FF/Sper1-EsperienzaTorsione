L'esperimento ha creato dei risultati che bene si accordano con le previsioni sperimentali (per esempio, si può vedere dal fatto
 che i coefficienti di smorzamento sono molto simili per tutte le prove effettuate). La curva di risonanza si rivela un buono
 strumento per lo studio delle ampiezze in funzione della frequenza, e il grafico da essa disegnato non si discosta molto da quello atteso.
  Per migliorare I risultati ottenuti, sarebbe stato necessario ridurre le interazioni dell'ambiente con il pendolo (impossibile con
 l'apparato strumnentale dato) oppure effettuare un amggior numero di indagini anche a frequenze differenti.
 L'analisi dati è stata fatta in modo da minimizzare gli errori, che risultano comunque molto buoni per tutti i risultati presentati.
