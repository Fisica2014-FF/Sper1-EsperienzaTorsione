L'esperimento ha creato dei risultati che bene si accordano con le rpevisioni sperimentali (Per esempio, si può vedere dal fatto
 che i coefficienti di smorzamento sono molto simili per tutte le prove effettuate). La curva di risonanza si rivela un buono
 strumento per lo studio della frequenza di risonanza, e il grafico da essa disegnato non si discosta molto da quello atteso.
 L'analisi dati è stata fatta in modo da minimizzare gli errori, che risultano comunque molto buoni per tutti i risultati presentati.
 
