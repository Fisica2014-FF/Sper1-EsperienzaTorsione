Per poter stimare la frequenza di risonanza si è proceduto azionando il motore e mettendo in oscillazione la piattaforma rotante.
Partendo dalle informazioni fornite e dall’apparecchiatura si è deciso di porre l’ampiezza a 10 millesimi di giro e di variare gli
intervalli delle frequenze di 0.020 Hz e acquisendo campioni di misure per un tempo totale di circa 10 secondi durante la fase a
regime, interrompendo la misurazione per eseguire lo store dei dati così ricavati, spegnendo il motore e intraprendendo una seconda
fase di registrazione dati per una durata di circa 20 secondi per la fase di smorzamento, al termine del quale è stato  eseguito
nuovamente lo store dei dati. Le misure sonos tate prese solo non appena è stato evidente dai grafici di riferimento il carattere
periodico del moto del pendolo. 

L'apparato strumentale è stato ricalibrato prima delle prese dati della giornata per permetterne un
funzionamento ottimale. L' ampiezza massima di oscillazione della forzante è stata scelta per permettere oscillazioni abbastanza
ampie da studiare ma non così ampie da rendere caotico il moto del pendolo, costringendolo a muoversi sul piano perpendicolare
all'asse di oscillazione. Attraverso questo metodo è stato possibile ottenere una panoramica del comportamento oscillatorio del 
materiale di studio e identificare efficacemente il settore in cui avveniva il fenomeno di risonanza. Tale settore è stato poi 
sondato ricorrendo al metodo di bisezione restringendosi in un intorno di valori della frequenza e aumentando l’esposizione 
dell’acquisizione dati. 

In questa fase (che si concentra nell’intervallo tra 0.965 e 0.970 Hz) sono stati registrati valori per la durata di circa 100
secondi per la fase a regime e di 40 secondi circa per la fase in smorzamento, una volta spento il motore. Questo offre agli
sperimentatori la possibilità di analizzare una serie di campioni più concentrati avente intervalli di frequenza di 0.001 Hz,
permettendo di stimare il più efficacemente possibile la frequenza di risonanza.
