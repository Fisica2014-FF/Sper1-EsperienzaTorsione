
\begin{verbatim}


#include <iostream>
#include <cmath>
using namespace std;

int main()
{
	double o0, so0, os, sos, gamma, sgamma;
	cerr << "Inserire in ordine omegas, suo errore, gamma, suo errore. " << endl;
	cin >> os >> sos >> gamma >> sgamma;
	o0 = sqrt( ( os * os ) + ( gamma * gamma ) );
	so0 = sqrt( 1 / ( (gamma * gamma ) + ( os * os ) ) * ( ( sgamma *  gamma * gamma ) 
	+ ( sos * os * os ) ) );
	cout << o0 << "\t" << so0 << endl;
	return 0;
}

%-------------------------------------------------------------------------------------------------

#include <iostream>
#include <vector>
#include <cmath>
#include <algorithm>
using namespace std;

int main()
{
 int n = 0;
 int m = 0;
 double temp;
 vector<double> data;
 vector<double> dist;
 while(cin >> temp)		//Riempie il vector DATA con i dati e li conta (n)
 {
	data.push_back(temp);
	n++;
 }


 sort(data.begin() , data.begin() + n);
 


 for (int i = 0 ; i < n - 1 ; i++)		//Trova i periodi
 {
	temp= ( data.at(i + 1) - data.at(i) ) * 2;	//Fattore 2 legato al fatto che serve un periodo, cioè il doppio della d
	dist.push_back(temp);	
	m++;
 }
 

 double sommasigma, somma, media, sigma = 0;
 for( int i = 0 ; i < m ; i++ )			//Calcolo media
 {
	somma += dist.at(i);
 }

 media = somma / m;

 for( int i = 0 ; i < m - 1; i++ )			//Calcolo sigma
 {
	sommasigma += ( dist.at(i) - media ) * ( dist.at(i) - media );
 }
 sigma = sqrt ( sommasigma / ( m - 1 ) );

 double puls, spuls;				//Converte tutto in frequenza [Hz]
 puls = (2 * M_PI ) / media; 
 spuls = ( 2 * M_PI * sigma ) / ( media * media);


 cout << puls << "\t" << spuls << endl;		//Output
 return 0;
}

%-------------------------------------------------------------------------------------------------

#include <iostream>
#include <cmath>
#include <vector>
using namespace std;
int main()
{ 
	int n = 0;
	vector <double> valori;
	double temp;						//Variabile temporanea contenente i valori
	double temp_neg;					//Variabile temporanea ausiliaria 1
	double temp_pos;					//Variabile temporanea ausiliaria 2
	while(cin >> temp)					//Mette i dati nel vector
	{
		valori.push_back(temp);
		n++;
	}
	int j = 0;
	vector <double> massimi;		//Crea vector in cui mettere i massimi
	vector <double> max_neg;		//Crea vector in cui mettere elemento precedente al massimo
	vector <double> max_pos;		//Crea vector in cui mettere elemento successivo al massimo
	vector <int> max_dist;			//Crea vector in cui scrivere posizione massimi
	vector <int> max_dist_neg;
	vector <int> max_dist_pos;
	double temp_dist;				//Crea variabile temporanea per la distanza dei valori
	double temp_dist_neg;
	double temp_dist_pos;

	for (int i = 1 ; i < ( n - 1 ) ; i++)
	{
		if (valori.at(i) > valori.at(i-1) && valori.at(i) >= valori.at(i+1) )	//Trova i massimi
		{
			temp = valori.at(i);
			temp_neg = valori.at(i-1);
			temp_pos = valori.at(i+1);
			temp_dist = (i + 1);
			temp_dist_neg = i ;
			temp_dist_pos = i + 2;
			massimi.push_back(temp);
			max_neg.push_back(temp_neg);
			max_pos.push_back(temp_pos);
			max_dist.push_back(temp_dist);
			max_dist_neg.push_back(temp_dist_neg);
			max_dist_pos.push_back(temp_dist_pos);

			j++;

		}
	}
	vector <double> minimi;
	vector <double> min_neg;
	vector <double> min_pos;
	vector <int> min_dist;
	vector <int> min_dist_neg;
	vector <int> min_dist_pos;
	j = 0;
	for (int i = 1 ; i < ( n - 1 ) ; i++)
	{
		if (valori.at(i) < valori.at(i-1) && valori.at(i) <= valori.at(i+1) )
		{
			temp = valori.at(i);
			temp_neg = valori.at(i-1);
			temp_pos = valori.at(i+1);
			temp_dist = i + 1;
			temp_dist_neg = i ;
			temp_dist_pos = i + 2;
			minimi.push_back(temp);
			min_neg.push_back(temp_neg);
			min_pos.push_back(temp_pos);
			min_dist.push_back(temp_dist);
			min_dist_neg.push_back(temp_dist_neg);
			min_dist_pos.push_back(temp_dist_pos);

			j++;
		}
	}



	double delta, da, db, dc, a, b, c, vortex, vortey;		//Parametri della parabola
	vector <double> xverticiMAX;					//Vector contenente i risultati del programma
	vector <double> yverticiMAX;

	double max_size = massimi.size();
	double min_size = minimi.size();

	for (int i = 0 ; i < max_size ; i++)
	{
		delta = (	( max_dist.at(i)     * max_dist_pos.at(i) * max_dist_pos.at(i) ) +
					( max_dist_neg.at(i) * max_dist.at(i)     * max_dist.at(i) )     +
					( max_dist_neg.at(i) * max_dist_neg.at(i) * max_dist_pos.at(i) ) -
					( max_dist_neg.at(i) * max_dist_neg.at(i) * max_dist.at(i) )     -
					( max_dist.at(i)     * max_dist.at(i)     * max_dist_pos.at(i) ) -
					( max_dist_neg.at(i) * max_dist_pos.at(i) * max_dist_pos.at(i) )	);


	da = (	( max_neg.at(i) * max_dist.at(i)     * max_dist_pos.at(i) * max_dist_pos.at(i) ) +
			( max_pos.at(i) * max_dist_neg.at(i) * max_dist.at(i)     * max_dist.at(i) )     +
			( massimi.at(i) * max_dist_neg.at(i) * max_dist_neg.at(i) * max_dist_pos.at(i) ) -
			( max_pos.at(i) * max_dist_neg.at(i) * max_dist_neg.at(i) * max_dist.at(i) )     -
			( max_neg.at(i) * max_dist.at(i)     * max_dist.at(i)     * max_dist_pos.at(i) ) - 
			( massimi.at(i) * max_dist_neg.at(i) * max_dist_pos.at(i) * max_dist_pos.at(i) ) );


	db = (	( massimi.at(i) * max_dist_pos.at(i) * max_dist_pos.at(i) ) +
			( max_neg.at(i) * max_dist.at(i)     * max_dist.at(i) )     +
			( max_pos.at(i) * max_dist_neg.at(i) * max_dist_neg.at(i) ) -
			( massimi.at(i) * max_dist_neg.at(i) * max_dist_neg.at(i) ) -
			( max_pos.at(i) * max_dist.at(i)     * max_dist.at(i) )     -
			( max_neg.at(i) * max_dist_pos.at(i) * max_dist_pos.at(i) ) 	);


	dc = (	( max_pos.at(i) * max_dist.at(i) )     +
			( massimi.at(i) * max_dist_neg.at(i) ) +
			( max_neg.at(i) * max_dist_pos.at(i) ) -
			( max_neg.at(i) * max_dist.at(i) )     -
			( massimi.at(i) * max_dist_pos.at(i) ) -
			( max_pos.at(i) * max_dist_neg.at(i) )	  );

		a = da / delta;
		b = db / delta;
		c = dc / delta;
		vortex = - b / ( 2 * c );
		vortey = - ( b * b - 4 * a * c ) / ( 4 * c );
		xverticiMAX.push_back(vortex * 0.05); //Presente valore di conversione delle x
		yverticiMAX.push_back(vortey); 
	}


	vector <double> xverticiMIN;
	vector <double> yverticiMIN;

	for (int i = 0 ; i < min_size ; i++)
	{
		delta = (	( min_dist.at(i)     * min_dist_pos.at(i) * min_dist_pos.at(i) ) +
					( min_dist_neg.at(i) * min_dist.at(i)     * min_dist.at(i) )    +
					( min_dist_neg.at(i) * min_dist_neg.at(i) * min_dist_pos.at(i) ) -
					( min_dist_neg.at(i) * min_dist_neg.at(i) * min_dist.at(i) )    -
					( min_dist.at(i)     * min_dist.at(i)     * min_dist_pos.at(i) ) -
					( min_dist_neg.at(i) * min_dist_pos.at(i) * min_dist_pos.at(i) )	);


	da = (	( min_neg.at(i) * min_dist.at(i)     * min_dist_pos.at(i) * min_dist_pos.at(i) ) +
			( min_pos.at(i) * min_dist_neg.at(i) * min_dist.at(i)     * min_dist.at(i) )     +
			( minimi.at(i)  * min_dist_neg.at(i) * min_dist_neg.at(i) * min_dist_pos.at(i) ) -
			( min_pos.at(i) * min_dist_neg.at(i) * min_dist_neg.at(i) * min_dist.at(i) )     -
			( min_neg.at(i) * min_dist.at(i)     * min_dist.at(i)     * min_dist_pos.at(i) ) - 
			( minimi.at(i)  * min_dist_neg.at(i) * min_dist_pos.at(i) * min_dist_pos.at(i) ) );


	db = (	( minimi.at(i)   * min_dist_pos.at(i) * min_dist_pos.at(i) ) +
			( min_neg.at(i)  * min_dist.at(i)     * min_dist.at(i) )     +
			( min_pos.at(i)  * min_dist_neg.at(i) * min_dist_neg.at(i) ) -
			( minimi.at(i)   * min_dist_neg.at(i) * min_dist_neg.at(i) ) -
			( min_pos.at(i)  * min_dist.at(i)     * min_dist.at(i) )     -
			( min_neg.at(i)  * min_dist_pos.at(i) * min_dist_pos.at(i) ) 	);


	dc = (	( min_pos.at(i) * min_dist.at(i) )     +
			( minimi.at(i)   * min_dist_neg.at(i) ) +
			( min_neg.at(i)  * min_dist_pos.at(i) ) -
			( min_neg.at(i)  * min_dist.at(i) )     -
			( minimi.at(i)   * min_dist_pos.at(i) ) -
			( min_pos.at(i) * min_dist_neg.at(i) )	  );


		a = da / delta;
		b = db / delta;
		c = dc / delta;		
		vortex = - b / ( 2 * c );
		vortey = - ( b * b - 4 * a * c ) / ( 4 * c );
		xverticiMIN.push_back(vortex * 0.05);
		yverticiMIN.push_back(vortey);
	}


	vector<int> eliminamassimi; //vector contenente la posizione dei valori da eliminare
	vector<int> eliminaminimi;
	int temperasemax;
	int temperasemin;

	

	for (int i = 0 ; i < max_size -1 ; i++) //Permette di trovare eventuali massimi fasulli
	{
		if (abs (xverticiMAX.at(i) - xverticiMAX.at(i+1)) < 0.8 ) //Intervallo considerato errore
		{
			if ( yverticiMAX.at(i)  -  yverticiMAX.at(i+1)  >= 0)
			{
				temperasemax = i+1;
				eliminamassimi.push_back(temperasemax);
			} else if ( yverticiMAX.at(i) -  yverticiMAX.at(i+1)  < 0 && temperasemax != i)
			 {
				temperasemax = i;
				eliminamassimi.push_back(temperasemax);
			 }
		}
	}


	for (int i = 0 ; i < min_size - 1; i++) //Permette di trovare eventuali minimi fasulli
	{
		if (abs (xverticiMIN.at(i) - xverticiMIN.at(i+1)) < 0.8 )
		{
			if ( yverticiMIN.at(i) - yverticiMIN.at(i+1) <= 0 && temperasemin != i )
			{
				temperasemin = i+1;
				eliminaminimi.push_back(temperasemin);
			} else if ( yverticiMIN.at(i) - yverticiMIN.at(i+1) > 0 )
			 {
				temperasemin = i;
				eliminaminimi.push_back(temperasemin);
			 }
		}
	} 


	int puliziamax = eliminamassimi.size();
	int puliziamin = eliminaminimi.size();
	int f = 0; //Variabile necessaria per regolare la modifica posizione vettori

	for (int i = 0 ; i < puliziamax ; i++) //Pulisce il vector definitivo di massimi
	{
		xverticiMAX.erase( xverticiMAX.begin() + eliminamassimi.at(i) - f );
		yverticiMAX.erase( yverticiMAX.begin() + eliminamassimi.at(i) - f );
		f++;
	}

	f = 0;

	for (int i = 0 ; i < puliziamin ; i++) //Pulisce il vector definitivo di minimi
	{
		xverticiMIN.erase( xverticiMIN.begin() + eliminaminimi.at(i) - f );
		yverticiMIN.erase( yverticiMIN.begin() + eliminaminimi.at(i) - f );
		f++;
	}

	double max_v_size  = xverticiMAX.size();
	double min_v_size = xverticiMIN.size();


		//cout << "Le x dei massimi valgono: " << endl;
	for (int i = 0 ; i < max_v_size ; i++)
	{
		cout << xverticiMAX.at(i) << "\t" << yverticiMAX.at(i) << endl;
	}


	//cout << "Le x dei minimi valgono: " << endl;
	for (int i = 0 ; i < min_v_size ; i++)
	{
		cout << xverticiMIN.at(i) << "\t" << yverticiMIN.at(i) << endl;
	}


/*

	 int q = 0;									//Algoritmo per la visualizzazione dei risultati
	cout << "Massimi: " << endl;
	for (double massimi)
	{
		cout << massimi.at(q) << endl;
		q++;
	}
	q = 0;
	cout << "A una posizione di: " << endl;
	while (max_dist.at(q) != 0)
	{
		cout << max_dist.at(q) << endl;
		q++;
	}


	q = 0;
	cout << "Minimi: " << endl;
	while (minimi.at(q) != 0)
	{
		cout << minimi.at(q) << endl;
		q++;
	}
	q = 0;
	cout << "A una posizione di: " << endl;
	while (min_dist.at(q) != 0)
	{
		cout << min_dist.at(q) << endl;
		q++;
	}
*/
	return 0;

}

%-------------------------------------------------------------------------------------------------

#include <iostream>
#include <vector>
#include <cmath>
using namespace std;
int main()
{
	double temp1, temp2;
	vector <double> datax;
	vector <double> datay;

	while(cin >> temp1)
	{
		cin >> temp2;
		datax.push_back(temp1);
		datay.push_back(temp2);
	}	

	int n =datax.size();

	for(int i = 0 ; i < n ; i++)
	{
		datay.at(i) = log( abs( datay.at(i) * 2 * M_PI ) );
		cout << datax.at(i) << "\t" << datay.at(i) << endl;
	}	
	return 0;
}

%-------------------------------------------------------------------------------------------------

Media pesata


#include <iostream>
#include <cmath>
#include <fstream>
#include <cstdlib>
#include <string>

using namespace std;

int main ()
{
    int n;
    cout << "dire di quanti valor si vuole calcolare la media" << endl;
    cin >> n;
    double* misure = new double [n];
    for (int i = 0 ; i < n ; i++)
    {
        cout << "inserire valore "<< i << " della media" << endl;
        cin >> misure[i];
    }

    double* sigme = new double [n];
    for (int i = 0 ; i < n ; i++)
    {
        cout << "inserire valore "<< i << " della sigma" << endl;
        cin >> sigme[i];
    }
    
    double sommavalsig;
    for (int i = 0 ; i < n ; i++)
    {
        sommavalsig += (misure[i]/sigme[i]);
    }
    
    double sommak;
    for (int i = 0 ; i < n ; i++)
    {
        sommak += (1/sigme[i]);
    }
    
    double mediapesata;
    mediapesata = (1/sommak)*(sommavalsig);
    
    cout << " Media pesata: " << mediapesata << endl;
    
    double errormediapesata;
    errormediapesata = sqrt(1/sommak);
    
    cout << " Error media pesata: " << errormediapesata << endl;
    
    return 0;
}



Calcolo Della Gamma

#include <vector>
#include <dirent.h>
#include <iostream>
#include <cmath>
#include <fstream>
#include <cstdlib>
#include <string>

using namespace std;


#include <vector>
using std::vector;

class funzione_punti_lineare
{
public:
	std::vector<double> vectx;
	std::vector<double> vecty;
    
	//x dev'essere ordinato?
	funzione_punti_lineare(std::vector<double> vx, std::vector<double> vy)
    {
		vectx = vx;
		vecty = vy;
	}
    
	long double operator()(double x)
    {
		long long i = 0;
		while(vectx.at(i) < x)
			i++;
        
		long double x0 = vectx.at(i);
		long double y0 = vecty.at(i);
        
		long double x1 = vectx.at(i+1);
		long double y1 = vecty.at(i+1);
        
		//Coeff. angolare, DeltaY/DeltaX
		long double m = (y1 - y0) / (x1 - x0);
		long double y = m*(x-x0)+y0;
        
		return y;
	}
};


%-------------------------------------------------------------------------------------------------


int main()
{
    
    fstream fout ("Valori.txt" , fstream::out);
    
    int n = 0;
    double x;
    double y;
	vector <double> valori_x;
    vector <double> valori_y;					//Variabile temporanea contenente i valori
	double temp_neg;					//Variabile temporanea ausiliaria 1
	double temp_pos = 0;				//Variabile temporanea ausiliaria 2
	while(cin >> x)					//Mette i dati nel vector
	{
        cin >> y;
		valori_x.push_back(x);
        valori_y.push_back(y);
		n++;
	}

	int j = 0;
	vector <double> zeri;		    //Crea vector in cui mettere zeri
    double temp_dist;				//Crea variabile temporanea per la distanza dei valori
	double temp_dist_neg;
	double temp_dist_pos;
    double temp_dist_no = 0;
    double temp_dist_forse;
    double temp_dist_si;

    for (int i = 0 ; i < ( n - 1 ) ; i++)
	{
        if (valori_y.at(i) == 0)
        {
            temp_neg = abs((valori_x.at(i) + valori_x.at(i + 1))/2) ;
			temp_dist = abs((temp_neg + temp_pos)/2);
            temp_pos = temp_neg;
            
            
            zeri.push_back(temp_dist);
            i++;
        }
		else if (valori_y.at(i) * valori_y.at(i+1) < 0 )	//Trova gli zeri
		{
        
            temp_neg = abs((valori_x.at(i) + valori_x.at(i + 1))/2) ;
			temp_dist = abs((temp_neg + temp_pos)/2);
            temp_pos = temp_neg;
            
            
            zeri.push_back(temp_dist);
        
			            
        
            if (zeri.size() > 1  )
            {
                
                
                cout << temp_dist << endl;
            }
            
            
			j++;
            
            
		}
            
	}
    
    funzione_punti_lineare f(valori_x,valori_y);
    
    for (double xzero : zeri)
    {
        fout << xzero << "\t" << f(xzero) << endl;
    }
    
    return 0;
}

\end{verbatim}
