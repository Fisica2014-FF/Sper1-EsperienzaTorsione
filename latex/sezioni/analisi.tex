Come detto nella descrizione dell'apparato strumentale, il tasso di rilevamento dei dati è di 20 al secondo. Questo corrisponde a una
frequenza di campionamento di 20 Hertz, di molto superiore al 
Nyquist rate necessario per il pendolo (il doppio della massima frequenza
necessaria), dato che come verificabile a vista ha una frequenza dell'ordine di 1 Hz. 
Non ci sono quindi problemi di aliasing e sottocampionamento.
Per quanto riguarda l'offset, è stato rifatta la calibrazione prima di ogni presa dati (inizio giornata) e si può vedere che era
 calibrato da un'evidente simmetria rispetto all'asse delle ascisse.
Per il calcolo dei massimi è stato utilizzato un programma che riconoscesse i punti di massimo e minimo approssimando
la funzione come una parabola in un intorno dei dati ``stazionari`` (dati massimi e minimi locali) usando il dato precedente e il successivo,
vincolando la parabola a passare per questi 3 punti e trovandone il vertice. L'errore legato all'utilizzo
 di questa approssimazione è, come noto dallo sviluppo di Taylor delle funzioni goniometriche, $o(x^3)$, che essendo lo step 0.05 è compreso tra $10^{-3}$ e $10^{-4}$,
  e trascurabile rispetto agli altri errori.
 Per una stima delle ampiezze legate alle frequenze di oscillazione sono stati presi i valori medi delle ordinate dei massimi (e dei valori assoluti dei minimi).

Una stima della pulsazione di risonanza è stata fatta con un processo di esplorazione iniziale che ha permesso, attraverso
il metodo di bisezione, di concentrarsi sull'area nella quale l'ampiezza era più alta. Il valore finale trovato risulta di...
Per stimare il coefficiente di smorzamento $\gamma$ legato al movimento dell'acqua è stato tentato un approccio diretto con gnuplot, ma i problemi del suo algoritmo 
(Levenberg–Marquardt, una forma di step gradient descent) nel caso di funzioni come questa, in cui il gradiente dei minimi quadrati è pieno di punti 
stazionari locali, ne hanno impedito l'applicabilità pratica. Quindi è stato scelto un'altro approccio che elimini questi problemi, in particolare limitando lo
studio a una semplice funzione esponenziale, che è stata ulteriormente semplificata in un fit lineare usando una scala logaritmica.
Di consequenza, sono stati cercati gli $x_i \mid f(x_i)=0$. Essendo la funzione
\begin{equation}
	\theta_0 e^{-\gamma t} sin(\omega_s t + \phi),
\end{equation}
poichè $e^{-\gamma t} > 0$ gli zeri della funzione sono solo gli zeri del seno. Quindi i punti medi 
$x_m = \frac{x_i + x_{i+1}}{2}$ fra gli zeri sono i punti in cui $sin(\omega_s t + \phi) = 1$.
Interpolando questi punti la funzione diventa dunque
\[
	\theta_0 e^{-\gamma t} \cdot 1 = \theta_0 e^{-\gamma t}.
\]
Interpolando questa funzione con i punti $(x_m,\log{f(x_m)})$ (le coordinate $f(x_m)$ sono state calcolate, nei casi in cui non fossero già un punto dei dati,
interpolando linearmente tra i due punti più prossimi).

%L'ampiezza massima della forzante è stata regolata a 10 milligiri perché altrimenti veniva
La pulsazione di smorzamento è stata ottenuta attraverso una media pesata delle pulsazioni ottenute dallo studio dei periodi dei
grafici durante la fase di smorzamento (vedasi tabella...)
La pulsazione propria è stata trovata attraverso la formula $\omega_0 =\sqrt{{\omega_s} ^ 2 + \gamma ^ 2 }$.
Gli errori sono stati stimati a partire da una stima diretta con la sommatoria degli scarti al quadrato diviso $N-1$.
I grafici rivelano che, entro gli errori casuali...

