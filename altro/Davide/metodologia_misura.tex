Come detto nella descrizione dell'apparato strumentale, il tasso di rilevamento dei dati è di 20 al secondo. Questo corrisponde a una
frequenza di campionamento di 20 Hertz, di molto superiore al 
limite di Nymquist necessario per il pendolo (il doppio della massima frequenza
necessaria), dato che come verificabile a vista ha una frequenza dell'ordine di 1 Hz. 
Non ci sono quindi problemi di aliasing e sottocampionamento.
Per quanto riguarda l'offset, è stato rifatta la calibrazione prima di ogni presa dati (inizio giornata) e si può vedere che era
 calibrato da un'evidente simmetria rispetto all'asse delle ascisse.
Per il calcolo dei massimi è stato utilizzato un programma che riconoscesse i punti di massimo e minimo approssimando
la funzione come una parabola in un intorno dei dati ``stazionari`` (dati massimi e minimi locali) usando il dato precedente e il successivo,
vincolando la parabola a passare per questi 3 punti e trovandone il vertice. L'errore legato all'utilizzo
 di questa approssimazione è $o(x^3)$, come noto dallo sviluppo di Taylor delle funzioni goniometriche.

Per una stima delle ampiezze legate alle frequenze di oscillazione sono stati presi i valori medi...

Una stima della pulsazione di risonanza è stata fatta con un processo di esplorazione iniziale che ha permesso, attraverso
il metodo di bisezione, di concentrarsi sull'area nella quale l'ampiezza era più alta. Il valore finale trovato risulta di...
Per stimare il coefficiente di smorzamento legato al movimento dell'acqua è stato ....
L'ampiezza massima della forzante è stata regolata a 10 milligiri perché...
La pulsazione di smorzamento è stata ottenuta attraverso una media pesata delle pulsazioni ottenute dallo studio dei periodi dei
grafici durante la fase di smorzamento (vedasi tabella...)
La pulsazione propria è stata trovata attraverso la formula $\omega_0 =\sqrt{{\omega_s} ^ 2 + \gamma ^ 2 }$
Gli errori sono stati stimati a partire da una stima diretta, infatti...
Per trovare i punti nel quale il seno è uguale a 1...
I grafici rivelano che, entro gli errori di...
