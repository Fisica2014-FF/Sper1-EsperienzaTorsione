% Adatta LaTeX alle convenzioni tipografiche italiane,
% e ridefinisce alcuni titoli in italiano, come "Capitolo" al posto di "Chapter",
% se il documento è in italiano
%\usepackage[italian]{babel}
%\usepackage[utf8]{inputenc} % Consente l'uso caratteri accentati italiani
%\usepackage{graphicx}		% Per le immagini
%\usepackage{gnuplot-lua-tikz}
%\usepackage[top=2.5cm, bottom=2cm, left=2cm, right=2cm]{geometry}

%\nonstopmode %non fermarti agli errori

%\usepackage{fancyhdr}
%\setlength{\headheight}{15.2pt}
%\pagestyle{fancy} % Solo le pagine normali, non i titoli nè la pagina iniziale


%%%%%%%%%%%%%%%%%%%%%%%%%%%%%%%%%%%%%%%%%%%%%%%%%%%%%%%%%%%%%%%%%%%%%%%%%%%%%%%%%%%%%%%%%

\usepackage{lipsum} % Package to generate dummy text throughout this template

%\usepackage[sc]{mathpazo} % Use the Palatino font
%\usepackage{tgpagella} % TeX Gyre Pagella, versione migliorata di Palatino. Si ma bo, no
%\usepackage{inconsolata}
\usepackage{textcomp}
\usepackage[scale=0.98,ttdefault]{AnonymousPro}

%%%%%
\usepackage{Alegreya} %% Option 'black' gives heavier bold face 
\renewcommand*\oldstylenums[1]{{\AlegreyaOsF #1}}

%\usepackage[euler-digits,euler-hat-accent]{eulervm}
\usepackage[euler-hat-accent]{eulervm}
%%%%%%

\usepackage[T1]{fontenc} % Use 8-bit encoding that has 256 glyphs

\usepackage[utf8]{inputenc} % Consente l'uso caratteri accentati italiani
\linespread{1.05} % Line spacing - Palatino needs more space between lines
\usepackage{amsmath, amsthm, amssymb, amsfonts}
\usepackage[italian]{babel}
\usepackage[kerning,spacing,babel]{microtype} % Slightly tweak font spacing for aesthetics

%%%%%%%%%%%%%%%%%%%%%%%%%%%%%%%%%%%%%%%%%%%%%
%Miei package

\usepackage{graphicx}		% Per le immagini
\usepackage{gnuplot-lua-tikz}
%%%%%%%%%%%%%%%%%%%%%%%%%%%%%%%%%%%%%%%%%%%%%
\usepackage[hmarginratio=1:1,top=32mm,columnsep=20pt]{geometry} % Document margins
\usepackage{multicol} % Used for the two-column layout of the document
\usepackage[hang, small,labelfont=bf,up,textfont=it,up]{caption} % Custom captions under/above floats in tables or figures
\usepackage{booktabs} % Horizontal rules in tables
\usepackage{float} % Required for tables and figures in the multi-column environment - they need to be placed in specific locations with the [H] (e.g. \begin{table}[H])
%\usepackage{hyperref} % For hyperlinks in the PDF
\usepackage{siunitx}

\usepackage{lettrine} % The lettrine is the first enlarged letter at the beginning of the text
\usepackage{paralist} % Used for the compactitem environment which makes bullet points with less space between them

\usepackage{abstract} % Allows abstract customization
\renewcommand{\abstractnamefont}{\normalfont\bfseries} % Set the "Abstract" text to bold
\renewcommand{\abstracttextfont}{\normalfont\small\itshape} % Set the abstract itself to small italic text

\usepackage{caption} % Per captions avanzate

\usepackage{listingsutf8} % Per includere codice sorgente meglio che con verbatim (e con caratteri non inglesi)
\lstset{ 
  %Preso anche questo da http://en.wikibooks.org/wiki/LaTeX/Source_Code_Listings
  %backgroundcolor=\color{white},   % choose the background color; you must add \usepackage{color} or \usepackage{xcolor}
  basicstyle=\footnotesize\ttfamily,        % the size of the fonts that are used for the code E MESSO IN MONOSPACE
  breakatwhitespace=false,         % sets if automatic breaks should only happen at whitespace
  breaklines=true,                 % sets automatic line breaking
  captionpos=b,                    % sets the caption-position to bottom
  %commentstyle=\color{mygreen},    % comment style
  %deletekeywords={...},            % if you want to delete keywords from the given language
  %escapeinside={\%*}{*)},          % if you want to add LaTeX within your code
  %extendedchars=true,              % lets you use non-ASCII characters; for 8-bits encodings only, does not work with UTF-8
  frame=l,                    % adds a frame around the code
				    %you can control the rules at the top, right, bottom, and left directly by using the four initial 
				    %letters for single rules and their upper case versions for double rules. http://mirror.hmc.edu/ctan/macros/latex/contrib/listings/listings.pdf
				    % Es frame frame=trBL ha doppia linea a sinistra e sotto, e singola a destra e sopra
  keepspaces=true,                 % keeps spaces in text, useful for keeping indentation of code (possibly needs columns=flexible)
  %keywordstyle=\color{blue},       % keyword style
  %language=Octave,                 % the language of the code
  %morekeywords={*,...},            % if you want to add more keywords to the set
  numbers=left,                    % where to put the line-numbers; possible values are (none, left, right)
  numbersep=5pt,                   % how far the line-numbers are from the code
  %numberstyle=\tiny\color{mygray}, % the style that is used for the line-numbers
  %rulecolor=\color{black},         % if not set, the frame-color may be changed on line-breaks within not-black text (e.g. comments (green here))
  showspaces=false,                % show spaces everywhere adding particular underscores; it overrides 'showstringspaces'
  showstringspaces=false,          % underline spaces within strings only
  showtabs=false,                  % show tabs within strings adding particular underscores
  stepnumber=1,                    % the step between two line-numbers. If it's 1, each line will be numbered
  %stringstyle=\color{mymauve},     % string literal style
  tabsize=2,                       % sets default tabsize to 2 spaces
  title=\lstname                   % show the filename of files included with \lstinputlisting; also try caption instead of title
}


\usepackage{titlesec} % Allows customization of titles
\renewcommand\thesection{\Roman{section}} % Roman numerals for the sections
\renewcommand\thesubsection{\Roman{subsection}} % Roman numerals for subsections
\titleformat{\section}[block]{\large\scshape\bfseries\centering}{\thesection.}{1em}{} % Change the look of the section titles
\titleformat{\subsection}[block]{\large}{\thesubsection.}{1em}{} % Change the look of the section titles

\usepackage{fancyhdr} % Headers and footers
\pagestyle{fancy} % All pages have headers and footers
\fancyhead{} % Blank out the default header
\fancyfoot{} % Blank out the default footer
\fancyhead[C]{Frau, Chiappara, Forcher - \textit{Pendolo a Torsione} $\bullet$ \thesection} % Custom header text
\fancyfoot[RO,LE]{\thepage} % Custom footer text

